\subsubsection{\centering \Large CHAPITRE VII}
\addcontentsline{toc}{section}{\protect\numberline{}{\scshape\bfseries VII} - \emph{Des termes complexes et de leur universalité ou particularité}}
\begin{center}\emph{\large\scshape Des termes complexes et de leur universalité ou particularité.}\end{center}

	\lettrine{O}{n} joint quelquefois à un terme divers autres termes qui composent dans notre esprit une idée totale, de laquelle il arrive souvent qu'on peut affirmer ou nier ce qu'on ne pourrait pas affirmer ou nier de chacun de ces termes étant séparés ; par exemple, ce sont des termes complexes, \emph{un homme prudent, un corps transparent; Alexandre, fils de Philippe}.

Cette addition se fait quelquefois par le pronom relatif, comme si je dis, \emph{un corps qui est transparent; Alexandre, qui est fils de Philippe; le pape, qui est vicaire de Jésus-Christ}.

Et on peut dire même que si ce relatif n'est pas toujours exprimé, il est toujours en quelque sorte sous-entendu, parce qu'il peut s'exprimer, si l'on veut, sans changer la proposition. Car c'est la même chose de dire, un corps transparent, ou un corps qui est transparent.

Ce qu'il y a de plus remarquable dans ces termes complexes, est que l'addition que l'on fait à un terme est de deux sortes : l'une qu'on peut appeler \emph{explication}, et l'autre \emph{détermination}.

Cette addition peut s'appeler seulement \emph{explication} quand elle ne fait que développer, ou ce qui était enfermé dans la compréhension de l'idée du premier terme, ou du moins ce qui lui convient comme un de ses accidents, pourvu qu'il lui convienne généralement et dans toute son étendue; comme si je dis : \emph{l'homme, qui est un animal doué de raison, ou l'homme qui désire naturellement d'être heureux}, ou \emph{l'homme, qui est mortel}. Ces additions ne sont que des explications, parce qu'elles ne changent point du tout l'idée du mot d'homme, et ne la restreignent point à ne signifier qu'une partie des hommes, mais marquent seulement ce qui convient à tous les hommes.

Toutes les additions qu'on ajoute aux noms qui marquent distinctement un individu, sont de cette sorte ; comme quand on dit : \emph{Paris, qui est la plus grande ville de l'Europe ; Jules César, qui été le plus grand capitaine du monde ; Aristote, le prince des philosophes; Louis XIV, roi de France}. Car les termes individuels, distinctement exprimés, se prennent toujours dans toute leur étendue, étant déterminés tout ce qu'ils le peuvent être.

L'autre sorte d'addition, qu'on peut appeler \emph{détermination}, est quand ce qu'on ajoute à un mot général en restreint la signification, et fait qu'il ne se prend plus pour ce mot général dans toute son étendue, mais seulement pour une partie de cette étendue ; comme si je dis : \emph{les corps transparents, les hommes savants, un animal raisonnable}. Ces additions ne sont point de simples explications, mais des déterminations, parce qu'elles restreignent l'étendue du premier terme, en faisant que le mot de corps ne signifie plus qu'une partie des corps, le mot d'homme, qu'une partie des hommes, le mot d'animal, qu'une partie des animaux.

Et ces additions sont quelquefois telles, qu'elles rendent individuel un mot général, quand on y ajoute des conditions individuelles, comme quand je dis, \emph{le pape qui est aujourd'hui}, cela détermine le mot général de Pape à la personne unique et singulière d'Alexandre VII le père d'Alexandre le Grand, cela détermine le mot général de père à un homme unique, parce qu'il n'y en peut avoir qu'un qui ait été père d'Alexandre.

On peut de plus distinguer deux sortes de termes complexes, les uns dans l'expression, et les autres dans le sens seulement.

Les premiers sont ceux dont l'addition est exprimée, tels que sont tous les exemples qu'on a rapportés jusqu'ici.

Les derniers sont ceux dont l'un des termes n'est point exprimé, mais seulement sous-entendu, comme quand nous disons en France le Roi, c'est un terme complexe dans le sens, parce que nous n'avons pas dans l'esprit, en prononçant ce mot de roi, la seule idée générale qui répond à ce mot; mais nous y joignons mentalement l'idée de Louis XIV, qui est maintenant roi de France. Il y a une infinité de termes dans les discours ordinaires des hommes qui sont complexes en cette manière, comme le nom de \emph{Monsieur} dans chaque famille, etc.

Il y a même des mots qui sont complexes dans l'expression pour quelque chose, et qui le sont encore dans le sens pour d'autres ; comme quand on dit, \emph{le prince des philosophes}, c'est un terme complexe dans l'expression, puisque le mot de prince est déterminé par celui de philosophe ; mais au regard d'Aristote, que l'on marque dans les écoles par ce mot, il n'est complexe que dans le sens, puisque l'idée d'Aristote n'est que dans l'esprit, sans être exprimée par aucun son qui le distingue en particulier.

Tous les termes connotatifs ou adjectifs, ou sont parties d'un terme complexe quand leur substantif est exprimé, ou sont complexes dans le sens quand il est sous-entendu ; car, comme il a été dit dans le Chapitre II, ces termes connotatifs marquent directement un sujet, quoique plus confusément, et indirectement une forme ou un mode, quoique plus distinctement; et ainsi ce sujet n'est qu'une idée fort générale et fort confuse, quelquefois d'un être, quelquefois d'un corps qui est pour l'ordinaire déterminé par l'idée distincte de la forme qui lui est jointe; comme \emph{Album} signifie une chose qui a de la blancheur ; ce qui détermine l'idée confuse de chose à ne représenter que celles qui ont cette qualité.

Mais ce qui est de plus remarquable dans ces termes complexes, est qu'il y en a qui sont déterminés dans la vérité à un seul individu, et qui ne laissent pas de conserver une certaine universalité équivoque qu'on peut appeler une équivoque d'erreur, parce que les hommes demeurant d'accord que ce terme ne signifie qu'une chose unique, faute de bien discerner quelle est véritablement cette chose unique, l'appliquent, les uns à une chose, et les autres à une autre ; ce qui fait qu'il a besoin d'être encore déterminé, ou par diverses circonstances, ou par la suite du discours, afin que l'on sache précisément ce qu'il signifie.

Ainsi le mot de \emph{véritable religion} ne signifie qu'une seule et unique religion, qui est dans la vérité la catholique, n'y ayant que celle-là de véritable. Mais parce que chaque peuple et chaque secte croit que sa religion est la véritable, ce mot est très équivoque dans la bouche des hommes, quoique par erreur. Et si on lit dans un historien qu'un prince a été zélé pour la véritable religion, on ne saurait dire ce qu'il a entendu par là, si on ne sait de quelle religion a été cet historien ; car si c'est un protestant, cela voudra dire la religion protestante ; si c'est un Arabe mahométan qui parlât ainsi de son prince, cela voudrait dire la religion mahométane, et on ne pourrait juger que ce serait la religion catholique, si on ne savait que cet historien était catholique.

Les termes complexes, qui sont ainsi équivoques par erreur, sont principalement ceux qui enferment des qualités dont les sens ne jugent point, mais seulement l'esprit, sur lesquelles il est facile que les hommes aient divers sentiments.

Si je dis par exemple : Il n'y avait que des hommes de six pieds qui fussent enrôlés dans l'armée de Marius, ce terme complexe d'hommes de six pieds n'est pas sujet à être équivoque par erreur, parce qu'il est bien aisé de mesurer des hommes, pour juger s'ils ont six pieds. Mais si l'on eût dit qu'on ne devait enrôler que de vaillants hommes, le terme de vaillants hommes eût été plus sujet à être équivoque par erreur, c'est-à-dire à être attribué à des hommes qu'on eût crus vaillants, et qui ne l'eussent pas été en effet.

Les termes de comparaison sont aussi fort sujets à être équivoques par erreur. \emph{Le plus grand géomètre de Paris, le plus savant homme, le plus adroit, le plus riche}. Car, quoique ces termes soient déterminés par des conditions individuelles, n'y ayant qu'un seul homme qui soit le plus grand géomètre de Paris, néanmoins ce mot peut être facilement attribué à plusieurs, quoiqu'il ne convienne qu'à un seul dans la vérité, parce qu'il est fort aisé que les hommes soient partagés de sentiments sur ce sujet, et qu'ainsi plusieurs donnent ce nom à celui que chacun croit avoir cet avantage par-dessus les autres.

Les mots de, \emph{sens d'un auteur, la doctrine d'un auteur sur un tel sujet}, sont encore de ce nombre, surtout quand un auteur n'est pas si clair qu'on ne dispute quelle a été son opinion, comme nous voyons que les philosophes disputent tous les jours touchant les opinions d'Aristote, chacun le tirant de son côté. Car, quoique Aristote n'ait qu'un seul et unique sens sur un tel sujet, néanmoins, comme il est différemment entendu, ces mots de sentiment d'Aristote sont équivoques par erreur, parce que chacun appelle \emph{sentiment d'Aristote} ce qu'il a compris être son véritable sentiment; et ainsi, l'un comprenant une chose et l'autre une autre, ces termes de sentiment d'Aristote sur un tel sujet, quelque individuels qu'ils soient en eux-mêmes, pourront convenir à plusieurs choses, à savoir : à tous les divers sentiments qu'on lui aura attribués, et ils signifieront dans la bouche de chaque personne ce que chaque personne aura conçu être le sentiment de ce philosophe.

Mais, pour mieux comprendre en quoi consiste l'équivoque de ces termes, que nous avons appelés équivoques par erreur, il faut remarquer que ces mots sont connotatifs, ou expressément, ou dans le sens. Or, comme nous avons déjà dit, on doit considérer, dans les mots connotatifs, le sujet, qui est directement, mais confusément exprimé, et la forme ou le mode, qui est distinctement, quoique indirectement exprimé. Ainsi, le blanc signifie confusément un corps, et la blancheur distinctement; sentiment d'Aristote signifie confusément quelque opinion, quelque pensée, quelque doctrine, et distinctement la relation de cette pensée à Aristote, auquel on l'attribue.

Or, quand il arrive de l'équivoque dans ces mots, ce n'est pas proprement à cause de cette forme ou de ce mode, qui étant distinct est invariable. Ce n'est pas aussi à cause du sujet confus, lorsqu'il demeure dans cette confusion. Car, par exemple, le mot de prince des philosophes ne peut jamais être équivoque, tant qu'on n'appliquera cette idée de prince des philosophes à aucun individu distinctement connu. Mais l'équivoque arrive seulement parce que l'esprit, au lieu de ce sujet confus, y substitue souvent un sujet distinct et déterminé, auquel il attribue la forme et le mode. Car, comme les hommes sont de différents avis sur ce sujet, ils peuvent donner cette qualité à diverses personnes, et les marquer ensuite par ce mot, qu'ils croient leur convenir, comme autrefois on entendait Platon par le nom de prince des philosophes, et maintenant on entend Aristote.

Le mot de, véritable religion, n'étant point joint avec l'idée distincte d'aucune religion particulière, et demeurant dans son idée confuse, n'est point équivoque: puisqu'il ne signifie que ce qui est en effet la véritable religion. Mais lorsque l'esprit a joint cette idée de véritable religion à une idée distincte d'un certain culte particulier distinctement connu, ce mot devient très équivoque, et signifie dans la bouche de chaque peuple le culte qu'il prend pour véritable.

Il en est de même de ces mots, \emph{sentiment d'un tel philosophe sur une telle matière}. Car, demeurant dans leur idée générale, ils signifient simplement et en général la doctrine que ce philosophe a enseignée sur cette matière, comme ce qu'a enseigné Aristote sur la nature de notre âme: \emph{id quod sensit talis scriptor} ; et cet \emph{id}, c'est-à-dire cette doctrine, demeurant dans son idée confuse sans être appliquée à une idée distincte, ces mots ne sont nullement équivoques ; mais lorsqu'au lieu de cet \emph{id} confus, de cette doctrine confusément conçue, l'esprit substitue une doctrine distincte et un sujet distinct, alors, selon les différentes idées distinctes qu'on y pourra substituer, ce terme deviendra équivoque. Ainsi, l'opinion d'Aristote touchant la nature de notre âme est un mot équivoque dans la bouche de Pomponace, qui prétend qu'il l'a crue mortelle, et dans celle de plusieurs autres interprètes de ce philosophe, qui prétendent, au contraire, qu'il l'a crue immortelle, aussi bien que ses maîtres Platon et Socrate. Et de là il arrive que ces sortes de mots peuvent souvent signifier une chose à qui la forme exprimée indirectement ne convient pas. Supposant, par exemple, que Philippe n'ait pas été véritablement père d'Alexandre, comme Alexandre lui-même le voulait faire croire, le mot de, \emph{fils de Philippe}, qui signifie en général celui qui a été engendré par Philippe, étant appliqué par erreur à Alexandre, signifiera une personne qui ne serait pas véritablement le fils de Philippe. Le mot de, \emph{sens de l'Écriture} étant appliqué par un hérétique à une erreur contraire à l'Écriture, signifiera dans sa bouche cette erreur qu'il aura crue être le sens de l'Écriture, et qu'il aura, dans cette pensée, appelée le sens de l'Écriture. C'est pourquoi les calvinistes n'en sont pas plus catholiques, pour protester qu'ils ne suivent que la parole de Dieu, car ces mots, de \emph{parole de Dieu}, signifient dans leur bouche toutes les erreurs qu'ils prennent faussement pour la parole de Dieu.

