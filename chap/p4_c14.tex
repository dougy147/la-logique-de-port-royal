\subsubsection{\centering \Large CHAPITRE XIV}
\addcontentsline{toc}{section}{\protect\numberline{}{\scshape\bfseries XIV} - \emph{Autre remarque sur le sujet de la croyance des événements}}
\begin{center}\emph{\large\scshape Autre remarque sur le sujet de la croyance des événements.}\end{center}


	\lettrine{I}{l} y a encore une autre remarque très importante à faire sur la croyance des événements. C'est qu'entre les circonstances qu'on doit considérer pour juger si on doit les croire, ou si on ne doit pas les croire, il y en a qu'on peut appeler des circonstances communes, parce qu'elles se rencontrent en beaucoup de faits, et qu'elles se trouvent incomparablement plus souvent jointes à la vérité qu'à la fausseté; et alors, si elles ne sont point contrebalancées par d'autres circonstances particulières qui affaiblissent ou qui ruinent dans notre esprit les motifs de croyance qu'il tirait de ces circonstances communes, nous avons raison de croire ces événements, sinon certainement, au moins très probablement: ce qui nous suffit quand nous sommes obligés d'en juger; car comme nous nous devons contenter d'une certitude morale dans les choses qui ne sont pas susceptibles d'une certitude métaphysique, lors aussi que nous ne pouvons pas avoir une entière certitude morale, le mieux que nous puissions faire, quand nous sommes engagés à prendre parti, est d'embrasser le plus probable, puisque ce serait un renversement de la raison d'embrasser le moins probable.

Que si, au contraire, ces circonstances communes, qui nous auraient portés à croire une chose, se trouvent jointes à d'autres circonstances particulières qui ruinent dans notre esprit, comme nous venons de dire, les motifs de croyance qu'il tirait de ces circonstances communes ; ou qui même soient telles qu'il soit fort rare que de semblables circonstances ne soient pas accompagnées de fausseté, nous n'avons plus alors la même raison de croire cet événement : mais, ou notre esprit demeure en suspens, si les circonstances particulières ne font qu'affaiblir le poids des circonstances communes ; ou il se porte à croire que le fait est faux, si elles sont telles qu'elles soient ordinairement des marques de fausseté. Voici un exemple qui peut éclaircir cette remarque.

C'est une circonstance commune à beaucoup d'actes d'être signés par deux notaires, c'est-à-dire par deux personnes publiques qui ont d'ordinaire grand intérêt à ne point commettre de fausseté, parce qu'il y va non seulement de leur conscience et de leur honneur, mais aussi de leur bien et de leur vie. Cette seule considération suffit, si nous ne savons point d'autres particularités d'un contrat, pour croire qu'il n'est point antidaté; non qu'il n'y en puisse avoir d'antidatés, mais parce qu'il est certain que de mille contrats, il y en a neuf cent quatre-vingt-dix-neuf qui ne le sont point : de sorte qu'il est incomparablement plus probable que ce contrat que je vois est l'un des neuf cent quatre-vingt-dix-neuf, que non pas qu'il soit cet unique qui entre mille peut se trouver antidaté. Que si la probité des notaires qui l'ont signé m'est parfaitement connue, je tiendrai alors pour très certain qu'ils n'y auront point commis de fausseté.

Mais si, à cette circonstance commune d'être signé par deux notaires, qui m'est une raison suffisante, quand elle n'est point combattue par d'autres, d'ajouter foi à la date d'un contrat, on y joint d'autres circonstances particulières, comme que ces notaires soient diffamés pour être sans honneur et sans conscience, et qu'ils aient pu avoir un grand intérêt à cette falsification, cela ne me fera pas encore conclure que ce contrat est antidaté, mais diminuera le poids qu'aurait eu sans cela dans mon esprit la signature des deux notaires pour me faire croire qu'il ne le serait pas. Que si, de plus, je puis découvrir d'autres preuves positives de cette antidate, ou par témoins, ou par des arguments très forts, comme serait l'impuissance où un homme aurait été de prêter vingt mille écus en un temps où l'on montrerait qu'il n'aurait pas eu cent écus vaillant, je me déterminerai alors à croire qu'il y a de la fausseté dans ce contrat; et ce serait une prétention très déraisonnable de vouloir m'obliger, ou à ne pas croire ce contrat antidaté, ou à reconnaître que j'avais tort de supposer que les autres où je ne voyais pas les marques mêmes de fausseté ne l'étaient pas, puisqu'ils pouvaient l'être comme celui-là.

On peut appliquer tout ceci à des matières qui causent souvent des disputes parmi les doctes. On demande si un livre est véritablement d'un auteur dont il a toujours porté le nom; ou si les actes d'un concile sont vrais ou supposés.

Il est certain que le préjugé est pour l'auteur, qui est depuis longtemps en possession d'un ouvrage, et pour la vérité des actes d'un concile que nous lisons tous les jours, et qu'il faut des raisons considérables pour nous faire croire le contraire, nonobstant ce préjugé.

C'est pourquoi un fort habile homme de ce temps ayant voulu montrer que la lettre de saint Cyprien au pape Étienne, sur le sujet de Martien, évêque d'Arles, n'est pas de ce saint martyr, il n'en a pu persuader les savants, ses conjectures ne leur ayant pas paru assez fortes pour ôter à saint Cyprien une pièce qui a toujours porté son nom, et qui a une parfaite ressemblance de style avec ses ouvrages.

C'est en vain aussi que Blondel et Saumaise, ne pouvant répondre à l'argument qu'on tire des lettres de saint Ignace pour la supériorité de l'évêque au-dessus des prêtres dès le commencement de l'Église, ont voulu prétendre que toutes ces lettres étaient supposées, selon même qu'elles ont été imprimées par Isaac Vossius et Ussérius sur l'ancien manuscrit grec de la bibliothèque de Florence ; et ils ont été réfutés par ceux même de leur parti, parce qu'avouant, comme ils font, que nous avons les mêmes lettres qui ont été citées par Eusèbe, par saint Jérôme, par Théodoret, et même par Origène, il n'y a nulle apparence que les lettres de saint Ignace, ayant été recueillies par saint Polycarpe, ces véritables lettres soient disparues, et qu'on en ait supposé d'autres dans le temps qui s'est passé entre saint Polycarpe et Origène, ou Eusèbe; outre que ces lettres de saint Ignace, que nous avons maintenant, ont un certain caractère de sainteté et de simplicité si propre à ces temps apostoliques, qu'elles se défendent toutes seules contre ces vaines accusations de supposition et de fausseté.

Enfin, toutes les difficultés que le cardinal du Perron a proposées contre la lettre du concile d'Afrique au pape saint Célestin, touchant les appellations au saint siège, n'ont point empêché que l'on n'ait cru depuis, comme auparavant, qu'elle a été véritablement écrite par ce concile.

Mais il y a néanmoins d'autres rencontres où les raisons particulières l'emportent sur cette raison générale d'une longue possession.

Ainsi, quoique la lettre de saint Clément à saint Jacques, évêque de Jérusalem, ait été traduite par Ruffin, il y a près de treize cents ans, et qu'elle soit alléguée comme étant de saint Clément par un concile de France, il y a plus de douze cents ans, il est toutefois difficile de ne pas avouer qu'elle est supposée, puisque ce saint évêque de Jérusalem ayant été martyrisé avant saint Pierre, il est impossible que saint Clément lui ait écrit depuis la mort de saint Pierre, comme le suppose cette lettre.

De même, quoique les commentaires sur saint Paul, attribués à saint Ambroise, aient été cités sous son nom par un très grand nombre d'auteurs, et l'œuvre imparfaite sur saint Mathieu sous celui de saint Chrysostome, tout le monde néanmoins convient aujourd'hui qu'ils ne sont pas de ces saints, mais d'autres auteurs anciens engagés dans beaucoup d'erreurs.

Enfin, les Actes que nous voyons dans les conciles de Sinuesse sous Marcellin, de deux ou trois de Rome sous saint Sylvestre, et d'un autre de Rome sous Sixte III, seraient suffisants pour nous persuader de la vérité de ces conciles, s'ils ne contenaient rien que de raisonnable, et qui eût du rapport au temps qu'on attribue à ces conciles; mais ils en contiennent tant de déraisonnables, et qui ne conviennent point à ces temps-là, qu'il y a grande apparence qu'ils sont faux et supposés.

Voilà quelques remarques qui peuvent servir en ces sortes de jugements : mais il ne faut pas s'imaginer qu'elles soient de si grand usage qu'elles empêchent toujours qu'on ne s'y trompe. Tout ce qu'elles peuvent au plus, est de faire éviter les fautes les plus grossières, et d'accoutumer l'esprit à ne pas se laisser emporter par des lieux communs, qui, ayant quelque vérité en général, ne laissent pas d'être faux en beaucoup d'occasions particulières, ce qui est une des plus grandes sources des erreurs des hommes.

