\subsubsection{\centering \Large CHAPITRE VII}
\addcontentsline{toc}{section}{\protect\numberline{}{\scshape\bfseries VII} - \emph{Règles, modes et fondements de la troisième figure}}
\begin{center}\emph{\large\scshape Règles, modes et fondements de la troisième figure.}\end{center}

	\lettrine{D}{ans} la troisième figure le moyen est deux fois sujet. D'où il s'ensuit :

\begin{center}{\bfseries\scshape\large 1. Règle}\end{center}

\emph{Que la mineure doit être affirmative.}

Ce que nous avons déjà prouvé par la première règle de la première figure ; parce que dans l'une et dans l'autre, l'attribut de la conclusion est aussi attribut dans la majeure.

\begin{center}{\bfseries\scshape\large 2. Règle}\end{center}

\emph{L'on n'y peut conclure que particulièrement.}

Car la mineure étant toujours affirmative le petit terme qui y est attribut est particulier. Donc, il ne peut être universel dans la conclusion où il est sujet, parce que ce serait conclure le général du particulier, contre la deuxième règle générale.

\bigbreak

\begin{center}{\bfseries Démonstration}\end{center}

\emph{Qu'il ne peut y avoir que six modes dans la troisième figure.}

Des dix modes concluants, A, E, E, et A, O, O, sont exclus par la première règle de cette figure, qui est, que la mineure ne peut être négative.

A, A, A, et E, A, E, sont exclus par la deuxième règle, qui est, que la conclusion n'y peut être générale.

Il ne reste donc que ces six modes :


\begin{center}
$ \text {3 Affirmatifs} \left \{
    \begin{array}{ccc}
	    \text {A,} & \text{A,} & \text{I} \\
  	    \text {A,} & \text{I,} & \text{I} \\
  	    \text {I,} & \text{A,} & \text{I} \\
    \end{array}
	    \right \} $
\end{center}
\begin{center}
$ \text {3 Négatifs} \left \{
    \begin{array}{ccc}
	    \text {E,} & \text{A,} & \text{O} \\
	    \text {E,} & \text{I,} & \text{O} \\
	    \text {O,} & \text{A,} & \text{O} \\
    \end{array}
	    \right \} $
\end{center}

Ce qu'il fallait démontrer.

C'est ce qu'on a réduit à ces six mots artificiels, quoique dans un autre ordre.

\newpage
	\begin{tabularx}{\textwidth}{lX}
		{\large\scshape Da— } & \emph{Tous les vrais Chrétiens sont contents de leur état:} \\
		{\large\scshape Rap—} & \emph{Tous les vrais Chrétiens sont persécutés:} \\
		{\large\scshape Ti. } & \emph{Donc il y a des personnes persécutées, qui sont contentes de leur état.} \\
		{\large\scshape Fe— } & \emph{Nul saint n'est haï de Dieu:} \\
		{\large\scshape Lap—} & \emph{Tout saint est affligé en ce monde:} \\
		{\large\scshape Ton.} & \emph{Donc il y a des personnes affligées en ce monde qui ne sont pas haïes de Dieu.} \\
		{\large\scshape Di— } & \emph{Il y a des vicieux qui sont estimés dans le monde:} \\
		{\large\scshape Sa— } & \emph{Tous les vicieux sont dignes de mépris:} \\
		{\large\scshape Mis.} & \emph{Donc il y a des personnes dignes de mépris qui sont estimées dans le monde:} \\
		{\large\scshape Da— } & \emph{Toute amitié est agréable:} \\
		{\large\scshape Ti— } & \emph{Il y a des amitiés dangereuses:} \\
		{\large\scshape Si. } & \emph{Donc il y a des choses dangereuses qui sont agréables.} \\
		{\large\scshape Bo— } & \emph{Il y a des avares qui ne sont pas riches:} \\
		{\large\scshape Car—} & \emph{Tous les avares sont passionnés pour le bien:} \\
		{\large\scshape Do. } & \emph{Donc il y a des personnes passionnées pour le bien qui ne sont pas riches:} \\
		{\large\scshape Fe— } & \emph{Nul saint n'est méprisable:} \\
		{\large\scshape Ri— } & \emph{Il y a des saints ignorants :} \\
		{\large\scshape Son.} & \emph{Donc il y a des ignorants qui ne sont pas méprisables.} \\
	\end{tabularx}


\begin{center}{\bfseries Fondement de la troisième figure}\end{center}

Les deux termes de la conclusion étant attribués dans les deux prémisses à un même terme qui sert de moyen, on peut déduire les modes affirmatifs de cette figure à ce principe :

\begin{center}{\bfseries Principe des modes affirmatifs}\end{center}

\emph{Lorsque deux termes se peuvent affirmer d'une même chose, ils se peuvent aussi affirmer l'un de l'autre pris particulièrement.}

Car étant unis ensemble dans cette chose, puisqu'ils lui conviennent, il s'ensuit qu'ils sont quelquefois unis ensemble, et, partant, que l'on peut les affirmer l'un de l'autre particulièrement ; mais, afin qu'on soit assuré que ces deux termes aient été affirmés d'une même chose, qui est le moyen, il faut que ce moyen soit pris au moins une fois universellement, car s'il était pris deux fois particulièrement, ce pourrait être deux diverses parties d'un terme commun, qui ne serait pas la même chose.

\begin{center}{\bfseries Principe des modes négatifs}\end{center}

\emph{Lorsque de deux termes l'un peut être nié et l'autre affirmé de la même chose, ils peuvent se nier particulièrement l'un de l'autre}. Car il est certain qu'ils ne sont pas toujours joints ensemble, puisqu'ils n'y sont pas joints dans cette chose : donc on peut les nier quelquefois l'un de l'autre, c'est-à-dire que l'on peut les nier l'un de l'autre pris particulièrement; mais il faut, par la même raison, qu'afin que ce soit la même chose, le moyen soit pris au moins une fois universellement.

