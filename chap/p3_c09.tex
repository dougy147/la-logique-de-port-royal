\subsubsection{\centering \Large CHAPITRE IX}
\addcontentsline{toc}{section}{\protect\numberline{}{\scshape\bfseries IX} - \emph{De la réduction des syllogismes}}
\begin{center}\emph{\large\scshape De la réduction des syllogismes.}\end{center}

\begin{center}{\footnotesize Ce Chapitre est fort inutile.}\end{center}

	\lettrine{R}{éduire} un syllogisme c'est le mettre dans une forme plus parfaite, plus évidente, et plus naturelle. Ainsi toutes les réductions doivent être fondées sur ce qu'il y a des arguments plus clairs et plus directs les uns que les autres, et que les moins directs peuvent être réduits aux plus directs, les moins clairs aux plus clairs. Ce qui se fait en changeant quelque proposition, ou simplement, en rendant seulement le sujet attribut, et l'attribut sujet, ou par accident; en changeant aussi la quantité de la proposition. On suppose ordinairement dans l'école que les arguments de la première figure étant les plus directs, il y faut réduire tous les autres. Et pour le faire on fait remarquer dans les mots artificiels qui comprennent les modes des trois autres figures.

\bigbreak
$1$. La première consone, à savoir C, ou D, ou F, qui marquent que les modes qui commencent par C, à savoir \emph{Camestres, Cesare, Calentes} se réduisent à \emph{Celarent}, etc., ceux qui commencent par D, à \emph{Dari}; et ceux par F, à \emph{Ferio}.

\bigbreak
$2$. On remarque des consones qui terminent les syllabes, ces trois lettres,

S, qui marque que la proposition où elle est se doit convertir simplement.

P, qui marque que la proposition où elle est se doit convertir avec changement de la quantité de la proposition.

M, qui marque que l'on doit transposer les propositions.

\bigbreak
$3$. On remarque les syllabes, et on en attribue une à chaque proposition. La première à la majeure; la deuxième à la mineure; la troisième à la conclusion. Et afin de retenir toutes ces règles plus facilement on en a fait ces deux vers :

\begin{center}
	\begin{tabular}{l}
		\emph{S. vult simpliciter verti, P. vero per acci:} \\
		\emph{M. vult transponi, C. per impossibile duci.} \\
	\end{tabular}
\end{center}

En observant ces règles on fera toutes sortes de réductions, comme il est facile de le voir si on en veut faire l'essai dans les exemples qu'on a proposés sur chaque figure.

On avait fait sur cette matière diverses observations nouvelles qui se trouveront encore dans les copies manuscrites que plusieurs avaient fait faire de cette Logique. Mais comme tout cela est de nul usage, et qu'on ne l'eut pu comprendre qu'avec une assez forte attention d'esprit, on a jugé plus à propos de le retrancher.

On remarquera seulement; qu'on peut bien dire qu'une conclusion n'est pas directement tirée des prémisses, mais qu'on ne peut pas dire proprement qu'elle soit indirecte; parce que l'on a supposé déjà formée avant que l'on ait songé à la prouver. Étant donc déjà formée on y peut bien appliquer des prémisses qui lui soient jointes indirectement, mais on ne peut pas dire pour cela qu'elle soit indirecte.

Et de là il s'ensuit que les arguments qui ne se peuvent réduire qu'en changeant la conclusion, ne se réduisent pas proprement. Et qu'ainsi \emph{Calentes} et \emph{Camestres} ne se peuvent réduire à \emph{Celarent}, ni \emph{Disamis} et \emph{Dibatis} à \emph{Dari}, parce qu'il faudrait changer la conclusion.

Quant à ce qu'on appelle réduction à l'impossible. Elle consiste à obliger une personne qui nie mal à propos une conclusion à accorder la contradictoire d'une proposition déjà accordée. Cela se fait en prenant la contradictoire de la conclusion niée, laquelle jointe avec quelqu'une des prémisses produit nécessairement la contradictoire d'une des propositions accordées. Ainsi si quelqu'un avait nié la conclusion de cet argument en Bocardo.

\begin{center}
	\begin{tabular}{l}
		Il y a des saints qui ne sont pas riches: \\
		Tous les saints sont heureux: \\
		Il y a donc des heureux qui ne sont pas riches. \\
	\end{tabular}
\end{center}

On pourrait prendre la contradictoire de la conclusion niée, et dire que s'il est faux qu'il y ait des heureux qui ne soient pas riches, il est donc vrai que tous les heureux sont riches.

Or tous les saints sont heureux, comme on l'a accordé dans la mineure : Donc tous les saints sont riches, ce qui est la contradictoire de la majeure du premier argument.

Si l'on y joignait la majeure accordée, on en ferait cet autre argument :

\begin{center}
	\begin{tabular}{l}
		Tous les heureux sont riches. \\
		Or il y a des saints qui ne sont pas riches : \\
		Il y a donc des saints qui ne sont pas heureux ; \\
	\end{tabular}
\end{center}

ce qui est la contradictoire de la mineure accordée.

\bigbreak
Il est facile de voir par le sens commun comment il faut arranger ces propositions pour en tirer la contradictoire d'une des prémisses accordées; c'est pourquoi on ne s'arrêtera pas à expliquer les règles que l'on en donne.

Mais l'on serait encore obligé de dire sur cette sorte de réduction ce que l'on a dit sur l'autre, qui est, qu'elle n'est presque de nul usage. C'est une chose fort rare que l'on nie la conclusion d'un argument fait selon les règles. Et si cela arrivait entre des personnes qui agiraient de bonne foi, ce ne pourrait être qu'à cause de l'embarras de quelques termes : et en ce cas le moyen ordinaire dont on se sert pour montrer que l'argument dont on doute est bon, n'est pas de réduire à l'impossible celui qui en a nié la conclusion, mais de faire un autre argument semblable composé de termes plus clairs et plus simples qui paraisse clairement bon. Comme la manière de montrer qu'un argument est mauvais n'est pas de faire voir qu'il est contre les règles, ce qui est toujours embarrassé et peu sensible, mais d'en faire un de même sorte qui soit évidemment mauvais.

