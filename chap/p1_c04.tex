\subsubsection{\centering \Large CHAPITRE IV}
\addcontentsline{toc}{section}{\protect\numberline{}{\scshape\bfseries IV} - \emph{Des idées considérées selon leur composition ou simplicité. Où il est parlé de la manière de connaître par abstraction ou précision}}

\begin{center}\emph{\large\scshape Des idées considérées selon leur composition ou simplicité. Où il est parlé de la manière de connaître par abstraction ou précision.}\end{center}

	\lettrine{C}{e} que nous avons dit en passant dans le Chapitre II, que nous Pouvions considérer un mode sans faire une réflexion distincte sur la substance dont il est mode, nous donne occasion d'expliquer ce qu'on appelle \emph{abstraction d'esprit}.

Notre esprit étant fini et borné ne peut comprendre parfaitement les choses un peu composées, qu'en les considérant par parties, et comme par les diverses faces qu'elles peuvent recevoir. C'est ce qu'on peut appeler généralement connaître par abstraction.

Mais comme les choses sont différemment composées, et qu'il y en a qui le sont de parties réellement distinctes, qu'on appelle parties intégrantes, comme le corps humain, les diverses parties d'un nombre, il est bien facile alors de concevoir que notre esprit peut s'appliquer à considérer une partie sans considérer l'autre, parce que ces parties sont réellement distinctes, et ce n'est pas même ce qu'on appelle \emph{abstraction}.

Or, il est si utile dans ces choses-là même de considérer plutôt les parties séparément que le tout, que sans cela on ne peut avoir presque aucune connaissance distincte. Car, par exemple, le moyen de pouvoir connaître le corps humain, qu'en le divisant en toutes ses parties similaires et dissimilaires, et en leur donnant à toutes différents noms? Toute l'arithmétique est aussi fondée sur cela; car on n'a pas besoin d'art pour compter les petits nombres, parce que l'esprit les peut comprendre tout entiers ; et ainsi tout l'art consiste à compter par parties ce qu'on ne pourrait compter par le tout, comme il serait impossible, quelque étendue d'esprit qu'on eût, de multiplier deux nombres de huit ou neuf caractères chacun, en les prenant tout entiers.

La seconde connaissance par parties, est quand on considère un mode sans faire attention à la substance, ou deux modes qui sont joints ensemble dans une même substance en les regardant chacun à part. C'est ce qu'ont fait les géomètres qui ont pris pour objet de leur science le corps étendu en longueur, largeur et profondeur. Car, pour le mieux connaître, ils se sont premièrement appliqués à le considérer, selon une seule dimension qui est la longueur; et alors ils lui ont donné le nom de ligne. Ils l'ont considéré ensuite selon deux dimensions, la longueur et la largeur, et ils l'ont appelé surface. Et puis, considérant toutes les trois dimensions ensemble, longueur, largeur et profondeur, ils l'ont appelé solide ou corps.

On voit par là combien est ridicule l'argument de quelques sceptiques qui veulent faire douter de la certitude de la géométrie, parce qu'elle suppose des lignes et des surfaces qui ne sont point dans la nature ; car les géomètres ne supposent point qu'il y ait des lignes sans largeur ou des surfaces sans profondeur ; mais ils supposent seulement qu'on peut considérer la longueur sans faire attention à la largeur ; ce qui est indubitable, comme lorsqu'on mesure la distance d'une ville à une autre, on ne mesure que la longueur des chemins, sans se mettre en peine de leur largeur.

Or, plus on peut séparer les choses en divers modes, et plus l'esprit devient capable de les bien connaître; et ainsi nous voyons que tant qu'on n'a point distingué dans le mouvement la détermination vers quelque endroit, du mouvement même, et même diverses parties dans une même détermination, on n'a pu rendre de raison claire de la réflexion et de la réfraction, ce qu'on a fait aisément par cette distinction, comme on peut voir dans le Chapitre 2 de la Dioptrique de Monsieur Descartes.

La troisième manière de concevoir les choses par abstraction est quand une même chose ayant divers attributs, on pense à l'un sans penser à l'autre, quoiqu'il n'y ait entre eux qu'une distinction de raison : et voici comme cela se fait. Si je fais, par exemple, réflexion que je pense, et que par conséquent je suis moi qui pense, dans l'idée que j'ai de moi qui pense, je puis m'appliquer à la considération d'une chose qui pense, sans faire attention que c'est moi, quoique en moi, moi et celui qui pense ne soit que la même chose; et ainsi l'idée que je concevrai d'une personne qui pense, pourra représenter, non seulement moi, mais toutes les autres personnes qui pensent. De même, ayant figuré sur un papier un triangle équilatère, si je m'attache à le considérer au lieu où il est avec tous les accidents qui le déterminent, je n'aurai l'idée que d'un seul triangle; mais si je détourne mon esprit de la considération de toutes ces circonstances particulières, et que je ne l'applique qu'à penser que c'est une figure bornée par trois lignes égales, l'idée que je m'en formerai me représentera d'une part plus nettement cette égalité des lignes, et de l'autre sera capable de me représenter tous les triangles équilatères. Que si je passe .plus avant, et que ne m'arrêtant plus à cette égalité des lignes, je considère seulement que c'est une figure terminée par trois lignes droites, je me formerai une idée qui peut représenter toutes sortes de triangles. Si ensuite, ne m'arrêtant point au nombre des lignes, je considère seulement que c'est une surface plate, bornée par des lignes droites, l'idée que je me formerai pourra représenter toutes les figures rectilignes, et ainsi je puis monter de degré en degré jusqu'à l'extension. Or, dans ces abstractions, on voit toujours que le degré inférieur comprend le supérieur avec quelque détermination particulière, comme \emph{moi} comprend ce qui pense, et le triangle équilatère comprend le triangle, et le triangle la figure rectiligne; mais que le degré supérieur étant moins déterminé peut représenter plus de choses.

Enfin, il est visible que par ces sortes d'abstractions, les idées, de singulières, deviennent communes, et les communes plus communes, et ainsi cela nous donnera lieu de passer à ce que nous avons à dire des idées considérées selon leur universalité ou particularité.
