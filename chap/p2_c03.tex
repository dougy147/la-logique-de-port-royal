\subsubsection{\centering \Large CHAPITRE III}
\addcontentsline{toc}{section}{\protect\numberline{}{\scshape\bfseries III} - \emph{Des propositions simples et composées. Qu'il y en a de simples qui paraissent composées et qui ne le sont pas, et qu'on peut appeler complexes. De celles qui sont complexes par le sujet ou par l'attribut}}
\begin{center}\emph{\large\scshape Des propositions simples et composées. Qu'il y en a de simples qui paraissent composées et qui ne le sont pas, et qu'on peut appeler complexes. De celles qui sont complexes par le sujet ou par l'attribut.}\end{center}


	\lettrine{N}{ous} avons dit que toute proposition doit avoir au moins un sujet et un attribut; mais il ne s'ensuit pas de là qu'elle ne puisse avoir plus d'un sujet et plus d'un attribut. Celles donc qui n'ont qu'un sujet et qu'un attribut s'appellent \emph{simples}, et celles qui ont plus d'un sujet ou plus d'un attribut s'appellent \emph{composées}, comme quand je dis : Les biens et les maux, la vie et la mort, la pauvreté et les richesses viennent du Seigneur ; cet attribut, \emph{venir du Seigneur}, est affirmé, non d'un seul sujet, mais de plusieurs; savoir, \emph{des biens et des maux}, etc.

Mais, avant que d'expliquer ces propositions composées, il faut remarquer qu'il y en a qui le paraissent, et qui sont néanmoins simples : car la simplicité d'une proposition se prend de l'unité du sujet et de l'attribut. Or, il y a plusieurs propositions qui n'ont proprement qu'un sujet et qu'un attribut, mais dont le sujet et l'attribut est un terme complexe, qui enferme d'autres propositions qu'on peut appeler incidentes, qui ne font que partie du sujet ou de l'attribut, y étant jointes par le pronom relatif, \emph{qui, lequel}, dont le propre est de joindre ensemble plusieurs propositions, en sorte qu'elles n'en composent toutes qu'une seule.

Ainsi, quand Jésus-Christ dit, \emph{Celui qui fera la volonté de mon Père, qui est dans le ciel, entrera dans le royaume des cieux}, le sujet de cette proposition contient deux propositions, puisqu'il comprend deux verbes ; mais comme ils sont joints par des \emph{qui}, ils ne font que partie du sujet ; au lieu que quand je dis, les biens et les maux viennent du Seigneur, il y a proprement deux sujets, parce que j'affirme également de l'un et de l'autre qu'ils viennent de Dieu.

Et la raison de cela est, que les propositions jointes à d'autres par des \emph{qui}, ou ne sont des propositions que fort imparfaitement, selon ce qui sera dit plus bas, ou ne sont pas tant considérées comme des propositions que l'on fasse alors, que comme des propositions qui ont été faites auparavant, et qu'alors on ne fait plus que concevoir, comme si c'étaient de simples idées. D'où vient qu'il est indifférent d'énoncer ces propositions incidentes par des noms adjectifs ou par des participes sans verbes et sans \emph{qui}, ou avec des verbes et des \emph{qui}. Car c'est la même chose de dire : \emph{Dieu invisible a créé le monde visible}, ou \emph{Dieu qui est invisible a créé le monde qui est visible : Alexandre, le plus généreux de tous les rois a vaincu Darius}, ou \emph{Alexandre qui a été le plus généreux de tous les rois, a vaincu Darius}. Et dans l'un et dans l'autre, mon but principal n'est pas d'affirmer que Dieu soit invisible, ou qu'Alexandre ait été le plus généreux de tous les rois ; mais supposant l'un et l'autre comme affirmé auparavant, j'affirme de Dieu conçu comme invisible, qu'il a créé le monde visible, et d'Alexandre conçu comme le plus généreux de tous les rois, qu'il a vaincu Darius.

Mais si je disais : \emph{Alexandre a été le plus généreux de tous les rois, et le vainqueur de Darius}, il est visible que j'affirmerais également d'Alexandre, et qu'il aurait été le plus généreux de tous les rois, et qu'il aurait été le vainqueur de Darius. Et ainsi c'est avec raison qu'on appelle ces dernières sortes de propositions des propositions composées, au lieu qu'on peut appeler les autres des propositions complexes.

Il faut encore remarquer que ces propositions complexes peuvent être de deux sortes : car la complexion, pour parler ainsi, peut tomber ou sur la matière de la proposition, c'est-à-dire sur le sujet ou sur l'attribut, ou sur tous les deux, ou bien sur la forme seulement.

\bigbreak
{1.} La complexion tombe sur le sujet, quand le sujet est un terme complexe, comme dans cette proposition : \emph{Tout homme qui ne craint rien, est roi :}\begin{center}\emph{Rex est qui metuit nihil. \\ Beatus ille qui procul negotiis, \\ Ut prisca gens mortalium, \\ Paterna rura bobus exercet suis, \\ Solutus omni faenore}.\end{center}

Car le verbe \emph{est} est sous-entendu dans cette dernière proposition, et \emph{beatus} en est l'attribut, et tout le reste le sujet.

\bigbreak
{2.} La complexion tombe sur l'attribut, lorsque l'attribut est un terme complexe, comme : \emph{La piété est un bien qui rend l'homme heureux dans les plus grandes adversités :}\begin{center}\emph{Sum pius Æneas fama super aethera notus.}\end{center}

Mais il faut particulièrement remarquer ici que toutes les propositions composées de verbes actifs et de leur régime, peuvent être appelées complexes, et qu'elles contiennent en quelque manière deux propositions. Si je dis, par exemple, Brutus a tué un tyran, cela veut dire que Brutus a tué quelqu'un, et que celui qu'il a tué était tyran. D'où vient que cette proposition peut être contredite en deux manières, ou en disant : Brutus n'a tué personne, ou en disant que celui qu'il a tué n'était pas tyran. Ce qu'il est très important de remarquer, parce que lorsque ces sortes de propositions entrent en des arguments, quelquefois on n'en prouve qu'une partie en supposant l'autre : ce qui oblige souvent, pour réduire ces arguments dans la forme la plus naturelle, de changer l'actif en passif, afin que la partie qui est prouvée soit exprimée directement, comme nous remarquerons plus au long quand nous traiterons des arguments composés de ces propositions complexes.

\bigbreak
{3.} Quelquefois la complexion tombe sur le sujet et sur l'attribut, l'un et l'autre étant un terme complexe; comme dans cette proposition : \emph{Les grands qui oppriment les pauvres seront punis de Dieu, qui est le protecteur des opprimés :}\begin{center}\emph{Ille ego qui quondam gracili modulatus avena \\ Carmen, et egressus silvis vicina coegi \\ Ut, quamvis avido parerent arva colono \\ Gratum opus agricolis. At nunc horrentia Martis \\ Arma virumque cano, Troiae qui primus ab oris \\ Italiam fato profugus lavinaque venit litora}.\end{center}

Les trois premiers vers et la moitié du quatrième composent le sujet de cette proposition ; le reste en compose l'attribut, et l'affirmation est enfermée dans le verbe \emph{cano}.

Voilà les trois manières selon lesquelles les propositions peuvent être complexes, quant à leur matière, c'est-à-dire quant à leur sujet et à leur attribut.

