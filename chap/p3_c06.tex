\subsubsection{\centering \Large CHAPITRE VI}
\addcontentsline{toc}{section}{\protect\numberline{}{\scshape\bfseries VI} - \emph{Règles, modes et fondements de la seconde figure}}
\begin{center}\emph{\large\scshape Règles, modes et fondements de la seconde figure.}\end{center}

	\lettrine{L}{a} deuxième figure est celle où le moyen est deux fois attribut, et de là il s'ensuit qu'afin qu'elle conclue nécessairement, il faut que l'on garde ces deux règles.

\begin{center}{\bfseries\scshape\large 1. Règle}\end{center}

	\emph{Il faut qu'il y ait une des deux premières propositions négatives, et par conséquent que la conclusion le soit aussi par la sixième règle générale}.

Car si elles étaient toutes deux affirmatives le moyen qui est toujours attribut, serait pris deux fois particulièrement contre la première règle générale.

\begin{center}{\bfseries\scshape\large 2. Règle}\end{center}

	\emph{Il faut que la majeure soit universelle}.

Car, la conclusion étant négative, le grand terme ou l'attribut est pris universellement. Or, ce même terme est sujet de la majeure. Donc il doit être universel, et par conséquent rendre la majeure universelle.

Il serait facile de produire ici des exemples des arguments qui péchant contre ces règles sont mauvais, et ne concluent rien. Mais il est plus utile de les laisser trouver par ceux qui liront ceci : afin qu'ils s'appliquent davantage à la considération de ces règles.

\begin{center}{\bfseries Démonstration}\end{center}

\emph{Qu'il ne peut y avoir que quatre modes dans la deuxième figure}.

Des dix modes concluants, les quatre affirmatifs sont exclus par la première règle de cette figure, qui est que l'une des prémisses doit être négative.

O, A, O, est exclu par la deuxième règle, qui est que la majeure doit être universelle.

E, A, O, est exclu pour la même raison qu'en la première figure, parce que le petit terme est aussi sujet en la mineure.

Il ne reste donc de ces dix modes que ces quatre.


\begin{center}
$ \text {2 Généraux} \left \{
    \begin{array}{cl}
  	\text{E, A, E} \\
  	\text{A, E, E} \\
    \end{array}
	    \right \} $
\end{center}
\begin{center}
$ \text {2 Particuliers} \left \{
    \begin{array}{cl}
	\text{E, I, O} \\
	\text{A, O, O} \\
    \end{array}
	    \right \} $
\end{center}

Ce qu'il fallait démontrer.

On a compris ces quatre modes sous ces mots artificiels.

\begin{center}
	\begin{tabularx}{\textwidth}{lX}
		{\large\scshape Ce—} & \emph{Nulle matière ne pense:} \\
		{\large\scshape Sa— } & \emph{Notre âme pense:}                           \\
		{\large\scshape Re. } & \emph{Donc notre âme n'est point matière.}        \\
		{\large\scshape Ca— } & \emph{Tous les sages sont contents de ce qu'ils ont:} \\
		{\large\scshape Mes—} &  \emph{Nul avare n'est content de ce qu'il a:}    \\
		{\large\scshape Tres.} & \emph{Donc nul avare n'est sage.}               \\
		{\large\scshape Fes—} & \emph{Nul malheur n'est souhaitable:}            \\
		{\large\scshape Ti— } & \emph{Quelque mort est souhaitable:}              \\
		{\large\scshape No. } & \emph{Donc quelque mort n'est pas malheur.}       \\
		{\large\scshape Ba— } & \emph{Toute vertu est louable:}                   \\
		{\large\scshape Ro— } & \emph{Quelque magnificence n'est pas louable:}    \\
		{\large\scshape Co. } & \emph{Donc quelque magnificence n'est pas vertu.} \\
	\end{tabularx}
\end{center}

\begin{center}{\bfseries Fondement de la deuxième figure}\end{center}

Il serait facile de réduire toutes ces diverses sortes d'arguments à un même principe par quelques détours ; mais il est plus avantageux d'en réduire deux à un principe, et deux à un autre, parce que la dépendance et la liaison qu'ils ont avec ces deux principes, est plus claire et plus immédiate.


\begin{center}{\bfseries 1. Principe des arguments en \emph{Cesare} et \emph{Festino}.}\end{center}

Le premier de ces principes est celui qui sert aussi de fondement aux arguments négatifs de la première figure, savoir, \emph{Que ce qui est nié d'une idée universelle, est aussi nié de tout ce dont cette idée est affirmée, c'est-à-dire de tous les sujets de cette idée}. Car il est clair que les arguments en \emph{Cesare} et \emph{Festino}, sont établis sur ce principe. Pour montrer, par exemple, que nul homme de bien n'est menteur, j'ai affirmé croyable de tout homme de bien, et j'ai nié menteur de tout homme croyable, en disant que nul menteur n'est croyable. Il est vrai que cette façon de nier est indirecte, puisqu'au lieu de nier menteur de croyable, j'ai nié croyable de menteur : mais comme les propositions négatives universelles se convertissent simplement en niant l'attribut d'un sujet universel, on nie ce sujet universel de l'attribut.

Cela fait voir néanmoins que les arguments en \emph{Cesare} sont en quelque manière indirects, puisque ce qui doit être nié n'y est nié qu'indirectement: mais, comme cela n'empêche pas que l'esprit ne comprenne facilement et clairement la force de l'argument, ils peuvent passer pour directs, entendant ce terme pour des arguments clairs et naturels.

Cela fait voir aussi que ces deux modes \emph{Cesare} et \emph{Festino} ne
sont différents des deux de la première figure, \emph{Celarent} et \emph{ferio},
qu'en ce que la majeure en est renversée. Mais quoique l'on
puisse dire que les modes négatifs de la première figure sont plus
directs, il arrive néanmoins souvent que ces deux de la deuxième
figure qui y répondent sont plus naturels, et que l'esprit s'y porte
plus facilement. Car, par exemple, dans celui que nous venons de
proposer, quoique l'ordre direct de la négation demandât que l'on
dit : Nulle chose qui pense n'est matière, ce qui eut fait un argument en \emph{Celarent}, néanmoins notre esprit se porte plus naturellement à dire que nulle matière ne pense.

\newpage

\begin{center}{\bfseries 2. Principe des arguments en \emph{Camestres} et \emph{Baroco}.}\end{center}

Dans ces deux modes le moyen est affirmé de l'attribut de la conclusion, et nié du sujet. Ce qui fait voir qu'ils sont établis directement sur ce principe : Tout ce qui est compris dans l'extension d'une idée universelle, ne convient à aucun des sujets dont on la nie. L'attribut d'une proposition négative étant pris selon toute son extension, comme on l'a prouvé dans la seconde partie, l'âme de l'homme est comprise dans l'extension de substance pensante, puisque l'âme de l'homme est une substance pensante. La substance pensante est niée de la matière. Donc l'âme de l'homme est niée de la matière. Ce qui fait cet argument :
\begin{center}
	\begin{tabular}{l}
		\emph{Toute âme humaine est une substance pensante.} \\
		\emph{Nulle matière n'est une substance pensante.} \\
		\emph{Donc nulle matière n'est l'âme de l'homme.} \\
	\end{tabular}
\end{center}

Si la mineure était particulière on ferait un argument en \emph{Baroco} établi sur le même principe.


