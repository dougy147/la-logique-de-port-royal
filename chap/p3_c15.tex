\subsubsection{\centering \Large CHAPITRE XV}
\addcontentsline{toc}{section}{\protect\numberline{}{\scshape\bfseries XV} - \emph{Des lieux ou de la méthode de trouver des arguments. Combien cette méthode est de peu d'usage}}
\begin{center}\emph{\large\scshape Des lieux ou de la méthode de trouver des arguments. Combien cette méthode est de peu d'usage.}\end{center}

	\lettrine{C}{e} que les rhétoriciens et les logiciens appellent lieux, \emph{loci argumentorum}, sont certains chefs généraux, auxquels on peut rapporter toutes les preuves dont on se sert dans les diverses matières que l'on traite : et la partie de la logique, qu'ils appellent invention, n'est autre chose que ce qu'ils enseignent de ces lieux.

Ramus fait une querelle sur ce sujet à Aristote et aux philosophes de l'école, parce qu'ils traitent des lieux après avoir donné les règles des arguments, et il prétend contre eux qu'il faut expliquer les lieux et ce qui regarde l'invention avant que de traiter de ces règles.

La raison de Ramus est, que l'on doit avoir trouvé la matière avant que de songer à la disposer. Or, l'explication des lieux enseigne à trouver cette matière, au lieu que les règles des arguments n'en peuvent apprendre que la disposition.

Mais cette raison est très faible, parce qu'encore qu'il soit nécessaire que la matière soit trouvée pour la disposer, il n'est pas nécessaire néanmoins d'apprendre à trouver la matière avant d'avoir appris à la disposer : car, pour apprendre à disposer la matière, il suffit d'avoir certaines matières générales pour servir d'exemples ; or, l'esprit et le sens commun en fournissent toujours assez, sans qu'il soit besoin d'en emprunter d'aucun art ni d'aucune méthode. Il est donc vrai qu'il faut avoir une matière pour y appliquer les règles des arguments ; mais il est faux qu'il soit nécessaire de trouver cette matière par la méthode des lieux.

On pourrait dire, au contraire, que comme on prétend enseigner dans les lieux l'art de tirer des arguments et des syllogismes, il est nécessaire de savoir auparavant ce que c'est qu'argument et syllogisme ; mais on pourrait peut-être aussi répondre que la nature seule nous fournit une connaissance générale de ce que c'est que raisonnement, qui suffit pour entendre ce qu'on en dit en parlant des lieux.

Il est donc assez inutile de se mettre en peine en quel ordre on doit traiter des lieux, puisque c'est une chose à peu près indifférente. Mais il serait peut-être plus utile d'examiner, s'il ne serait pas plus à propos de n'en point traiter du tout.

On sait que les anciens ont fait un grand mystère de cette méthode, et que Cicéron la préfère même à toute la dialectique, telle qu'elle était enseignée par les Stoïciens, parce qu'ils ne parlaient point des lieux. Laissons, dit-il, toute cette science, qui ne nous dit rien de l'art de trouver des arguments, et qui ne nous fait que trop de discours pour nous instruire à en juger. \emph{Istam artem totam relinquamus quae in excogitandis argumentis muta nimium est, in judicandis nimium loquax}. Quintilien et tous les autres rhétoriciens, Aristote et tous les philosophes en parlent de même; de sorte que l'on aurait peine à n'être pas de leur sentiment, si l'expérience générale n'y paraissait entièrement opposée.

On peut en prendre à témoin presque autant de personnes qu'il y en a qui ont passé par le cours ordinaire des études, et qui ont appris de cette méthode artificielle de trouver des preuves, ce qu'on en apprend dans les collèges ; car, y en a-t-il un seul qui puisse dire véritablement que, lorsqu'il a été obligé de traiter quelque sujet, il ait fait réflexion sur ces lieux et y ait cherché les raisons qui lui étaient nécessaires? Qu'on consulte tant d'avocats et de prédicateurs qui sont au monde, tant de gens qui parlent et qui écrivent, et qui ont toujours de la matière de reste ; et je ne sais si on en pourra trouver quelqu'un qui ait jamais pensé à faire un argument \emph{a causa, ab effectu, ab adjunctis}, pour prouver ce qu'il désirait persuader.

Aussi, quoique Quintilien fasse paraître de l'estime pour cet art, il est obligé néanmoins de reconnaître qu'il ne faut pas, lorsqu'on traite une matière, aller frapper à la parte de tous ces lieux pour en tirer des arguments et des preuves, \emph{Illud quoque}, dit-il, \emph{studiosi eloquentiae cogitent, non esse cum proposita fuerit materia dicendi scrutanda singula et velut ostiatim pulsanda, ut sciant an ad id probandum quod intendimus, forte respondeant}.


Il est vrai que tous les arguments qu'on fait sur chaque sujet peuvent se rapporter à ces chefs et à ces termes généraux qu'on appelle lieux ; mais ce n'est point par cette méthode qu'on les trouve. La nature, la considération attentive du sujet, la connaissance des diverses vérités les fait produire et ensuite l'art les rapporte à certains genres. De sorte que l'on peut dire véritablement des lieux ce que saint Augustin dit en général des préceptes de la rhétorique. On trouve, dit-il, que les règles de l'éloquence sont observées dans les discours des personnes éloquentes, quoiqu'ils n'y pensent pas en les faisant, soit qu'ils les sachent, soit qu'ils les ignorent. Ils pratiquent ces règles, parce qu'ils sont éloquents ; mais ils ne s'en servent pas pour être éloquents. \emph{Implent quippe illa, quia sunt eloquentes, non adhibent ut sint eloquentes}.

L'on marche naturellement, comme ce même père le remarque en un autre endroit, et en marchant on fait certains mouvements réglés du corps; mais il ne servirait de rien pour apprendre à marcher, de dire, par exemple, qu'il faut envoyer des esprits en certains nerfs, remuer certains muscles, faire certains mouvements dans les jointures, mettre un pied devant l'autre, et se reposer sur l'un pendant que l'autre avance. On peut bien former des règles en observant ce que la nature nous fait faire ; mais on ne fait jamais ces actions par le secours de ces règles : ainsi l'on traite tous les lieux dans les discours les plus ordinaires, et l'on ne saurait rien dire qui ne s'y rapporte: mais ce n'est point en y faisant une réflexion expresse que l'on produit ces pensées ; cette réflexion ne pouvant servir qu'à ralentir la chaleur de l'esprit et à l'empêcher de trouver les raisons vives et naturelles, qui sont les vrais ornements de toute sorte de discours,

Virgile, dans le neuvième Livre de l'Énéide, après avoir représenté Euryale surpris et environné de ses ennemis, qui étaient près de venger sur lui la mort de leurs compagnons que Nisus, ami d'Euryale, avait tués, met ces paroles pleines de mouvement et de passion dans la bouche de Nisus :

	\begin{tabularx}{\textwidth}{X}
		\emph{Me, me, adsum qui feci : in me convertite ferrum,} \\
		\emph{O Rutuli ! mea fraus omnis ; nihil iste, nec ausus,} \\
		\emph{Nec potuit : caelum hoc, et sidera conscia testor :} \\
		\emph{Tantum infelicem nimium dilexit amicum.} \\
	\end{tabularx}

C'est un argument, dit Ramus, à cause efficiente; mais on pourrait bien juger avec assurance, que jamais Virgile ne songea, lorsqu'il fit ces vers, au lieu de la cause efficiente. Il ne les aurait jamais faits, s'il s'était arrêté à y chercher cette pensée ; et il faut nécessairement que, pour produire des vers si nobles et si animés, il ait, non seulement oublié ces règles, s'il les savait, mais qu'il se soit, en quelque sorte, oublié lui-même pour prendre la passion qu'il représentait.


En vérité, le peu d'usage que le monde a fait de cette méthode des lieux depuis tant de temps qu'elle est trouvée et qu'on l'enseigne dans les écoles, est une preuve évidente qu'elle n'est pas de grand usage; mais quand on se serait appliqué à en tirer tout le fruit qu'on en peut tirer, on ne voit pas qu'on puisse arriver par là à quelque chose qui soit véritablement utile et estimable ; car tout ce qu'on peut prétendre par cette méthode, est de trouver sur chaque sujet diverses pensées générales, ordinaires, éloignées, comme les Lullistes en trouvent par le moyen de leurs tables : or, tant s'en faut qu'il soit utile de se procurer cette sorte d'abondance, qu'il n'y a rien qui gâte davantage le jugement.

Rien n'étouffe plus les bonnes semences que l'abondance des mauvaises herbes ; rien ne rend un esprit plus stérile en pensées justes et solides, que cette mauvaise fertilité de pensées communes. L'esprit s'accoutume à cette facilité, et ne fait plus d'efforts pour trouver les raisons propres, particulières et naturelles, qui ne se découvrent que dans la considération attentive de son sujet.

Au reste on ne voit pas pourquoi on se met tant en peine de devenir abondant. Ce n'est pas ce qui manque à la plupart du monde, on pèche beaucoup plus par excès que par défaut; et les discours que l'on fait ne sont que trop remplis de matière. Ainsi, pour former les hommes dans une éloquence judicieuse et solide, il serait bien plus utile de leur apprendre à se taire qu'à parler, c'est-à-dire à supprimer et à retrancher les pensées basses, communes et fausses, qu'à produire, comme ils font, un amas confus de raisonnements bons et mauvais, dont on remplit les livres et les discours.

Et comme l'usage des lieux ne peut guère servir qu'à trouver de ces sortes de pensées, on peut dire que s'il est bon de savoir ce qu'on en dit, parce que tant de personnes célèbres en ont parlé qu'ils ont formé une espèce de nécessité de ne pas ignorer une chose si commune, il est encore beaucoup plus important d'être très-persuadé qu'il n'y a rien de plus ridicule que de les employer pour discourir de tout à perte de vue, comme les Lullistes font par le moyen de leurs attributs généraux, qui sont des espèces de lieux; et que cette mauvaise facilité de parler de tout, et de trouver raison partout, dont quelques personnes font vanité, est un si mauvais caractère d'esprit, qu'il est beaucoup au-dessous de la bêtise.

C'est pourquoi tout l'avantage qu'on peut tirer de ces lieux se
réduit au plus à en avoir une teinture générale, qui sert peut-être
un peu, sans qu'on y pense, à envisager la matière que l'on traite
par plus de faces et de parties.
