\subsubsection{\centering \Large CHAPITRE II}
\addcontentsline{toc}{section}{\protect\numberline{}{\scshape\bfseries II} - \emph{Division des syllogismes en simples et en conjonctifs, et des simples en incomplexes et complexes}}
\begin{center}\emph{\large\scshape Division des syllogismes en simples et en conjonctifs, et des simples en incomplexes et complexes.}\end{center}


	\lettrine{L}{es} syllogismes sont simples ou conjonctifs. Les simples, sont ceux où le moyen n'est joint à la fois qu'à un des termes de la conclusion : les conjonctifs sont ceux où il est joint à tous les deux ; ainsi cet argument est simple :

\begin{center}
	\begin{tabular}{l}
		\emph{Tout bon prince est aimé de ses sujets :} \\
		\emph{Tout roi pieux est bon prince :} \\
		\emph{Donc tout roi pieux est aimé de ses sujets ;} \\
	\end{tabular}
\end{center}

parce que le moyen est joint séparément avec roi pieux, qui est le sujet de la conclusion, et avec aimé de ses sujets, qui en est l'attribut; mais celui-ci est conjonctif par une raison contraire :

	\begin{tabularx}{\textwidth}{X}
		\emph{Si un état électif est sujet aux divisions, il n'est pas de longue durée :} \\
		\emph{Or un état électif est sujet aux divisions :} \\
		\emph{Donc un état électif n'est pas de longue durée ;} \\
	\end{tabularx}

puisque \emph{état électif}, qui est le sujet, et \emph{de longue durée}, qui est l'attribut, entrent dans la majeure.

Comme ces deux sortes de syllogismes ont leurs règles séparées, nous en parlerons séparément.

Les syllogismes simples, qui sont ceux où le moyen est joint séparément avec chacun des termes de la conclusion, sont encore de deux sortes.

Les uns, où chaque terme est joint tout entier avec le moyen, à savoir, avec l'attribut tout entier dans la majeure, et avec le sujet tout entier dans la mineure.

Les autres, où la conclusion étant complexe, c'est-à-dire composée de termes complexes, on ne prend qu'une partie du sujet, ou une partie de l'attribut, pour joindre avec le moyen dans l'une des propositions, et on prend tout le reste, qui n'est plus qu'un seul terme, pour joindre avec le moyen dans l'autre proposition, comme dans cet argument :
\begin{center}
	\begin{tabular}{l}
		\emph{La loi divine oblige d'honorer les rois :} \\
		\emph{Louis XIV est roi :} \\
		\emph{Donc la loi divine oblige d'honorer Louis XIV.} \\
	\end{tabular}
\end{center}

Nous appellerons les premières sortes d'arguments, démêlés et incomplexes, et les autres impliqués ou complexes ; non que tous ceux où il y a des propositions complexes soient de ce dernier genre, mais parce qu'il n'y en a point de ce dernier genre où il n'y ait des propositions complexes.

Or quoique les règles qu'on donne ordinairement pour les syllogismes simples puissent avoir lieu dans tous les syllogismes complexes en les renversant, néanmoins; parce que la force de la conclusion ne dépend point de ce renversement-là, nous n'appliquerons ici les règles des syllogismes simples qu'aux incomplexes, en nous réservant de traiter à part des syllogismes complexes.
