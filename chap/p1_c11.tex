\subsubsection{\centering \Large CHAPITRE XI}
\addcontentsline{toc}{section}{\protect\numberline{}{\scshape\bfseries XI} - \emph{Observations importantes touchant la définition des noms}}
\begin{center}\emph{\large\scshape Observations importantes touchant la définition des noms.}\end{center}

	\lettrine{A}{près} avoir expliqué ce que c'est que les définitions des noms, et combien elles sont utiles et nécessaires, il est important de faire quelques observations sur la manière de s'en servir, afin de ne pas en abuser.

La première est qu'il ne faut pas entreprendre de définir tous les mots, parce que souvent cela serait inutile, et qu'il est même impossible de le faire. Je dis qu'il serait souvent inutile de définir de certains noms; car, lorsque l'idée que les hommes ont de quelque chose est distincte, et que tous ceux qui entendent une langue forment la même idée en entendant prononcer un mot, il serait inutile de le définir, puisqu'on a déjà la fin de la définition, qui est que le mot soit attaché à une idée claire et distincte. C'est ce qui arrive dans les choses fort simples dont tous les hommes ont naturellement la même idée; de sorte que les mots par lesquels on les signifie sont entendus de la même sorte par tous ceux qui s'en servent, ou, s'ils y mêlent quelquefois quelque chose d'obscur, leur principale attention néanmoins va toujours à ce qu'il y a de clair; et ainsi ceux qui ne s'en servent que pour en marquer l'idée claire, n'ont pas sujet de craindre qu'ils ne soient pas entendus. Tels sont les mots \emph{d'être}, de \emph{pensée}, \emph{d'étendue}, \emph{d'égalité}, de \emph{durée} ou de \emph{temps}, et autres semblables. Car, encore que quelques-uns obscurcissent l'idée du temps par diverses propositions qu'ils en forment, et qu'ils appellent définitions, comme que le temps est la mesure du mouvement selon l'antériorité et la postériorité, néanmoins ils ne s'arrêtent pas eux-mêmes à cette définition, quand ils entendent parler du temps, et n'en conçoivent autre chose que ce que naturellement tous les autres en conçoivent : et ainsi les savants et les ignorants entendent la même chose, et avec la même facilité, quand on leur dit qu'un cheval est moins de temps à faire une lieue qu'une tortue.

Je dis de plus qu'il serait impossible de définir tous les mots ; car, pour définir un mot, on a nécessairement besoin d'autres mots qui désignent l'idée à laquelle on veut attacher ce mot; et, si l'on voulait aussi définir les mots dont on se serait servi pour l'explication de celui-là, on en aurait encore besoin d'autres, et ainsi à l'infini. Il faut donc nécessairement s'arrêter à des termes primitifs qu'on ne définisse point ; et ce serait un aussi grand défaut de vouloir trop définir, que de ne pas assez définir, parce que, par l'un et par l'autre, on tomberait dans la confusion que l'on prétend éviter.

La seconde observation est qu'il ne faut point changer les définitions déjà reçues, quand on n'a point sujet d'y trouver à redire; car il est toujours plus facile de faire entendre un mot, lorsque l'usage déjà reçu, au moins parmi les savants, l'a attaché à une idée, que lorsqu'il l'y faut attacher de nouveau, et le détacher de quelque autre idée avec laquelle on a accoutumé de le joindre. C'est pourquoi ce serait une faute de changer les définitions reçues par les mathématiciens, si ce n'est qu'il y en eût quelqu'une d'embrouillée, et dont l'idée n'aurait pas été désignée assez nettement, comme peut être celle de l'angle et de la proportion dans Euclide.

La troisième observation est que, quand on est obligé de définir un mot, on doit, autant que l'on peut, s'accommoder à l'usage, en ne donnant pas aux mots des sens tout à fait éloignés de ceux qu'ils ont, et qui pourraient même être contraires à leur étymologie, comme qui dirait : J'appelle parallélogramme une figure terminée par trois lignes ; mais se contentant pour l'ordinaire de dépouiller les mots qui ont deux sens, de l'un de ces sens, pour l'attacher uniquement à l'autre. Comme la chaleur signifiant, dans l'usage commun, et le sentiment que nous avons, et une qualité que nous nous imaginons dans le feu tout à fait semblable à ce que nous sentons ; pour éviter cette ambiguïté, je puis me servir du nom de chaleur, en l'appliquant à l'une de ces idées, et le détachant de l'autre ; comme si je dis : J'appelle chaleur le sentiment que j'ai quand je m'approche du feu, et donnant à la cause de ce sentiment, ou un nom tout à fait différent, comme serait celui d'ardeur, ou ce même nom, avec quelque addition qui le détermine et qui le distingue de chaleur prise pour le sentiment, comme qui dirait la chaleur virtuelle.

La raison de cette observation est que les hommes, ayant une fois attaché une idée à un mot, ne s'en défont pas facilement; et ainsi leur ancienne idée revenant toujours, leur fait aisément oublier la nouvelle que vous voulez leur donner en définissant ce mot; de sorte qu'il serait plus facile de les accoutumer à un mot qui ne signifierait rien du tout, comme qui dirait j'appelle bara une figure terminée par trois lignes, que de les accoutumer à dépouiller le mot de \emph{parallélogramme} de l'idée d'une figure dont les côtés opposés sont parallèles, pour lui faire signifier une figure dont les côtés ne peuvent être parallèles.

C'est un défaut dans lequel sont tombés tous les chimistes, qui ont pris plaisir de changer les noms à la plupart des choses dont ils parlent, sans aucune utilité, et de leur en donner qui signifient déjà d'autres choses qui n'ont nul véritable rapport avec les nouvelles idées auxquelles ils les lient. Ce qui donne même lieu à quelques-uns de faire des raisonnements ridicules, comme est celui d'une personne qui, s'imaginant que la peste est un mal saturnien, prétendait qu'on avait guéri des pestiférés en leur pendant au col un morceau de plomb, que les chimistes appellent Saturne, sur lequel on avait gravé un jour de samedi, qui porte aussi le nom de Saturne, la figure dont les astronomes se servent pour marquer cette planète ; comme si des rapports arbitraires et sans raison entre le plomb et la planète de Saturne, et entre cette même planète et le jour du samedi, et la petite marque dont on la désigne, pouvaient avoir des effets réels, et guérir effectivement des maladies.


