\subsubsection{\centering \Large CHAPITRE V}
\addcontentsline{toc}{section}{\protect\numberline{}{\scshape\bfseries V} - \emph{Des idées, considérées selon leur généralité, particularité et singularité}}
\begin{center}\emph{\large\scshape Des idées, considérées selon leur généralité, particularité et singularité.}\end{center}

	\lettrine{Q}{uoique} toutes les choses qui existent soient singulières, néanmoins, par le moyen des abstractions que nous venons d'expliquer, nous ne laissons pas d'avoir tous plusieurs sortes d'idées, dont les unes ne nous représentent qu'une seule chose, comme l'idée que chacun a de soi-même, et les autres en peuvent également représenter plusieurs, comme lorsque quelqu'un conçoit un triangle sans y considérer autre chose, sinon que c'est une figure à trois lignes et à trois angles; l'idée qu'il en a formée peut lui servir à concevoir tous les autres triangles.

Les idées qui ne représentent qu'une seule chose s'appellent singulières ou individuelles, et ce qu'elles représentent, \emph{des individus}; et celles qui en représentent plusieurs s'appellent universelles, communes, générales.

Les noms qui servent à marquer les premières s'appellent propres, \emph{Socrate, Rome, Bucéphale}, et ceux qui servent à marquer les dernières, communs et appellatifs, comme \emph{homme, ville, cheval}. Et tant les idées universelles, que les noms communs, se peuvent appeler termes généraux.

Mais il faut remarquer que les mots sont généraux en deux manières : l'une, que l'on appelle \emph{univoque}, qui est lorsqu'ils sont liés avec des idées générales; de sorte que le même mot convient à plusieurs, et selon le son, et selon une même idée qui y est jointe : tels sont les mots dont on vient de parler, d'homme, de ville, de cheval.

L'autre, qu'on appelle \emph{équivoque}, qui est lorsqu'un même son à été lié par les hommes à des idées différentes, de sorte que le même son convient à plusieurs, non selon une même idée, mais selon les idées différentes auxquelles il se trouve joint dans l'usage : ainsi le mot de \emph{canon} signifie une machine de guerre, et un décret de Concile, et une sorte d'ajustement ; mais il ne les signifie que selon des idées toutes différentes.

Néanmoins cette universalité équivoque est de deux sortes. Car les différentes idées jointes à un même son, ou n'ont aucun rapport naturel entre elles, comme dans le mot de \emph{canon}, ou en ont quelqu'un, comme lorsqu'un mot étant principalement joint à une idée, on ne le joint à une autre idée que parce qu'elle a un rapport de cause ou d'effet, ou de signe, on de ressemblance à la première; et alors ces sortes de mots équivoques s'appellent \emph{analogues}; comme quand le mot de \emph{sain} s'attribue à l'animal, à l'air et aux viandes. Car l'idée jointe à ce mot est principalement la santé qui ne convient qu'à l'animal; mais on y joint une autre idée approchante de celle-là, qui est d'être cause de la santé, qui fait qu'on dit qu'un air est sain, qu'une viande est saine, parce qu'ils servent à conserver la santé.

Mais quand nous parlons ici de mots généraux, nous entendons les univoques qui sont joints à des idées universelles et générales.

Or, dans ces idées universelles, il y a deux choses qu'il est très important de bien distinguer, \emph{la compréhension} et \emph{l'étendue}.

J'appelle \emph{compréhension} de l'idée, les attributs qu'elle enferme en soi, et qu'on ne peut lui ôter sans la détruire, comme la compréhension de l'idée du triangle enferme extension, figure, trois lignes, trois angles, et l'égalité de ces trois angles à deux droits, etc.

J'appelle \emph{étendue} de l'idée les sujets à qui cette idée convient, ce qu'on appelle aussi les inférieurs d'un terme général, qui, à leur égard, est appelé supérieur, comme l'idée du triangle en général s'étend à toutes les diverses espèces de triangles.

Mais, quoique l'idée générale s'étende indistinctement à tous les sujets à qui elle convient, c'est-à-dire à tous ses inférieurs, et que le nom commun les signifie tous, il y a néanmoins cette différence entre les attributs qu'elle comprend et les sujets auxquels elle s'étend, qu'on ne peut lui ôter aucun de ses attributs sans la détruire, comme nous avons déjà dit ; au lieu qu'on peut la resserrer, quant à son étendue, ne l'appliquant qu'à quelqu'un des sujets auxquels elle convient, sans que pour cela on la détruise.

Or, cette restriction ou resserrement de l'idée générale, quant à son étendue, peut se faire en deux manières.

La première est, par une autre idée distincte et déterminée qu'on y joint, comme lorsqu'à l'idée générale du triangle, je joins celle d'avoir un angle droit ; ce qui resserre cette idée à une seule espèce de triangle, qui est le triangle rectangle.

L'autre, en y joignant seulement une idée indistincte et indéterminée de partie, comme quand je dis, quelque triangle ; et on dit alors que le terme commun devient particulier, parce qu'il ne s'étend plus qu'à une partie des sujets auxquels il s'étendait auparavant, sans que néanmoins on ait déterminé quelle est cette partie à laquelle on l'a resserré.

