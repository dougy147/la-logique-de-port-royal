\subsubsection{\centering \Large CHAPITRE V}
\addcontentsline{toc}{section}{\protect\numberline{}{\scshape\bfseries V} - \emph{Des règles qui regardent les axiomes, c'est-à-dire les propositions claires et évidentes par elles-mêmes}}
\begin{center}\emph{\large\scshape Des règles qui regardent les axiomes, c'est-à-dire les propositions claires et évidentes par elles-mêmes.}\end{center}


	\lettrine{T}{out} le monde demeure d'accord qu'il y a des propositions si claires et si évidentes d'elles-mêmes, qu'elles n'ont pas besoin d'être démontrées; et que toutes celles qu'on ne démontre point doivent être telles pour être principes d'une véritable démonstration : car si elles sont tant soit peu incertaines, il est clair qu'elles ne peuvent être le fondement d'une conclusion tout à fait certaine.

Mais plusieurs ne comprennent pas assez en quoi consiste cette clarté et cette évidence d'une proposition, car, premièrement, il ne faut pas s'imaginer qu'une proposition ne soit claire et certaine, que lorsque personne ne la contredit; et qu'elle doive passer pour douteuse, ou qu'au moins on soit obligé de la prouver, lorsqu'il se trouve quelqu'un qui la nie. Si cela était, il n'y aurait rien de certain ni de clair, puisqu'il s'est trouvé des philosophes qui ont fait profession de douter généralement de tout, et qu'il y en a même qui ont prétendu qu'il n'y avait aucune proposition qui fût plus vraisemblable que sa contraire. Ce n'est donc point par les contestations des hommes qu'on doit juger de la certitude ni de la clarté; car il n'y a rien qu'on ne puisse contester, surtout de parole : mais il faut tenir pour clair ce qui paraît tel à tous ceux qui veulent prendre la peine de considérer les choses avec attention, et qui sont sincères à dire ce qu'ils en pensent intérieurement. C'est pourquoi il y a une parole de très grand sens dans Aristote, qui est que la démonstration ne regarde proprement que le discours intérieur, et non pas le discours extérieur, parce qu'il n'y a rien de si bien démontré qui ne puisse être nié par un homme opiniâtre, qui s'engage à contester de paroles les choses mêmes dont il est intérieurement persuadé, ce qui est une très mauvaise disposition, et très indigne d'un esprit bien fait; quoiqu'il soit vrai que cette humeur se prend souvent dans les écoles de philosophie, par la coutume qu'on y a introduite de disputer de toutes choses, et de mettre son honneur à ne se rendre jamais, celui-là étant jugé avoir le plus d'esprit qui est le plus prompt à trouver des défaites pour s'échapper; au lieu que le caractère d'un honnête homme est de rendre les armes à la vérité, aussitôt qu'il l'aperçoit, et de l'aimer dans la bouche même de son adversaire.

Secondement, les mêmes philosophes, qui tiennent que toutes nos idées viennent de nos sens, soutiennent aussi que toute la certitude et toute l'évidence des propositions vient, ou immédiatement, ou médiatement des sens. \emph{Car}, disent-ils, \emph{cet axiome même, qui passe pour le plus clair et le plus évident que l'on puisse désirer : Le tout est plus grand que sa partie, n'a trouvé de créance dans notre esprit que parce que, dès notre enfance, nous avons observé en particulier, et que tout l'homme est plus grand que sa tête, et toute une maison qu'une chambre, et toute une forêt qu'un arbre, et tout le ciel qu'une étoile}.

Cette imagination est aussi fausse que celle que nous avons réfutée dans la première partie, \emph{que toutes nos idées viennent de nos sens}. Car si nous n'étions assurés de cette vérité, \emph{le tout est plus grand que sa partie}, que par les diverses observations que nous en avons faites depuis notre enfance, nous n'en serions que probablement assurés; puisque l'induction n'est un moyen certain de connaître une chose, que quand nous sommes assurés que l'induction est entière, n'y ayant rien de plus ordinaire que de découvrir la fausseté de ce que nous avions cru vrai sur des inductions qui nous paraissaient si générales, qu'on ne s'imaginait point pouvoir y trouver d'exception.

Ainsi, il n'y a pas longtemps qu'on croyait indubitable que l'eau contenue dans un vaisseau courbé, dont un côté était beaucoup plus large que l'autre, se tenait toujours au niveau, n'étant pas plus haute dans le petit côté que dans le grand, parce qu'on s'en était assuré par une infinité d'observations : et néanmoins on a trouvé depuis peu que cela est faux, quand l'un des côtés est extrêmement étroit, parce qu'alors l'eau s'y tient plus haute que dans l'autre côté. Tout cela fait voir que les seules inductions ne sauraient nous donner une certitude entière d'aucune vérité, à moins que nous ne fussions assurés qu'elles fussent générales, ce qui est impossible ; et par conséquent nous ne serions que probablement assurés de la vérité de cet axiome, \emph{le tout est plus grand que sa partie}, si nous n'en étions assurés que pour avoir vu qu'un homme est plus grand que sa tête, une forêt qu'un arbre, une maison qu'une chambre, le ciel qu'une étoile, puisque nous aurions toujours sujet de douter s'il n'y aurait point quelque autre tout auquel nous n'aurions pas pris garde, qui ne serait pas plus grand que sa partie.

Ce n'est donc point de ces observations que nous avons faites depuis notre enfance, que la certitude de cet axiome dépend ; puisqu'au contraire il n'y a rien de plus capable de nous entretenir dans l'erreur, que de nous arrêter à ces préjugés de notre enfance ; mais elle dépend uniquement de ce que les idées claires et distinctes que nous avons d'un tout et d'une partie renferment clairement, et que le tout est plus grand que la partie, et que la partie est plus petite que le tout : et tout ce qu'ont pu faire les diverses observations que nous avons faites d'un homme plus grand que sa tète, d'une maison plus grande qu'une chambre, a été de nous servir d'occasion pour faire attention aux idées de \emph{tout} et de \emph{partie}. Mais il est absolument faux qu'elles soient cause de la certitude absolue et inébranlable que nous avons de la vérité de cet axiome, comme je crois l'avoir démontré.

Ce que nous avons dit de cet axiome, peut se dire de tous les autres, et ainsi je crois que la certitude et l'évidence de la connaissance humaine dans les choses naturelles dépend de ce principe :

\begin{center}
\emph{Tout ce qui est contenu dans l'idée claire et distincte d'une chose, peut s'affirmer avec vérité de cette chose}.
\end{center}

Ainsi, parce qu'\emph{être animal} est renfermé dans l'idée de \emph{l'homme}, je puis affirmer de l'homme qu'il est animal; parce qu'avoir tous ses diamètres égaux est renfermé dans l'idée d'un cercle, je puis affirmer de tout cercle que tous ses diamètres sont égaux ; parce qu'avoir tous ses angles égaux à deux droits est renfermé dans l'idée d'un triangle, je dois l'affirmer de tout triangle.

Et l'on ne peut contester ce principe sans détruire toute l'évidence de la connaissance humaine, et établir un pyrrhonisme ridicule; car nous ne pouvons juger des choses que par les idées que nous en avons, puisque nous n'avons aucun moyen de les concevoir qu'autant qu'elles sont dans notre esprit, et qu'elles n'y sont que par leurs idées. Or, si les jugements que nous formons en considérant ces idées ne regardaient pas les choses en elles-mêmes, mais seulement nos pensées ; c'est-à-dire si de ce que je vois clairement qu'avoir trois angles égaux à deux droits est renfermé dans l'idée d'un triangle, je n'avais pas droit de conclure que, dans la vérité, tout triangle a trois angles égaux à deux droits, mais seulement que je le pense ainsi, il est visible que nous n'aurions aucune connaissance des choses, mais seulement de nos pensées : et par conséquent, nous ne saurions rien des choses que nous nous persuadons savoir le plus certainement; mais nous saurions seulement que nous les pensons être de telle sorte, ce qui détruirait manifestement toutes les sciences.

Et il ne faut pas craindre qu'il y ait des hommes qui demeurent sérieusement d'accord de cette conséquence, que nous ne savons d'aucune chose si elle est vraie ou fausse en elle-même; car il y en a de si simples et de si évidentes, comme, \emph{Je pense; donc je suis : Le tout est plus grand que sa partie}, qu'il est impossible de douter sérieusement si elles sont telles en elles-mêmes que nous les concevons. La raison est qu'on ne saurait en douter sans y penser, et on ne saurait y penser sans les croire vraies, et par conséquent on ne saurait en douter.

Néanmoins ce principe seul ne suffit pas pour juger de ce qui doit être reçu pour axiome; car il y a des attributs qui sont véritablement renfermés dans l'idée des choses qui s'en peuvent néanmoins et s'en doivent démontrer, comme l'égalité de tous les angles d'un triangle à deux droits, et de tous ceux d'un hexagone à huit droits, mais il faut prendre garde si l'on n'a besoin que de considérer l'idée d'une chose avec une attention médiocre, pour voir clairement qu'un tel attribut y est renfermé ; ou si, de plus, il est nécessaire d'y joindre quelque autre idée pour s'apercevoir de cette liaison. Quand il n'est besoin que de considérer l'idée, la proposition peut être prise pour axiome, surtout si cette considération ne demande qu'une attention médiocre dont tous les esprits ordinaires soient capables : mais si l'on a besoin de quelque autre idée que de l'idée de la chose, c'est une proposition qu'il faut démontrer. Ainsi, l'on peut donner ces deux règles pour les axiomes :

\begin{center}{\scshape\bfseries\large 1. Règle}\end{center}

	\emph{Lorsque, pour voir clairement qu'un attribut convient à un sujet, comme pour voir qu'il convient au tout d'être plus grand que sa partie, on n'a besoin que de considérer les deux idées du sujet et de l'attribut avec une médiocre attention, en sorte qu'on ne puisse le faire sans s'apercevoir que l'idée de l'attribut est véritablement renfermée dans l'idée du sujet : on a droit alors de prendre cette proposition pour un axiome qui n'a pas besoin d'être démontré, parce qu'il a de lui-même toute l'évidence que pourrait lui donner la démonstration, qui ne pourrait faire autre chose, sinon de montrer que cet attribut convient au sujet en se servant d'une troisième idée pour montrer cette liaison; ce qu'on voit déjà sans l'aide d'aucune troisième idée.}

Mais il ne faut pas confondre une simple explication, quand même elle aurait quelque forme d'argument, avec une vraie démonstration; car il y a des axiomes qui ont besoin d'être expliqués pour mieux les faire entendre, quoiqu'ils n'aient pas besoin d'être démontrés; l'explication n'étant autre chose que de dire en autres termes et plus au long ce qui est contenu dans l'axiome ; au lieu que la démonstration demande quelque moyen nouveau que l'axiome ne contienne pas clairement.

\begin{center}{\scshape\bfseries\large 2. Règle}\end{center}

	\emph{Quand la seule considération des idées du sujet et de l'attribut ne suffit pas pour voir clairement que l'attribut convient au sujet, la proposition qui l'affirme ne doit point être prise pour axiome; mais elle doit être démontrée, en se servant de quelques autres idées pour faire voir cette liaison, comme on se sert de l'idée des lignes parallèles pour montrer que les trois angles d'un triangle sont égaux à deux droits}.

Ces deux règles sont plus importantes que l'on ne pense, car c'est un des défauts les plus ordinaires aux hommes, de ne pas assez se consulter eux-mêmes dans ce qu'ils assurent ou qu'ils nient; de s'en rapporter à ce qu'ils en ont ouï dire, ou à ce qu'ils en ont autrefois pensé, sans prendre garde à ce qu'ils en penseraient eux-mêmes, s'ils considéraient avec plus d'attention ce qui se passe dans leur esprit, de s'arrêter plus au son des paroles qu'à leurs véritables idées ; d'assurer comme clair et évident ce qu'il leur est impossible de concevoir, et de nier comme faux ce qu'il leur serait impossible de ne pas croire vrai, s'ils voulaient prendre la peine d'y penser sérieusement.

Par exemple, ceux qui disent que dans un morceau de bois, outre ses parties et leur situation, leur figure, leur mouvement ou leur repos, et les pores qui se trouvent entre ces parties, il y a encore une forme substantielle distinguée de tout cela, croient ne rien dire que de certain, et cependant ils disent une chose que ni eux ni personne n'a jamais comprise et ne comprendra jamais.

Que si, au contraire, on veut leur expliquer les effets de la nature par les parties insensibles dont les corps sont composés, et par leur différente situation, grandeur, figure, mouvement ou repos, et par les pores qui se trouvent entre ces parties, et qui donnent ou ferment le passage à d'autres matières, ils croient qu'on ne leur dit que des chimères, quoiqu'on ne leur dise rien qu'ils ne conçoivent très facilement ; et même, par un renversement d'esprit assez étrange, la facilité qu'ils ont à concevoir ces choses les porte à croire que ce ne sont pas les vraies causes des effets de la nature, mais qu'elles sont plus mystérieuses et plus cachées; de sorte qu'ils sont plus disposés à croire ceux qui les leur expliquent par des principes qu'ils ne conçoivent point, que ceux qui ne se servent que des principes qu'ils entendent.

Et ce qui est encore assez plaisant est que, quand on leur parle de parties insensibles, ils croient être bien fondés à les rejeter, parce qu'on ne peut les leur faire voir ni toucher, et cependant ils se contentent de formes substantielles, de pesanteur, de vertu attractive, etc., que non seulement ils ne peuvent voir ni toucher, mais qu'ils ne peuvent même concevoir.
