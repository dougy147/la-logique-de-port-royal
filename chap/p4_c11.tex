\subsubsection{\centering \Large CHAPITRE XI}
\addcontentsline{toc}{section}{\protect\numberline{}{\scshape\bfseries XI} - \emph{De ce que nous connaissons par la foi, soit humaine, soit divine}}
\begin{center}\emph{\large\scshape De ce que nous connaissons par la foi, soit humaine, soit divine.}\end{center}


	\lettrine{T}{out} ce que nous avons dit jusqu'ici regarde les sciences humaines, purement humaines, et les connaissances qui sont fondées sur l'évidence de la raison ; mais, avant de finir, il est bon de parler d'une autre sorte de connaissance, qui souvent n'est pas moins certaine ni moins évidente en sa manière, qui est celle que nous tirons de l'autorité.

Car il y a deux voies générales qui nous font croire qu'une chose est vraie. La première est la connaissance que nous en avons par nous-mêmes, pour en avoir reconnu et recherché la vérité, soit par nos sens, soit par notre raison ; ce qui peut s'appeler généralement \emph{raison}, parce que les sens mêmes dépendent du jugement de la raison ; ou science, prenant ici ce nom plus généralement qu'on ne le prend dans les écoles, pour toute connaissance d'un objet tiré de l'objet même.

L'autre voie est l'autorité des personnes dignes de croyance qui nous assurent qu'une telle chose est, quoique par nous-mêmes nous n'en sachions rien; ce qui s'appelle foi ou croyance, selon cette parole de saint Augustin : \emph{Quod scimus, debemus rationi ; Quod credimus, auctoritati}.

Mais comme cette autorité peut être de deux sortes, de Dieu ou des hommes, il y a aussi deux sortes de foi, divine et humaine.

La foi divine ne peut être sujette à erreur, parce que Dieu ne peut ni nous tromper ni être trompé.

La foi humaine est de soi-même sujette à erreur, parce que tout homme est menteur, selon l'Écriture, et qu'il peut se faire que celui qui nous assurera une chose comme véritable sera lui-même trompé. Et néanmoins, ainsi que nous avons déjà marqué ci-dessus, il y a des choses que nous ne connaissons que par une foi humaine, que nous devons tenir pour aussi certaines et aussi indubitables, que si nous en avions des démonstrations mathématiques: comme ce que l'on sait, par une relation constante de tant de personnes qu'il est moralement impossible qu'elles eussent pu conspirer ensemble pour assurer la même chose, si elle n'était vraie. Par exemple, les hommes ont assez de peine naturellement à concevoir qu'il y ait des antipodes: cependant, quoique nous n'y ayons pas été, et qu'ainsi nous n'en sachions rien que par une foi humaine, il faudrait être fou pour ne pas croire, ou pour douter qu'il y ait un royaume appelé \emph{Pérou}, dont les Espagnols sont les maîtres, et un détroit de mer en ces quartiers-là à qui Magellan a donné son nom. Il faudrait de même avoir perdu le sens, pour douter si jamais César, Pompée, Cicéron, Virgile ont été; et si ce ne sont point des personnages feints, comme ceux des Amadis.

Il est vrai qu'il est souvent assez difficile de marquer précisément quand la foi humaine est parvenue à cette certitude, et quand elle n'y est pas encore parvenue ; et c'est ce qui fait tomber les hommes en deux égarements opposés : dont l'un est de ceux qui croient trop légèrement sur les moindres bruits, et l'autre de ceux qui mettent ridiculement la force de l'esprit à ne pas croire les choses les mieux attestées lorsqu'elles choquent les préventions de leur esprit; mais on peut néanmoins marquer de certaines bornes qu'il faut avoir passées pour avoir cette certitude humaine, et d'autres au delà desquelles on l'a certainement, en laissant un milieu entre ces deux sortes de bornes, qui approche plus de la certitude ou de l'incertitude, selon qu'il approche plus des unes ou des autres.

Que si l'on compare ensemble les deux voies générales qui nous font croire qu'une chose est, la raison et la foi, il est certain que la foi suppose toujours quelque raison ; car, comme dit saint Augustin dans sa Lettre 122, et en beaucoup d'autres lieux, nous ne pourrions pas nous porter à croire ce qui est au-dessus de notre raison, si la raison même ne nous avait persuadés qu'il y a des choses que nous faisons bien de croire, quoique nous ne soyons pas encore capables de les comprendre : ce qui est principalement vrai à l'égard de la foi divine, parce que la vraie raison nous apprend que Dieu étant la vérité même, il ne peut nous tromper en ce qu'il nous révèle de sa nature ou de ses mystères. D'où il paraît qu'encore que nous soyons obligés de captiver notre entendement pour obéir à Jésus-Christ, comme dit saint Paul, nous ne le faisons pas néanmoins aveuglément et déraisonnablement, ce qui est l'origine de toutes les fausses religions ; mais avec connaissance de cause, et parce que c'est une action raisonnable que de se captiver de la sorte sous l'autorité de Dieu, lorsqu'il nous a donné des preuves suffisantes, comme sont les miracles et autres événements prodigieux, qui nous obligent de croire que c'est lui-même qui a découvert aux hommes les vérités que nous devons croire.

Il est certain, en second lieu, que la foi divine doit avoir plus de force sur notre esprit que notre propre raison ; et cela par la raison même qui nous fait voir qu'il faut toujours préférer ce qui est plus certain à ce qui l'est moins ; et qu'il est plus certain que ce que Dieu dit est véritable, que ce que notre raison nous persuade, parce que Dieu est plus incapable de nous tromper que notre raison d'être trompée.

Néanmoins, à considérer les choses exactement, jamais ce que nous voyons évidemment et par la raison ou par le fidèle rapport des sens n'est opposé à ce que la foi divine nous enseigne; mais ce qui fait que nous le croyons, c'est que nous ne prenons pas garde à quoi doit se terminer l'évidence de notre raison et de nos sens. Par exemple, nos sens nous montrent clairement dans l'Eucharistie de la rondeur et de la blancheur ; mais nos sens ne nous apprennent point si c'est la substance du pain qui fait que nos yeux y aperçoivent de la rondeur et de la blancheur : et ainsi la foi n'est point contraire à l'évidence de nos sens, lorsqu'elle nous dit que ce n'est point la substance du pain qui n'y est plus, ayant été changée au corps de Jésus-Christ par le mystère de la Transsubstantiation, et que nous n'y voyons plus que les espèces et les apparences du pain qui demeurent, quoique la substance n'y soit plus.

Notre raison, de même, nous fait voir qu'un seul corps n'est pas en même temps en divers lieux ni deux corps en un même lieu ; mais cela doit s'entendre de la condition naturelle des corps; parce que ce serait un défaut de raison de s'imaginer que notre esprit étant fini, il pût comprendre jusqu'où peut aller la puissance de Dieu qui est infinie ; et ainsi lorsque les hérétiques, pour détruire les mystères de la foi, comme la Trinité, l'Incarnation et l'Eucharistie, opposent ces prétendues impossibilités qu'ils tirent de la raison, ils s'éloignent en cela même visiblement de la raison, en prétendant pouvoir comprendre par leur esprit l'étendue infinie de la puissance de Dieu. C'est pourquoi il suffit de répondre à toutes ces objections ce que saint Augustin dit sur le sujet même de la pénétration des corps, \emph{sed noua sunt, sed insolita sunt, sed contra naturae cursum notissimum sunt, quia magna, quia mira, quia divina, et eo magis vera, certa firma}.
