La première édition de \emph {La Logique ou l'art de penser} fut publiée en 1662 sans nom d'auteur. Elle est le fruit du travail d'Antoine Arnauld (1612-1694) et de Pierre Nicole (1625-1695), qui la rééditèrent à quatre reprises, jusqu'à la cinquième et dernière édition (1683).
D'aucuns considèrent ce manuel de logique comme l'un des plus influents des débuts de l'ère moderne, ayant occupé les bibliothèques personnelles de Locke, Leibniz, Hume ou encore Newton\footnote{Grey, J. (2017). The modal equivalence rules of the Port-Royal Logic. \emph{History and Philosophy of Logic, 38}(3), 210-221.}.
Il nous a semblé important de remettre en lumière cette première édition, car elle seyait grandement à ses auteurs. En effet, s'ils se sont pliés aux requêtes et remarques de leurs lecteurs contemporains pour en compléter, reformuler et réorganiser quelques passages, ce serait en partie pour contenter les esprits qui ne seraient pas habiles ou intelligents, comme on peut le lire de leur plume dans \emph{Le second discours} de la deuxième édition (1664).

Cette première édition n'aurait donc nullement eu besoin d'être retouchée! En dépit de cette remarque, sans vouloir faire offense aux auteurs d'un ouvrage de plus de trois siècles, nous avons pris la liberté de retranscrire l'ancien français en français moderne, sans que cela ne modifie le sens des propos tenus. Nous avons tenté de respecter autant que possible la mise en page d'origine quand elle ne nuisait pas à la lisibilité du texte, toute subjective que soit cette impression. C'est pourquoi nous nous sommes permis quelques aérations, sauts de pages et de lignes, çà et là. Certaines illustrations (enluminures et décorations de chapitres ou de parties) ont été redessinées par nos soins, à partir de l'édition originale, pour tenter de conserver l'aspect artisanal de cet ouvrage d'antan.
Aussi, des coquilles repérées durant la première impression de l'ouvrage avaient amené les auteurs à mentionner des corrections dans une page additionnelle. Nous les avons directement appliquées dans le corps du texte.

Nous espérons que ce modeste travail de mise en page pourra satisfaire les lecteurs qui souhaiteraient suivre les esprits \mbox{d'Arnauld} et de Nicole dans la maîtrise de ce nécessaire et subtil art de penser.

 \begin{flushright}
	 {\small LD}
 	\\{\small \texttt{https://humanbooks.xyz}}
 \end{flushright}
