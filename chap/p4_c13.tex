\subsubsection{\centering \Large CHAPITRE XIII}
\addcontentsline{toc}{section}{\protect\numberline{}{\scshape\bfseries XIII} - \emph{Application de la règle précédente à la croyance des miracles}}
\begin{center}\emph{\large\scshape Application de la règle précédente à la croyance des miracles.}\end{center}

	\lettrine{L}{a} règle qui vient d'être expliquée est, sans doute, très importante pour bien conduire sa raison dans la croyance des faits particuliers ; et, faute de l'observer, on est en danger de tomber en des extrémités dangereuses de crédulité et d'incrédulité.

Car il y en a, par exemple, qui feraient conscience de douter d'aucun miracle, parce qu'ils se sont mis dans l'esprit qu'ils seraient obligés de douter de tous s'ils doutaient d'aucuns, et qu'ils se persuadent que ce leur est assez de savoir que tout est possible à Dieu, pour croire tout ce qu'on leur dit des effets de sa toute-puissance.

D'autres, au contraire, s'imaginent ridiculement qu'il y a de la force d'esprit à douter de tous les miracles, sans en avoir d'autre raison, sinon qu'on en a souvent raconté qui ne se sont pas trouvés véritables, et qu'il n'y a pas plus de sujet de croire les uns que les autres.

La disposition des premiers est bien meilleure que celle des derniers ; mais il est vrai néanmoins que les uns et les autres raisonnent également mal.

Ils se jettent de part et d'autre sur les lieux communs. Les premiers en font sur la puissance et sur la bonté de Dieu, sur les miracles certains qu'ils apportent pour preuve de ceux dont on doute, et sur l'aveuglement des libertins, qui ne veulent croire que ce qui est proportionné à leur raison. Tout cela est fort bon en soi, mais très faible pour nous persuader d'un miracle en particulier, puisque Dieu ne fait pas tout ce qu'il peut faire ; que ce n'est pas un argument qu'un miracle soit arrivé, de ce qu'il en est arrivé de semblables en d'autres occasions, et qu'on peut être fort bien disposé à croire ce qui est au-dessus de la raison, sans être obligé de croire tout ce qu'il plaît aux hommes de nous raconter, comme étant au-dessus de la raison.

Les derniers font des lieux communs d'une autre sorte. \emph{La vérité}, (dit l'un d'eux) \emph{et le mensonge ont leurs visages conformes, le port, le goût et les allures pareilles; nous les regardons de même œil. J'ai vu la naissance de plusieurs miracles de mon temps. Encore qu'ils s'étouffent en naissant, nous ne laissons pas de prévoir le train qu'ils eussent pris, s'ils eussent vécu leur âge : car il n'est que de trouver le bout du fil, on dévide tant qu'on veut, et il y a plus loin de rien à la plus petite chose du monde ; qu'il n'y a de celle-là jusqu'à la plus grande. Or, les premiers qui sont abreuvés de ce commencement d'étrangeté, venant à semer leur histoire, sentent, par les oppositions qu'on leur fait, où loge la difficulté de la persuasion, et vont calfeutrant cet endroit de quelque pièce fausse. L'erreur particulière fait premièrement l'erreur publique, et, à son tour, après, l'erreur publique fait l'erreur particulière. Ainsi va tout ce bâtiment, s'étoffant et se formant de main en main, de manière que le plus éloigné témoin en est mieux instruit que le plus voisin, et le dernier informé mieux persuadé que le premier}.

Ce discours est ingénieux, et peut être utile pour ne pas se laisser emporter à toutes sortes de bruits: mais il y aurait de l'extravagance d'en conclure généralement qu'on doit tenir pour suspect tout ce qui se dit des miracles ; car il est certain que cela ne regarde au plus que ce qu'on ne sait que par des bruits communs, sans remonter jusqu'à l'origine ; et il faut avouer qu'il n'y a pas grand sujet de s'assurer de ce qu'on ne saurait que de cette sorte.

Mais qui ne voit qu'on peut faire aussi un lieu commun opposé à celui-là, qui sera pour le moins aussi bien fondé? Car, comme il y a quelques miracles qui se trouveraient peu assurés si l'on remontait jusqu'à la source, il y en a aussi qui s'étouffent dans la mémoire des hommes, ou qui trouvent peu de croyance dans leur esprit, parce qu'ils ne veulent pas prendre la peine de s'en informer. Notre esprit n'est pas sujet à une seule espèce de maladie, il en a de différentes et de toutes contraires. Il y a une sotte simplicité qui croit les choses les moins croyables ; mais il y a aussi une sotte présomption qui condamne comme faux tout ce qui passe les bornes étroites de son esprit. On a souvent de la curiosité pour des bagatelles, et l'on n'en a point pour des choses importantes. De fausses histoires se répandent partout, et de très véritables n'ont point de cours.

Combien y a-t-il peu de personne qui sachent le miracle arrivé de notre temps à Farmoutier en la personne d'une religieuse tellement aveugle qu'il lui restait à peine la forme des yeux, qui recouvra la vue en un moment par l'attouchement des reliques de sainte Fare ?

Saint Augustin dit qu'il y avait, de son temps, beaucoup de miracles très certains, qui étaient connus de peu de personnes, et qui, quoique très remarquables et très étonnants, ne passaient pas d'un bout de la ville à l'autre. C'est ce qui le porta à faire écrire et réciter devant le peuple ceux qui se trouvaient assurés, et il remarque, dans le vingt-deuxième livre de la Cité de Dieu, qu'il s'en était fait dans la seule ville d'Hippone près de soixante et dix depuis deux ans qu'on y avait bâti une chapelle en l'honneur de saint Étienne, sans beaucoup d'autres qu'on n'avait pas écrits, qu'il témoigne néanmoins avoir sus très certainement.

On voit donc assez qu'il n'y a rien de moins raisonnable que de se conduire par des lieux communs en ces rencontres, soit pour embrasser tous les miracles, soit pour les rejeter tous, mais qu'il faut les examiner par leurs circonstances particulières et par la fidélité et la lumière des témoins qui les rapportent.

La piété n'oblige pas un homme de bon sens de croire tous les miracles rapportés dans la Légende dorée, ou dans Métaphraste; parce que ces auteurs sont remplis de tant de fables qu'il n'y a pas sujet de s'assurer de rien sur leur témoignage seul, comme le cardinal Bellarmin n'a pas fait difficulté de l'avouer du dernier.

Mais je soutiens que tout homme de bon sens, quand il n'aurait point de piété, doit reconnaître pour véritables les miracles que saint Augustin raconte dans ses Confessions ou dans la Cité de Dieu être arrivés devant ses yeux, ou dont il témoigne avoir été très particulièrement informé par les personnes mêmes à qui les choses étaient arrivées, comme d'un aveugle guéri à Milan en présence de tout le peuple, par l'attouchement des reliques de saint Gervais et de saint Protais, qu'il rapporte dans ses Confessions, et dont il dit, dans le vingt-deuxième livre de la Cité de Dieu, Chapitre 8. \emph{Miraculum quod Mediolani factum est cum illic essemus, quando illuminatus est caecus, ad multorum notitiam potuit pervenire ; quia et grandis est civitas, et ibi erat tunc imperator, et immenso populo teste res gesta est, concurrente ad corpora martyrum Gervasii et Protasii}.

D'une femme guérie en Afrique par des fleurs qui avaient touché aux reliques de saint Étienne, comme il le témoigne au même lieu.

D'une dame de qualité, guérie d'un cancer jugé incurable, par le signe de la croix qu'elle y fit faire par une nouvelle baptisée, selon la révélation qu'elle en avait eue.

D'un enfant mort sans baptême, dont la mère obtint la résurrection par les prières qu'elle en fit à saint Étienne, en lui disant, avec une grande foi : \emph{Saint Martyr, rendez-moi mon fils. Vous savez que je ne demande sa vie qu'afin qu'il ne soit pas éternellement séparé de Dieu}. Ce que ce saint rapporte comme une chose dont il était très assuré, dans un sermon qu'il fit à son peuple, sur le sujet d'un autre miracle très insigne qui venait d'arriver en ce moment-là même dans l'église où il prêchait, lequel il décrit fort, au long dans cet endroit de la Cité de Dieu.

Il dit que sept frères et trois sœurs, d'une honnête famille de Césarée en Cappadoce, ayant été maudits par leur mère pour une injure qu'ils lui avaient faite, Dieu les avait punis de cette peine, qu'ils étaient continuellement agités, et dans le sommeil même, par un horrible tremblement de tout le corps, ce qui était si difforme, que, ne pouvant plus souffrir la vue des personnes de leur connaissance, ils avaient tous quitté leur pays pour s'en aller de divers côtés, et qu'ainsi l'un de ces frères, appelé Paul, et l'une de ses sœurs, appelée Palladie, étaient venus à Hippone, et s'étant fait remarquer par toute la ville, on avait appris d'eux la cause de leur malheur ; que le propre jour de Pâques, le frère, priant Dieu devant les barreaux de la chapelle de Saint-Étienne, tomba tout d'un coup dans un assoupissement pendant lequel on s'aperçut qu'il ne tremblait plus; et s'étant réveillé parfaitement sain, il se fit dans l'église un grand bruit du peuple, qui louait Dieu de ce miracle et qui courait à saint Augustin, lequel se préparait à dire la messe, pour l'avertir de ce qui s'était passé.

\emph{Après}, dit-il, \emph{que les cris de réjouissance furent passés et que l'Écriture sainte eut été lue, je leur dis peu de chose sur la fête et sur ce grand sujet de joie, parce que j'aimai mieux leur laisser, non pas entendre, mais considérer l'éloquence de Dieu dans cet ouvrage divin. Je menai ensuite chez moi le frère qui avait été guéri ; je lui fis conter toute son histoire, je l'obligeai de l'écrire, et le lendemain je promis au peuple que je la lui ferais réciter le jour d'après. Ainsi le troisième jour d'après Pâques, ayant fait mettre le frère et la sœur sur les degrés du jubé, afin que tout le peuple pût voir dans la sœur, qui avait encore cet horrible tremblement, de quel mal le frère avait été délivré par la bonté de Dieu ; je fis lire le récit de leur histoire devant le peuple, et je les laissai aller. Je commençai alors à prêcher sur ce sujet (on en a le sermon), et voici que tout d'un coup lorsque je parlais encore, un grand cri de joie s'élève du côté de la chapelle, et on m'amène la sœur, qui, étant sortie de devant moi, y était allée et y avait été parfaitement guérie en la même manière que son frère; ce qui causa une telle joie parmi le peuple, qu'à peine pouvait-on supporter le bruit qu'ils faisaient}.

J'ai voulu rapporter toutes les particularités de ce miracle pour convaincre les plus incrédules qu'il y aurait de la folie à le révoquer en doute, aussi bien que tant d'autres que ce saint raconte au même endroit; car, supposé que les choses soient arrivées comme il le rapporte, il n'y a point de personnes raisonnables qui n'y doivent reconnaître le doigt de Dieu, et ainsi tout ce qui resterait à l'incrédulité serait de douter du témoignage même de saint Augustin, de s'imaginer qu'il a altéré la vérité pour autoriser la religion chrétienne dans l'esprit des païens ; or, c'est ce qui ne peut se dire avec la moindre couleur.

Premièrement, parce qu'il n'est point vraisemblable qu'un homme judicieux eût voulu mentir en des choses si publiques, où il aurait pu être convaincu de mensonge par une infinité de témoins, ce qui n'aurait pu tourner qu'à la honte de la religion chrétienne. Secondement, parce qu'il n'y eut jamais personne plus ennemi du mensonge que ce saint, surtout en matière de religion, ayant établi par des livres entiers, non seulement qu'il n'est jamais permis de mentir, mais que c'est un crime horrible de le faire, sous prétexte d'attirer plus facilement les hommes à la foi.

Et c'est ce qui doit causer un extrême étonnement de voir que les hérétiques de ce temps, qui regardent saint Augustin comme un homme très éclairé et très sincère, n'aient pas considéré que la manière dont ils parlent de l'invocation des saints et de la vénération des reliques, comme d'un culte superstitieux, et qui tient de l'idolâtrie, va à la ruine de toute la religion ; car il est visible que c'est lui ôter un de ses plus solides fondements que d'ôter aux vrais miracles l'autorité qu'il doivent avoir pour la confirmation de la vérité ; et il est clair que c'est détruire entièrement cette autorité des miracles que de dire que Dieu en fasse pour récompenser un culte superstitieux et idolâtre. Or, c'est proprement ce que les hérétiques font, en traitant, d'une part, le culte que les catholiques rendent aux saints et à leurs reliques, d'une superstition criminelle; et ne pouvant nier, de l'autre, que les plus grands amis de Dieu, tel qu'a été saint Augustin, par leur propre confession, ne nous aient assuré que Dieu a guéri des maux incurables, illuminé des aveugles, et ressuscité des morts pour récompenser la dévotion de ceux qui invoquaient les saints et révéraient leurs reliques. En vérité, cette seule considération devrait faire reconnaître à tout homme de bon sens la fausseté de la religion prétendue réformée.

Je me suis un peu étendu sur cet exemple célèbre du jugement qu'on doit faire de la vérité des faits, pour servir de règle dans les rencontres semblables, parce qu'on s'y égare de la même sorte. Chacun croit que c'est assez pour les décider de faire un lieu commun, qui n'est souvent composé que de maximes, lesquelles, non seulement ne sont pas universellement vraies, mais qui ne sont pas même probables, lorsqu'elles sont jointes avec les circonstances particulières des faits que l'on examine. Il faut joindre les circonstances et non les séparer, parce qu'il arrive souvent qu'un fait qui est peu probable selon une seule circonstance, qui est ordinairement une marque de fausseté, doit être estimé certain selon d'autres circonstances ; et, qu'au contraire, un fait qui nous paraîtrait vrai selon une certaine circonstance, qui est d'ordinaire jointe avec la vérité; doit être jugé faux selon d'autres qui affaiblissent celle-là, comme on l'expliquera dans le chapitre suivant.

