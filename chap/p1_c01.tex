\subsubsection{\centering \Large CHAPITRE I}
\addcontentsline{toc}{section}{\protect\numberline{}{\scshape\bfseries I} - {\emph{Des idées selon leur nature et leur origine}}}
\begin{center}\emph{\large\scshape Des idées selon leur nature et leur origine.}\end{center}

	\lettrine{L}{e} mot d'idée est du nombre de ceux qui sont si clairs qu'on ne peut les expliquer par d'autres, parce qu'il n'y en a point de plus clairs et de plus simples.

Mais tout ce qu'on peut faire pour empêcher qu'on ne s'y trompe, est de marquer la fausse intelligence qu'on pourrait donner à ce mot, en le restreignant à cette seule façon de concevoir les choses, qui se fait par l'application de notre esprit aux images qui sont peintes dans notre cerveau, et qui s'appelle imagination.

Car, comme saint Augustin remarque souvent, l'homme, depuis le péché, s'est tellement accoutumé à ne considérer que les choses corporelles dont les images entrent par les sens dans notre cerveau, que la plupart croient ne pouvoir concevoir une chose quand ils ne se la peuvent imaginer, c'est-à-dire se la représenter sous une image corporelle, comme s'il n'y avait en nous que cette seule manière de penser et de concevoir.

Au lieu qu'on ne peut faire réflexion sur ce qui se passe dans notre esprit, qu'on ne reconnaisse que nous concevons un très grand nombre de choses sans aucune de ces images, et qu'on ne s'aperçoive de la différence qu'il y a entre l'imagination et la pure intellection. Car lors, par exemple, que je m'imagine un triangle, je ne le conçois pas seulement comme une figure terminée par trois lignes droites; mais, outre cela, je considère ces trois lignes comme présentes par la force et l'application intérieure de mon esprit, et c'est proprement ce qui s'appelle imaginer. Que si je veux penser à une figure de mille angles, je conçois bien, à la vérité, que c'est une figure composée de mille côtés ; aussi facilement que je conçois qu'un triangle est une figure composée de trois côtés seulement ; mais je ne puis m'imaginer les mille côtés de cette figure, ni, pour ainsi dire, les regarder comme présents avec les yeux de mon esprit.

Il est vrai néanmoins que la coutume que nous avons de nous servir de notre imagination, lorsque nous pensons aux choses corporelles, fait souvent qu'en concevant une figure de mille angles, on se représente confusément quelque figure ; mais il est évident que cette figure, qu'on se représente alors par l'imagination, n'est point une figure de mille angles, puisqu'elle ne diffère nullement de ce que je me représenterais si je pensais à une figure de dix mille angles, et qu'elle ne sert en aucune façon à découvrir les propriétés qui font la différence d'une figure de mille angles d'avec tout autre polygone.

Je ne puis donc proprement m'imaginer une figure de mille angles; puisque l'image que j'en voudrais peindre dans mon imagination me représenterait toute autre figure d'un grand nombre d'angles, aussitôt que celle de mille angles ; et néanmoins je puis la concevoir très clairement et très distinctement, puisque j'en puis démontrer toutes les propriétés, comme, que tous ses angles ensemble sont égaux à 1996 angles droits: et, par conséquent, c'est autre chose de s'imaginer, et autre chose de concevoir.

Cela est encore plus clair par la considération de plusieurs choses que nous concevons très clairement, quoiqu'elles ne soient en aucune sorte du nombre de celles que l'on peut s'imaginer. Car, que concevons-nous plus clairement que notre pensée lorsque nous pensons ? Et cependant il est impossible de s'imaginer une pensée, ni d'en peindre aucune image dans notre cerveau. Le \emph{oui} et le \emph{non} n'y peuvent aussi en avoir aucune: celui qui juge que Ta terre est ronde, et celui qui juge qu'elle n'est pas ronde, ayant tous deux les mêmes choses peintes dans le cerveau, à savoir, la Terre et la rondeur; mais l'un y ajoutant l'affirmation, qui est une action de son esprit, laquelle il conçoit sans aucune image corporelle, et l'autre une action contraire, qui est la négation, laquelle peut encore moins avoir d'image.

Lors donc que nous parlons des idées, nous n'appelons point de ce nom les images qui sont peintes en la fantaisie, mais tout ce qui est dans notre esprit, lorsque nous pouvons dire avec vérité que nous concevons une chose, de quelque manière que nous la concevions.

D'où il s'ensuit que nous ne pouvons rien exprimer par nos paroles, lorsque nous entendons ce que nous disons, que de cela même il ne soit certain que nous avons en nous l'idée de la chose que nous signifions par nos paroles, quoique cette idée soit quelquefois plus claire et plus distincte, et quelquefois plus obscure et plus confuse, comme nous l'expliquerons plus bas ; car il y aurait de la contradiction entre dire que je sais ce que je dis en prononçant un mot, et que néanmoins je ne conçois rien en le prononçant que le son même du mot.

Et c'est ce qui fait voir la fausseté de deux opinions très dangereuses qui ont été avancées par des Philosophes de ce temps.

La première est que nous n'avons aucune idée de Dieu, car si nous n'en avions aucune idée, en prononçant le nom de Dieu nous n'en concevrions que ces quatre lettres D, i, e, u, et un Français n'aurait rien davantage dans l'esprit en entendant le nom de Dieu, que si, entrant dans une synagogue et étant entièrement ignorant de la langue hébraïque, il entendait prononcer en hébreu Adonaï, ou Eloha.

Et quand les hommes ont pris le nom de Dieu, comme Caligula, et Domitien, ils n'auraient commis aucune impiété, puisqu'il n'y a rien dans ces lettres ou ces deux syllabes \emph{Deus}, qui ne puisse être attribué à un homme, si on n'y attachait aucune idée. D'où vient qu'on n'accuse point un Hollandais d'être impie pour s'appeler \emph{Ludovicus Dieu}? En quoi donc consistait l'impiété de ces princes, sinon en ce que laissant à ce mot \emph{Deus} une partie au moins de son idée, comme est celle d'une nature excellente et adorable, ils s'appropriaient ce nom avec cette idée ?

Mais, si nous n'avions point d'idée de Dieu, sur quoi pourrions-nous fonder tout ce que nous disons de Dieu? comme, qu'il n'y en a qu'un: qu'il est tout-puissant, tout bon, tout sage, éternel; puisqu'il n'y a rien de tout cela enfermé dans ce son \emph{Dieu}, mais seulement dans l'idée que nous avons de Dieu, et que nous avons jointe à ce son.

Et ce n'est aussi que par là que nous refusons le nom de Dieu à toutes les fausses divinités, non pas que ce mot ne puisse leur être attribué, s'il était pris matériellement, puisqu'il leur a été attribué par les païens ; mais parce que l'idée qui est en nous du souverain Être, et que l'usage a liée à ce mot de Dieu, ne convient qu'au seul vrai Dieu.

La seconde de ces fausses opinions est ce qu'un Anglais a dit, \emph{que le raisonnement n'est peut-être autre chose qu'un assemblage et enchaînement de noms par ce mot} est. \emph{D'où il s'ensuivrait que par la raison nous ne concluons rien du tout touchant la nature des choses, mais seulement touchant leurs appellations ; c'est-à-dire que nous voyons simplement si nous assemblons bien ou mal les noms des choses selon les conventions que nous avons faites à notre fantaisie, touchant leurs significations.}

À quoi cet auteur ajoute : \emph{Si cela est, comme il peut être, le raisonnement dépendra des mots, les mots de l'imagination, et l'imagination dépendra peut-être, comme je le crois, du mouvement des organes corporels ; et ainsi notre âme (mens) ne sera autre chose qu'un mouvement dans quelques parties du corps organique}.

Il faut croire que ces paroles ne contiennent qu'une objection éloignée du sentiment de celui qui la propose; mais comme, étant prises assertivement, elles iraient à ruiner l'immortalité de l'âme, il est important d'en faire voir la fausseté, ce qui ne sera pas difficile, car les conventions dont parle ce philosophe ne peuvent avoir été que l'accord que les hommes ont fait de prendre de certains sons pour être signes des idées que nous avons dans l'esprit. De sorte que si, outre les noms, nous n'avions en nous-même les idées des choses, cette convention aurait été impossible, comme il est impossible par aucune convention de faire entendre à un aveugle ce que veut dire le mot de rouge, de vert, de bleu, parce que, n'ayant point ces idées, il ne peut les joindre à aucun son.

De plus, les diverses nations ayant donné divers noms aux choses, et même aux plus claires et aux plus simples, comme à celles qui sont les objets de la géométrie, ils n'auraient pas les mêmes raisonnements touchant les mêmes vérités, si le raisonnement n'était qu'un assemblage de noms par le mot \emph{est}.

Et comme il paraît par ces divers mots, que les Arabes, par exemple, ne sont point convenus avec les Français pour donner les mêmes significations aux sons, ils ne pourraient aussi convenir dans leurs jugements et leurs raisonnements, si leurs raisonnements dépendaient de cette convention.

Enfin, il y a une grande équivoque dans ce mot d'\emph{arbitraire}, quand on dit que la signification des mots est arbitraire, car il est vrai que c'est une chose purement arbitraire que de joindre une telle idée à un tel son plutôt qu'à un autre ; mais les idées ne sont point des choses arbitraires et qui dépendent de notre fantaisie, au moins celles qui sont claires et distinctes, et, pour le montrer évidemment, c'est qu'il serait ridicule de s'imaginer que des effets très réels pussent dépendre de choses purement arbitraires. Or, quand un homme a conclu par son raisonnement que l'axe de fer qui passe par les deux meules du moulin pourrait tourner sans faire tourner celle de dessous, si, étant rond, il passait par un trou rond; mais qu'il ne pourrait tourner sans faire tourner celle de dessus, si, étant carré, il était emboîté dans un trou carré de cette meule de dessus, l'effet qu'il a prétendu s'ensuit infailliblement. Et par conséquent son raisonnement n'a point été un assemblage de noms, selon une convention qui aurait entièrement dépendu de la fantaisie des hommes, mais un jugement solide et effectif de la nature des choses par la considération des idées qu'il en a dans l'esprit, lesquelles il a plu aux hommes de marquer par de certains noms.

Nous voyons donc assez ce que nous entendons par le mot d'idée ; il ne reste plus qu'à dire un mot de leur origine.

Toute la question est de savoir si toutes nos idées viennent de nos sens, et si l'on doit passer pour vraie cette maxime commune: \emph{Nihil est in intellectu quod non prius fuerit in sensu}.

C'est le sentiment d'un philosophe qui est estimé dans le monde, et qui commence sa Logique par cette proposition : \emph{Omnis idea ortum ducit a sensibus, Toute idée tire son origine des sens}. Il avoue néanmoins que toutes nos idées n'ont pas été dans nos sens telles qu'elles sont dans notre esprit, mais il prétend qu'elles ont au moins été formées de celles qui ont passé par nos sens, ou par composition, comme lorsque des images séparées de l'or et d'une montagne, on s'en fait une montagne d'or ; ou par ampliation et diminution, comme lorsque de l'image d'un homme d'une grandeur ordinaire, on s'en forme un géant ou un pygmée; ou par accommodation et proportion, comme lorsque de l'idée d'une maison qu'on a vue, on s'en forme l'image d'une maison qu'on n'a pas vue. Et ainsi, dit-il, nous concevons Dieu, qui ne peut tomber sous le sens, sous l'image d'un vénérable vieillard.

Quoique cette opinion lui soit commune avec plusieurs des philosophes de l'École, je ne craindrai point de dire qu'elle est très absurde et aussi contraire à la religion qu'à la véritable philosophie. Car, pour ne rien dire que de clair que le jour, il n'y a rien que nous concevions plus distinctement que notre pensée même, ni de proposition qui puisse nous être plus claire que celle-là : \emph{Je pense, donc je suis}. Or, nous ne pourrions avoir aucune certitude de cette proposition, si nous ne concevions distinctement ce que c'est qu'\emph{être} et ce que c'est que \emph{penser}, et il ne nous faut point demander que nous expliquions ces termes, parce qu'ils sont du nombre de ceux qui sont si bien entendus par tout le monde qu'on les obscurcirait en les voulant expliquer. Si donc on ne peut nier que nous n'ayons en nous les idées de l'être et de la pensée, je demande par quel sens elles sont entrées ? Sont-elles lumineuses ou colorées, pour être entrées par la vue ? d'un son grave ou aigu, pour être entrées par l'ouïe? d'une bonne ou mauvaise odeur, pour être entrées par l'odorat ? de bon ou de mauvais goût, pour être entrées par le goût? froides ou chaudes, dures ou molles, pour être entrées par l'attouchement? Que si l'on dit qu'elles ont été formées d'autres images sensibles, qu'on nous dise quelles sont ces autres images sensibles dont on prétend que les idées de l'être et de la pensée ont été formées, et comment elles ont pu être formées, ou par composition, ou par ampliation, ou par diminution, ou par proportion. Que si l'on ne peut rien répondre à tout cela qui ne soit déraisonnable, il faut avouer que les idées de l'être et de la pensée ne tirent en aucune sorte leur origine des sens, mais que notre âme a la faculté de les former de soi-même, quoiqu'il arrive souvent qu'elle est excitée à le faire par quelque chose qui frappe les sens; comme un peintre peut être porté à faire un tableau par l'argent qu'on lui promet, sans qu'on puisse dire pour cela que le tableau a tiré son origine de l'argent.

Mais ce qu'ajoutent ces mêmes auteurs, que l'idée que nous avons de Dieu tire son origine des sens, parce que nous le concevons sous l'idée d'un vieillard vénérable, est une pensée qui n'est digne que des Anthropomorphites; ou qui confond les véritables idées que nous avons des choses spirituelles avec les fausses imaginations que nous en formons par une mauvaise accoutumance de se vouloir tout imaginer, au lieu qu'il est aussi absurde de se vouloir imaginer ce qui n'est point corporel que de vouloir ouïr des couleurs, et voir des sons.

Pour réfuter cette pensée, il ne faut que considérer que si nous n'avions point d'autre idée de Dieu que celle d'un vieillard vénérable, tous les jugements que nous ferions de Dieu nous devraient paraître faux, lorsqu'ils seraient contraires à cette idée. Car nous sommes portés naturellement à croire que nos jugements sont faux, quand nous voyons clairement qu'ils sont contraires aux idées que nous avons des choses; et ainsi nous ne pourrions juger avec certitude que Dieu n'a point de parties, qu'il n'est point corporel, qu'il est partout, qu'il est invisible, puisque tout cela n'est point conforme à l'idée d'un vénérable vieillard. Que si Dieu s'est quelquefois représenté sous cette forme, cela ne fait pas que ce soit là l'idée que nous en devions avoir, puisqu'il faudrait aussi que nous n'eussions point d'autre idée du Saint Esprit que celle d'une colombe, parce qu'il s'est représenté sous la forme d'une colombe; ou que nous conçussions Dieu comme un son, parce que le son du nom de Dieu nous sert à nous en réveiller l'idée.

Il est donc faux que toutes nos idées viennent de nos sens ; mais on peut dire, au contraire, que nulle idée qui est dans notre esprit ne tire son origine des sens, sinon par occasion, en ce que les mouvements qui se font dans notre cerveau, qui est tout ce que peuvent faire nos sens, donnent occasion à l'âme de se former diverses idées qu'elle ne se formerait pas sans cela, quoique presque toujours ces idées n'aient rien de semblable à ce qui se fait dans les sens et dans le cerveau, et qu'il y ait de plus un très grand nombre d'idées qui, ne tenant rien du tout d'aucune image corporelle, ne peuvent sans une absurdité visible être rapportées à nos sens.

