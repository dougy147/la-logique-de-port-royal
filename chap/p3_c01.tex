\subsubsection{\centering \Large CHAPITRE I}
\addcontentsline{toc}{section}{\protect\numberline{}{\scshape\bfseries I} - \emph{De la nature du raisonnement, et des diverses espèces qu'il peut y en avoir}}
\begin{center}\emph{\large\scshape De la nature du raisonnement, et des diverses espèces qu'il peut y en avoir.}\end{center}

	\lettrine{L}{a} nécessité du raisonnement n'est fondés que sur les bornes étroites de l'esprit humain, qui ayant à juger de la vérité ou de la fausseté d'une proposition, qu'alors on appelle \emph{question}, ne peut pas toujours le faire par la considération des deux idées qui la composent, dont celle qui en est le sujet est aussi appelée \emph{le petit terme}, parce que le sujet est d'ordinaire moins étendu que l'attribut, et celle qui en est l'attribut est aussi appelée \emph{le grand terme} par une raison contraire. Lors donc que la seule considération de ces deux idées ne suffit pas pour faire juger si l'on doit affirmer ou nier l'une de l'autre, il a besoin de recourir à une troisième idée, ou incomplexe ou complexe (suivant ce qui a été dit des termes complexes), et cette troisième idée s'appelle \emph{moyen}.

Or, il ne servirait de rien, pour faire cette comparaison de deux idées ensemble par l'entremise de cette troisième idée, de la comparer seulement avec un des deux termes. Si je veux savoir, par exemple, si l'âme est spirituelle, et que ne le pénétrant pas d'abord, je choisisse, pour m'en éclaircir, l'idée de pensée, il est clair qu'il me sera utile de comparer la pensée avec l'âme, si je ne conçois dans la pensée aucun rapport avec l'attribut de spirituelle, par le moyen duquel je puisse juger s'il convient ou ne convient pas à l'âme. Je dirai bien, par exemple, l'âme pense; mais je n'en pourrai pas conclure, donc elle est spirituelle, si je ne conçois aucun rapport entre le terme de \emph{penser} et celui de \emph{spirituelle}.

Il faut donc que ce terme moyen soit comparé, tant avec le sujet ou le petit terme, qu'avec l'attribut ou le grand terme, soit qu'il ne le soit que séparément avec chacun de ces termes, comme dans les syllogismes, qu'on appelle \emph{simples} pour cette raison, soit qu'il le soit tout à la fois avec tous les deux, comme dans les arguments qu'on appelle \emph{conjonctifs}.

Mais en l'une ou l'autre manière, cette comparaison demande deux propositions.

Nous parlerons en particulier des arguments conjonctifs, mais pour les simples cela est clair, parce que le moyen étant une fois comparé avec l'attribut de la conclusion (ce qui ne peut être qu'en affirmant ou niant) fait la proposition qu'on appelle \emph{majeure}, à cause que cet attribut de la conclusion s'appelle \emph{grand terme}.

Et, étant une autre fois comparé avec le sujet de la conclusion, fait celle qu'on appelle \emph{mineure}, à cause que le sujet de la conclusion s'appelle \emph{petit terme}.

Et puis la conclusion, qui est la proposition même qu'on avait à prouver, et qui, avant que d'être prouvée, s'appelait \emph{question}.

Il est bon de savoir que les deux premières propositions s'appellent aussi \emph{prémisses} \emph{(praemissae)}, parce qu'elles sont mises au moins dans l'esprit avant la conclusion, qui en doit être une suite nécessaire si le syllogisme est bon ; c'est-à-dire que, supposé la vérité des prémisses, il faut nécessairement que la conclusion soit vraie, de sorte qu'elle ne puisse être niée avec raison par celui qui les aurait accordées.

Il est vrai que l'on n'exprime pas toujours les deux prémisses, parce que souvent une seule suffit pour en faire concevoir deux à l'esprit; et, quand on n'exprime ainsi que deux propositions, cette sorte de raisonnement s'appelle \emph{enthymème}, qui est un véritable syllogisme dans l'esprit, parce qu'il supplée la proposition qui n'est pas exprimée; mais qui est imparfait dans l'expression, et ne conclut qu'en vertu de cette proposition sous-entendue.

J'ai dit qu'il y avait au moins trois propositions dans un raisonnement; mais il pourrait y en avoir beaucoup davantage, sans qu'il fût pour cela défectueux, pourvu qu'on garde toujours les règles. Car, si, après avoir consulté une troisième idée, pour savoir si un attribut convient ou ne convient pas à un sujet, et l'avoir comparée avec un des termes, je ne sais pas encore s'il convient ou ne convient pas au second terme, j'en pourrais choisir une quatrième pour m'en éclaircir, et une cinquième si celle-là ne suffit pas, jusqu'à ce que je vinsse à une idée qui liât l'attribut de la conclusion avec le sujet.

Si je doute, par exemple, \emph{Si les avares sont misérables}, je pourrai considérer d'abord que les avares sont pleins de désirs et de passions: Si cela ne me donne pas lieu de conclure, \emph{donc ils sont misérables}, j'examinerai ce que c'est que d'être plein de désirs, et je trouverai dans cette idée celle de manquer de beaucoup de choses que l'on désire, et la misère dans cette privation de ce que l'on désire, ce qui me donnera lieu de former ce raisonnement : \emph{Les avares sont pleins de désirs : Ceux qui sont pleins de désirs manquent de beaucoup de choses, parce qu'il est impossible qu'ils satisfassent tous leurs désirs : Ceux qui manquent de ce qu'ils désirent sont misérables: Donc les avares sont misérables}.

Ces sortes de raisonnements composés de plusieurs propositions, dont la seconde dépend de la première, et ainsi du reste, s'appellent \emph{sorites}. Et ce sont ceux qui sont les plus ordinaires dans les mathématiques ; mais parce que, quand ils sont longs, l'esprit a plus de peine à les suivre, et que le nombre des trois propositions est assez proportionné avec l'étendue de notre esprit, on a pris plus de soin d'examiner les règles des bons et des mauvais syllogismes, c'est-à-dire des arguments de trois propositions; ce qu'il est bon de suivre, parce que les règles qu'on en donne peuvent facilement s'appliquer à tous les raisonnements composés de plusieurs propositions, d'autant qu'ils peuvent tous se réduire en syllogismes s'ils sont bons.


