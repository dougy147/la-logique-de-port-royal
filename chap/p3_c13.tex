\subsubsection{\centering \Large CHAPITRE XIII}
\addcontentsline{toc}{section}{\protect\numberline{}{\scshape\bfseries XIII} - \emph{Des dilemmes}}
\begin{center}\emph{\large\scshape Des dilemmes.}\end{center}

	\lettrine{N}{ous} avons dit dans le Chapitre Premier qu'il y avait des raisonnements composés de plus de trois propositions qu'on appelait généralement \emph{Sorites}.

Or comme entre ces sortes de raisonnements il n'y a guère que les dilemmes qui aient besoin d'une réflexion particulière, nous avons jugé à propos de les expliquer ici.

On peut définir un dilemme, un raisonnement composé, où après avoir divisé un tout en ses parties, on conclut affirmativement ou négativement du tout, ce qu'on a conclu de chaque partie.

Je dis \emph{ce qu'on a conclu de chaque partie}, et non pas seulement ce qu'on en aurait affirmé. Car on n'appelle proprement dilemme que quand ce que l'on dit de chaque partie est appuyé de la raison particulière.

Par exemple, ayant à prouver \emph{qu'on ne saurait être heureux en ce monde}, on le peut faire par ce dilemme :

	\begin{tabularx}{\textwidth}{X}
		\emph{On ne peut vivre en ce monde qu'en s'abandonnant à ses passions, ou en les combattant :} \\
		\emph{Si on s'y abandonne, c'est un état malheureux, parce qu'il est honteux, et qu'on n'y saurait être content :} \\
		\emph{Si on les combat, c'est aussi un état malheureux, parce qu'il n'y a rien de plus pénible que cette guerre intérieure qu'on est continuellement obligé de se faire à soi-même:} \\
		\emph{Il ne peut donc y avoir en cette vie de véritable bonheur.} \\
	\end{tabularx}

Si l'on veut prouver \emph{que les évêques qui ne travaillent point au salut des âmes qui leur sont commises sont inexcusables devant Dieu}, on le peut faire par un dilemme :

	\begin{tabularx}{\textwidth}{X}
		\emph{Ou ils sont capables de cette charge, ou ils en sont incapables:} \\
		\emph{S'ils en sont capables, ils sont inexcusables de ne s'y pas employer.} \\
		\emph{S'ils en sont incapables, ils sont inexcusables, d'avoir accepté une charge si importante dont ils ne pouvaient pas s'acquitter.} \\
		\emph{Et par conséquent en quelques manière que ce soit ils sont inexcusables devant Dieu, s'ils ne travaillent au salut des âmes qui leur sont commises.} \\
	\end{tabularx}

Mais on peut faire quelques observations sur ces sortes de raisonnements.

La première est que l'on n'exprime pas toujours toutes les propositions qui y entrent. Car, par exemple, le dilemme que nous venons de proposer est renfermé en ce peu de paroles dans une harangue de S. Charles à l'entrée de l'un de ses Conciles provinciaux : \emph{Si tanto muneri impares, cur tam ambitiosi? si pares, cur tam negigentes ?}

Ainsi il y a beaucoup de choses sous-entendues dans le dilemme célèbre par lequel un ancien philosophe prouvait qu'on ne se devait point mêler des affaires de la République.

\emph{Si on y agit bien, on offensera les hommes; si on y agit mal, on offensera les Dieux: Donc on ne s'en doit point mêler}.

Et de même en celui par lequel un autre prouvait qu'il ne se fallait point marier : \emph{Si la femme qu'on épouse est belle, elle cause de la jalousie : Si elle est laide elle déplaît : Donc il ne se faut point marier}.

Car dans l'un et l'autre de ces dilemmes la proposition qui devait contenir la partition est sous-entendue. Et c'est ce qui est fort ordinaire, parce qu'elle le sous-entend facilement, étant assez marquée par les propositions particulières où l'on traite chaque partie.

Et de plus, afin que la conclusion soit renfermée dans les prémisses, il faut sous-entendre par tout quelque chose de général qui puisse convenir à tout, comme dans le premier :

	\begin{tabularx}{\textwidth}{X}
		\emph{Si on agit bien, on offensera les hommes, ce qui est fâcheux:} \\
		\emph{Si on agit mal, on offensera les Dieux, ce qui est fâcheux aussi:} \\
		\emph{Donc il est fâcheux en toutes manières de se mêler des affaires de la} \\
		\emph{\space\space\space\space\space République.} \\
\end{tabularx}

Cet avis est fort important pour bien juger de la force d'un dilemme. Car ce qui fait, par exemple, que celui-là n'est pas concluant, est qu'il n'est point fâcheux d'offenser les hommes, quand on ne le peut éviter qu'en offensant Dieu.

La deuxième observation est, qu'un dilemme peut être vicieux principalement par deux défauts. L'un est quand la disjonctive sur laquelle il est fondé est défectueuse, ne comprenant pas tous les membres du tout que l'on divise.

Ainsi le dilemme pour ne se point marier ne conclut pas; parce qu'il peut y avoir des femmes, qui ne seront pas si belles qu'elles causent de la jalousie, ni si laides qu'elles déplaisent.

C'est aussi par cette raison un très faux dilemme que celui dont se servaient les anciens philosophes pour ne point craindre la mort. \emph{Ou notre âme}, disaient-ils, \emph{périt avec le corps, et ainsi n'ayant plus de sentiment, nous serons incapables de mal : ou si l'âme survit au corps, elle sera plus heureuse qu'elle n'était dans le corps : Donc la mort n'est point à craindre}. Car comme Montaigne même a fort bien remarqué, c'était un grand aveuglement de ne pas voir qu'on peut concevoir un troisième état entre ces deux-là, qui est que l'âme demeurant après le corps, se trouvait dans un état de tourment et de misère, ce qui donne un juste sujet d'appréhender la mort de peur de tomber dans cet état.

L'autre défaut, qui empêche que les dilemmes ne concluent, est, quand les conclusions particulières de chaque partie ne sont pas nécessaires. Ainsi il n'est pas nécessaire qu'une belle femme cause de la jalousie; puisqu'elle peut être si sage et si vertueuse, qu'on n'aura aucun sujet de se défier de sa fidélité.

Il n'est point nécessaire aussi qu'étant laide elle déplaise à son mari : puisqu'elle peut avoir d'autres qualités si avantageuses d'esprit et de vertu, qu'elle ne laissera pas de lui plaire.

La troisième observation est, que celui qui se sert d'un dilemme doit prendre garde qu'on ne le puisse retourner contre lui-même. Ainsi Aristote témoigne qu'on retourna contre le philosophe qui ne voulait pas qu'on se mêla des affaires publiques, le dilemme dont il se servait pour le prouver. Car on lui dit :

	\begin{tabularx}{\textwidth}{X}
		\emph{Si on s'y gouverne selon les règles corrompues des hommes, on contentera les hommes:} \\
		\emph{Si on garde la vraie justice, on contentera les Dieux:} \\
		\emph{Donc on s'en doit mêler.} \\
	\end{tabularx}

Néanmoins ce retour n'était pas raisonnable. Car il n'est pas avantageux de contenter les hommes en offensant Dieu.


