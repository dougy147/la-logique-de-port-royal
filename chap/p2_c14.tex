\subsubsection{\centering \Large CHAPITRE XIV}
\addcontentsline{toc}{section}{\protect\numberline{}{\scshape\bfseries XIV} - \emph{De la conversion des propositions affirmatives}}
\begin{center}\emph{\large\scshape De la conversion des propositions affirmatives.}\end{center}

	\lettrine{O}{n} appelle conversion d'une proposition, lorsqu'on change le sujet en attribut, et l'attribut en sujet, sans que la proposition cesse d'être vraie, si elle l'était auparavant, ou plutôt en sorte qu'il s'ensuive nécessairement de la conversion qu'elle est vraie, supposé qu'elle le fût.

Or, ce que nous venons de dire fera entendre facilement comment cette conversion doit se faire. Car, comme il est impossible qu'une chose soit jointe et unie à une autre, que cette autre ne soit jointe aussi à la première, et qu'il s'ensuit fort bien que si A est joint à B, B aussi est joint à A, il est clair qu'il est impossible que deux choses soient conçues comme identifiées, qui est la plus parfaite de toutes les unions, que cette union ne soit réciproque, c'est-à-dire, que l'on ne puisse faire une affirmation mutuelle des deux termes unis en la manière qu'ils sont unis; ce qui s'appelle conversion.

Ainsi, comme dans les propositions particulières affirmatives; par exemple
, lorsqu'on dit, \emph{quelque homme est juste}, le sujet et l'attribut sont tous deux particuliers, le sujet d'\emph{homme} étant particulier par la marque de particularité que l'on y ajoute, et l'attribut \emph{juste} l'étant aussi, parce que son étendue étant resserré par celle du sujet, il ne signifie que la seule justice qui est en quelque homme; il est évident que si quelque homme est identifié avec quelque juste, quelque juste aussi est identifié avec quelque homme ; et qu'ainsi il n'y a qu'à changer simplement l'attribut en sujet, en gardant la même particularité, pour convertir ces sortes de propositions.

On ne peut pas dire la même chose des propositions universelles affirmatives, à cause que, dans ces propositions, il n'y a que le sujet qui soit universel, c'est-à-dire qui soit pris selon toute son étendue, et que l'attribut, au contraire, est limité et restreint ; et partant, lorsqu'on le rendra sujet par la conversion, il faudra lui garder sa même restriction, et y ajouter une marque qui le détermine, de peur qu'on ne le prenne généralement. Ainsi, quand je dis que \emph{l'homme est animal}, j'unis l'idée d'\emph{homme} avec celle d'\emph{animal} restreinte et resserrée aux seuls hommes. Et partant, quand je voudrai envisager cette union comme par une autre face, et commençant par l'\emph{animal}, et affirmer ensuite l'\emph{homme}, il faut conserver à ce terme sa même restriction, et de peur que l'on ne s'y trompe, y ajouter quelque note de détermination. Car dans la première proposition, \emph{Tout homme est animal}, il était resserré par son sujet; mais devenant sujet, pour retenir sa même restriction il faut nécessairement y ajouter quelque terme qui l'exprime.

De sorte que de ce que les propositions universelles affirmatives ne peuvent se convertir qu'en particulières affirmatives, on ne doit pas conclure qu'elles se convertissent moins proprement que les autres; mais comme elles sont composées d'un sujet général et d'un attribut restreint, il est clair que lorsqu'on les convertit, en changeant l'attribut en sujet, elles doivent avoir un sujet restreint et resserré, c'est-à-dire particulier.

De là on doit tirer ces deux règles.

\begin{center}{\bfseries\scshape\large 1. Règle}\end{center}

	\emph{Les propositions universelles affirmatives se peuvent convertir, en ajoutant une marque de particularité à l'attribut devenu sujet}.




\begin{center}{\bfseries\scshape\large 2. Règle}\end{center}

	\emph{Les propositions particulières affirmatives doivent se convertir sans aucune addition, ni changement}, c'est-à-dire en retenant pour l'attribut devenu sujet, la marque de particularité qui était au premier sujet.

Mais il est aisé de voir que ces deux règles peuvent se réduire à une seule qui les comprendra toutes deux.

L'attribut étant restreint par le sujet dans toutes les propositions affirmatives, où on veut le faire devenir sujet, il faut lui conserver sa restriction, et par conséquent lui donner une marque de particularité, soit que le premier sujet fût universel, soit qu'il fût particulier.

Néanmoins il arrive assez souvent que des propositions universelles affirmatives peuvent se convertir en d'autres universelles ; mais c'est seulement lorsque l'attribut n'a pas de soi-même plus d'étendue que le sujet, comme lorsqu'on affirme la différence ou le propre de l'espèce, ou la définition du défini ; car alors l'attribut, n'étant pas restreint, peut se prendre dans la conversion aussi généralement que se prenait le sujet: \emph{Tout homme est raisonnable. Tout raisonnable est homme}.

Mais ces conversions n'étant véritables qu'en des rencontres particulières, on ne les compte point pour de vraies conversions, qui doivent être certaines et infaillibles par la seule transposition des termes.

