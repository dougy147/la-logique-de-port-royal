\subsubsection{\centering \Large CHAPITRE I}
\addcontentsline{toc}{section}{\protect\numberline{}{\scshape\bfseries I} - \emph{Ce que c'est qu'une proposition; et des quatre sortes de propositions}}
\begin{center}\emph{\large\scshape Ce que c'est qu'une proposition; et des quatre sortes de propositions.}\end{center}

	\lettrine{A}{près} avoir conçu les choses par nos idées, nous comparons ces idées ensemble ; et, trouvant que les unes conviennent entre elles, et que les autres ne conviennent pas, nous les lions ou délions, ce qui s'appelle \emph{affirmer} ou \emph{nier}, et généralement \emph{juger}.

Ce jugement s'appelle aussi \emph{proposition}, et il est aisé de voir qu'elle doit avoir deux termes : l'un de qui l'on affirme ou de qui l'on nie, lequel on appelle \emph{sujet}; et l'autre que l'on affirme ou que l'on nie, lequel s'appelle \emph{attribut} ou \emph{praedicatum}.

Et il ne suffit pas de concevoir ces deux termes ; mais il faut que l'esprit les lie ou les sépare : et cette action de notre esprit est marquée dans le discours par le verbe \emph{est}, ou seul quand nous affirmons, ou avec une particule négative quand nous nions. Ainsi quand je dis \emph{Dieu est juste}, Dieu est le sujet de cette proposition , et juste en est l'attribut; et le mot \emph{est} marque l'action de mon esprit qui affirme, c'est-à-dire qui lie ensemble les deux idées de \emph{Dieu} et de \emph{juste} comme convenant l'un à l'autre. Que si je dis \emph{Dieu n'est pas injuste}, \emph{est} étant joint avec les particules, \emph{ne pas}, signifie l'action contraire à celle d'affirmer, savoir : celle de nier par laquelle je regarde ces idées comme répugnantes l'une à l'autre, parce qu'il y a quelque chose d'enfermé dans l'idée d'\emph{injuste} qui est contraire à ce qui est enfermé dans l'idée de \emph{Dieu}.

Mais, quoique toute proposition enferme nécessairement ces trois choses, néanmoins, comme l'on a dit dans le chapitre précédent, elle peut n'avoir que deux mots ou même qu'un.

Car les hommes, voulant abréger leurs discours ont fait une infinité de mots qui signifient tout ensemble l'affirmation, c'est-à-dire ce qui est signifié par le verbe substantif, et de plus un certain attribut qui est affirmé. Tels sont tous les verbes, hors celui qu'on appelle substantif, comme \emph{Dieu existe}, c'est-à-dire \emph{est existant}, \emph{Dieu aime les hommes}, c'est-à-dire \emph{Dieu est aimant les hommes}. Et, le verbe substantif quand il est seul, comme quand je dis, \emph{je pense; Donc je suis}, cesse d'être purement substantif, parce qu'alors on y joint le plus général des attributs qui est l'être; car \emph{je suis} veut dire, \emph{je suis un être, je suis quelque chose}.

Il y a aussi d'autres rencontres où le sujet et l'affirmation sont renfermés dans un même mot, comme dans les premières et secondes personnes des verbes, surtout en latin ; comme quand je dis, \emph{sum Christianus}. Car le sujet de cette proposition est \emph{ego}, qui est renfermé dans \emph{sum}.

D'où il paraît que, dans cette même langue, un seul mot fait une proposition dans les premières et les secondes personnes des verbes, qui, par leur nature, renferment déjà l'affirmation avec l'attribut; comme \emph{veni, vidi, vici}, sont trois propositions.

On voit par là que toute proposition est affirmative ou négative, et que c'est ce qui est marqué par le verbe, qui est affirmé ou nié.

Mais il y a une autre différence dans les propositions, laquelle naît de leur sujet, qui est d'être universelles ou particulières, ou singulières.

Car les termes, comme nous avons déjà dit dans la première partie, sont ou singuliers, ou communs et universels.

Et les termes universels peuvent être pris, ou selon toute leur étendue, en les joignant aux signes universels exprimés, ou sous-entendus, comme \emph{omnis, tout}, pour l'affirmation ; \emph{Nullus, nul}, pour la négation : \emph{tout homme, nul homme}.

Ou selon une partie indéterminée de leur étendue, qui est lorsqu'on y joint le mot \emph{aliquis, quelque}, comme \emph{quelque homme}, quelques hommes, ou d'autres, selon l'usage des langues.

D'où il arrive une différence notable dans les propositions. Car, lorsque le sujet d'une proposition est un terme commun qui est pris dans toute son étendue ; la proposition s'appelle universelle, soit qu'elle soit affirmative, comme \emph{tout impie est fou}; ou négative, comme \emph{nul vicieux n'est heureux}.

Et, lorsque le terme commun n'est pris que selon une partie indéterminée de son étendue, à cause qu'il est resserré par le mot indéterminé \emph{quelque}, la proposition s'appelle particulière, soit qu'elle affirme, comme \emph{quelque cruel est lâche}; soit qu'elle nie, comme \emph{quelque pauvre n'est pas malheureux}.

Que si le sujet d'une proposition est singulier, comme quand je dis, \emph{Louis XIII a pris la Rochelle}, on l'appelle singulière.

Mais, quoique cette proposition singulière soit différente de l'universelle, en ce que son sujet n'est pas commun, elle doit néanmoins plutôt s'y rapporter qu'à la particulière ; parce que son sujet, par cela même qu'il est singulier, est nécessairement pris dans toute son étendue; ce qui fait l'essence d'une proposition universelle, et qui la distingue de la particulière; car il importe peu pour l'universalité d'une proposition, que l'étendue de son sujet soit grande ou petite, pourvu que, telle qu'elle soit, on la prenne tout entière ; et c'est pourquoi les propositions singulières tiennent lieu d'universelles dans l'argumentation. Ainsi l'on peut réduire toutes les propositions à quatre sortes, que l'on a marquées par ces quatre voyelles A, E, I, O.


\begin{table}[!htbp]
	\centering\begin{tabularx}{\textwidth}{lX}
	    A & L'universelle affirmative, comme \emph{tout vicieux est esclave}. \\
	    E & L'universelle négative, comme \emph{nul vicieux n'est heureux}. \\
	    I & La particulière affirmative, comme \emph{quelque vicieux est riche}. \\
	    O & La particulière négative, comme \emph{quelque vicieux n'est pas riche}. \\
    \end{tabularx}
\end{table}

Et pour les faire mieux retenir, on a fait ces deux vers :

\begin{center}
	\emph{Asserit A, negat E, verum generaliter ambo.}
	\\\emph{Asserit I, negat O, sed particulariter ambo.}
\end{center}

On a aussi accoutumé d'appeler quantité, l'universalité ou la particularité des proportions.

Et on appelle qualité, l'affirmation ou la négation qui dépendent du verbe, qui est regardé comme la forme de la proposition.

Et ainsi, A et E conviennent selon la quantité, et diffèrent selon la qualité, et de même I et O.

Mais A et I conviennent selon la qualité, et diffèrent selon la quantité, et de même E et O.

Les propositions se divisent encore, selon la matière, en vraies et en fausses ; et il est clair qu'il n'y en peut point avoir qui ne soient ni vraies ni fausses, puisque toute proposition marquant le jugement que nous faisons des choses, elle est vraie quand ce jugement est conforme à la vérité, et fausse lorsqu'il n'y est pas conforme.

Mais, parce que nous manquons souvent de lumière pour reconnaître le vrai et le faux, outre les propositions qui nous paraissent certainement vraies, et celles qui nous paraissent certainement fausses, il y en a qui nous semblent vraies, mais dont la vérité ne nous est pas si évidente que nous n'ayons quelque appréhension qu'elles ne soient fausses; ou bien qui nous semblent fausses, mais dont nous ne nous tenons pas assurés de la fausseté. Ce sont les propositions qu'on appelle probables; dont les premières sont plus probables, et les dernières moins probables. Nous dirons quelque chose dans la quatrième partie de ce qui nous fait juger avec certitude qu'une proposition est vraie.
