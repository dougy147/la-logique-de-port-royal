\subsubsection{\centering \Large CHAPITRE II}

\addcontentsline{toc}{section}{\protect\numberline{}{\scshape\bfseries II} - \emph{Des idées, considérées selon leurs objets}}
\begin{center}\emph{\large\scshape Des idées, considérées selon leurs objets.}\end{center}

	\lettrine{T}{out} ce que nous concevons est représenté à notre esprit, ou comme chose, ou comme manière de chose, ou comme chose modifiée.

J'appelle chose ce que l'on conçoit comme subsistant par soi-même, et comme le sujet de tout ce que l'on y conçoit. C'est ce que l'on appelle autrement substance.

J'appelle manière de chose, ou mode, ou attribut, ou qualité, ce qui étant conçu dans la chose, et comme ne pouvant subsister sans elle, la détermine à être d'une certaine façon, et la fait nommer telle.

J'appelle chose modifiée, lorsqu'on considère la substance comme déterminée par une certaine manière, ou mode.

C'est ce qui se comprendra mieux par des exemples.

Quand je considère un corps, l'idée que j'en ai me représente une chose ou une substance, parce que je le considère comme une chose qui subsiste par soi-même, et qui n'a point besoin d'aucun sujet pour exister.

Mais quand je considère que ce corps est rond, l'idée que j'ai de la rondeur ne me représente qu'une manière d'être, ou un mode que je conçois ne pouvoir subsister naturellement sans le corps dont il est rondeur.

Et enfin, quand, joignant le mode avec la chose, je considère un corps rond, cette idée me représente une chose modifiée.

Les noms qui servent à exprimer les choses, s'appellent substantifs ou absolus, comme terre, soleil, esprit, Dieu.

Ceux aussi qui signifient premièrement et directement les modes, parce qu'en cela ils ont quelque rapport avec les substances, sont aussi appelés substantifs et absolus, comme dureté, chaleur, justice, prudence.

Les noms qui signifient les choses comme modifiées, marquant premièrement et directement la chose, quoique plus confusément, et indirectement le mode, quoique plus distinctement, sont appelés adjectifs ou connotatifs; comme rond, dur, juste, prudent.

Mais il faut remarquer que notre esprit, étant accoutumé de connaître la plupart des choses comme modifiées, parce qu'il ne les connaît presque que par les accidents ou qualités qui nous frappent les sens, divise souvent la substance même dans son essence en deux idées, dont il regarde l'une comme sujet, et l'autre comme mode. Ainsi, quoique tout ce qui est en Dieu soit Dieu même, on ne laisse pas de le concevoir comme un être infini, et de regarder l'infinité comme un attribut de Dieu, et l'être comme sujet de cet attribut. Ainsi l'on considère souvent l'homme comme le sujet de l'humanité, \emph{habens humanitatem}, et par conséquent comme une chose modifiée.

Et alors on prend pour mode l'attribut essentiel qui est la chose même, parce qu'on le conçoit comme dans un sujet. C'est proprement ce qu'on appelle abstrait des substances, comme humanité, corporéité, raison.

Il est néanmoins très important de savoir ce qui est véritablement mode, et ce qui ne l'est qu'en apparence, parce qu'une des principales causes de nos erreurs est de confondre les modes avec les substances, et les substances avec les modes. Il est donc de la nature du véritable mode, qu'on puisse concevoir sans lui clairement et distinctement la substance dont il est mode, et que néanmoins on ne puisse pas réciproquement concevoir clairement ce mode, sans concevoir en même temps le rapport qu'il a à la substance dont il est mode, et sans laquelle il ne peut naturellement exister.

Ce n'est pas qu'on ne puisse concevoir le mode sans faire une attention distincte et expresse à son sujet: mais ce qui montre que la notion du rapport à la substance est enfermée au moins confusément dans celle du mode, c'est qu'on ne saurait nier ce rapport du mode, qu'on ne détruise l'idée qu'on en avait : au lieu que, quand on conçoit deux choses et deux substances, l'on peut nier l'une de l'autre sans détruire les idées qu'on avait de chacune.

Par exemple, je puis bien concevoir la prudence, sans faire attention distincte à un homme qui soit prudent ; mais je ne puis concevoir la prudence en niant le rapport qu'elle a à un homme ou à une autre nature intelligente qui ait cette vertu.

Et, au contraire, lorsque j'ai considéré tout ce qui convient à une substance étendue qu'on appelle corps, comme l'extension, la figure, la mobilité, la divisibilité, et que d'autre part je considère tout ce qui convient à l'esprit et à la substance qui pense, comme de penser, de douter, de se souvenir, de vouloir, de raisonner, je puis nier de la substance étendue tout ce que je conçois de la substance qui pense, sans cesser pour cela de concevoir très distinctement la substance étendue et tous les autres attributs qui y sont joints, et je puis réciproquement nier de la substance qui pense tout ce que j'ai conçu de la substance étendue, sans cesser pour cela de concevoir très distinctement tout ce que je conçois dans la substance qui pense.

Et c'est ce qui fait voir aussi que la pensée n'est point un mode de la substance étendue, parce que l'étendue et toutes les propriétés qui la suivent se peuvent nier de la pensée, sans qu'on cesse pour cela de bien concevoir la pensée.

