\subsubsection{\centering \Large CHAPITRE XII}
\addcontentsline{toc}{section}{\protect\numberline{}{\scshape\bfseries XII} - \emph{Quelques règles pour bien conduire sa raison dans la croyance des événements qui dépendent de la foi humaine}}
\begin{center}\emph{\large\scshape Quelques règles pour bien conduire sa raison dans la croyance des événements qui dépendent de la foi humaine.}\end{center}

	\lettrine{L}{'usage} le plus ordinaire du bon sens et de cette puissance de notre âme qui nous fait discerner le vrai d'avec le faux, n'est pas dans les sciences spéculatives auxquelles il y a si peu de personnes qui soient obligées de s'appliquer ; mais il n'y a guère d'occasion où on l'emploie plus souvent, et où elle soit plus nécessaire, que dans le jugement que l'on porte de ce qui se passe tous les jours parmi les hommes.

Je ne parle point du jugement que l'on fait si une action est bonne ou mauvaise, digne de louange ou de blâme, parce que c'est à la morale à le régler, mais seulement de celui que l'on porte touchant la vérité ou la fausseté des événements humains; ce qui seul peut regarder la logique, soit qu'on les considère comme passés, comme lorsqu'il ne s'agit que de savoir si on doit les croire ou ne pas les croire ; ou qu'on les considère dans le temps à venir, comme lorsqu'on appréhende qu'ils n'arrivent, ou qu'on espère qu'ils arriveront : ce qui règle nos craintes et nos espérances.

Il est certain qu'on peut faire quelques réflexions sur ce sujet, qui ne seront peut-être pas inutiles, et qui pourront au moins servir à éviter des fautes où plusieurs personnes tombent pour n'avoir pas assez consulté les règles de la raison.

La première réflexion est qu'il faut mettre une extrême différence entre deux sortes de vérités : les unes qui regardent seulement la nature des choses et leur essence immuable, indépendamment de leur existence ; et les autres qui regardent les choses existantes, et surtout les événements humains et contingents, qui peuvent être et n'être pas quand il s'agit de l'avenir, et qui pouvaient n'avoir pas été quand il s'agit du passé. J'entends tout ceci selon leurs causes prochaines, en faisant abstraction de leur ordre immuable dans la providence de Dieu ; parce que, d'une part, il n'empêche pas la contingence, et que, de l'autre, ne nous étant pas connu, il ne contribue en rien à nous faire croire les choses.

Dans la première sorte de vérités, comme tout y est nécessaire, rien n'est vrai qui ne soit universellement vrai ; et ainsi nous devons conclure qu'une chose est fausse, si elle est fausse en un seul cas. Et au contraire la possibilité est une marque assurée de la vérité au regard de ce qui est reconnu possible, lorsqu'il ne s'agit que de l'essence des choses : car notre esprit ne saurait rien concevoir comme possible, qu'il ne le conçoive comme véritable selon son essence.

Ainsi quand un géomètre a conçu qu'une ligne peut être décrite par 4 ou 5 mouvements différents, il ne se met point en peine que cette ligne soit actuellement décrite, parce qu'il lui suffit que cela soit possible pour le regarder comme vrai, et pour raisonner sur cette supposition.

Mais si l'on pense se servir des mêmes règles dans la croyance des événements humains, on n'en jugera jamais que faussement, si ce n'est par hasard, et on y fera mille faux raisonnements.

Car ces événements étant contingents de leur nature, il serait ridicule d'y chercher une vérité nécessaire ; et ainsi un homme serait tout à fait déraisonnable qui n'en voudrait croire aucun, que quand on lui aurait fait voir qu'il serait absolument nécessaire que la chose se fût passée de la sorte.

Et il ne serait pas moins déraisonnable s'il voulait m'obliger d'en croire quelqu'un, comme serait la conversion du roi de la Chine à la religion chrétienne, par cette seule raison que cela n'est pas impossible ; car un autre qui m'assurerait du contraire, pouvant se servir de la même raison, il est clair que cela ne pourrait me déterminer à croire l'un plutôt que l'autre.

Il faut donc poser pour une maxime certaine et indubitable dans cette rencontre, que la seule possibilité d'un événement n'est pas une raison suffisante pour me le faire croire ; et que je puis aussi avoir raison de le croire, quoique je ne juge pas impossible que le contraire soit arrivé : de sorte que de deux événements je pourrai avoir raison de croire l'un et de ne pas croire l'autre, quoique je les croie tous deux possibles.

Mais par où me déterminerai-je donc à croire l'un plutôt que l'autre, si je les juge tous deux possibles ? Ce sera par cette maxime.

Pour juger de la vérité d'un événement, et me déterminer à le croire ou à ne pas le croire, il ne faut pas le considérer nûment et en lui-même, comme on ferait une proposition de géométrie ; mais il faut prendre garde à toutes les circonstances qui l'accompagnent, tant intérieures qu'extérieures. J'appelle circonstances intérieures celles qui appartiennent au fait même, et extérieures celles qui regardent les personnes par le témoignage desquelles nous sommes portés à le croire. Cela étant fait, si toutes ces circonstances sont telles qu'il n'arrive jamais, ou fort rarement, que de pareilles circonstances soient accompagnées de fausseté, notre esprit se porte naturellement à croire que cela est vrai, et il a raison de le faire, surtout dans la conduite de la vie, qui ne demande pas une plus grande certitude que cette certitude morale, et qui doit même se contenter en plusieurs rencontres de la plus grande probabilité.

Que si, au contraire, ces circonstances ne sont pas telles qu'elles ne se trouvent fort souvent avec la fausseté, la raison veut ou que nous demeurions en suspens, ou que nous tenions pour faux ce qu'on nous dit, quand nous ne voyons aucune apparence que cela soit vrai, encore que nous n'y voyons pas une entière impossibilité.

On demande, par exemple, si l'histoire du baptême de Constantin par saint Sylvestre est vraie ou fausse ? Baronius la croit vraie ; le cardinal du Perron, l'évêque Sponde, le P. Pétau, le P. Morin et les plus habiles gens de l'Église la croient fausse. Si on s'arrêtait à la seule possibilité; on n'aurait pas droit de la rejeter, car elle ne contient rien d'absolument impossible ; et il est même possible, absolument parlant, qu'Eusèbe qui témoigne le contraire, ait voulu mentir pour favoriser les Ariens, et que les Pères qui l'ont suivi aient été trompés par son témoignage : mais si l'on se sert de la règle que nous venons d'établir, qui est de considérer quelles sont les circonstances de l'un ou de l'autre baptême de Constantin, et qui sont celles qui ont plus de marques de vérité, on trouvera que ce sont celles du dernier : car, d'une part, il n'y a pas grand sujet de s'appuyer sur le témoignage d'un écrivain aussi fabuleux qu'est l'auteur des Actes de saint Sylvestre, qui est le seul ancien qui ait parlé du baptême de Constantin à Rome; et de l'autre, il n'y a aucune apparence qu'un homme aussi habile qu'Eusèbe eût osé mentir en rapportant une chose aussi célèbre qu'était le baptême du premier empereur qui avait rendu la liberté à l'Église, et qui devait être connue de toute la terre, lorsqu'il l'écrivait, puisque ce n'était que quatre ou cinq ans après la mort de cet empereur.

Il y a néanmoins une exception à cette règle, dans laquelle on doit se contenter de la possibilité et de la vraisemblance ; c'est quand un fait, qui est d'ailleurs suffisamment attesté, est combattu par des inconvénients et des contrariétés apparentes avec d'autres histoires : car alors il suffit que les solutions qu'on apporte à ces contrariétés soient possibles et vraisemblables ; et c'est agir contre la raison que d'en demander des preuves positives, parce que le fait en soi étant suffisamment prouvé, il n'est pas juste de demander qu'on en prouve de la même sorte toutes les circonstances : autrement on pourrait douter de mille histoires très assurées qu'on ne peut accorder avec d'autres qui ne le sont pas moins, que par des conjectures qu'il est impossible de prouver positivement.

On ne saurait, par exemple, accorder ce qui est rapporté dans les Livres des Rois et dans ceux des Paralipomènes des années des règnes de divers rois de Judas et d'Israël, qu'en donnant à quelques-uns de ces rois deux commencements de règne, l'un, du vivant, et l'autre après la mort de leurs pères. Que si l'on demande quelle preuve on a qu'un tel roi ait régné quelque temps avec son père, il faut avouer qu'on n'en a point de positive ; mais il suffit que ce soit une chose possible, et qui est arrivée assez souvent en d'autres rencontres, pour avoir droit de la supposer comme une circonstance nécessaire pour allier des histoires d'ailleurs très certaines.

C'est pourquoi il n'y a rien de plus ridicule que les efforts qu'ont faits quelques hérétiques de ce dernier siècle pour prouver que saint Pierre n'a jamais été à Rome. Ils ne peuvent nier que cette vérité ne soit attestée par tous les auteurs ecclésiastiques, et même les plus anciens, comme Papias, saint Denis de Corinthe ; Caïus, saint Irénée, Tertullien, sans qu'il s'en trouve aucun qui l'ait niée ; et néanmoins ils s'imaginent pouvoir la ruiner par des conjectures, comme par exemple, que saint Paul ne fait pas mention de saint Pierre dans ses Épîtres écrites de Rome; et quand on leur répond que saint Pierre pouvait être alors hors de Rome, parce qu'on ne prétend pas qu'il y ait été tellement attaché qu'il n'en soit souvent sorti pour aller prêcher l'Évangile en d'autres lieux, ils répliquent que cela se dit sans preuve ; ce qui est impertinent, parce que le fait qu'ils contestent étant une des vérités les plus assurées de l'histoire ecclésiastique, c'est à ceux qui le combattent de faire voir qu'il contient des contrariétés avec l'Écriture, et il suffit à ceux qui le soutiennent de résoudre ces prétendues contrariétés, comme on fait celles de l'Écriture même, à quoi nous avons montré que la possibilité suffisait.



