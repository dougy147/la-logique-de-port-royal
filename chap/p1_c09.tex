\subsubsection{\centering \Large CHAPITRE IX}
\addcontentsline{toc}{section}{\protect\numberline{}{\scshape\bfseries IX} - \emph{D'une autre cause qui met de la confusion dans nos pensées et dans nos discours, qui est que nous les attachons à des mots}}
\begin{center}\emph{\large\scshape D'une autre cause qui met de la confusion dans nos pensées et dans nos discours, qui est que nous les attachons à des mots.}\end{center}

	\lettrine{N}{ous} avons déjà dit que la nécessité que nous avons d'user de signes extérieurs pour nous faire entendre, fait que nous attachons tellement nos idées aux mots, que souvent nous considérons plus les mots que les choses. Or, c'est une des causes les plus ordinaires de la confusion de nos pensées et de nos discours.

Car il faut remarquer que, quoique les hommes aient souvent de différentes idées des mêmes choses, ils se servent néanmoins des mêmes mots pour les exprimer, comme l'idée qu'un philosophe païen a de la vertu, n'est pas la même que celle qu'en a un théologien, et néanmoins chacun exprime son idée par le même mot de vertu.

De plus, les mêmes hommes en différents âges ont considéré les mêmes choses en des manières très différentes, et néanmoins ils ont toujours rassemblé toutes ces idées sous un même nom : ce qui fait que prononçant ce mot, ou l'entendant prononcer, on se brouille facilement, le prenant tantôt selon une idée, tantôt selon l'autre. Par exemple, l'homme ayant reconnu qu'il y avait en lui quelque chose, quoi que ce fût, qui faisait qu'il se nourrissait et qu'il croissait, a appelé cela \emph{âme}, et à étendu cette idée à ce qui est de semblable, non seulement dans les animaux, mais même dans les plantes. Et ayant vu encore qu'il pensait, il a encore appelé du nom d'\emph{âme} ce qui était en lui le principe de la pensée ; d'où il est arrivé que, par cette ressemblance de nom, il a pris pour la même chose ce qui pensait et ce qui faisait que le corps se nourrissait et croissait. De même on a étendu également le mot de vie à ce qui est cause des opérations des animaux, et à ce qui nous fait penser, qui sont deux choses absolument différentes.

Il y a de même beaucoup d'équivoques dans les mots de \emph{sens} et de \emph{sentiment}, lors même qu'on ne prend ces mots que pour quelqu'un des cinq sens corporels; car il se passe ordinairement trois choses en nous lorsque nous usons de nos sens, comme lorsque nous voyons quelque chose. La première est qu'il se fait de certains mouvements dans les organes corporels, comme dans l'œil et dans le cerveau. La deuxième, que ces mouvements donnent occasion à notre âme de concevoir quelque chose, comme lorsque ensuite du mouvement qui se fait dans notre œil par la réflexion de la lumière dans des gouttes de pluie opposées au soleil, elle a des idées du rouge, du bleu et de l'orangé. La troisième est le jugement que nous faisons de ce que nous voyons, comme l'arc-en-ciel, à qui nous attribuons ces couleurs, et que nous concevons d'une certaine grandeur, d'une certaine figure et en une certaine distance. La première de ces trois choses est uniquement dans notre corps. Les deux autres sont seulement en notre âme, quoiqu'à l'occasion de ce qui se passe dans notre corps. Et néanmoins nous comprenons toutes les trois, quoique si différentes, sous le même nom de \emph{sens et de sentiment}, ou de \emph{vue}, d'\emph{ouïe}, etc. Car quand on dit que l'œil voit, que l'oreille oit, cela ne peut s'entendre que selon le mouvement de l'organe corporel, étant bien clair que l'œil n'a aucune perception des objets qui le frappent, et que ce n'est pas lui qui en juge. On dit au contraire qu'on n'a pas vu une personne qui s'est présentée devant nous, et qui nous a frappé les yeux, lorsque nous n'y avons pas fait réflexion. Et alors on prend le mot de voir pour la pensée qui se forme en notre âme, ensuite de ce qui se passe dans notre œil et dans notre cerveau ; et selon cette signification du mot de \emph{voir}, c'est l'âme qui voit et non pas le corps, comme Platon le soutient, et Cicéron après lui par ces paroles : \emph{Nos enim ne nunc quidem oculis cernimus ea quae videmus. Neque enim est nullus sensus in corpore. Viae quasi quaedam sunt ad oculos, ad aures, ad nares ad sedem animi perforatae itaque sape aut cogitatione, aut aliqua vi morbi impediti apertis atque integris et oculis et auribus, nec videmus, nec audimus ; ut facile intelligi possit, animum et videre et audire non eas partes quae quasi fenestrae sunt animi}. Enfin, on prend les mots de sens, de la vue, de l'ouïe, etc., pour la dernière de ces trois choses, c'est-à-dire pour les jugements que notre âme fait ensuite des perceptions qu'elle a eues à l'occasion de ce qui s'est passé dans les organes corporels, lorsque l'on dit que les sens se trompent, comme quand ils voient dans l'eau un bâton courbé, et que le soleil ne nous paraît que de deux pieds de diamètre. Car il est certain qu'il ne peut y avoir d'erreur ou de fausseté ni en tout ce qui se passe dans l'organe corporel, ni dans la seule perception de notre âme, qui n'est qu'une simple appréhension ; mais que toute l'erreur ne vient que de ce que nous jugeons mal, en concluant, par exemple, que le soleil n'a que deux pieds de diamètre, parce que sa grande distance fait que l'imago qui s'en forme dans le fond de notre œil est à peu près de la même grandeur que celle qu'y formerait un objet de deux pieds à une certaine distance plus proportionnée à notre manière ordinaire de voir. Mais parce que nous avons fait ce jugement dès l'enfance, et que nous y sommes tellement accoutumés qu'il se fait au même instant que nous voyons le soleil, sans presque aucune réflexion, nous l'attribuons à la vue, et nous disons que nous voyons les objets petits ou grands, selon qu'ils sont plus proches et plus éloignés de nous, quoique ce soit notre esprit et non notre œil qui juge de leur petitesse et de leur grandeur.

Toutes les langues sont pleines d'une infinité de mots semblables, qui, n'ayant qu'un même son, sont néanmoins signes d'idées entièrement différentes.

Mais il faut remarquer que quand un nom équivoque signifie deux choses qui n'ont nul rapport entre elles, et que les hommes n'ont jamais confondues dans leur pensée, il est presque impossible alors qu'on s'y trompe, et qu'il soit cause d'aucune erreur ; comme on ne se trompera pas, si l'on a un peu de sens commun, par l'équivoque du mot \emph{bélier}, qui signifie un animal, et un signe du zodiaque. Au lieu que quand l'équivoque est venue de l'erreur même des hommes, qui ont confondu par méprise des idées différentes, comme dans le mot d'âme, il est difficile de s'en détromper, parce qu'on suppose que ceux qui se sont les premiers servis de ces mots, les ont bien entendus; et ainsi nous nous contentous souvent de les prononcer, sans examiner jamais si l'idée que nous en avons est claire et distincte ; et nous attribuons même à ce que nous nommons d'un même nom ce qui ne convient qu'à des idées de choses incompatibles, sans nous apercevoir que cela ne vient que de ce que nous avons confondu deux choses différentes sous un même nom.

