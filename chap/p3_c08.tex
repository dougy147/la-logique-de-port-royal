\subsubsection{\centering \Large CHAPITRE VIII}
\addcontentsline{toc}{section}{\protect\numberline{}{\scshape\bfseries VIII} - \emph{Des modes la quatrième figure}}
\begin{center}\emph{\large\scshape Des modes de la quatrième figure.}\end{center}

	\lettrine{L}{a} quatrième figure est celle où le moyen est attribut dans la majeure, et sujet dans la mineure ; elle est si peu naturelle, qu'il est assez inutile d'en donner les règles. Les voilà néanmoins, afin qu'il ne manque rien à la démonstration de toutes les manières simples de raisonner.

\begin{center}{\bfseries\scshape\large 1. Règle}\end{center}

	\emph{Quand la majeure est affirmative, la mineure est toujours universelle.}

Car le moyen est pris particulièrement dans la majeure affirmative, parce qu'il en est l'attribut. Il faut donc (par la première règle générale) qu'il soit pris généralement dans la mineure, et que par conséquent, il la rende universelle, parce qu'il en est le sujet.

\begin{center}{\bfseries\scshape\large 2. Règle}\end{center}

	\emph{Quand la mineure est affirmative, la conclusion est toujours particulière.}

Car le petit terme est attribut dans la mineure, et par conséquent il y est pris particulièrement, quand elle est affirmative ; d'où il s'ensuit (par la deuxième règle générale) qu'il doit être aussi particulier dans la conclusion, ce qui la rend particulière, parce qu'il en est le sujet.

\begin{center}{\bfseries\scshape\large 3. Règle}\end{center}

	\emph{Dans les modes négatifs, la majeure doit être générale.}

Car la conclusion étant négative, le grand terme y est pris généralement. Il faut donc (par la deuxième règle générale) qu'il soit pris aussi généralement dans les prémisses, or, il est le sujet de la majeure aussi bien que dans la deuxième figure, et par conséquent il faut, aussi bien que dans la deuxième figure, qu'étant pris généralement il rende la majeure générale.


\begin{center}{\bfseries Démonstration.}\end{center}

	\emph{Qu'il ne peut y avoir que cinq modes dans la quatrième figure.}

Des dix modes concluants, A, I, I, et A, O, O, sont exclus par la première règle.

A, A, A, et E, A, E, sont exclus par la deuxième. O, A, O, par la troisième.

Il ne reste donc que ces cinq :

\begin{center}
$ \text {2 Affirmatifs} \left \{
    \begin{array}{ccc}
	    \text {A,} & \text{A,} & \text{I} \\
  	    \text {I,} & \text{A,} & \text{I} \\
    \end{array}
	    \right \} $
\end{center}
\begin{center}
$ \text {3 Négatifs} \left \{
    \begin{array}{ccc}
	    \text {A,} & \text{E,} & \text{E} \\
	    \text {E,} & \text{A,} & \text{O} \\
	    \text {E,} & \text{I,} & \text{O} \\
    \end{array}
	    \right \} $
\end{center}

Ces cinq modes peuvent se renfermer dans ces mots artificiels :

\newpage

	\begin{tabularx}{\textwidth}{lX}
		{\scshape\large Bar—} & \emph{Tout homme sage est modéré:} \\
		{\scshape\large Ba— } & \emph{Tout homme modéré est ennemi des grandes fortunes:} \\
		{\scshape\large Ri. } & \emph{Donc quelque ennemi des grandes fortunes est sage.} \\
		{\scshape\large Ca— } & \emph{Tout vice est blâmable:} \\
		{\scshape\large Len—} & \emph{Nulle chose blâmable n'est à imiter:} \\
		{\scshape\large Tes.} & \emph{Donc nulle chose à imiter n'est blâmable.} \\
		{\scshape\large Di— } & \emph{Quelque fou dit vrai:} \\
		{\scshape\large Ba— } & \emph{Quiconque dit vrai mérite d'être écouté:} \\
		{\scshape\large Tis.} & \emph{Donc quelqu'un qui mérite d'être écouté est fou.} \\
		{\scshape\large Fes—} & \emph{Nul avare n'est content:} \\
		{\scshape\large Pa— } & \emph{Tout homme content est riche:} \\
		{\scshape\large Mo. } & \emph{Donc quelque riche n'est pas avare.} \\
		{\scshape\large Fre—} & \emph{Nul esclave n'est libre:} \\
		{\scshape\large Si— } & \emph{Quelque libre est misérable:} \\
		{\scshape\large Som.} & \emph{Donc quelque misérable n'est pas esclave.} \\
	\end{tabularx}

Il est bon d'avertir que l'on exprime ordinairement ces cinq modes en cette façon : \emph{Baralipton, Celantes, Dibatis, Fapesmo, Frisesomorum}; ce qui est venu de ce qu'Aristote n'ayant pas fait une figure séparée de ces modes, on ne les a regardés que comme des modes indirects de la première figure, parce qu'on a prétendu que la conclusion en était renversée, et que l'attribut en était le véritable sujet. C'est pourquoi ceux qui ont suivi cette opinion ont mis pour première proposition celle où le sujet de la conclusion entre, et pour mineure celle où entre l'attribut.

Et ainsi ils ont donné neuf modes à la première figure, quatre directs et cinq indirects, qu'ils ont renfermés dans ces deux vers.

\emph{Barbara, Celarent, Darii, Ferio: Baralipton, Celantes, Dabitis, Fapesmo, Frisesomorum.}

Et pour les deux autres figures.

\emph{Cesare, Camestres, Festino, Baroco: Darapti, Felapton, Disamis, Datisi, Bocardo, Ferison.}

Mais, comme la conclusion étant toujours supposée, puisque c'est ce qu'on veut prouver, on ne peut pas dire proprement qu'elle soit jamais renversée, nous avons cru qu'il était plus avantageux de prendre toujours pour majeure la proposition où entre l'attribut de la conclusion : ce qui nous a obligés, pour mettre la majeure la première, de renverser ces mots artificiels. De sorte que, pour mieux les retenir, on peut les renfermer en ce vers.

\emph{Barbari, Calentes, Dibatis, Fespamo, Frisesom.}

\newpage

\begin{center}{\bfseries Récapitulation}\end{center}

	\begin{center}\emph{Des diverses espèces de syllogismes.}\end{center}

De tout ce qu'on vient de dire, on peut conclure qu'il y a dix-neuf espèces de syllogismes, qu'on peut diviser en diverses manières.

\begin{center}
$ \text {$1$. En} \left  \{
    \begin{array}{lc}
	    \text {Généraux} & \text{5} \\
	    \text {Particuliers} & \text{14} \\
    \end{array}
    \right \} $
\end{center}

\begin{center}
$ \text {$2$. En} \left \{
    \begin{array}{lc}
	    \text {Affirmatifs} & \text{7} \\
	    \text {Négatifs} & \text{12} \\
    \end{array}
    \right \} $
\end{center}

\begin{center}
$ \text {$3$. En ceux qui concluent}
	\left \{
    	\begin{array}{cc}
		\text {A} & \text{1} \\
  		\text {E} & \text{4} \\
  		\text {I} & \text{6} \\
		\text {O} & \text{8} \\
    	\end{array}
	\right \}$
\end{center}

$4$. Selon les différentes figures en les subdivisant par les modes, ce qui a déjà été assez fait dans l'explication de chaque figure.

\smallbreak
$5$. Ou, au contraire, selon les modes, en les subdivisant par les figures.

Car comme on a fait voir dans le Chapitre IV qu'il n'y avait que dix modes de concluants, en n'y considérant point la diverse disposition du moyen, mais qu'un seul servant à différentes figures pouvait faire différentes espèces de syllogisme, on peut encore trouver ce même nombre de 19 par cette sorte de division, parce qu'il y en a 3 dont chacun ne conclut qu'en une seule figure : 6 dont chacun conclut en 2 figures : et 1 qui conclut en toutes les 4 figures, ce qui fait en tout 19 espèces de syllogismes. Je laisse à en faire la table à ceux qui en voudrait prendre la peine pour le soulagement de la mémoire. Car cela ne mérite pas qu'on la mette ici.

