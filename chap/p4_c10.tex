\subsubsection{\centering \Large CHAPITRE X}
\addcontentsline{toc}{section}{\protect\numberline{}{\scshape\bfseries X} - \emph{La méthode des sciences réduite à huit règles principales}}
\begin{center}\emph{\large\scshape La méthode des sciences réduite à huit règles principales.}\end{center}


	\lettrine{O}{n} peut conclure de tout ce que nous venons de dire, que, pour avoir une méthode qui soit encore plus parfaite que celle qui est en usage parmi les géomètres, on doit ajouter deux ou trois règles aux cinq que nous avons proposées dans le Chapitre II. De sorte que toutes ces règles peuvent se réduire à huit.

Dont les deux premières regardent les idées, et peuvent se rapporter à la première partie de cette Logique.

La troisième et la quatrième regardent les axiomes, et peuvent se rapporter à la deuxième partie.

La cinquième et la sixième regardent les raisonnements, et peuvent se rapporter à la troisième partie.

Et les deux dernières regardent l'ordre, et peuvent se rapporter à la quatrième partie.


\begin{center}\textbf{Deux règles touchant les définitions.}\end{center}

$1$. Ne laisser aucun des termes un peu obscurs ou équivoques sans le définir.
\smallbreak
$2$. N'employer dans les définitions que des termes parfaitement connus ou déjà expliqués.

\begin{center}\textbf{Deux règles pour les axiomes.}\end{center}

$3$. Ne demander en axiomes que des choses parfaitement évidentes.
\smallbreak
$4$. Recevoir pour évident ce qui n'a besoin que d'un peu d'attention pour être reconnu véritable.

\begin{center}\textbf{Deux règles pour les démonstrations.}\end{center}

$5$. Prouver toutes les propositions un peu obscures, en n'employant à leur preuve que les définitions qui auront précédé, et les axiomes qui auront été accordés, ou les propositions qui auront déjà été démontrées.
\smallbreak
$6$. N'abuser jamais de l'équivoque des termes, en manquant de substituer mentalement les définitions qui les restreignent et qui les expliquent.

\begin{center}\textbf{Deux règles pour la méthode.}\end{center}

$7$. Traiter les choses, autant qu'il se peut, dans leur ordre naturel, en commençant par les plus générales et les plus simples, et expliquant tout ce qui appartient à la nature du genre avant que de passer aux espèces particulières.
\smallbreak
$8$. Diviser, autant qu'il se peut, chaque genre en toutes ses espèces, chaque tout en toutes ses parties, et chaque difficulté en tous ses cas.

J'ai ajouté à ces deux règles, \emph{autant qu'il se peut}, parce qu'il est vrai qu'il arrive beaucoup de rencontres où on ne peut pas les observer à la rigueur, soit à cause des bornes de l'esprit humain, soit à cause de celles qu'on a été obligé de donner à chaque science. Ce qui fait qu'on y traite souvent d'une espèce, sans qu'on puisse y traiter tout ce qui appartient au genre; comme on traite du cercle dans la géométrie commune, sans rien dire en particulier de la ligne courbe qui en est le genre, qu'on se contente seulement de définir.

On ne peut pas aussi expliquer d'un genre tout ce qui pourrait s'en dire, parce que cela serait souvent trop long; mais il suffit d'en dire tout ce qu'on veut en dire avant que de passer aux espèces.

Enfin je crois qu'une science ne peut être traitée parfaitement, qu'on n'ait pas grand égard à ces deux dernières règles, aussi bien qu'aux autres, et qu'on ne se résolue à ne s'en point dispenser que par nécessité, ou pour une grande utilité.

On avoue néanmoins qu'on ne s'y est pas beaucoup astreint dans cet ouvrage : Que s'il y en a qui s'en plaignent on leur confessera franchement, que cette Logique ayant été augmentée près de moitié depuis les premiers essais qui en furent fait en 4 ou 5 jours, il ne faut pas s'étonner si les diverses pièces qu'on y a ajoutées en divers temps, et pendant même qu'on l'imprimait, ne sont pas toujours si bien placées qu'elles auraient pu être si on les y avait mises d'abord. C'est pourquoi même on a dit dans le \emph{Discours} qui est à l'entrée, que plusieurs personnes se pouvaient contenter de la première et de la quatrième partie, en mettant ainsi toute la troisième au nombre des choses plus subtiles qu'agréables. Et cependant on y a fait depuis des additions qui en rendent la dernière moitié aussi utile et plus divertissante qu'aucune autre.


