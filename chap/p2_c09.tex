\subsubsection{\centering \Large CHAPITRE IX}
\addcontentsline{toc}{section}{\protect\numberline{}{\scshape\bfseries IX} - \emph{Quelques observations pour reconnaître dans quelques propositions exprimées d'une manière moins ordinaire, quel en est le sujet et quel en est l'attribut}}
\begin{center}\emph{\large\scshape Quelques observations pour reconnaître dans quelques propositions exprimées d'une manière moins ordinaire, quel en est le sujet et quel en est l'attribut.}\end{center}

	\lettrine{C}{'est} un des plus grands défauts de la Logique ordinaire, qu'on n'accoutume point ceux qui l'apprennent à reconnaître la nature des propositions et des raisonnements, qu'en les attachant à l'ordre et à l'arrangement dont on les forme dans les écoles, qui est souvent très différent de celui dont on les forme dans le monde et dans les livres, soit d'éloquence, soit de morale, soit des autres sciences.

Ainsi on n'a presque point d'autre idée d'un sujet et d'un attribut, sinon que l'un est le premier terme d'une proposition, et l'autre le dernier; et de l'universalité ou particularité, sinon qu'il y a dans l'une \emph{omnis} ou \emph{nullus, tout} ou \emph{nul}, et dans l'autre, \emph{aliquis, quelque}.

Cependant tout cela trompe très souvent, et il est besoin de jugement pour discerner ces choses en plusieurs propositions. Commençons par le sujet et l'attribut.

L'unique et véritable règle est de regarder par le sens ce dont on affirme, et ce qu'on affirme; car le premier est toujours le sujet, et le dernier l'attribut, en quelque ordre qu'ils se trouvent.

Ainsi il n'y a rien de plus commun en latin que ces sortes de propositions : \emph{Turpe est obsequi libidini: Il est honteux d'être esclave de ses passions}: où il est visible par le sens, que \emph{turpe, honteux}, est ce qu'on affirme, et par conséquent l'attribut, \emph{et obsequi lubidini, être esclave de ses passions}, ce dont on affirme, c'est-à-dire, ce qu'on assure être honteux, et par conséquent le sujet. De même dans saint Paul, \emph{Est quaestus magnus pietas cum sufficientia}, le vrai ordre serait, \emph{pietas cum sufficientia est quaestus magnus}.

Et de même dans ces vers :
	\begin{tabularx}{\textwidth}{X}
		\emph{Felix qui potuit rerum cognoscere causas ;} \\
		\emph{Atque metus omnes, et inexorabile fatum} \\
		\emph{Subjecit pedibus, strepitumque Acherontis avari.} \\
	\end{tabularx}
\emph{Felix} est l'attribut, et le reste le sujet.

Le sujet et l'attribut sont souvent encore plus difficiles à reconnaître dans les propositions complexes; et nous avons déjà vu qu'on ne peut quelquefois juger que par la suite du discours et l'intention d'un auteur, quelle est la proposition principale, et quelle est l'incidente dans ces sortes de propositions.

Mais, outre ce que nous avons dit, on peut encore remarquer que, dans ces propositions complexes, où la première partie n'est que la proposition incidente, et la dernière est la principale, comme dans la majeure et la conclusion de ce raisonnement :
	\begin{tabularx}{\textwidth}{X}
		\emph{Dieu commande d'honorer les rois :} \\
		\emph{Louis XIV est roi :} \\
		\emph{Donc Dieu commande d'honorer Louis XIV.} \\
	\end{tabularx}

Il faut souvent changer le verbe actif en passif, pour avoir le vrai sujet de cette proposition principale, comme dans cet exemple même ; car il est visible que, raisonnant de la sorte, mon intention principale, dans la majeure, est d'affirmer quelque chose des rois, dont je puisse conclure qu'il faut honorer Louis XIV ; et ainsi ce que je dis du commandement de Dieu n'est proprement qu'une proposition incidente qui confirme cette affirmation, \emph{les rois doivent être honorés: Reges sunt honorandi}. D'où il s'ensuit que \emph{les rois} est le sujet de la majeure, et \emph{Louis XIV} le sujet de la conclusion, quoiqu'à ne considérer les choses que superficiellement, l'un et l'autre semblent n'être qu'une partie de l'attribut.

Ce sont aussi des propositions fort ordinaires à notre langue : \emph{C'est une folie que de s'arrêter à des flatteurs: C'est de la grêle qui tombe : C'est un Dieu qui nous à rachetés}. Or, le sens doit faire encore juger que, pour les remettre dans l'arrangement naturel, en plaçant le sujet avant l'attribut, il faudrait les exprimer ainsi : \emph{S'arrêter à des flatteurs est une folie: Ce qui tombe est de la grêle : Celui qui nous a rachetés est Dieu}. Et cela est presque universel dans toutes les propositions qui commencent par \emph{c'est}, et où l'on trouve après un \emph{qui} ou un \emph{que}, d'avoir leur attribut au commencement et le sujet à la fin. C'est assez d'en avoir averti une fois, et tous ces exemples ne sont que pour faire voir qu'on en doit juger par le sens, et non par l'ordre des mots. Ce qui est un avis très nécessaire pour ne pas se tromper, en prenant des syllogismes pour vicieux qui sont en effet très bons ; parce que, faute de discerner dans les propositions le sujet et l'attribut, on croit qu'ils sont contraires aux règles lorsqu'ils y sont très conformes.


