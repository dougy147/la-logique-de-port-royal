\subsubsection{\centering \Large CHAPITRE X}
\addcontentsline{toc}{section}{\protect\numberline{}{\scshape\bfseries X} - \emph{Des syllogismes complexes; et comment on les peut réduire aux syllogismes communs, et en juger par les mêmes règles}}
\begin{center}\emph{\large\scshape Des syllogismes complexes; et comment on les peut réduire aux syllogismes communs, et en juger par les mêmes règles.}\end{center}

	\lettrine{N}{ous} avons déjà dit que dans les syllogismes communs il y en avait que l'on pouvait appeler \emph{complexes}, non pas simplement parce qu'il s'y trouvait des propositions complexes, mais parce que les termes de la conclusion étaient complexes, n'étant pas pris tout entiers dans chacune des prémisses pour être joints avec le moyen, mais seulement une partie de l'un des termes, comme en cet exemple :

\begin{center}
	\begin{tabular}{l}
		\emph{Le Soleil est une chose insensible :} \\
		\emph{Les Perses adoraient le Soleil :} \\
		\emph{Donc les Perses adoraient une chose insensible}. \\
	\end{tabular}
\end{center}

Où l'on voit que la conclusion ayant pour attribut, \emph{adoraient une chose insensible} on n'en met qu'une partie dans la majeure, à savoir \emph{une chose insensible}; et \emph{adoraient}, dans la mineure.

Or, nous ferons deux choses touchant ces sortes de syllogismes. Nous montrerons premièrement comment on peut les réduire aux syllogismes incomplexes dont nous avons parlé jusqu'ici pour en juger par les mêmes règles.

Et nous ferons voir, en second lieu, que l'on peut donner des règles plus générales pour juger tout d'un coup de la bonté ou du vice de ces syllogismes complexes, sans avoir besoin d'aucune réduction.

C'est une chose assez étrange que, quoique l'on fasse peut-être beaucoup plus d'état de la logique qu'on ne devrait, jusqu'à soutenir qu'elle est absolument nécessaire pour acquérir les sciences, on la traite néanmoins avec si peu de soin, que l'on ne dit presque rien de ce qui peut avoir quelque usage ; car on se contente d'ordinaire de donner des règles des syllogismes simples, et presque tous les exemples qu'on en apporte sont composés de propositions incomplexes, qui sont si claires, que personne ne s'est jamais avisé de les proposer sérieusement dans aucun discours. Car à qui a-t-on jamais ouï faire ces syllogismes qui leur sont si ordinaires : Tout homme est animal : Pierre est homme : Donc Pierre est animal ?

Mais on se met peu en peine d'appliquer les règles des syllogismes aux arguments dont les propositions sont complexes, quoique cela soit souvent assez difficile, et qu'il y ait plusieurs arguments de cette nature qui paraissent mauvais, et qui sont néanmoins fort bons ; et que d'ailleurs l'usage de ces sortes d'arguments soit beaucoup plus fréquent que celui des syllogismes entièrement simples. C'est ce qu'il sera plus aisé de faire voir par des exemples que par des règles.

\begin{center}{\scshape 1. Exemple}\end{center}

Nous avons dit, par exemple, que toutes les propositions composées de verbes actifs sont complexes en quelque manière ; et de ces propositions on en fait souvent des arguments dont la forme et la force sont difficiles à reconnaître, comme celui-ci que nous avons déjà proposé en exemple.

	\begin{tabularx}{\textwidth}{X}
		\emph{La loi divine commande d'honorer les rois : } \\
		\emph{Louis XIV est roi :}  \\
		\emph{Donc la loi divine commande d'honorer Louis XIV.} \\
	\end{tabularx}

Quelques personnes peu intelligentes ont accusé ces sortes de syllogismes d'être défectueux, parce que, disaient-elles, ils sont composés de pures affirmatives dans la deuxième figure, ce qui est un défaut essentiel; mais ces personnes ont bien montré qu'elles consultaient plus la lettre et l'écorce des règles, que non pas la lumière de la raison, par laquelle ces règles ont été trouvées ; car cet argument est tellement vrai et concluant que, s'il était contre la règle, ce serait une preuve que la règle serait fausse et non pas que l'argument fût mauvais.

Je dis donc premièrement que cet argument est bon ; car dans cette proposition : \emph{La loi divine commande d'honorer les rois}, ce mot de \emph{rois} est pris généralement pour tous les rois en particulier, et par conséquent Louis XIV est du nombre de ceux que la loi divine commande d'honorer.

Je dis, en second lieu, que \emph{roi}, qui est le moyen, n'est point attribut dans cette proposition : \emph{La loi divine commande d'honorer les rois}, quoiqu'il soit joint à l'attribut \emph{commande}, ce qui est bien différent. Car, ce qui est véritablement attribut est affirmé et convient. Or \emph{roi} n'est point affirmé, et ne convient point à la loi de Dieu. Deuxièmement, l'attribut est restreint par le sujet. Or, le mot de \emph{roi} n'est point restreint dans cette proposition, \emph{La loi divine commande d'honorer les rois}, puisqu'il se prend généralement.

Mais si l'on demande ce qu'il est donc, il est facile de répondre qu'il est sujet d'une autre proposition enveloppée dans celle-là ; car, quand je dis que la loi divine commande d'honorer les rois ; comme j'attribue à la loi de commander, j'attribue aussi l'honneur aux rois, car c'est comme si je disais : \emph{La loi divine commande que les rois soient honorés}.

De même, dans cette conclusion : \emph{La loi divine commande d'honorer Louis XIV}. Louis XIV n'est point l'attribut, quoique joint à l'attribut, et il est, au contraire, le sujet de la proposition enveloppée. Car c'est autant que si je disais : \emph{La loi divine commande que Louis XIV soit honoré}.

Ainsi, ces propositions étant développées en cette manière :

	\begin{tabularx}{\textwidth}{X}
		\emph{La loi divine commande que les rois soient honorés :}  \\
		\emph{Louis XIV est roi :}  \\
		\emph{Donc la loi divine commande que Louis XIV soit honoré;} \\
	\end{tabularx}

Il est clair que tout l'argument consiste dans ces propositions.

	\begin{tabularx}{\textwidth}{X}
		\emph{Les rois doivent être honorés :} \\
		\emph{Louis XIV est roi :} \\
		\emph{Donc Louis XIV doit être honoré.} \\
	\end{tabularx}

Et que cette proposition, \emph{La loi divine commande}, qui paraissait la principale, n'est qu'une proposition incidente à cet argument, qui est jointe à l'affirmation à qui la loi divine sert de preuve.

Il est clair de même que cet argument est de la première figure en \emph{Barbara}, les termes singuliers, comme Louis XIV, passant pour universels, parce qu'ils sont pris dans toute leur étendue, comme nous avons déjà marqué.

\begin{center}{\scshape 2. Exemple}\end{center}

Par la même raison, cet argument, qui paraît de la deuxième figure, et conforme aux règles de cette figure, ne vaut rien.

	\begin{tabularx}{\textwidth}{X}
		\emph{Nous devons croire l'Écriture :} \\
		\emph{La tradition n'est point l'Écriture :} \\
		\emph{Donc nous ne devons point croire la tradition}. \\
	\end{tabularx}

Car il doit se réduire à la première figure, comme s'il y avait :

	\begin{tabularx}{\textwidth}{X}
		\emph{L'Écriture doit être crue :} \\
		\emph{La tradition n'est point l'Écriture :} \\
		\emph{Donc la tradition ne doit pas être crue}. \\
	\end{tabularx}

Or, l'on ne peut rien conclure dans la première figure d'une mineure négative.

\begin{center}{\scshape 3. Exemple}\end{center}

Il arrive encore de ces propositions complexes composées, que des arguments qui sont très bons semblent tout à fait contraires aux règles communes, comme celui-ci :

	\begin{tabularx}{\textwidth}{X}
		\emph{Les seuls amis de Dieu sont heureux:} \\
		\emph{Il y a des riches qui ne sont point amis de Dieu:} \\
		\emph{Donc il y a des riches qui ne sont point heureux.} \\
	\end{tabularx}

Cet argument examiné selon les règles communes semblerait mauvais, parce qu'il paraît de la première figure, où l'on ne conclue rien d'une mineure négative; mais la première proposition étant composée dans le sens à cause de la particule \emph{seuls} qui la rend exclusive, vaut ces deux propositions : \emph{Les amis de Dieu sont heureux, et tous les autres hommes qui ne sont point amis de Dieu ne sont point heureux}. Or comme c'est de cette seconde proposition que dépend la force de ce raisonnement, la mineure qui semblait négative devient affirmative, parce que le sujet de la majeure, qui doit être attribut dans la mineure, n'est pas, \emph{amis de Dieu}, mais, \emph{ceux qui ne sont pas amis de Dieu}, de sorte que tout l'argument se doit prendre ainsi :

	\begin{tabularx}{\textwidth}{X}
		\emph{Tous ceux qui ne sont point amis de Dieu ne sont point heureux:} \\
		\emph{Or il y a des riches qui sont du nombre de ceux qui ne sont point amis de Dieu:} \\
		\emph{Donc il y a des riches qui ne sont point heureux.} \\
	\end{tabularx}

Mais ce qui fait qu'il n'est point nécessaire d'exprimer la mineure de cette sorte, et que l'on lui laisse l'apparence d'une proposition négative, c'est que c'est la même chose de dire négativement qu'un homme n'est pas ami de Dieu, et de dire affirmativement qu'il est non ami de Dieu, c'est-à-dire, du nombre de ceux qui ne sont pas amis de Dieu.

\newpage

\begin{center}{\scshape 4. Exemple}\end{center}

Il y a beaucoup d'arguments semblables dont toutes les propositions paraissent négatives, et qui néanmoins sont très bons, parce qu'il y en a une qui n'est négative qu'en apparence, et qui est affirmative en effet, comme nous venons de le faire voir, et comme on verra encore par cet exemple :

	\begin{tabularx}{\textwidth}{X}
		\emph{Ce qui n'a point de parties ne peut périr par la dissolution de ses parties:} \\
		\emph{Notre âme n'a point de parties:} \\
		\emph{Donc notre âme ne peut périr par la dissolution de ses parties.} \\
	\end{tabularx}

Il y a des personnes qui apportent ces sortes de syllogismes pour montrer que l'on ne doit pas prétendre que cet axiome de Logique, \emph{On ne conclut rien de pures négatives}, soit vrai généralement et sans distinction. Mais ils n'ont pas pris garde que dans le sens, la mineure de ce syllogisme et autres semblables est affirmative, parce que le milieu, qui est le sujet de la majeure, en est l'attribut. Or le sujet de la majeure n'est pas, \emph{ce qui a des parties}, mais, \emph{ce qui n'a point de parties}. Et ainsi le sens de la mineure est, \emph{Notre âme est une chose qui n'a point de parties}, ce qui est une proposition affirmative d'un attribut négatif.

Ces mêmes personnes prouvent encore que les arguments négatifs sont quelque fois concluants, par ces exemples : \emph{Jean n'est point raisonnable : Donc il n'est point homme. Nul animal ne voit : Donc nul homme ne voit}. Mais ils devaient considérer que ces exemples ne sont que des enthymèmes, et que nul enthymème ne conclut qu'en vertu d'une proposition sous-entendue, et qui par conséquent doit être dans l'esprit quoi qu'elle ne soit pas exprimée. Or dans l'un et l'autre de ces exemples la proposition sous-entendue est nécessairement affirmative. Dans le premier, celle-ci : \emph{Tout homme est raisonnable : Jean n'est point raisonnable : Donc Jean n'est point homme}. Et dans l'autre : \emph{Tout homme est animal : Nul animal ne voit : Donc nul homme ne voit}. Or on ne peut pas dire que ces syllogismes soient de pures négatives. Et par conséquent les enthymèmes qui ne concluent que parce qu'ils enferment ces syllogismes entiers dans l'esprit de celui qui les fait, ne peuvent être apportés en exemple pour faire voir qu'il y a quelque fois des arguments de pures négatives qui concluent.


