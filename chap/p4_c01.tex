\subsubsection{\centering \Large CHAPITRE I}
\addcontentsline{toc}{section}{\protect\numberline{}{\scshape\bfseries I} - \emph{De deux sortes de méthodes, Analyse et Synthèse. Exemple de l'Analyse}}
\begin{center}\emph{\large\scshape De deux sortes de méthodes, Analyse et Synthèse. Exemple de l'Analyse.}\end{center}

	\lettrine{O}{n} peut appeler généralement méthode l'art de bien disposer une suite de plusieurs pensées, ou pour découvrir la vérité quand nous l'ignorons, ou pour la prouver aux autres, quand nous la connaissons déjà.

Ainsi il y a deux sortes de méthodes ; l'une pour découvrir la vérité, qu'on appelle \emph{analyse} ou \emph{méthode de résolution}, et qu'on peut aussi appeler \emph{méthode d'invention}: et l'autre pour la faire entendre aux autres, quand on l'a trouvée, qu'on appelle \emph{synthèse} ou \emph{méthode de composition}, et qu'on peut aussi appeler \emph{méthode de doctrine}.

Dans l'analyse ou méthode d'invention, on suppose que ce qu'on cherche n'est pas entièrement connu, mais qu'on le peut connaître en l'examinant plus particulièrement, et se servant des connaissances que cet examen nous donnera pour arriver à celle que nous cherchons; comme si on propose, \emph{si l'âme de l'homme est immortelle}, et que, pour le chercher on s'applique à considérer la nature de notre âme, on y remarque, premièrement, que c'est le propre de l'âme de penser, et qu'elle pourrait douter de tout, sans pouvoir douter si elle pense, puisque le doute même est une pensée. On examine ensuite ce que c'est que de penser; et, ne voyant point que dans l'idée de la pensée il y ait rien d'enfermé de ce qui est enfermé dans l'idée de la substance étendue qu'on appelle corps, et qu'on peut même nier de la pensée tout ce qui appartient au corps, comme d'être long, large, profond, d'avoir diversité de parties, d'être d'une telle ou d'une telle figure, d'être divisible, etc., Sans détruire pour cela l'idée qu'on a de la pensée ; on en conclut que la pensée n'est point un mode de la substance étendue, parce qu'il est de la nature du mode de ne pouvoir être conçu en niant de lui la chose dont il serait mode. D'où l'on infère encore que la pensée n'étant point un mode de la substance étendue, il faut que ce soit l'attribut d'une autre substance ; et qu'ainsi la substance qui pense et la substance étendue soient deux substances réellement distinctes. D'où il s'ensuit que la destruction de l'une ne doit point emporter la destruction de l'autre ; puisque même la substance étendue n'est point proprement détruite, mais que tout ce qui arrive, en ce que nous appelons destruction, n'est autre chose que le changement ou la dissolution de quelques parties de la matière qui demeure toujours dans la nature, comme nous jugeons fort bien qu'en rompant toutes les roues d'une horloge, il n'y a point de substance détruite, quoique l'on dise que cette horloge est détruite : ce qui fait voir que l'âme, n'étant point divisible et composée d'aucunes parties, ne peut périr, et par conséquent qu'elle est immortelle.

Voilà ce qu'on appelle \emph{analyse} ou \emph{résolution}, où il faut remarquer $1$. qu'on doit y pratiquer, aussi bien que dans la méthode qu'on appelle \emph{de composition}, de passer toujours de ce qui est plus connu à ce qui l'est moins. Car il n'y a point de vraie méthode qui puisse se dispenser de cette règle.

$2$. Mais qu'elle diffère de celle de composition, en ce que l'on prend ces vérités connues dans l'examen particulier de la chose que l'on se propose de connaître, et non dans les choses plus générales, comme on fait dans la méthode de doctrine. Ainsi, dans l'exemple que nous avons proposé, on ne commence pas par l'établissement de ces maximes générales : Que nulle substance ne périt à proprement parler ; que ce qu'on appelle destruction n'est qu'une dissolution de parties ; qu'ainsi ce qui n'a point de parties ne peut être détruit, etc. ; mais on monte par degrés à ces connaissances générales.

$3$. On n'y propose les maximes claires et évidentes qu'à mesure qu'on en a besoin, au lieu que dans l'autre on les établit d'abord, ainsi que nous dirons plus bas.

$4$. Enfin ces deux méthodes ne différent que comme le chemin qu'on fait en montant d'une vallée en une montagne, de celui que l'on fait en descendant de la montagne dans la vallée; ou comme diffèrent les deux manières dont on peut se servir pour prouver qu'une personne est descendue de saint Louis, dont l'une est de montrer que cette personne a tel pour père, qui était fils d'untel, et celui-là d'un autre, et ainsi jusqu'à saint Louis; et l'autre de commencer par saint Louis, et montrer qu'il a eu tels enfants, et ces enfants d'autres, en descendant jusqu'à la personne dont il s'agit: et cet exemple est d'autant plus propre, en cette rencontre, qu'il est certain que, pour trouver une généalogie inconnue, il faut remonter du fils au père: au lieu que, pour l'expliquer après l'avoir trouvée, la manière la plus ordinaire est de commencer par le tronc pour en faire voir les descendants ; qui est aussi ce qu'on fait d'ordinaire dans les sciences, où, après s'être servi de l'analyse pour trouver quelque vérité, on se sert de l'autre méthode pour expliquer ce qu'on a trouvé.

On peut comprendre par là ce que c'est que l'analyse des géomètres : car voici en quoi elle consiste. Une question leur ayant été proposée, dont ils ignorent la vérité ou la fausseté, si c'est un théorème, la possibilité ou l'impossibilité, si c'est un problème, ils supposent que cela est comme il est proposé; et, examinant ce qui s'ensuit de là, s'ils arrivent, dans cet examen, à quelque vérité claire dont ce qui leur est proposé soit une suite nécessaire, ils en concluent que ce qui leur est proposé est vrai ; et reprenant ensuite par où ils avaient fini, ils le démontrent par l'autre méthode qu'on appelle \emph{de composition}. Mais s'ils tombent, par une suite nécessaire de ce qui leur est proposé, dans quelque absurdité ou impossibilité, ils en concluent que ce qu'on leur avait proposé est faux et impossible.

Voilà ce qu'on peut dire généralement de l'analyse, qui consiste plus dans le jugement et dans l'adresse de l'esprit que dans des règles particulières. Ces quatre néanmoins, que Descartes propose dans sa Méthode, peuvent être utiles pour se garder de l'erreur en voulant rechercher la vérité dans les sciences humaines, quoique, à dire vrai, elles soient générales pour toutes sortes de méthodes, et non particulières pour la seule analyse.

La 1. est de \emph{ne recevoir jamais aucune chose pour vraie, qu'on ne la connaisse évidemment être telle, c'est-à-dire d'éviter soigneusement la précipitation et la prévention, et de ne comprendre rien de plus en ses jugements que ce qui se présente si clairement à l'esprit, qu'on n'ait aucune occasion de le mettre en doute}.

La 2. de \emph{diviser chacune des difficultés qu'on examine en autant de parcelles qu'il se peut, et qu'il est requis pour les résoudre}.

La 3. de \emph{conduire par ordre ses pensées, en commençant par les objets les plus simples et les plus aisés à connaître, pour monter peu à peu, comme par degrés, jusqu'à la connaissance des plus composés, et supposant même de l'ordre entre ceux qui ne se précèdent point naturellement les uns les autres}.

La 4. de \emph{faire partout des dénombrements si entiers et des revues si générales, qu'on puisse s'assurer de ne rien omettre}.

Il est vrai qu'il y a beaucoup de difficulté à observer ces règles ; mais il est toujours avantageux de les avoir dans l'esprit, et de les garder autant que l'on peut lorsqu'on veut trouver la vérité par la voie de la raison, et autant que notre esprit est capable de la connaître.

