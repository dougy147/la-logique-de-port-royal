\subsubsection{\centering \Large CHAPITRE III}
\addcontentsline{toc}{section}{\protect\numberline{}{\scshape\bfseries III} - \emph{Des dix catégories d'Aristote}}
\begin{center}\emph{\large\scshape Des dix catégories d'Aristote.}\end{center}

	\lettrine{O}{n} peut rapporter à cette considération des idées selon leurs objets, les dix catégories d'Aristote, puisque ce ne sont que diverses classes auxquelles ce philosophe a voulu réduire tous les objets de nos pensées, en comprenant toutes les substances sous la première, et tous les accidents sous les neuf autres. Les voici.

\bigbreak
{\bfseries\scshape I. La substance}, qui est ou spirituelle, ou corporelle, etc.
\bigbreak

{\bfseries\scshape II. La quantité}, qui s'appelle discrète, quand les parties n'en sont point liées, comme le nombre.

Continue, quand elles sont liées ; et alors elle est ou successive comme le temps, le mouvement :

Ou permanente, qui est ce qu'on appelle autrement l'espace, ou l'étendue en longueur, largeur, profondeur ; la longueur seule faisant les lignes ; la longueur et la largeur les surfaces, et les trois ensemble les solides.

\bigbreak
{\bfseries\scshape III. La qualité}, dont Aristote fait quatre espèces.

La première comprend \emph{les habitudes}, c'est-à-dire les dispositions d'esprit ou de corps, qui s'acquièrent par des actes réitérés, comme les sciences, les vertus, les vices, l'adresse de peindre, d'écrire, de danser.

La deuxième \emph{les puissances naturelles}, telles que sont les facultés de l'âme ou du corps, l'entendement, la volonté, la mémoire, les cinq sens, la puissance de marcher.

La troisième \emph{les qualités sensibles}, comme la dureté, la mollesse, la pesanteur, le froid, le chaud, les couleurs, le son, les odeurs, les divers goûts.

La quatrième la forme et la figure qui est la détermination extérieure de la quantité, comme être rond, carré, sphérique, cubique.

\bigbreak
{\bfseries\scshape IV. La relation}, ou le rapport d'une chose à une autre, comme de père, de fils, de maître, de valet, de roi, de sujet; de la puissance à son objet, de la vue à ce qui est visible ; et tout ce qui marque comparaison, comme semblable, égal, plus grand, plus petit.

\bigbreak
{\bfseries\scshape V. L'agir}, ou en soi-même, comme marcher, danser, connaître, aimer; ou hors de soi, comme battre, couper, rompre, éclairer, échauffer.

\bigbreak
{\bfseries\scshape VI. Pâtir}, être battu, être rompu, être éclairé, être échauffé.

\bigbreak
{\bfseries\scshape VII. Où}, c'est-à-dire ce qu'on répond aux questions qui regardent le lieu, comme être à Rome, à Paris, dans son cabinet, dans son lit, dans sa chaise.

\bigbreak
{\bfseries\scshape VIII. Quand}, c'est-à-dire ce qu'on répond aux questions qui regardent le temps, comme : quand a-t-il vécu? il y a cent ans: quand cela s'est-il fait ? hier.

\bigbreak
{\bfseries\scshape IX. La situation}, être assis, debout, couché, devant, derrière, à droite, à gauche.

\bigbreak
{\bfseries\scshape X. Avoir}, c'est-à-dire avoir quelque chose autour de soi pour servir de vêtement, ou d'ornement, ou d'armure, comme être habillé, être couronné, être chaussé, être armé.

\bigbreak
Voilà les dix catégories d'Aristote, dont on fait tant de mystères, quoique à dire le vrai, ce soit une chose de soi très peu utile, et qui non seulement ne sert guère à former le jugement, ce qui est le but de la vraie logique, mais qui souvent y nuit beaucoup, pour deux raisons qu'il est important de remarquer.

La première est qu'on regarde ces catégories comme une chose établie sur la raison et sur la vérité, au lieu que c'est une chose tout arbitraire, et qui n'a de fondement que l'imagination d'un homme qui n'a eu aucune autorité de prescrire une loi aux autres, qui ont autant de droit que lui d'arranger d'une autre sorte les objets de leurs pensées, chacun selon sa manière de philosopher. Et, en effet, il y en a qui ont compris en ce distique tout ce que l'on considère selon une nouvelle philosophie en toutes les choses du monde :

\begin{center}
\emph{Mens, mensura, quies, motus positura, figura, Sunt cum materia cunctarum exordia rerum.}
\end{center}

C'est-à-dire que ces gens-là se persuadent que l'on peut rendre raison de toute la nature en n'y considérant que ces sept choses, ou modes: {1. }\emph{Mens}, l'esprit ou la substance qui pense. {2. }\emph{Materia}, le corps ou la substance étendue. {3. }\emph{Mensura}, la grandeur ou la petitesse de chaque partie de la matière. {4. }\emph{Positura}, leur situation à l'égard les unes des autres. {5. }\emph{Figura}, leur figure. {6. }\emph{Motus}, leur mouvement. {7. }\emph{Quies}, leur repos ou moindre mouvement.

La seconde raison qui rend l'étude des catégories dangereuse, est qu'elle accoutume les hommes à se payer de mots, à s'imaginer qu'ils savent toutes choses quand ils n'en connaissent que des noms arbitraires qui n'en forment dans l'esprit aucune idée claire et distincte, comme on le fera voir en un autre endroit.

On pourrait encore parler ici des attributs des Lullistes, \emph{bonté, puissance, grandeur, etc}. ; mais en vérité c'est une chose si ridicule que l'imagination qu'ils ont, qu'appliquant ces mots métaphysiques à tout ce qu'on leur propose, ils pourront rendre raison de tout, qu'elle ne mérite seulement pas d'être réfutée.

Un auteur de ce temps a dit avec grande raison que les règles de la logique d'Aristote servaient seulement à prouver à un autre ce que l'on savait déjà, mais que l'art de Lulle ne servait qu'à faire discourir sans jugement de ce qu'on ne savait pas. L'ignorance vaut beaucoup mieux que cette fausse science qui fait que l'on s'imagine savoir ce qu'on ne sait point. Car, comme saint Augustin a très judicieusement remarqué dans le livre de l'utilité de la créance, cette disposition d'esprit est très blâmable pour deux raisons : L'une, que celui qui s'est faussement persuadé qu'il connaît la vérité, se rend par là incapable de s'en faire instruire : L'autre, que cette présomption et cette témérité est une marque d'un esprit qui n'est pas bien fait : \emph{Opinari} (c'est le nom qui signifie dans la pureté de la langue latine la disposition d'un esprit qui croit savoir ce qu'il ne sait pas) \emph{duas ob res turpissimum est : quod discere non potest qui sibi jam se scire persuasit: et per se ipsa temeritas non bene affecti animi signum est}.
