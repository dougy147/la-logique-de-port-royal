\subsubsection{\centering \Large CHAPITRE II}
\addcontentsline{toc}{section}{\protect\numberline{}{\scshape\bfseries II} - \emph{De l'opposition entre les propositions qui ont même sujet et même attribut}}
\begin{center}\emph{\large\scshape De l'opposition entre les propositions qui ont même sujet et même attribut.}\end{center}

	\lettrine{N}{ous} venons de dire qu'il y a quatre sortes de propositions A, E, I, O; on demande maintenant quelle convenance ou disconvenance elles ont ensemble, lorsqu'on fait du même sujet et du même attribut diverses sortes de propositions. C'est ce qu'on appelle opposition.

Et il est aisé de voir que cette opposition ne peut être que de trois sortes, quoique l'une des trois se divise en deux autres.

Car, si elles sont opposées en quantité et en qualité tout ensemble, comme A, O; et E, I, on les appelle contradictoires, comme \emph{Tout homme est animal, Quelque homme n'est pas animal : Nul n'est impeccable, Quelque homme est impeccable}.

Si elles diffèrent en quantité seulement, et qu'elles conviennent en qualité, comme A, I ; et E, O, on les appelle subalternes, comme \emph{Tout homme est animal, Quelque homme est animal : Nul homme n'est impeccable, Quelque homme n'est pas impeccable}.

Et si elles diffèrent en qualité, et qu'elles conviennent en quantité, alors elles sont appelées \emph{contraires}, ou \emph{subcontraires}: \emph{contraires}, quand elles sont universelles, comme \emph{Tout homme est animal, Nul homme n'est animal}.

\emph{Subcontraires}, quand elles sont particulières, comme \emph{Quelque homme est animal, Quelque homme n'est pas animal}.

En regardant maintenant ces propositions opposées selon la vérité ou la fausseté, il est aisé de juger :

\bigbreak
{$1$.} Que les contradictoires ne sont jamais ni vraies, ni fausses ensemble ; mais si l'une est vraie, l'autre est fausse ; et si l'une est fausse, l'autre est vraie : car s'il est vrai que tout homme soit animal, il ne peut pas être vrai que quelque homme n'est pas animal ; et si, au contraire, il est vrai que quelque homme n'est pas animal, il n'est donc pas vrai que tout homme soit animal. Cela est si clair, qu'on ne pourrait que l'obscurcir en l'expliquant davantage.

\bigbreak
{$2$.} Les contraires ne peuvent jamais être vraies ensemble ; mais elles peuvent être toutes deux fausses. Elles ne peuvent être vraies, parce que les contradictoires seraient vraies; car s'il est vrai que tout homme soit animal, il est faux que quelque homme n'est pas animal, qui est la contradictoire, et par conséquent encore plus faux que nul homme ne soit animal, qui est la contraire.

Mais la fausseté de l'une n'emporte pas la vérité de l'autre; car il peut être faux que tous les hommes soient justes, sans qu'il soit vrai pour cela que nul homme ne soit juste, puisqu'il peut y avoir des hommes justes, quoique tous ne soient pas justes.

\bigbreak
{$3$.} Les subcontraires, par une règle tout opposée à celle des contraires, peuvent être vraies ensemble, comme ces deux-ci, \emph{Quelque homme est juste, quelque homme m'est pas juste}, parce que la justice peut convenir à une partie des hommes, et ne pas convenir à l'autre; et ainsi l'affirmation et la négation né regardent pas le même sujet, puisque \emph{quelque homme} est pris pour une partie des hommes dans l'une des propositions, et pour une autre partie dans l'autre. Mais elles ne peuvent être toutes deux fausses ; puisque autrement les contradictoires seraient toutes deux fausses, car s'il était faux que quelque homme fût juste, il serait donc vrai que nul homme n'est juste, qui est la contradictoire, et à plus forte raison que quelque homme n'est pas juste, qui est la subcontraire.

\bigbreak
{$4$.} Pour les subalternes, ce n'est pas une véritable opposition, puisque la particulière est une suite de la générale ; car, si tout homme est animal, quelque homme est animal : Si nul homme n'est singe, quelque homme n'est pas singe. C'est pourquoi la vérité des universelles emporte celle des particulières ; mais la vérité des particulières n'emporte pas celle des universelles : car il ne s'ensuit pas que, parce qu'il est vrai que quelque homme est juste, il soit vrai aussi que tout homme est juste ; et, au contraire, la fausseté des particulières emporte la fausseté des universelles : car, s'il est faux que quelque homme soit impeccable, il est encore plus faux que tout homme soit impeccable. Mais la fausseté des universelles n'emporte pas la fausseté des particulières ; car, quoiqu'il soit faux que tout homme soit juste, il ne s'ensuit pas que ce soit une fausseté de dire que quelque homme est juste. D'où il s'ensuit qu'il y a plusieurs rencontres où ces propositions subalternes sont toutes deux vraies, et d'autres où elles sont toutes deux fausses.

\bigbreak
Je ne dis rien de la réduction des propositions opposées en un même sens, parce que cela est tout à fait inutile, et que les règles qu'on en donne ne sont la plupart vraies qu'en latin.
