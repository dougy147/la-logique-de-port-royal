\subsubsection{\centering \Large CHAPITRE IV}
\addcontentsline{toc}{section}{\protect\numberline{}{\scshape\bfseries IV} - \emph{De la nature des propositions incidentes, qui font partie des propositions complexes}}
\begin{center}\emph{\large\scshape De la nature des propositions incidentes, qui font partie des propositions complexes.}\end{center}

	\lettrine{M}{ais}, avant que de parler des propositions dont la complexion tombe sur la forme, c'est-à-dire sur l'affirmation ou la négation, il y a plusieurs remarques importantes à faire sur la nature des propositions incidentes, qui font partie du sujet ou de l'attribut de celles qui sont complexes selon la matière.

\bigbreak
{1.} On a déjà vu que ces propositions incidentes sont celles dont le sujet est le relatif qui, comme, \emph{les hommes qui sont créés pour connaître et pour aimer Dieu}, ou \emph{les hommes qui sont pieux}, ôtant le terme \emph{d'hommes}, le reste est une proposition incidente.

Mais il faut se souvenir de ce qui a été dit dans le Chapitre VI de la première partie, que les additions des termes complexes sont de deux sortes : les unes qu'on peut appeler de simples explications, qui est lorsque l'addition ne change rien dans l'idée du terme, parce que ce qu'on y ajoute lui convient généralement et dans toute son étendue, comme dans le premier exemple, \emph{les hommes qui sont créés pour connaître et pour aimer Dieu}.

Les autres qui se peuvent appeler des déterminations, parce que ce qu'on ajoute à un terme ne convenant pas à ce terme dans toute son étendue, en restreint et en détermine la signification, comme dans le second exemple, \emph{les hommes qui sont pieux}. Selon cela, on peut dire qu'il y a un \emph{qui} explicatif, et un \emph{qui} déterminatif.

Or, quand le \emph{qui} est explicatif, l'attribut de la proposition incidente est affirmé du sujet auquel le \emph{qui} se rapporte, quoique ce ne soit qu'incidemment au regard de la proposition totale, de sorte qu'on peut substituer le sujet même au \emph{qui}, comme on peut voir dans le premier exemple, \emph{Les hommes qui ont été créés pour connaître et pour aimer Dieu}. Car on peut dire, \emph{Les hommes ont été créés pour connaître et pour aimer Dieu}.

Mais quand le \emph{qui} est déterminatif, l'attribut de la proposition incidente n'est point proprement affirmé du sujet auquel le \emph{qui} se rapporte. Car si, après avoir dit, \emph{les hommes qui sont pieux sont charitables}, on voulait substituer le mot \emph{d'hommes} au \emph{qui} en disant, \emph{les hommes sont pieux}, la proposition serait fausse, parce que ce serait affirmer le mot de pieux des hommes comme hommes; mais en disant, \emph{les hommes qui sont pieux sont charitables}, on n'affirme ni des hommes en général, ni d'aucuns hommes en particulier, qu'ils soient pieux ; mais l'esprit, joignant ensemble l'idée de \emph{pieux} avec celle \emph{d'homme}, et en faisant une idée totale, juge que l'attribut de \emph{charitable} convient à cette idée totale, et ainsi, tout le jugement qui est exprimé dans la proposition incidente est seulement celui par lequel notre esprit juge que l'idée de \emph{pieux} n'est pas incompatible avec celle \emph{d'homme}, et qu'ainsi il peut les considérer comme jointes ensemble et examiner ensuite ce qui leur convient selon cette union.

\bigbreak
{2.} Il y a souvent des termes qui sont doublement et triplement complexes, étant composés de plusieurs parties dont chacune à part est complexe ; et ainsi il peut s'y rencontrer diverses propositions incidentes et de diverse espèce, le \emph{qui} de l'une étant déterminatif, et le \emph{qui} de l'autre explicatif. C'est ce qu'on verra mieux par cet exemple : \emph{La doctrine qui met le souverain bien dans la volupté du corps, laquelle a été enseignée par Épicure est indigne d'un philosophe} : cette proposition a pour attribut, \emph{indigne d'un philosophe}, et tout le reste pour sujet ; et ainsi ce sujet est un terme complexe qui enferme deux propositions incidentes: la première est, \emph{qui met le souverain bien dans la volupté du corps}: le \emph{qui}, dans cette proposition incidente, est déterminatif, car il détermine le mot de doctrine, qui est général, à celle qui affirme que le souverain bien de l'homme est dans la volupté du corps ; d'où vient qu'on ne pourrait sans absurdité substituer au \emph{qui} le mot de doctrine, en disant, \emph{la doctrine met le souverain bien dans la volupté du corps}. La seconde proposition incidente est, \emph{qui a été enseignée par Épicure}, et le sujet auquel ce \emph{qui} se rapporte, est tout le terme complexe, \emph{la doctrine met le souverain bien dans la volupté du corps}, qui marque une doctrine singulière et individuelle, capable de divers accidents, comme d'être soutenue par diverses personnes, quoiqu'elle soit déterminée en elle-même à être toujours prise de la même sorte, au moins dans ce point précis, selon lequel on l'entend. Et c'est pourquoi le \emph{qui} de la seconde proposition incidente, \emph{qui a été enseignée par Épicure}, n'est point déterminatif, mais seulement explicatif: d'où vient qu'on peut substituer le sujet auquel ce \emph{qui} se rapporte en la place du \emph{qui}, en disant, \emph{la doctrine qui met le souverain bien dans la volupté du corps a été enseignée par Épicure}.

\bigbreak
{3.} La dernière remarque est que, pour juger de la nature de ces propositions, et pour savoir si le \emph{qui} est déterminatif ou explicatif, il faut souvent avoir plus d'égard au sens et à l'intention de celui qui parle qu'à la seule expression.

Car il y a souvent des termes complexes qui paraissent incomplexes, ou qui paraissent moins complexes qu'ils ne le sont en effet, parce qu'une partie de ce qu'ils enferment dans l'esprit de celui qui parle est sous-entendue et non exprimée, selon ce qui a été dit dans le Chapitre VII de la première partie, où l'on a fait voir qu'il n'y avait rien de plus ordinaire dans les discours des hommes, que de marquer des choses singulières par des noms communs, parce que les circonstances du discours font assez voir qu'on joint à cette idée commune qui répond à ce mot une idée singulière et distincte, qui le détermine à ne signifier qu'une seule et unique chose.

J'ai dit que cela se reconnaissait d'ordinaire par les circonstances, comme, dans la bouche des Français le mot de \emph{roi} signifie \emph{Louis XIV}. Mais voici encore une règle qui peut servir à faire juger quand un terme commun demeure dans son idée générale, ou quand il est déterminé par une idée distincte et particulière, quoique non exprimée.

Quand il y a une absurdité manifeste à lier un attribut avec un sujet demeurant dans son idée générale, on doit croire que celui qui fait cette proposition n'a pas laissé ce sujet dans son idée générale. Ainsi, si j'entends dire à un homme : \emph{Rex hoc mihi imperavit, le roi m'a commandé telle chose}, je suis assuré qu'il n'a pas laissé le mot de roi dans son idée générale, car le roi en général ne fait point de commandement particulier.

Si un homme m'avait dit : \emph{La gazette de Bruxelles, du 14 janvier 1662 touchant ce qui se passe à Paris est fausse}, je serais assuré qu'il aurait quelque chose dans l'esprit de plus que ce qui serait signifié par ces termes, parce que tout cela n'est point capable de faire juger si cette gazette est vraie ou fausse, et qu'ainsi il faudrait qu'il eût conçu une nouvelle distincte et particulière, laquelle il jugeât contraire à la vérité, comme si cette gazette avait dit, \emph{que le roi a fait cent chevaliers de l'ordre du Saint Esprit}.

De même dans les jugements que l'on fait des opinions des philosophes, quand on dit que la doctrine d'un tel philosophe est fausse, sans exprimer distinctement quelle est cette doctrine, comme \emph{que la doctrine de Lucrèce touchant la nature de notre âme est fausse}, il faut nécessairement que, dans ces sortes de jugements, ceux qui les font conçoivent une opinion distincte et particulière sous le mot général de doctrine d'un tel philosophe, parce que la qualité de fausse ne peut pas convenir à une doctrine ; comme étant d'un tel auteur, mais seulement comme étant une telle opinion en particulier, contraire à la vérité; et ainsi ces sortes de propositions se résolvent nécessairement en celles-ci : \emph{Une telle opinion, qui a été enseignée par un tel auteur, est fausse : l'opinion que notre âme soit composée d'atomes, qui a été enseignée par Lucrèce, est fausse}. De sorte que ces jugements enferment toujours deux affirmations, lors même qu'elles ne sont pas distinctement exprimées : l'une principale, qui regarde la vérité en elle-même, qui est que c'est une grande erreur de vouloir que notre âme soit composée d'atomes; l'autre incidente, qui ne regarde qu'un point d'histoire, qui est que cette erreur a été enseignée par Lucrèce.
