\subsubsection{\centering \Large CHAPITRE XVI}
\addcontentsline{toc}{section}{\protect\numberline{}{\scshape\bfseries XVI} - \emph{De la conversion des propositions négatives}}
\begin{center}\emph{\large\scshape De la conversion des propositions négatives.}\end{center}

	\lettrine{C}{omme} il est impossible qu'on sépare deux choses totalement, que cette séparation ne soit mutuelle et réciproque, il est clair que si je dis que nul homme n'est pierre, je puis dire aussi que nulle pierre n'est homme ; car si quelque pierre était homme, cet homme serait pierre, et par conséquent il ne serait pas vrai que nul homme ne fût pierre. Et partant :

\bigbreak
\begin{center}{\bfseries\scshape\large 3. Règle}\end{center}

	\emph{Les propositions universelles négatives se peuvent convertir simplement en changeant l'attribut en sujet, et conservant à l'attribut, devenu sujet, la même universalité qu'avait le premier sujet}.

Car l'attribut dans les propositions négatives est toujours pris universellement, parce qu'il est nié selon toute son étendue, ainsi que nous l'avons montré ci-dessus. Et par conséquent il ne faut pas s'étonner qu'il conserve dans la conversion la même généralité qu'il avait avant la conversion.

Mais, par cette même raison, on ne peut faire de conversion des propositions négatives particulières, et on ne peut pas dire, par exemple, que \emph{quelque médecin n'est pas homme}, parce que l'on dit que \emph{quelque homme n'est pas médecin}. Cela vient comme j'ai dit, de la nature même de la négation que nous venons d'expliquer, qui est que dans les propositions négatives l'attribut est toujours pris universellement et selon toute son extension ; de sorte que lorsqu'un sujet particulier devient attribut par la conversion dans une proposition négative particulière, il devient universel, et change de nature contre les règles de la véritable conversion, qui ne doit point changer la restriction ou l'étendue des termes. Ainsi, dans celte proposition, \emph{Quelque homme n'est pas médecin}, le terme d'\emph{homme} est pris particulièrement. Mais dans cette fausse conversion, \emph{quelque médecin n'est pas homme}, le mot d'homme est pris universellement.

Or, il ne s'ensuit nullement de ce que la qualité de médecin est séparée de quelque homme, dans cette proposition, \emph{Quelque homme n'est pas médecin}, et de ce que l'idée de triangle est séparée de celle de quelque figure en cette autre proposition, \emph{Quelque figure n'est pas triangle}, il ne s'ensuit, dis-je, nullement qu'il y ait des médecins qui ne soient pas hommes, ni des triangles qui ne soient pas figures.

Il arrive néanmoins souvent que ces propositions se convertissent dans l'usage ordinaire, comme l'on dit que \emph{quelques savants ne sont pas vertueux} et \emph{quelques vertueux ne sont pas savants}; et même on le peut toujours faire, hors un seul cas, qui est lorsqu'on nie d'un sujet pris particulièrement un et un accident qui ne convienne qu'à ce seul sujet, quoi qu'il ne convienne pas à tout ce sujet: \emph{Quod convenit soli, sed non omni}. Car si un attribut convenait à tout un sujet, on ne l'en pourrait pas nier; et s'il convenait à quelqu'autre chose qu'à ce sujet, on le pourrait prendre pour cette autre chose, et ainsi le sujet en pourrait être nié. Exemple, parce que les richesses ne conviennent pas aux seuls vertueux, comme je puis dire que \emph{quelques vertueux ne sont pas riches}, je peux dire que \emph{quelques vertueux ne sont pas riches}, je puis dire aussi que \emph{quelques riches ne sont pas vertueux}.

Néanmoins il suffit qu'il y ait un seul cas où ces conversions soient fausses, pour les rejeter absolument, parce que, comme nous avons déjà dit, la véritable conversion doit être telle, que supposé que la première proposition soit véritable, l'autre le soit aussi nécessairement, comme il s'ensuit nécessairement que \emph{si nul cruel n'est vertueux, nul vertueux n'est cruel}.

Ce serait ces sortes de conversions qui dépendent de la matière qu'on devrait proprement appeler accidentelles, mais il a plu à l'usage de donner ce nom aux conversions où l'on change la quantité de la proposition, comme nous avons dit qu'il arrivait dans les propositions affirmatives universelles. Ce qui a lieu aussi dans les propositions négatives universelles, qui se peuvent convertir non seulement en négatives universelles, mais aussi en particulières négatives; parce que la particulière négative est enfermée dans la négative universelle. \emph{Nul vicieux n'est saint}: donc, \emph{nul saint n'est vicieux}. Et  plus forte raison : \emph{quelque saint n'est pas vicieux}.
