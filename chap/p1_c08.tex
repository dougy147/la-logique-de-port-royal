\subsubsection{\centering \Large CHAPITRE VIII}
\addcontentsline{toc}{section}{\protect\numberline{}{\scshape\bfseries VIII} - \emph{De la clarté et distinction des idées, et de leur obscurité et confusion}}
\begin{center}\emph{\large\scshape De la clarté et distinction des idées, et de leur obscurité et confusion.}\end{center}

	\lettrine{O}{n} peut distinguer dans une idée la clarté d'avec la distinction, et l'obscurité d'avec la confusion ; car on peut dire qu'une idée nous est claire quand elle nous frappe vivement, quoiqu'elle ne soit point distincte, comme l'idée de la douleur nous frappe très vivement, et, selon cela, peut être appelée claire ; et néanmoins elle est fort confuse, en ce qu'elle nous représente la douleur comme dans la main blessée, quoiqu'elle ne soit que dans notre esprit.

Néanmoins, on peut dire que toute idée est distincte en tant que claire, et que leur obscurité ne vient que de leur confusion, comme dans la douleur le seul sentiment qui nous frappe est clair et est distinct aussi ; mais ce qui est confus, qui est que ce sentiment soit dans notre main, ne nous est point clair.

Prenant donc pour une même chose la clarté et la distinction des idées, il est très important d'examiner pourquoi les unes sont claires, et les autres obscures.

Mais c'est ce qui se connaît mieux par des exemples que par tout autre moyen, et ainsi nous allons faire un dénombrement des principales de nos idées qui sont claires et distinctes, et des principales de celles qui sont confuses et obscures.

L'idée que chacun a de soi-même, comme d'une chose qui pense, est très claire, et de même aussi l'idée de toutes les dépendances de notre pensée, comme juger, raisonner, douter, vouloir, désirer, sentir, imaginer.

Nous avons aussi des idées fort claires de la substance étendue et de ce qui lui convient, comme figure, mouvement, repos; car quoique nous puissions feindre qu'il n'y a aucun corps ni aucune figure, ce que nous ne pouvons pas feindre de la substance qui pense tant que nous pensons, néanmoins nous ne pouvons pas nous dissimuler à nous-même que nous ne concevions clairement l'étendue et la figure.

Nous concevons aussi clairement l'être, l'existence, la durée, l'ordre, le nombre, pourvu que nous pensions seulement que la durée de chaque chose est un mode ou une façon dont nous considérons cette chose en tant qu'elle continue d'être, et que pareillement l'ordre et le nombre ne diffèrent pas en effet des choses ordonnées et nombrées.

Toutes ces idées-là sont si claires, que souvent, en voulant les éclaircir davantage et ne pas se contenter de celles que nous formons naturellement, on les obscurcit.

Nous pouvons aussi dire que l'idée que nous avons de Dieu est claire, quoiqu'elle soit très imparfaite, en ce que notre esprit étant fini ne peut concevoir que très imparfaitement un objet infini. Mais ce sont différentes conditions en une idée d'être parfaite et d'être claire. Car elle est parfaite quand elle nous représente tout ce qui est en son objet, et elle est claire quand elle nous en représente assez pour le concevoir clairement et distinctement.

Les idées confuses et obscures sont celles que nous avons des qualités sensibles, comme des couleurs, des sons, des odeurs, des goûts, du froid, du chaud, de la pesanteur, etc., comme aussi de nos appétits, de la faim, de la soif, de la douleur corporelle, etc., et voici ce qui fait que ces idées sont confuses.

Comme nous avons été plutôt enfants qu'hommes, et que les choses extérieures ont agi sur nous en causant divers sentiments dans notre âme par les impressions qu'elles faisaient sur notre corps, l'âme, qui voyait que ce n'était pas par sa volonté que ces sentiments s'excitaient en elle, mais qu'elle ne les avait qu'à l'occasion de certains corps, comme qu'elle sentait de la chaleur en s'approchant du feu, ne s'est pas contentée de juger qu'il y avait quelque chose hors d'elle qui était cause qu'elle avait ces sentiments, en quoi elle ne se serait pas trompée; mais elle a passé plus outre, ayant cru que ce qui était dans ces objets était entièrement semblable aux sentiments ou aux idées qu'elle avait à leur occasion ; et de ces jugements elle en forme des idées, en transportant ces sentiments de chaleur, de couleur, etc., dans les choses même qui sont hors d'elle ; et ce sont là ces idées obscures et confuses que nous avons des qualités sensibles, l'âme ayant ajouté ses faux jugements à ce que la nature lui faisait connaître.

Et comme ces idées ne sont point naturelles, mais arbitraires, on y a agi avec une grande bizarrerie. Car quoique la chaleur et la brûlure ne soient que deux sentiments, l'un plus faible et l'autre plus fort, on a mis la chaleur dans le feu, et l'on a dit que le feu a de la chaleur ; mais on n'y a pas mis la brûlure ou la douleur qu'on sent en s'en approchant de trop près, et on ne dit point que le feu a de la douleur.

Il en est arrivé de même sur le sujet de la pesanteur. Les enfants voyant des pierres et autres choses semblables qui tombent en bas aussitôt qu'on cesse de les soutenir ; ils ont formé de là l'idée d'une chose qui tombe, laquelle idée est naturelle et vraie, et de plus de quelque cause de cette chute, ce qui est encore vrai. Mais parce qu'ils ne voyaient rien que la pierre, et qu'ils ne voyaient point ce qui la poussait, par un jugement précipité ils ont conclu que ce qu'ils ne voyaient point n'était point, et qu'ainsi la pierre tombait d'elle-même par un principe intérieur qui était en elle sans que rien autre chose la poussait en bas, et c'est à cette idée confuse, et qui n'était née que de leur erreur qu'ils ont attaché le nom de gravité et de pesanteur.

Et il leur est encore ici arrivé la même chose que dans l'autre exemple, qui est de faire des jugements tout différents de choses dont ils devaient juger de la même force. Car comme ils ont vu des pierres qui se remuaient en bas vers la terre, ils ont vu des pailles qui se remuaient dans l'ambre, et des morceaux de fer ou d'acier qui se remuaient vers l'aimant. Ils avaient donc autant de raison de mettre une qualité dans les pailles et dans le fer pour se porter vers l'ambre ou l'aimant, que dans les pierres pour se porter vers la terre. Néanmoins il ne leur a pas plu de le faire, mais ils ont mis une qualité dans l'ambre pour attirer les pailles, et une dans l'aimant pour attirer le fer qu'ils ont appelé des qualités attractives, comme s'il ne leur eut pas été aussi facile d'en mettre une dans la terre pour attirer les choses pesantes. Mais quoi qu'il en soit, ces qualités attractives ne sont nées de même que la pesanteur que d'un faux raisonnement, qui a fait croire qu'il fallait que le fer attirait l'aimant, parce qu'on ne voyait rien qui poussait l'aimant vers le fer : quoi qu'il soit impossible de concevoir qu'un corps en puisse attirer un autre, si le corps qui attire ne se meut lui-même, et si celui qui est attiré ne lui est joint ou attaché par quelque lien.

On pourrait étendre cela beaucoup plus loin, mais c'est assez pour faire entendre toutes les autres idées confuses, qui ont presque toutes quelques causes semblables à ce que nous venons de dire.

L'unique remède à cet inconvénient, est de nous défaire des préjugés de notre enfance, et de ne croire rien de ce qui est du ressort de notre raison, par ce que nous en avons jugé autrefois, mais par ce que nous en jugeons maintenant. Et ainsi nous nous réduirons à nos idées naturelles, et pour les confuses nous n'en retiendrons que ce qu'elles ont de clair, comme qu'il y a quelque chose dans le feu qui est cause que je sens de la chaleur, que toutes les choses qu'on appelle pesantes sont poussées en bas par quelque cause, ne déterminant rien de ce qui peut être dans le feu qui me cause ce sentiment, ou de la cause qui fait tomber une pierre en bas, que je n'aie des raisons claire qui m'en donnent la connaissance.

