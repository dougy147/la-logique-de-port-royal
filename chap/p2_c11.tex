\subsubsection{\centering \Large CHAPITRE XI}
\addcontentsline{toc}{section}{\protect\numberline{}{\scshape\bfseries XI} - \emph{De deux sortes de propositions qui sont de grand usage dans les sciences, la division et la définition. Et premièrement de la division}}
\begin{center}\emph{\large\scshape De deux sortes de propositions qui sont de grand usage dans les sciences, la division et la définition. Et premièrement de la division.}\end{center}


	\lettrine{I}{l} est nécessaire de dire quelque chose en particulier de deux sortes de propositions qui sont de grand usage dans les sciences, la division et la définition.

La division est le partage d'un tout en ce qu'il contient.

Mais comme il y a deux sortes de \emph{tout}, il y a aussi deux sortes de divisions. Il y a un tout composé de plusieurs parties réellement distinctes, appelé en latin \emph{totum}, et dont les parties sont appelées \emph{parties intégrantes}. La division de ce tout s'appelle proprement \emph{partition}. Comme quand on divise une maison en ses appartements, une ville en ses quartiers, un royaume ou un état en ses provinces, l'homme en corps et en âme, le corps en ses membres. La seule règle de cette division est de faire des dénombrements bien exacts et auxquels il ne manque rien.

L'autre \emph{tout} est appelé en latin \emph{omne}, et ses parties \emph{parties subjectives} ou \emph{inférieures}; parce que ce tout est un terme commun, et ses parties sont les sujets compris dans son étendue. Le mot d'\emph{animal} est un tout de cette nature, dont les inférieurs comme \emph{homme} et \emph{bête}, qui sont compris dans son étendue, sont les parties subjectives. Cette division retient proprement le nom de division, et on en peut remarquer de quatre sortes.

La première est quand on divise le genre par ses espèces : \emph{Toute substance est corps ou esprit : Tout animal est homme ou bête}.

La deuxième est quand on divise le genre par ses différences : \emph{Tout animal est raisonnable ou privé de raison : Tout nombre est pair ou impair: Toute proposition est vraie ou fausse : Toute ligne est droite ou courbe}.

La troisième quand on divise un sujet commun par les accidents opposés dont il est capable, ou selon ses divers inférieurs, ou en divers temps, comme, \emph{Tout astre est lumineux par soi-même, ou seulement par réflexion: Tout corps est en mouvement ou en repos : Tous les Français sont nobles ou roturiers : Tout homme est sain ou malade : Tous les peuples se servent pour s'exprimer, ou de la parole seulement, ou de l'écriture outre la parole}.

La quatrième d'un accident en ses divers sujets, comme la division des biens en ceux de l'esprit et du corps.

Les règles de la division sont premièrement qu'elle soit entière, c'est-à-dire que les membres de la division comprennent toute l'étendue du terme que l'on divise, comme \emph{pair} et \emph{impair} comprennent toute l'étendue du terme de \emph{nombre}, n'y en ayant point qui ne soit pair ou impair. Il n'y a presque rien qui fasse faire tant de faux raisonnements que le défaut d'attention à cette règle; et ce qui trompe est qu'il y a souvent des termes qui paraissent tellement opposés, qu'ils semblent ne point souffrir de milieu, et qui ne laissent pas d'en avoir. Ainsi, entre ignorant et savant, il y a une certaine médiocrité de savoir qui tire un homme du rang des ignorants, et qui ne le met pas encore au rang des savants. Entre vicieux et vertueux, il y a aussi un certain état dont on peut dire ce que Tacite dit de Galba, \emph{magis extra vitia quam cum virtutibus}: car il y a des gens qui, n'ayant point de vices grossiers ne sont pas appelés vicieux, et qui ne faisant point de bien, ne peuvent point être appelés vertueux, quoique devant Dieu ce soit un grand vice que de n'avoir point de vertu. Entre sain et malade, il y a l'état d'un homme indisposé ou convalescent : entre le jour et la nuit, il y a le crépuscule : entre les vices opposés, il y a le milieu de la vertu, comme la piété entre l'impiété et la superstition ; et quelquefois ce milieu est double, comme entre l'avarice et la prodigalité, il y a libéralité, et une épargne louable : entre la timidité qui craint tout et la témérité qui ne craint rien, il y a la générosité qui ne s'étonne point des périls, et une précaution raisonnable, qui fait éviter ceux auxquels il n'est pas à propos de s'exposer.

La deuxième règle, qui est une suite de la première, est que les membres de la division soient opposés, comme \emph{pair, impair ; raisonnable, privé de raison}. Mais il faut remarquer ce qu'on a déjà dit dans la première partie, qu'il n'est pas nécessaire que toutes les différences qui font ses membres opposés soient positives ; mais qu'il suffit que l'une le soit, et que l'autre soit le genre seul avec la négation de l'autre différence. Et c'est même par là qu'on fait que les membres sont plus certainement opposés. Ainsi, la différence de la bête d'avec l'homme n'est que la privation de la raison, qui n'est rien de positif : l'imparité n'est que la négation de la divisibilité en deux parties égales. Le nombre premier n'a rien que n'ait le nombre composé ; l'un et l'autre ayant l'unité pour mesure, et celui qu'on appelle premier n'étant différent du composé, qu'en ce qu'il n'a point d'autre mesure que l'unité.

Néanmoins, il faut avouer que c'est le meilleur d'exprimer les différences opposées par des termes positifs, quand cela se peut ; parce que cela fait mieux entendre la nature des membres de la division. C'est pourquoi la division de la substance en celle qui pense et celle qui est étendue, est beaucoup meilleure que la commune, en celle qui est matérielle, et celle qui est immatérielle, ou bien en celle qui est corporelle, et celle qui n'est pas corporelle; parce que les mots \emph{d'immatérielle} et \emph{d'incorporelle}, ne nous donnent qu'une idée fort imparfaite et fort confuse de ce qui se comprend beaucoup mieux par les mots de \emph{substance qui pense}.

La troisième règle, qui est une suite de la seconde, est que l'un des membres ne soit pas tellement enfermé dans l'autre, que l'autre en puisse être affirmé, quoiqu'il puisse quelquefois y être enfermé en une autre manière ; car la ligne est enfermée dans la surface comme le terme de la surface, et la surface dans le solide comme le terme du solide. Mais cela n'empêche pas que l'étendue ne se divise en ligne, surface et solide, parce qu'on ne peut pas dire que la ligne soit surface, ni la surface solide. On ne peut pas, au contraire, diviser le nombre en pair, impair et carré, parce que tout nombre carré étant pair ou impair, il est enfermé dans les deux premiers membres.

On ne doit pas aussi diviser les opinions en vraies, fausses et probables, parce que toute opinion probable est vraie ou fausse. Mais on peut les diviser premièrement en vraies et en fausses, et puis diviser les unes et les autres en certaines et en probables.

Ramus et ses partisans se sont fort tourmentés pour montrer que toutes les divisions ne doivent avoir que deux membres. Tant qu'on peut le faire commodément, c'est le meilleur ; mais la clarté et la facilité étant ce qu'on doit le plus considérer dans les sciences, on ne doit pas rejeter les divisions en trois membres, et plus encore, quand elles sont plus naturelles, et qu'on aurait besoin de subdivisions forcées pour les faire toujours en deux membres : car alors, au lieu de soulager l'esprit, ce qui est le principal fruit de la division, on l'accable par un grand nombre de subdivisions, qu'il est bien plus difficile de retenir, que si tout d'un coup on avait fait plus de membres à ce que l'on divise. Par exemple, n'est-il pas plus court, plus simple et plus naturel de dire : \emph{Toute étendue est, ou ligne, ou surface, ou solide}, que de dire comme Ramus, \emph{magnitudo est linea, vel lineatum : Lineatum est superficies vel solidum}.

Enfin, on peut remarquer que c'est un égal défaut de ne faire pas assez et de faire trop de divisions. L'un accable l'esprit, l'autre le dissipe. Grassot qui est un philosophe estimable entre les interprètes d'Aristote, a nui à son livre par le trop grand nombre de divisions. On retombe par là dans la confusion que l'on prétend éviter. \emph{Confusum est quidquid in pulverem sectum est}.

