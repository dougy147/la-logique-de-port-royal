\subsubsection{\centering \Large CHAPITRE XII}
\addcontentsline{toc}{section}{\protect\numberline{}{\scshape\bfseries XII} - \emph{Application de ce principe général à plusieurs syllogismes qui paraîssent embarrassés}}
\begin{center}\emph{\large\scshape Application de ce principe général à plusieurs syllogismes qui paraîssent embarrassés.}\end{center}

	\lettrine{S}{achant} donc, par ce que nous avons dit dans la seconde partie, ce que c'est que l'étendue et la compréhension des termes, par où l'on peut juger quand une proposition en contient ou n'en contient pas une autre, on peut juger de la bonté ou du défaut de tout syllogisme, sans considérer s'il est simple ou composé, complexe ou incomplexe, sans prendre garde aux figures ni aux modes, par ce seul principe général : \emph{Que l'une des deux propositions doit contenir la conclusion, et l'autre faire voir qu'elle la contient}. C'est ce qui se comprendra mieux par des exemples.

\begin{center}{\scshape 1. Exemple}\end{center}

Je doute si ce raisonnement est bon :

	\begin{tabularx}{\textwidth}{X}
		\emph{Le devoir d'un chrétien est de ne point louer ceux qui commettent des actions criminelles:} \\
		\emph{Or, ceux qui se battent en duel commettent une action criminelle:} \\
		\emph{Donc le devoir d'un chrétien est de ne point louer ceux qui se battent en duel.} \\
	\end{tabularx}

Je n'ai que faire de me mettre en peine pour savoir à quelle figure ni à quel mode on peut le réduire; mais il me suffit de considérer si la conclusion est contenue dans l'une des deux premières propositions, et si l'autre le fait voir, et je trouve d'abord que la première n'ayant rien de différent de la conclusion, sinon qu'il y a en l'une, \emph{ceux qui commettent des actions criminelles}, et en l'autre, \emph{ceux qui se battent en duel}; celle où il y a, \emph{commettre des actions criminelles} contiendra celle où il y a, \emph{se battre en duel}, pourvu que \emph{commettre des actions criminelles}, contienne \emph{se battre en duel}.

Or, il est visible, par le sens, que le terme de, \emph{ceux qui commettent des actions criminelles}, est pris universellement ; et que cela s'entend de tous ceux qui en commettent quelles qu'elles soient. Et ainsi la mineure, \emph{ceux qui se battent en duel commettent une action criminelle}, faisant voir que \emph{se battre en duel} est contenu sous ce terme de, \emph{commettre des actions criminelles} elle fait voir aussi que la première proposition contient la conclusion.

\begin{center}{\scshape 2. Exemple}\end{center}

Je doute si ce raisonnement est bon :

\begin{center}
	\begin{tabular}{l}
		\emph{L'Évangile promet le salut aux chrétiens :} \\
		\emph{Il y a des méchants qui sont chrétiens :} \\
		\emph{Donc l'Évangile promet le salut aux méchants.} \\
	\end{tabular}
\end{center}

Pour en juger, je n'ai qu'à regarder que la majeure ne peut contenir la conclusion, si le mot de \emph{chrétiens} n'y est pris généralement pour \emph{tous les chrétiens}, et non pour \emph{quelques chrétiens} seulement. Car, si l'Évangile ne promet le salut qu'à quelques chrétiens, il ne s'ensuit pas qu'il le promette à des méchants qui seraient chrétiens, parce que ces méchants peuvent n'être pas du nombre de ces chrétiens auxquels l'Évangile promet le salut ; c'est pourquoi ce raisonnement conclut bien, mais la majeure est fausse, si le mot de \emph{chrétiens} se prend dans la majeure pour \emph{tous les chrétiens} ; et il conclut mal, s'il ne se prend que pour \emph{quelques chrétiens}. Car alors la première proposition ne contiendrait point la conclusion.

Mais, pour savoir s'il doit se prendre universellement, cela doit se juger par une autre règle que nous avons donnée dans la seconde partie, qui est que \emph{hors les faits, ce dont on affirme, est pris universellement, quand il est exprimé indéfiniment}. Or quoique \emph{ceux qui commettent des actions criminelles} dans le premier exemple, et \emph{chrétiens} dans le deuxième, soient partie d'un attribut, ils tiennent lieu néanmoins de sujet au regard de l'autre partie du même attribut; car ils sont ce dont on affirme, qu'on ne doit pas les louer, ou qu'on leur promet le salut : et par conséquent, n'étant point restreints, ils doivent être pris universellement, et ainsi, l'un et l'autre argument est bon dans la forme; mais la majeure du second est fausse, si ce n'est qu'on entendit par le mot de \emph{chrétiens}, ceux qui vivent conformément à l'Évangile, auquel cas la mineure serait fausse, parce qu'il n'y a point de méchants qui vivent conformément à l'Évangile.

\begin{center}{\scshape 3. Exemple}\end{center}

Il est aisé de voir par le même principe que ce raisonnement ne vaut rien :

	\begin{tabularx}{\textwidth}{X}
		\emph{La loi divine commande d'obéir aux magistrats séculiers :} \\
		\emph{Les évêques ne sont point des magistrats séculiers :} \\
		\emph{Donc la loi divine ne commande point d'obéir aux évêques.} \\
	\end{tabularx}

Car nulle des premières propositions ne contient la conclusion, puisqu'il ne s'ensuit pas que la loi divine, commandant une chose, n'en commande pas une autre : et ainsi, la mineure fait bien voir que \emph{les évêques} ne sont pas compris sous le nom de \emph{magistrats séculiers}, et que le commandement d'honorer les magistrats séculiers ne comprend point les évêques ; mais la majeure ne dit pas que Dieu n'ait fait d'autres commandements que celui-là, comme il faudrait qu'elle fit pour enfermer la conclusion en vertu de cette mineure : ce qui fait que cet autre argument est bon :


\begin{center}{\scshape 4. Exemple}\end{center}

	\begin{tabularx}{\textwidth}{X}
		\emph{Le christianisme n'oblige les serviteurs de servir leurs maîtres que dans les choses qui ne sont point contre la loi de Dieu :} \\
		\emph{Or, un mauvais commerce est contre la loi de Dieu :} \\
		\emph{Donc le christianisme n'oblige point les serviteurs de servir leurs maîtres dans un mauvais commerce.} \\
	\end{tabularx}

Car la majeure contient la conclusion, puisque la mineure, \emph{mauvais commerce}, est contenue dans le nombre des choses qui sont contre la loi de Dieu, et que la majeure étant exclusive, vaut autant que si on disait, \emph{La loi divine n'oblige point les serviteurs de servir leurs maîtres dans toutes les choses qui sont contre la loi de Dieu}.

\begin{center}{\scshape 5. Exemple}\end{center}

On peut résoudre facilement ce sophisme commun par ce seul principe :

	\begin{tabularx}{\textwidth}{X}
		\emph{Celui qui dit que vous êtes un animal dit vrai :} \\
		\emph{Celui qui dit que vous êtes un oison dit que vous êtes un animal :} \\
		\emph{Donc celui qui dit que vous êtes un oison dit vrai.} \\
	\end{tabularx}

Car il suffit de dire que nulle de ces deux premières propositions ne contient la conclusion ; puisque, si la majeure la contenait, n'étant différente de la conclusion qu'en ce qu'il y a \emph{animal} dans la majeure, et \emph{oison} dans la conclusion, il faudrait qu'\emph{animal} contint \emph{oison}. Mais \emph{animal} est pris particulièrement dans cette majeure, puisqu'il est attribut de cette proposition incidente affirmative, \emph{vous êtes un animal}; et par conséquent il ne pourrait contenir \emph{oison} que dans sa compréhension. Ce qui obligerait, pour le faire voir, de prendre le mot d'\emph{animal} universellement dans la mineure, en affirmant \emph{oison} de tout animal. Ce qu'on ne peut faire, et ce qu'on ne fait pas aussi, puisque \emph{animal} est encore pris particulièrement dans la mineure, étant encore, aussi bien que dans la majeure, l'attribut de cette proposition affirmative incidente, \emph{vous êtes un animal}.

\newpage

\begin{center}{\scshape 6. Exemple}\end{center}

On peut encore résoudre par là cet ancien sophisme, qui est rapporté par saint Augustin :

	\begin{tabularx}{\textwidth}{X}
		\emph{Vous n'êtes pas ce que je suis:} \\
		\emph{Je suis homme:} \\
		\emph{Donc vous n'êtes pas homme.} \\
	\end{tabularx}

Cet argument ne vaut rien par les règles des figures, parce qu'il est de la première, et que la première proposition, qui en est la mineure, est négative : mais il suffit de dire que la conclusion n'est point contenue dans la première de ces propositions, et que l'autre proposition (\emph{je suis homme}) ne fait point voir qu'elle y soit contenue; car la conclusion étant négative, le terme d'\emph{homme} y est pris universellement, et ainsi n'est point contenu dans le terme \emph{ce que je suis}, parce que celui qui parle ainsi n'est pas \emph{tout homme}, mais seulement \emph{quelque homme}, comme il paraît en ce qu'il dit seulement dans la proposition applicative, \emph{je suis homme}, où le terme d'homme est restreint à une signification particulière, parce qu'il est attribut d'une proposition affirmative : or, le général n'est pas contenu dans le particulier.


