\subsubsection{\centering \Large CHAPITRE VII}
\addcontentsline{toc}{section}{\protect\numberline{}{\scshape\bfseries VII} - \emph{Des diverses sortes de propositions composées}}
\begin{center}\emph{\large\scshape Des diverses sortes de propositions composées.}\end{center}

	\lettrine{N}{ous} avons déjà dit que les propositions composées sont celles qui ont ou un double sujet ou un double attribut. Or, il y en a de deux sortes : les unes où la composition est expressément marquée, et les autres où elle est plus cachée, et que les logiciens, pour cette raison, appellent \emph{exponibiles}, qui ont besoin d'être exposées ou expliquées.

On peut réduire celles de la première sorte à six espèces : Les copulatives et les disjonctives, les conditionnelles et les causales, les relatives et les discrétives.

\bigbreak
{1.} On appelle {\bfseries\scshape Copulatives} celles qui enferment ou plusieurs sujets, ou plusieurs attributs joints par une conjonction affirmative ou négative. Car \emph{ni} fait la même chose que \emph{et} en ces sortes de propositions :
	\begin{tabularx}{\textwidth}{X}
		\emph{La foi et la bonne vie sont nécessaires au salut. Le méchant n'est heureux ni en ce monde ni en l'autre}. \\
	\end{tabularx}

La vérité de ces propositions dépend de la vérité de toutes les deux parties. C'est pourquoi ces deux-ci sont vraies, parce que non seulement la foi, et non seulement la bonne vie, mais que l'un et l'autre est nécessaire pour le salut; et les méchants sont malheureux en ce monde ici aussi bien que l'autre.

Ces propositions ne sont regardées comme négatives et contradictoires aux autres, que quand la négation tombe sur la conjonction; ce qui le fait en latin en mettant la négation à la tête de la proposition :

	\begin{tabularx}{\textwidth}{X}
\emph{Non et fides et bona opera necessaria sunt ad salutem.} \\
	\end{tabularx}

Mais en Français on fait le même effet, quoiqu'on mette la négation auprès du verbe :

    \begin{tabularx}{\textwidth}{X}
		\emph{La science ni les richesses ne sont pas nécessaires pour le salut.} \\
		\emph{Les méchants ne sont point heureux en ce monde, et ne le sont point aussi en l'autre.} \\
    \end{tabularx}

Ou pour la mieux marquer on peut mettre au lieu de la négation latine, \emph{il n'est pas vrai} :
	\begin{tabularx}{\textwidth}{X}
	\emph{Il n'est pas vrai que la science et les richesses soient nécessaires au salut.} \\
	\emph{Il n'est pas vrai que les méchants soient malheureux en ce monde et en l'autre.} \\
	\end{tabularx}

\bigbreak
{2.} Les {\bfseries\scshape Disjonctives} sont de grand usage, et ce sont celles où entre la conjonction disjonctive \emph{vel}, \emph{ou}.
	\begin{tabularx}{\textwidth}{X}
		\emph{Toute ligne est droite, ou courbe.} \\
		\emph{Tout homme sera éternellement heureux, ou éternellement malheureux.} \\
		\emph{Toute action faite avec jugement est bonne ou mauvaise.} \\
	\end{tabularx}

La vérité de ces propositions dépend de l'opposition nécessaire
des parties, qui ne doivent point souffrir de milieu, mais chaque partie séparée n'est point nécessairement vraie.

Car il n'est point nécessaire qu'un homme soit heureux, ni aussi qu'il soit malheureux éternellement, mais il est nécessaire qu'il soit l'un, l'autre.

Elles sont négatives, quand on nie la nécessité de la disjonctive. Ce qu'on fait en latin comme dans toutes les autres propositions composées, en mettant la négation à la tête : \emph{Non omnis actio est bona vel mala}; et en français, \emph{Il n'est pas vrai que toute action soit bonne ou mauvaise}.


\bigbreak
{3.} Les {\bfseries\scshape Conditionnelles} sont celles qui sont liées par la condition \emph{si}, comme :
	\begin{tabularx}{\textwidth}{X}
		\emph{Si on ne vit selon l'Évangile, on ne sera point sauvé.} \\
		\emph{Si on aime Dieu, on trouvera tout en lui.} \\
	\end{tabularx}

On ne regarde pour la vérité ces propositions que la vérité de la conséquence. Car quoique l'un et l'autre partie fût fausse, si néanmoins la conséquence est vraie, la proposition en tant que conditionnelle passe pour vraie; comme :
	\begin{tabularx}{\textwidth}{X}
		\emph{Si le singe est homme, il est raisonnable.} \\
	\end{tabularx}

Ces propositions ne sont considérées comme négatives et contradictoires aux affirmatives, que quand la condition est niée. Ce qui se fait en latin en mettant une négation à la tête.
	\begin{tabularx}{\textwidth}{X}
		\emph{Non si miserum fortuna sinonem} \\
		\emph{Finxit vanum etiam mendacemque improba singet.} \\
	\end{tabularx}

Mais en français on exprime ces contradictoires par \emph{quoique}, et une négation.
	\begin{tabularx}{\textwidth}{X}
		\emph{Si vous mangez du fruit défendu, vous mourrez.} \\
		\emph{Quoique vous mangiez du fruit défendu, vous ne mourrez pas.} \\
	\end{tabularx}

Ou bien par, \emph{il n'est pas vrai}.
	\begin{tabularx}{\textwidth}{X}
		\emph{Il n'est pas vrai que si vous mangez du fruit défendu, vous mourrez.} \\
	\end{tabularx}

\bigbreak
{4.} Les {\bfseries\scshape Causales} sont celles qui contiennent deux propositions liées
par un mot de cause, \emph{quia, parce que, à cause que}.
	\begin{tabularx}{\textwidth}{X}
		\emph{Il a été puni, parce qu'il a commis un crime.} \\
		\emph{Il sera sauvé parce qu'il vit selon Dieu.} \\
	\end{tabularx}

Il est nécessaire pour la vérité de ces propositions, que l'une et l'autre partie soit vraie, et que l'une soit cause de l'autre.

On les contredit en niant ou l'une ou l'autre des deux parties, ou toutes les deux, ou seulement que l'une soit cause de l'autre; comme :
	\begin{tabularx}{\textwidth}{X}
\emph{Il a été puni, et il a commis un crime, mais ce n'a pas été pour cela qu'il a été puni.}
	\end{tabularx}

\bigbreak
{5.} Les {\bfseries\scshape Relatives} sont celles qui enferment quelque comparaison ou quelque rapport.
	\begin{tabularx}{\textwidth}{X}
		\emph{Où est le trésor, là est le cœur.} \\
		\emph{Quelle est la vie, telle est la mort.} \\
	\end{tabularx}

La vérité dépend de la justesse du rapport. Et on les contredit en niant le rapport.
	\begin{tabularx}{\textwidth}{X}
		\emph{Il n'est pas vrai, que quelle est la vie telle est la mort.} \\
	\end{tabularx}

\bigbreak
{6.} Les {\bfseries\scshape Discrétives} sont celles où on énonce diverses choses de différentes choses, en les joignant par \emph{sed, mais} ou \emph{tamen, néanmoins}; comme :
	\begin{tabularx}{\textwidth}{X}
		\emph{Non omnis qui dicit mihi, Domine, Domine, intrabit in regnum caelorum, sed qui facit voluntatem Patris mei:} \\
		\emph{Ne croyez pas que quiconque me dira, Seigneur, Seigneur, doive entrer dans le royaume des cieux, mais ce sera celui qui fera la volonté de mon Père, qui y entrera.} \\
	\end{tabularx}

La vérité de cette sorte de proposition dépend de la vérité de toutes les deux parties, et de la séparation qu'on y met. Car quoique les deux parties fussent vraies, une proposition de cette sorte serait ridicule, s'il n'y avait point entre elles d'opposition, comme si je disais :
	\begin{tabularx}{\textwidth}{X}
		\emph{Judas était un larron, et néanmoins il ne put souffrir que la Madelaine répandit ses parfums sur Jésus-Christ.} \\
	\end{tabularx}

Il peut y avoir plusieurs contradictoires d'une proposition de cette sorte, comme si on disait :
	\begin{tabularx}{\textwidth}{X}
\emph{Ce n'est pas des richesses, mais de la science que dépend le bonheur.}
	\end{tabularx}

On peut contredire cette proposition en toutes ces manières :
	\begin{tabularx}{\textwidth}{X}
		\emph{Le bonheur dépend des richesses, et non pas de la science.} \\
		\emph{Le bonheur ne dépend ni des richesses ni de la science.} \\
		\emph{Le bonheur dépend des richesses et de la science.} \\
	\end{tabularx}

Ainsi l'on voit que les copulatives sont contradictoires des discrétives. Car ces deux dernières propositions sont copulatives.
