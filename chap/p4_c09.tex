\subsubsection{\centering \Large CHAPITRE IX}
\addcontentsline{toc}{section}{\protect\numberline{}{\scshape\bfseries IX} - \emph{Réponse à ce que disent les géomètres sur ce sujet}}
\begin{center}\emph{\large\scshape Réponse à ce que disent les géomètres sur ce sujet.}\end{center}


	\lettrine{I}{l} y a des géomètres qui croient avoir justifié ces défauts, en disant qu'ils ne se mettent pas en peine de cela ; qu'il leur suffit de ne rien dire qu'ils ne prouvent d'une manière convaincante ; et qu'ils sont par là assurés d'avoir trouvé la vérité, qui est leur unique but.

On avoue aussi que ces défauts ne sont pas si considérables, qu'on ne soit obligé de reconnaître que, de toutes les sciences humaines, il n'y en a point qui aient été mieux traitées que celles qui sont comprises sous le nom général de mathématiques ; mais on prétend seulement qu'on pourrait encore y ajouter quelque chose qui les rendrait plus parfaites; et que, quoique la principale chose qu'ils aient dû y considérer soit de ne rien avancer que de véritable, il aurait été néanmoins à souhaiter qu'ils eussent eu plus d'attention à la manière la plus naturelle de faire entrer la vérité dans l'esprit.

Car ils ont beau dire qu'ils ne se soucient pas du vrai ordre, ni de prouver par des voies naturelles ou éloignées, pourvu qu'ils fassent ce qu'ils prétendent, qui est de convaincre ; ils ne peuvent pas changer par là la nature de notre esprit, ni faire que nous n'ayons une connaissance beaucoup plus nette, plus entière et plus parfaite des choses que nous savons par leurs vraies causes et leurs vrais principes, que de celles qu'on ne nous a prouvées que par des voies obliques et étrangères.

Et il est de même indubitable qu'on apprend avec une facilité incomparablement plus grande, et qu'on retient beaucoup mieux ce qu'on enseigne dans le vrai ordre; parce que les idées qui ont une suite naturelle s'arrangent bien mieux dans notre mémoire, et se réveillent bien plus aisément les unes les autres.

On peut dire même que ce qu'on a su une fois pour en avoir pénétré la vraie raison, ne se retient pas par mémoire, mais par jugement, et que cela devient tellement propre, qu'on ne peut l'oublier : au lieu que ce qu'on ne sait que par des démonstrations qui ne sont point fondées sur des raisons naturelles, s'échappe aisément, et se retrouve difficilement quand il nous est une fois sorti de la mémoire, parce que notre esprit ne nous donne point de voie pour le retrouver.

Il faut donc demeurer d'accord qu'il est en soi beaucoup mieux de garder cet ordre que de ne point le garder ; mais tout ce que pourraient dire des personnes équitables, est qu'il faut négliger un petit inconvénient, lorsqu'on ne peut l'éviter sans tomber dans un plus grand ; qu'ainsi c'est un inconvénient de ne pas toujours garder le vrai ordre; mais qu'il vaut mieux néanmoins ne pas le garder, que de manquer à prouver invinciblement ce que l'on avance, et s'exposer à tomber dans quelque erreur et quelque paralogisme, en recherchant de certaines preuves qui peuvent être plus naturelles, mais qui ne sont pas si convaincantes ni si exemptes de tout soupçon de tromperie.

Cette réponse est très raisonnable. Et j'avoue qu'il faut préférer à toutes choses l'assurance de ne point se tromper, et qu'il faut négliger le vrai ordre si on ne peut le suivre sans perdre beaucoup de la force des démonstrations, et s'exposer à l'erreur. Mais je ne demeure pas d'accord qu'il soit impossible d'observer l'un et l'autre, pourvu que l'on se contente d'une certitude raisonnable; c'est-à-dire, qu'on accorde pour vrai ce que nul homme ne pourra croire être faux, pourvu qu'il apporte tant soit peu d'attention à le considérer.

Car je m'imagine qu'on pourrait faire des éléments de géométrie, où toutes choses seraient traitées dans leur ordre naturel, toutes les propositions prouvées par des voies très simples et très naturelles, et où tout néanmoins serait très clairement démontré.
