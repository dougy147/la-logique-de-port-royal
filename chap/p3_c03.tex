\subsubsection{\centering \Large CHAPITRE III}
\addcontentsline{toc}{section}{\protect\numberline{}{\scshape\bfseries III} - \emph{Règles générales des syllogismes simples incomplexées}}
\begin{center}\emph{\large\scshape Règles générales des syllogismes simples incomplexées.}\end{center}

	\begin{center}{\footnotesize Ce chapitre et les suivants jusqu'au onzième, sont de ceux dont il est parlé dans le Discours qui contiennent des choses subtiles, et nécessaires pour la spéculation de la Logique, mais qui sont de peu d'usage dans la pratique, si ce n'est qu'elles peuvent servir à exercer l'esprit.}\end{center}

		\lettrine{N}{ous} avons déjà vu dans les chapitres précédents qu'un syllogisme simple ne doit avoir que trois termes, les deux termes de la conclusion et un seul moyen, dont chacun étant répété deux fois, il s'en fait trois propositions : la majeure où entre le moyen et l'attribut de la conclusion appelé le grand terme; la mineure, où entre aussi le moyen et le sujet de la conclusion, appelée le petit terme; et la conclusion, dont le petit terme est le sujet, et le grand terme l'attribut.

Mais parce qu'on ne peut pas tirer toutes sortes de conclusions de toutes sortes de prémisses, il y a des règles générales qui font voir qu'une conclusion ne saurait être bien tirée dans un syllogisme où elles ne sont pas observées : et ces règles sont fondées sur les axiomes qui ont été établis dans la seconde partie, touchant la nature des propositions affirmatives et négatives, universelles et particulières, tels que sont ceux-ci, qu'on ne fera que proposer, ayant été prouvés ailleurs.

\smallbreak
$1$. Les propositions particulières sont enfermées dans les générales de même nature, et non les générales dans les particulières. I dans A, et O dans E et non A dans I ni E dans O.

\smallbreak
$2$. Le sujet d'une proposition, pris universellement ou particulièrement, est ce qui la rend universelle ou particulière.

\smallbreak
$3$. L'attribut d'une proposition affirmative n'ayant jamais plus d'étendue que le sujet, est toujours considéré comme pris particulièrement, parce que ce n'est que par accident s'il est quelquefois pris généralement.

\smallbreak
$4$. L'attribut d'une proposition négative est toujours pris généralement.

C'est principalement sur ces axiomes que sont fondées les règles générales des syllogismes, qu'on ne saurait violer sans tomber dans de faux raisonnements.

\bigbreak
\bigbreak
\bigbreak

\begin{center}{\bfseries\scshape\large 1. Règle}\end{center}

	\emph{Le moyen ne peut être pris deux fois particulièrement; mais il doit être pris au moins une fois universellement}.

Car, devant unir ou désunir les deux termes de la conclusion, il est clair qu'il ne peut le faire s'il est pris pour deux parties différentes d'un même tout, parce que ce ne sera pas peut-être ; la même partie qui sera unie ou désunie de ces deux termes. Or, étant pris deux fois particulièrement, il peut être pris pour deux différentes parties du même tout; et par conséquent on n'en pourra rien conclure, au moins nécessairement; ce qui suffit pour rendre un argument vicieux, puisqu'on n'appelle bon syllogisme, comme on vient de le dire, que celui dont la conclusion ne peut être fausse, les prémisses étant vraies. Ainsi, dans cet argument : \emph{Quelque homme est saint: Quelque homme est voleur : Donc quelque voleur est saint}, le mot d'\emph{homme} étant pris pour diverses parties des hommes, ne peut unir \emph{voleur} avec \emph{saint}, parce que ce n'est pas le même homme qui est saint et qui est voleur.

On ne peut pas dire de même du sujet et de l'attribut de la conclusion ; car, encore qu'ils soient pris deux fois particulièrement, on peut néanmoins les unir ensemble en unissant un de ces termes au moyen dans toute l'étendue du moyen; car il s'ensuit de là fort bien que si ce moyen est uni dans quelqu'une de ses parties à quelque partie de l'autre terme, ce premier terme, que nous avons dit être joint à tout le moyen, se trouvera joint aussi avec le terme auquel quelque partie du moyen est jointe. S'il y a quelques Français dans chaque maison de Paris, et qu'il y ait des Allemands en quelques maisons de Paris, il y a des maisons ou il y a tout ensemble un Français et un Allemand.

\begin{center}
	\begin{tabular}{l}
\emph{Si quelques riches sont fourbes:} \\
\emph{Et que tout riche soit honoré:} \\
\emph{Il y a des fourbes honorés}. \\
	\end{tabular}
\end{center}
Car ces riches qui sont sots, sont aussi honorés, puisque tous les riches sont honorés, et par conséquent, dans ces riches sots et honorés, les qualités de sot et d'honoré sont jointes ensemble.

\bigbreak
\bigbreak

\begin{center}{\bfseries\scshape\large 2. Règle}\end{center}

	\emph{Les termes de la conclusion ne peuvent point être pris plus universellement dans la conclusion que dans les prémisses.}

C'est pourquoi, lorsque l'un ou l'autre est pris universellement dans la conclusion, le raisonnement sera faux s'il est pris particulièrement dans les deux premières propositions.

La raison est qu'on ne peut rien conclure du particulier au général (selon le premier axiome) ; car de ce que quelque homme est noir, on ne peut pas conclure que tout homme est noir.

De ces deux règles on tire ces corollaires.

\begin{center}{\bfseries\scshape 1. Corollaire}\end{center}

	Il doit toujours y avoir dans les prémisses un terme universel de plus que dans la conclusion. Car tout terme qui est général dans la conclusion, doit aussi être dans les prémisses. Et de plus, le moyen doit y être pris au moins une fois généralement.

\begin{center}{\bfseries\scshape 2. Corollaire}\end{center}

	Lorsque la conclusion est négative, il faut nécessairement que le grand terme soit pris généralement dans la majeure ; car il est pris généralement dans la conclusion négative (par le quatrième axiome), et par conséquent il doit aussi être pris généralement dans la majeure (par la deuxième règle).

\begin{center}{\bfseries\scshape 3. Corollaire}\end{center}

	La majeure d'un argument, dont la conclusion est négative, ne peut jamais être une particulière affirmative, car le sujet et l'attribut d'une proposition affirmative sont tous deux pris particulièrement (par le deuxième et le troisième axiome) : et ainsi le grand terme n'y serait pris que particulièrement contre le deuxième corollaire.

\bigbreak

\begin{center}{\bfseries\scshape 4. Corollaire}\end{center}

	Le petit terme est toujours dans la conclusion comme dans les prémisses, c'est-à-dire que, comme il ne peut être que particulier dans la conclusion quand il est particulier dans les prémisses, il peut au contraire être toujours général dans la conclusion quand il l'est dans les prémisses.
	Car le petit terme ne saurait être général dans la mineure lorsqu'il en est le sujet, qu'il ne soit généralement uni ou défini avec le moyen. Et il n'en peut être l'attribut et y être pris généralement que la proposition ne soit négative, parce que l'attribut d'une proposition affirmative est toujours pris particulièrement. Or les propositions négatives marquent que l'attribut pris selon toute son étendue, est défini d'avec le sujet.

	Et par conséquent une proposition où le petit terme est général, marque ou une union du moyen avec tout ce petit terme, ou une désunion du moyen d'avec tout le petit terme.

	Or si par cette union du moyen avec le petit terme on conclut qu'une autre idée est jointe avec ce petit terme, on doit conclure qu'elle est jointe à tout le petit terme, et non seulement à une partie. Car le moyen étant joint à tout le petit terme ne peut prouver rien par cette union d'une partie qu'il ne le prouve aussi des autres, puisqu'il est joint à toutes.

	De même si la désunion du moyen d'avec le petit terme prouve quelque chose de quelque partie du petit terme, elle le prouve de toutes les parties, puisqu'il est également désuni de toutes les parties.

	Car toute conclusion doit être en vertu dans les prémisses. Or c'est proprement du petit terme que l'on conclut que de telle chose lui convient, ou ne lui convient pas. Et ainsi on le doit toujours prendre dans le même sens, et lui donner toujours la même étendue.

\begin{center}{\bfseries\scshape 5. Corollaire}\end{center}

	Lorsque la mineure est une négative universelle, si on en peut tirer une conclusion légitime, elle peut être toujours générale. C'est une suite du précédent corollaire ; car le petit terme ne saurait manquer d'être pris généralement dans la mineure, lorsqu'elle est négative universelle, soit qu'il en soit le sujet (par le deuxième axiome), soit qu'il en soit l'attribut (par le quatrième axiome).

\begin{center}{\bfseries\scshape\large 3. Règle}\end{center}

	\emph{On ne peut rien conclure de deux propositions négatives.}

Car deux propositions négatives séparent le sujet du moyen, et l'attribut du même moyen ; or, de ce que deux choses sont séparées de la même chose, il ne s'ensuit, ni qu'elles soient, ni qu'elles ne soient pas la même chose. De ce que les Espagnols ne sont pas Turcs, et de ce que les Turcs ne sont pas chrétiens, il ne s'ensuit pas que les Espagnols ne soient pas chrétiens, et il ne s'ensuit pas aussi que les Chinois le soient, quoiqu'ils ne soient pas plus Turcs que les Espagnols.

\begin{center}{\bfseries\scshape\large 4. Règle}\end{center}

	\emph{On ne peut prouver une proposition négative par deux propositions affirmatives.}

Car de ce que les deux termes de la conclusion sont unis avec un troisième, on ne peut pas prouver qu'ils soient désunis entre eux.

\begin{center}{\bfseries\scshape\large 5. Règle}\end{center}

	\emph{La conclusion suit toujours la plus faible partie, c'est-à-dire que, s'il y a une des deux propositions qui soit négative, elle doit être négative, et s'il y en a une particulière, elle doit être particulière.}

La preuve en est que, s'il y a une proposition négative, le moyen est désuni de l'une des parties de la conclusion, et ainsi il est incapable de les unir, ce qui est nécessaire pour conclure affirmativement.

Et s'il y a une proposition particulière, la conclusion n'en peut être générale; car si la conclusion est générale et affirmative, le sujet étant universel, il doit aussi être universel dans la mineure, et par conséquent il en doit être le sujet, l'attribut n'étant jamais pris généralement dans les propositions affirmatives : donc le moyen, joint à ce sujet, sera particulier dans la mineure : donc il sera général dans la majeure, parce qu'autrement, il serait deux fois particulier: donc il en sera le sujet, et le terme ne saurait être général dans la mineure, lorsqu'il en est le sujet, qu'il ne le soit généralement, et par conséquent cette majeure sera aussi universelle; et ainsi il ne peut y avoir de proposition particulière dans un argument affirmatif dont la conclusion est générale.

Cela est encore plus clair dans les conclusions universelles négatives ; car de là il s'ensuit qu'il doit y avoir trois termes universels dans les deux prémisses, suivant le premier corollaire ; or, comme il doit y avoir une proposition affirmative, par la troisième règle, dont l'attribut est pris particulièrement, il s'ensuit que tous les autres trois termes sont pris universellement, et par conséquent les deux sujets des deux propositions, ce qui les rend universelles : ce qu'il fallait démontrer.

\begin{center}{\bfseries\scshape 6. Corollaire}\end{center}

\emph{Ce qui conclut le général, conclut le particulier}. Ce qui conclut A conclut I, ce qui conclut E conclut O. Mais ce qui conclut le particulier ne conclut pas pour cela le général : c'est une suite de la règle précédente et du premier axiome ; mais il faut remarquer qu'il a plu aux hommes de ne considérer les espèces d'un syllogisme que selon sa plus noble conclusion, qui est la générale: de sorte qu'on ne compte point pour une espèce particulière de syllogisme celui où on ne conclut le particulier que parce qu'on en peut aussi conclure le général.

C'est pourquoi il n'y a point de syllogisme où la majeure étant A et la mineure E, la conclusion soit O. Car O (par le cinquième corollaire) la conclusion d'une mineure universelle négative peut toujours être générale. De sorte que si on ne peut pas la tirer générale, ce sera parce qu'on n'en pourra tirer aucune. Ainsi A, E, O, n'est jamais un syllogisme à part, mais seulement en tant qu'il peut être enfermé dans A, E, E.

\begin{center}{\bfseries\scshape\large 6. Règle}\end{center}

	\emph{De deux propositions particulières il ne s'ensuit rien.}

Car si elles sont toutes deux affirmatives, le moyen y sera pris deux fois particulièrement, soit qu'il soit sujet (par le deuxième axiome), soit qu'il soit attribut (par le troisième axiome). Or, par la première règle, on ne conclut rien par un syllogisme dont le moyen est pris deux fois particulièrement.

Et, s'il y en avait une négative, la conclusion l'étant aussi (par la règle précédente), il doit y avoir au moins deux termes universels dans les prémisses (suivant le deuxième corollaire) ; donc il doit y avoir une proposition universelle dans ces deux prémisses, étant impossible de disposer trois termes en deux propositions où il doit y avoir deux termes pris universellement, en sorte que l'on ne fasse ou deux attributs négatifs, ce qui serait contre la troisième règle, ou quelqu'un des sujets universels, ce qui fait la proposition universelle.


