\subsubsection{\centering \Large CHAPITRE XVIII}
\addcontentsline{toc}{section}{\protect\numberline{}{\scshape\bfseries XVIII} - \emph{Des mauvais raisonnements que l'on commet dans la vie civile}}
\begin{center}\emph{\large\scshape Des mauvais raisonnements que l'on commet dans la vie civile.}\end{center}

	\lettrine{V}{oilà} quelques exemples de fautes les plus ordinaires que l'on commet en raisonnant. Il serait à souhaiter qu'on fit autant d'attention pour les remarquer dans les choses qui regardent les mœurs et la conduite de la vie, que l'on en fait pour les découvrir dans les matières de science, puisque d'une part l'on y faut encore plus de mauvais raisonnements; et que de l'autre ils y sont encore bien plus dangereux, n'étant pas seulement des erreurs, mais souvent aussi des fautes très importantes.

Ce serait sans doute une étude non seulement très utile, mais aussi très agréable, que de considérer en détail ce qui engage les hommes dans tous les faux jugements qu'ils forment dans les matières des mœurs. Mais parce que ce sujet demanderait un ouvrage à part plus considérable que celui-ci, on se contentera de remarquer ici diverses manières de mal raisonner qui sont communes dans la vie des hommes, dont chacun ensuite pourra trouver une infinité d'exemples particuliers, pour peu d'attention qu'il y veuille faire.

\bigbreak
{\bfseries\scshape I.} Un des défauts ordinaires des hommes, est de juger témérairement des actions et des intentions des autres: et l'on n'y tombe guère que par un mauvais raisonnement, par lequel on attribue précisément un effet à une cause, qui peut avoir été produit par plusieurs autres.

Un homme de lettres se trouve de même sentiment qu'un hérétique sur une matière de critique indépendantes des controverses de la religion. Un adversaire malicieux en conclura qu'il a de l'inclination pour les hérétiques, mais il le conclura témérairement et malicieusement.

Un écrivain parlera avec quelque force contre une opinion qu'il croit dangereuse. On l'accusera sur cela de haine et d'animosité contre les auteurs qui l'ont avancée, mais ce sera injustement et témérairement : cette force pouvant naître de zèle pour la vérité, aussi bien que de haine contre les personnes.

Un homme est ami d'un méchant; donc, conclut-on, il est lié d'intérêt avec lui, et il est participant de ses crimes. Cela ne s'ensuit pas; peut-être les a-t-il ignorés, et peut-être n'y a-t-il point pris part.

On manque de rendre quelque civilité à ceux à qui on en doit. C'est, dit-on, un orgueilleux et un insolent : mais ce n'est peut-être qu'une inadvertance, ou un simple oubli.

Des personnes ne sont pas de l'opinion d'un autre. Il en conclut, que ce sont des opiniâtres; qu'ils trahissent leur conscience, qu'ils sont lâches, intéressés, vains, présomptueux. Tous ces jugements sont manifestement injustes; car peut-être ces personnes ont raison de n'être pas de ce sentiment. Parce qu'il est faux; ou s'il vrai, ce n'est peut-être qu'un simple défaut de lumière, et non aucun mauvais motif qui les empêche de l'embrasser.

Toutes ces choses extérieures ne sont que des signes équivoques, c'est-à-dire, qui peuvent signifier plusieurs choses; et c'est juger témérairement que de déterminer ce signe à une chose particulière. Le silence est quelque fois signe de modestie et de jugement, et quelque fois de bêtise. La lenteur marque quelque fois la prudence, et quelque fois la pesanteur de l'esprit. Le changement est quelque fois signe d'inconstance, et quelque fois de sincérité. Ainsi c'est mal raisonner, que de conclure qu'un homme est inconstant; de cela seul, qu'il a changé de sentiment : car il peut avoir eu raison d'en changer.

\bigbreak
{\bfseries\scshape II.} C'est encore une faute ordinaire dans les disputes des hommes, que de se faire réciproquement de certains reproches communs, et qui n'étant vrais que d'un côté, se peuvent néanmoins alléguer de part et d'autre avant la décision du fond.

Il n'y a presque point de plaideurs qui ne se reprochent mutuellement d'être des chicaneurs; d'allonger les procès, de couvrir la vérité par des adresses artificieuses. On ne voit autre chose que des gens qui s'entr'accusent d'opiniâtreté, de témérité, d'aveuglement, de manquer de sens commun. Voyez un peu, disent ceux-ci, en quelles absurdités l'on tombe quand on s'éloigne de la vérité. Les autres en disent autant de leur côté. Et ce n'est rien avancer, mais perdre des paroles; c'est pourquoi il n'y a rien de plus judicieux, ni de plus sage que ce sentiment de saint Augustin : \emph{Omittamu ista communia que dici ex utraque parte possunt, licet veredici ex utraque parte non possint}. Il est visible que ce défaut se peut réduire à la pétition de principe, puisque l'on y suppose ce que l'on doit prouver.

\bigbreak
{\bfseries\scshape III.} Les fausses inductions par lesquelles on tire des propositions générales de quelques expériences particulières, sont une des plus communes sources des faux raisonnements des hommes. Il ne leur faut que trois ou quatre exemples pour en former une maxime et un lieu commun, et pour s'en servir ensuite de principe pour décider toutes choses.

Il y a beaucoup de maladies cachées aux plus habiles médecins, et souvent les remèdes ne réussissent pas; des esprits excessifs en concluent, que la médecine est absolument inutile, et que c'est un métier de charlatans.

Il y a des femmes légères et déréglées : cela suffit à des jaloux pour concevoir des soupçons injustes contre les plus honnêtes; et à des écrivains licencieux, pour les condamner toutes généralement.

Il y a souvent des personnes qui cachent de grands vices sous une apparence de piété; des libertins en concluent que toute la dévotion n'est qu'hypocrisie.

Il y a des choses obscures et cachées, et l'on se trompe quelque fois grossièrement. Toutes choses sont obscures et incertaines, disent les anciens et nouveaux Pyrrhoniens, et nous ne pouvons connaître la vérité d'aucune avec certitude.

Il y a de l'inégalité dans quelques actions des hommes : cela suffit pour en faire un lieu commun dont personne ne soit excepté. Ce n'est qu'inconstance, disent-ils, que légèreté, qu'instabilité, que la conduite des hommes, même les plus sages. \emph{Nous ne pensons ce que nous voulons, qu'à l'instant que nous le voulons; nous ne voulons rien librement, rien absolument, rien constamment}. Il y a des philosophes peu religieux. Tous les philosophes sont des libertins, et des impies, disent quelques personnes téméraires et indiscrètes.

La plupart du monde ne saurait présenter les défauts ou les bonnes qualités des autres, que par des propositions générales et excessives. De quelques actions particulières on en conclut l'habitude ; de trois ou quatre fautes on en fait une coutume; ce qui arrive une fois le mois, une fois l'an arrive tous les jours, à toutes heures, à tout moment dans les discours des hommes : tant ils ont peu de soin de garder dans leurs paroles les bornes de la vérité et de la justice.

\bigbreak
{\bfseries\scshape IV.} C'est une faiblesse et une injustice que l'on condamne souvent et que l'on évite peu, de juger des conseils par les événements, et de rendre coupables ceux qui ont pris une résolution prudente selon les circonstances qu'ils pouvaient voir, de toutes les mauvaises suites qui en sont arrivées, ou par un simple hasard, ou par la malice de ceux qui l'ont traversée, ou par quelques autres rencontres qu'il ne leur était pas possible de prévoir. Non seulement les hommes aiment autant être heureux que sages, mais ils ne font pas de différence entre heureux et sages, ni entre malheureux et coupables. Cette distinction leur paraît trop subtile. On est ingénieux pour trouver les fautes que l'on s'imagine avoir attiré les mauvais succès ; et comme les astrologues, lorsqu'ils savent un certain accident, ne manquent jamais de trouver l'aspect des astres qui l'a produit, on ne manque aussi jamais de trouver, après les disgrâces et les malheurs, que ceux qui y sont tombés les ont mérités par quelque imprudence. Il n'a pas réussi, il a donc tort. C'est ainsi que l'on raisonne dans le monde, et qu'on y a toujours raisonné, parce qu'il y a toujours eu peu d'équité dans les jugements des hommes.

\bigbreak
{\bfseries\scshape V.} La passion est plutôt une source générale de mauvais raisonnements, qu'une manière particulière de mal raisonner; néanmoins on y peut rapporter certains faux jugements qui ne semblent être tirés que de la passion même, sans l'entremise d'aucune autre erreur. Combien voit-on de personnes qui ne peuvent plus reconnaître aucune bonne qualité, ni naturelle, ni acquise, dans ceux contre qui ils ont conçu de l'aversion, ou qui ont été contraires en quelque chose à leurs sentiments, à leurs désirs, à leurs intérêts ? Cela suffit pour devenir tout d'un coup à leur égard téméraire, orgueilleux, ignorant, sans foi, sans honneur, sans conscience. Leurs affections et leurs désirs ne sont pas plus justes ni plus modérés que leur haine. S'ils aiment quelqu'un, il est exempt de toute sorte de défaut ; tout ce qu'ils désirent est juste et facile, tout ce qu'ils ne désirent pas est injuste et impossible, sans qu'ils puissent alléguer aucune raison de tous ces jugements, que la passion même qui les possède : de sorte qu'encore qu'ils ne fassent pas dans leur esprit ce raisonnement formel : je l'aime ; donc c'est le plus habile homme du monde : je le hais ; donc c'est un homme de néant, ils le font en quelque sorte dans leur cœur ; et c'est pourquoi on peut appeler ces sortes d'égarement des sophismes et des illusions du cœur, qui consistent à transporter nos passions dans les objets de nos passions, et à juger qu'ils sont ce que nous voulons ou désirons qu'ils soient : ce qui est sans doute très déraisonnable, puisque nos désirs ne changent rien dans l'être de ce qui est hors de nous, et qu'il n'y a que Dieu, dont la volonté soit tellement efficace, que les choses sont tout ce qu'il veut qu'elles soient.

On voit assez combien ce défaut est ridicule; et néanmoins il n'y a rien de plus commun. On croit le faux parce qu'on le veut croire; ce qui nous plaît nous paraît vrai; et nos intérêts, et nos passions donnent ordinairement le plus grand branle à nos jugements: c'est le poids qui emporte la balance, et qui nous détermine dans la plupart de nos doutes. Il n'en faut point d'autres preuves que ce que nous voyons tous les jours, que de choses tenues partout ailleurs pour douteuses, ou même pour fausses, sont tenues pour très certaines par tous ceux d'une nation, ou d'une profession ou d'un institut. Car n'étant pas possible que ce qui est vrai en Espagne soit faux en France, ni que l'esprit de tous les Espagnols soit tellement autrement tourné que celui de tous les Français, qu'à ne juger des choses que par les règles de la raison, ce qui paraît vrai généralement aux uns, paraisse faux généralement aux autres; il est visible que cette diversité de jugement ne peut venir d'autre cause, sinon qu'il plaît aux uns de tenir pour vrai ce qui leur est avantageux, et que les autres n'y ayant point d'intérêt en jugent d'une autre sorte.

Cependant qu'y a-t-il de moins raisonnable que de prendre notre intérêt pour motif de croire une chose ? Tout ce qu'il peut faire au plus, est de nous porter à considérer avec plus d'attention les raisons qui peuvent nous faire découvrir la vérité de ce que nous désirons être vrai : mais il n'y a que cette vérité, qui doit se trouver dans la chose même indépendamment de nos désirs, qui doive nous persuader. Je suis d'un tel pays ; donc je dois croire qu'un tel saint y a prêché l'Évangile. Je suis d'un tel ordre ; donc je crois qu'un tel privilège est véritable. Ce ne sont pas là des raisons. De quelque ordre et de quelque pays que vous soyez, vous ne devez croire que ce qui est vrai, et que ce que vous seriez disposé à croire si vous étiez d'un autre pays, d'un autre ordre, d'une autre profession.

On peut rapporter à la même illusion celle de ceux qui décident tout par un principe fort général, qui est qu'ils ont raison, qu'ils connaissent la vérité; d'où il ne leur est pas difficile de conclure que les autres qui ne sont pas de leur sentiment ont tort : en effet la conclusion est nécessaire. Mais le défaut de ces personnes n'est pas de croire qu'ils ont raison, puisque c'est une chose commune à tous ceux qui sont persuadés de quelque chose; mais de s'en servir de principe à l'égard des autres qui ne sont d'un autre sentiment qu'eux, que parce qu'ils sont persuadés qu'ils n'ont pas raison.

Il y en a de même qui n'ont point d'autre fondement, pour rejeter certaines opinions, que ce plaisant raisonnement : Si cela était, je ne serais pas un habile homme : or, je suis un habile homme ; donc cela n'est pas. C'est la principale raison qui a fait rejeter longtemps certains remèdes très utiles et des expériences très certaines ; parce que ceux qui ne s'en étaient point encore avisés concevaient qu'ils se seraient donc trompés jusqu'alors. Quoi! Si le sang, disaient-ils, avait une révolution circulaire dans le corps ; si l'aliment ne se portait pas au foie par les veines mézarraïques ; si l'artère veineuse portait le sang au cœur ; si le sang montait par la veine cave descendante ; si la nature n'avait point d'horreur du vide ; si l'air était pesant et avait un mouvement en bas, j'aurais ignoré des choses importantes dans l'anatomie et dans la physique : il faut donc que cela ne soit pas. Mais pour les guérir de cette fantaisie, il ne faut que leur bien représenter que c'est un très petit inconvénient qu'un homme se trompe, et qu'ils ne laisseront pas d'être habiles en d'autres choses, quoiqu'ils ne l'aient pas été en celles qui auraient été nouvellement découvertes.

\bigbreak
{\bfseries\scshape VI.} Mais il n'y a point de faux raisonnements plus fréquents parmi les hommes, que ceux où l'on tombe, ou en jugeant témérairement de la vérité des choses par une autorité qui n'est pas suffisante pour nous en assurer, ou en décidant le fond par la manière. Nous appellerons l'un le sophisme de l'autorité, et l'autre le sophisme de la manière.

Pour comprendre combien ils sont ordinaires, il ne faut que considérer que la plupart des hommes ne se déterminent point à croire un sentiment plutôt qu'un autre, par des raisons solides et essentielles qui en feraient connaître la vérité, mais par certaines marques extérieures et étrangères qui sont plus convenables, ou qu'ils jugent plus convenables à la vérité qu'à la fausseté.

La raison en est que la vérité intérieure des choses est souvent assez cachée ; que les esprits des hommes sont ordinairement faibles et obscurs, pleins de nuages et de faux jours, au lieu que ces marques extérieures sont claires et sensibles : de sorte que, comme les hommes se portent aisément à ce qui leur est le plus facile, ils se rangent presque toujours du côté où ils voient ces marques extérieures qu'ils discernent facilement.

Elles peuvent se réduire à deux principales : l'autorité de celui qui propose la chose, et la manière dont elle est proposée ; et ces deux voies de persuader sont si puissantes qu'elles emportent presque tous les esprits.

Ainsi Dieu, qui voulait que la connaissance certaine des mystères de la foi pût s'acquérir par les plus simples d'entre les fidèles, a eu la bonté de s'accommoder à cette faiblesse de l'esprit des hommes, en ne la faisant pas dépendre d'un examen particulier de tous les points qui nous sont proposés à croire ; mais en nous donnant pour règle certaine de la vérité l'autorité de l'Église universelle qui nous les propose, qui, étant claire et évidente, retire les esprits de tous les embarras où les engageraient nécessairement les discussions particulières de ces mystères.

Ainsi, dans les choses de la foi, l'autorité de l'Église universelle est entièrement décisive; et tant s'en faut qu'elle puisse être un sujet d'erreur, qu'on ne tombe dans l'erreur qu'en s'écartant de son autorité, et en refusant de s'y soumettre.

On tire aussi dans les matières de religion des arguments convaincants, de la manière dont elles sont proposées. Quand on a vu, par exemple, en divers siècles de l'Église, et principalement dans le dernier, des hommes qui tâchaient de planter leurs opinions par le fer et par le sang; quand on les a vus armés contre l'Église par le schisme, contre les puissances temporelles par la révolte; quand on a vu des gens sans mission ordinaire, sans miracles, sans aucunes marques extérieures de piété, et plutôt avec des marques sensibles de dérèglement, entreprendre de changer la foi et la discipline de l'Église, une manière si criminelle étant plus que suffisante pour les faire rejeter par toutes les personnes raisonnables, et pour empêcher les plus grossières de les écouter.

Mais dans les choses dont la connaissance n'est pas absolument nécessaire, et que Dieu a laissées davantage au discernement de la raison de chacun en particulier, l'autorité et la manière ne sont pas si considérables, et elles servent souvent à engager plusieurs personnes à des jugements contraires à la vérité.

On n'entreprend pas ici de donner des règles et des bornes précises de la déférence qu'on doit à l'autorité dans les choses humaines, mais de marquer seulement quelques fautes grossières que l'on commet en cette matière.

Souvent on ne regarde que le nombre des témoins, sans considérer si ce nombre fait qu'il soit plus probable qu'on ait rencontré la vérité, ce qui n'est pas raisonnable. Car, comme un auteur de ce temps a judicieusement remarqué, dans les choses difficiles et qu'il faut que chacun trouve par soi-même, il est plus vraisemblable qu'un seul trouve la vérité, que non pas qu'elle soit découverte par plusieurs. Ainsi ce n'est pas une bonne conséquence ; cette opinion est suivie du plus grand nombre des philosophes, donc elle est la plus vraie.

Souvent on se persuade par certaines qualités qui n'ont aucune liaison avec la vérité des choses dont il s'agit. Ainsi, il y a quantité de gens qui croient, sans autre examen, ceux qui sont les plus âgés, et qui ont plus d'expérience dans les choses mêmes qui ne dépendent ni de l'âge ni de l'expérience, mais de la lumière de l'esprit.

La piété, la sagesse, la modération sont sans doute les qualités les plus estimables qui soient au monde, et elles doivent donner beaucoup d'autorité aux personnes qui les possèdent, dans les choses qui dépendent de la piété, de la sincérité, et même d'une lumière de Dieu, qu'il est plus probable que Dieu communique davantage à ceux qui le servent plus purement; mais il y a une infinité de choses qui ne dépendent que d'une lumière humaine, d'une expérience humaine, d'une pénétration humaine, et dans ces choses, ceux qui ont l'avantage de l'esprit et de l'étude méritent plus de créance que les autres. Cependant il arrive souvent le contraire, et plusieurs estiment qu'il est plus sûr de suivre dans ces choses mêmes le sentiment des plus gens de bien.

Cela vient en partie de ce que ces avantages d'esprit ne sont pas si sensibles que le règlement extérieur qui paraît dans les personnes de piété, et en partie aussi de ce que les hommes n'aiment point à faire des distinctions ; le discernement les embarrasse ; ils veulent tout ou rien. S'ils ont créance à une personne pour quelque chose, ils la croient en tout ; s'ils n'en ont point pour une autre, ils ne la croient en rien ; ils aiment les voies courtes, décisives et abrégées; mais cette humeur, quoique ordinaire, ne laisse pas d'être contraire à la raison qui nous fait voir que les mêmes personnes ne sont pas croyables en tout, parce qu'elles ne sont pas éminentes en tout, et que c'est mal raisonner que de conclure : C'est un homme grave ; donc il est intelligent et habile en toutes choses.

Il est vrai que s'il y a des erreurs pardonnables, ce sont celles où l'on s'engage en déférant plus qu'il ne faut au sentiment de ceux qu'on estime gens de bien ; mais il y a une illusion beaucoup plus absurde en soi, et qui est néanmoins très ordinaire, qui est de croire qu'un homme dit vrai, parce qu'il est de condition, qu'il est riche ou élevé en dignité.

Ce n'est pas que personne fasse expressément ces sortes de raisonnements : Il a cent mille livres de rente, donc il a raison ; il est de grande naissance, donc on doit croire ce qu'il avance comme véritable; c'est un homme qui n'a point de bien, il a donc tort : néanmoins il se passe quelque chose de semblable dans l'esprit de la plupart des hommes, et qui emporte leur jugement sans qu'ils y pensent.

Qu'une même chose soit proposée par une personne de qualité, ou par un homme de néant, on l'approuvera souvent dans la bouche de cette personne de qualité, lorsqu'on ne daignera pas même l'écouter dans celle d'un homme de basse condition. L'Écriture a voulu nous instruire de cette humeur des hommes, en la présentant parfaitement dans le livre de l'Ecclésiastique : Si le riche parle, dit-elle, tout le monde se tait, et on élève ses paroles jusqu'aux nues; si le pauvre parle, on demande qui est celui-là? \emph{Diues locutus est et omnes tacuerunt, et verbum illius usque ad nubes perducent : pauper locutus est, et dicunt quis est hic?} Lorsque le riche, dit encore l'Écriture en ce même lieu, avance des choses folles et arrogantes, on l'approuve et on le justifie; et quand le pauvre parle sagement, on ne veut pas même l'écouter. \emph{Diues locutus est superba, et iustificaverunt illum : humilis locutus est sensate, et non est datus ei locus}.

Il est certain que la complaisance et la flatterie ont beaucoup de part dans l'approbation que l'on donne aux actions et aux paroles des personnes de condition, et qu'ils l'attirent souvent aussi par une certaine grâce extérieure et par une manière d'agir noble, libre et naturelle, qui leur est quelquefois si particulière qu'elle est presque inimitable à ceux qui sont de basse naissance ; mais il est certain aussi qu'il y en a plusieurs qui approuvent tout ce que font et disent les grands, par un abaissement intérieur de leur esprit, qui plie sous le fait de la grandeur, et qui n'a pas la vue assez ferme pour en soutenir l'éclat ; et que cette pompe extérieure qui les environne en impose toujours un peu, et fait quelque impression sur les âmes les plus fortes.

La raison de cette tromperie vient de la corruption du cœur des hommes, qui, ayant une passion ardente pour l'honneur et les plaisirs, conçoivent nécessairement beaucoup d'amour pour les richesses et les autres qualités par le moyen desquelles on obtient ces honneurs et ces plaisirs. Or, l'amour que l'on a pour toutes ces choses que le monde estime fait que l'on juge heureux ceux qui les possèdent ; et en les jugeant heureux, on les place au-dessus de soi, et on les regarde comme des personnes éminentes et élevées. Cette accoutumance de les regarder avec estime passe insensiblement de leur fortune à leur esprit. Les hommes ne font pas d'ordinaire les choses à demi. On leur donne donc une âme aussi élevée que leur rang, on se soumet à leurs opinions, et c'est la raison de la créance qu'ils trouvent ordinairement dans les affaires qu'ils traitent.

Mais cette illusion est encore bien plus forte dans les grands mêmes, qui n'ont pas eu soin de corriger l'impression que leur fortune fait naturellement dans leur esprit, qu'elle n'est dans ceux qui leur sont inférieurs. Il y en a peu qui ne fassent une raison de leur condition et de leurs richesses, et qui ne prétendent que leurs sentiments doivent prévaloir sur celui de ceux qui sont au-dessous d'eux. Ils ne peuvent souffrir que ces gens qu'ils regardent avec mépris prétendent avoir autant de jugement et de raison qu'eux ; et c'est ce qui les rend si impatients la moindre contradiction qu'on leur fait.

Tout cela vient encore de la même source, c'est-à-dire des fausses idées qu'ils ont de leur grandeur, de leur noblesse et de leurs richesses. Au lieu de les considérer comme des choses entièrement étrangères à leur être, qui n'empêchent pas qu'ils ne soient parfaitement égaux à tout le reste des hommes, selon l'âme et selon le corps, et qui n'empêchent pas qu'ils n'aient le jugement aussi faible et aussi capable de se tromper que celui de tous les autres, ils incorporent en quelque manière dans leur essence toutes ces qualités de grand, de noble, de riche, de maître, de seigneur, de prince ; ils en grossissent leur idée, et ne se représentent jamais à eux-mêmes sans tous leurs titres, tout leur attirail et tout leur train.

Ils s'accoutument à se regarder dès leur enfance comme une espèce séparée des autres hommes: leur imagination ne les mêle jamais dans la foule du genre humain ; ils sont toujours comtes ou ducs à leurs yeux et jamais simplement hommes. Ainsi, ils se taillent une âme et un jugement selon la mesure de leur fortune. C'est ce que l'Écriture nous a voulu marquer par ces paroles : \emph{Facultates et virtutes exaltant cor}, les biens et la puissance réhaussent l'âme à ses propres yeux, et font qu'elle s'estime plus grande.

La sottise de l'esprit humain est telle qu'il n'y a rien qui ne lui serve à grandir l'idée qu'il a de lui-même. Une belle maison, un habit magnifique; une grande barbe, font qu'il s'en croit plus habile, et, si l'on y prend garde, il s'estime davantage à cheval ou en carrosse qu'à pied. Il est facile de persuader à tout le monde qu'il n'y a rien de plus ridicule que ces jugements ; mais il est très difficile de se garantir entièrement de l'impression secrète que toutes ces choses extérieures font dans l'esprit. Tout ce qu'on peut faire est de s'accoutumer, autant qu'on le peut, à ne donner aucune autorité à toutes les qualités qui ne peuvent en rien contribuer à trouver la vérité, et de n'en donner à celles mêmes qui y contribuent qu'autant qu'elles y contribuent effectivement. L'âge, la science, l'étude, l'expérience, l'esprit, la vivacité, la retenue, l'exactitude, le travail, servent pour trouver la vérité des choses cachées, et ainsi ces qualités méritent qu'on y ait égard ; mais il faut pourtant les peser avec soin, et ensuite en faire comparaison avec les raisons contraires, car de chacune de ces choses en particulier on ne conclut rien de certain, puisqu'il y a des opinions très fausses qui ont été approuvées par des personnes de fort bon esprit et qui avaient une grande partie de ces qualités.

\bigbreak
{\bfseries\scshape VII.} Il y a encore quelque chose de plus trompeur dans les surprises qui naissent de la manière. Car on est porté naturellement à croire qu'un homme a raison lorsqu'il parle avec grâce, avec facilité, avec gravité, avec modération et avec douceur; et à croire, au contraire, qu'un homme a tort, lorsqu'il parle désagréablement, ou qu'il fait paraître de l'emportement, de l'aigreur, de la présomption dans ses actions et dans ses paroles : étant difficile de reconnaître une vérité qui est environnée de tant de marque de mensonge, et de rejeter un mensonge qui est couvert des couleurs naturelles de la vérité.

Cependant, si l'on ne juge du fond des choses que par ces manières extérieures et sensibles, il est impossible qu'on n'y soit souvent trompé. Car il y a des gens qui débitent gravement et modestement des sottises ; et d'autres, au contraire, qui, étant d'un naturel prompt, ou qui, étant même possédés de quelque passion qui paraît dans leur visage et dans leurs paroles, ne laissent pas d'avoir la vérité de leur côté. Il y a des esprits fort médiocres et très superficiels, qui, pour avoir été nourris à la cour, où l'on étudie et où l'on pratique mieux l'art de plaire que partout ailleurs, ont des manières fort agréables, sous lesquelles ils font passer beaucoup de faux jugements ; il y en a d'autres, au contraire, qui, n'ayant aucun extérieur, ne laissent pas d'avoir l'esprit grand et solide dans le fond. Il y en a qui parlent mieux qu'ils ne pensent, et d'autres qui pensent mieux qu'ils ne parlent. Ainsi, la raison veut que ceux qui en sont capables n'en jugent point par ces choses extérieures, et qu'ils ne laissent pas de se rendre à la vérité, non seulement lorsqu'elle est proposée avec ces manières choquantes et désagréables, mais lors même qu'elle est mêlée avec quantité de faussetés : car une même personne peut dire vrai en une chose et faux dans une autre, avoir raison en ce point et tort en celui-là.

Il faut donc considérer chaque chose séparément, c'est-à-dire qu'il faut juger de la manière par la manière, et du fond par le fond, et non du fond par la manière, ni de la manière par le fond. Une personne a tort de parler avec colère, et elle a raison de dire vrai; et, au contraire, une autre a raison de parler sagement et civilement, et elle a tort d'avancer des faussetés.

Mais, comme il est raisonnable d'être sur ses gardes, pour ne pas conclure qu'une chose est vraie ou fausse, parce qu'elle est proposée de telle ou telle façon, il est juste aussi que ceux qui désirent persuader les autres de quelque vérité qu'ils ont reconnue s'étudient à la revêtir des manières favorables qui sont propres à la faire approuver, et à éviter les manières odieuses qui ne sont capables que d'en éloigner les hommes.

Ils doivent se souvenir que, quand il s'agit d'entrer dans l'esprit du monde, c'est peu de chose que d'avoir raison ; et que c'est un grand mal de n'avoir que raison, et de n'avoir pas ce qui est nécessaire pour faire goûter la raison.

S'ils honorent sérieusement la vérité, ils ne doivent pas la déshonorer, en la couvrant des marques de la fausseté et du mensonge; et, s'ils l'aiment sincèrement, ils ne doivent pas attirer sur elle la haine et l'aversion des hommes par la manière choquante dont ils la proposent. C'est le plus grand précepte de la rhétorique, qui est d'autant plus utile, qu'il sert à régler l'âme aussi bien que les paroles; car, encore que ce soient deux choses différentes d'avoir tort dans la manière et d'avoir tort dans le fond, néanmoins les fautes de la manière sont souvent plus grandes et plus considérables que celles du fond.

En effet, toutes ces manières fières, présomptueuses, aigres, opiniâtres, emportées, viennent toujours de quelque dérèglement d'esprit, qui est souvent plus considérable que le défaut d'intelligence et de lumière que l'on reprend dans les autres ; et même il est toujours injuste de vouloir persuader les hommes de cette sorte : car il est bien juste que l'on se rende à la vérité, quand on la connaît; mais il est injuste qu'on exige des autres qu'ils tiennent pour vrai tout ce que l'on croit, et qu'ils défèrent à notre seule autorité; et c'est néanmoins ce que l'on fait en proposant la vérité avec ces manières choquantes : car l'air du discours entre ordinairement dans l'esprit avec les raisons, l'esprit étant plus prompt pour apercevoir cet air, qu'il ne l'est pour comprendre la solidité des preuves, qui souvent ne se comprennent point du tout. Or, l'air du discours étant ainsi séparé des preuves, ne marque que l'autorité que celui qui parle s'attribue ; de sorte que s'il est aigre et impérieux, il rebute nécessairement l'esprit des autres, parce qu'il paraît qu'on veut emporter par autorité, et par une espèce de tyrannie, ce qu'on ne doit obtenir que par la persuasion et par la raison.

Cette injustice est encore plus grande, s'il arrive qu'on emploie ces manières choquantes pour combattre des opinions communes et reçues; car la raison d'un particulier peut bien être préférée à celle de plusieurs, lorsqu'elle est plus vraie : mais un particulier ne doit jamais prétendre que son autorité doive prévaloir à celle de tous les autres.

Ainsi, non seulement la modestie et la prudence, mais la justice même oblige de prendre un air rabaissé quand on combat des opinions communes, ou une autorité affermie, parce qu'autrement on ne peut éviter cette injustice, d'opposer l'autorité d'un particulier à une autorité, ou publique, ou plus grande et plus établie. On ne peut témoigner trop de modération, quand il s'agit de troubler la possession d'une opinion reçue, ou d'une créance acquise depuis longtemps. Ce qui est si vrai, que saint Augustin l'étend même aux vérités de la religion, ayant donné cette excellente règle à tous ceux qui sont obligés d'instruire les autres.

Voici de quelle sorte, dit-il, les catholiques sages et religieux enseignent ce qu'ils doivent enseigner aux autres. Si ce sont des choses communes et autorisées, ils les proposent d'une manière pleine d'assurance, et qui ne témoigne aucun doute, en l'accompagnant de toute la douceur qui leur est possible ; mais si ce sont des choses extraordinaires, quoiqu'ils en reconnaissent très clairement la vérité, ils les proposent plutôt comme des doutes et comme des questions à examiner, que comme des dogmes et des décisions arrêtées, pour s'accommoder en cela à la faiblesse de ceux qui les écoutent. Que si une vérité est si haute qu'elle surpasse les forces de ceux à qui l'on parle, ils aiment mieux la retenir pour quelque temps, pour leur donner lieu de croître et de s'en rendre capables, que de la leur découvrir en cet état de faiblesse, où elle ne ferait que les accabler.

\begin{center}
	\includegraphics[scale=0.155]{images/fin-de-partie.png}
\end{center}

