\subsubsection{\centering \Large CHAPITRE XIV}
\addcontentsline{toc}{section}{\protect\numberline{}{\scshape\bfseries XIV} - \emph{Des syllogismes composés, ou conjonctifs}}
\begin{center}\emph{\large\scshape Des syllogismes composés, ou conjonctifs.}\end{center}

	\lettrine{N}{ous} n'avons plus qu'à expliquer les syllogismes composés, ou conjonctifs, qui ne sont pas tous ceux dont les propositions sont conjonctives ou composées, mais ceux dont la majeure est tellement composée qu'elle enferme toute la conclusion. On les peut réduire à trois genres, \emph{les conditionnels, les disjonctifs}, et \emph{les copulatifs}.

\begin{center}\emph{\scshape\bfseries Des syllogismes conditionnels}\end{center}

Les syllogismes conditionnels sont ceux où la majeure est une proposition conditionnelle qui contient toute la conclusion, comme,

	\begin{tabularx}{\textwidth}{X}
		\emph{S'il y a un Dieu, il faut l'aimer :} \\
		\emph{Or, il y a un Dieu :} \\
		\emph{Donc il le faut aimer.} \\
	\end{tabularx}

La majeure a deux parties : la première s'appelle l'antécédent, \emph{S'il y a un Dieu}: la deuxième, le conséquent, \emph{il le faut aimer}. Ce syllogisme peut être de deux sortes, parce que de la même majeure on peut former deux conclusions.

La première est, quand ayant affirmé le conséquent dans la majeure, on affirme l'antécédent dans la mineure, selon cette règle : \emph{en posant l'antécédent, on pose le conséquent}.

	\begin{tabularx}{\textwidth}{X}
		\emph{Si la matière ne peut se mouvoir d'elle-même, il faut que le premier mouvement lui ait été donné de Dieu :} \\
		\emph{Or, la matière ne peut se mouvoir d'elle-même :} \\
		\emph{Il faut donc que le premier mouvement lui ait été donné de Dieu.} \\
	\end{tabularx}

La deuxième sorte est, quand on ôte le conséquent pour ôter l'antécédent, selon cette règle, \emph{ôtant le conséquent, on ôte l'antécédent}.

	\begin{tabularx}{\textwidth}{X}
		\emph{Si les bêtes pensent, la matière pense:}  \\
		\emph{Or la matière est incapable de penser:} \\
		\emph{Donc les bêtes ne pensent point.} \\
	\end{tabularx}

Les arguments conditionnels sont faux en deux manières : La première lorsqu'on infère l'antécédent du conséquent; comme si on disait :

	\begin{tabularx}{\textwidth}{X}
		\emph{Si les Chinois sont Mahométans, ils sont infidèles:} \\
		\emph{Or ils sont infidèles:} \\
		\emph{Donc ils sont Mahométans.} \\
	\end{tabularx}

La deuxième sorte d'arguments conditionnels qui sont faux, est quand de la négation de l'antécédent on infère la négation du conséquent, comme dans le même exemple.

	\begin{tabularx}{\textwidth}{X}
		\emph{Si les Chinois sont Mahométans, ils sont infidèles :} \\
		\emph{Or, ils ne sont pas Mahométans :} \\
		\emph{Donc ils ne sont pas infidèles.} \\
	\end{tabularx}


\begin{center}\emph{\scshape\bfseries Des syllogismes disjonctifs}\end{center}

	On appelle syllogismes disjonctifs ceux dont la première proposition est disjonctive, c'est-à-dire dont les parties sont jointes par \emph{vel, ou}; comme celui-ci de Cicéron :

	\begin{tabularx}{\textwidth}{X}
		\emph{Ceux qui ont tué César sont parricides ou défenseurs de la liberté:} \\
		\emph{Or, ils ne sont point parricides : } \\
		\emph{Donc ils sont défenseurs de la liberté.} \\
	\end{tabularx}

Il y en a de deux sortes : la première, quand on ôte une partie pour garder l'autre; comme dans celui que nous venons de proposer, ou dans celui-ci :

	\begin{tabularx}{\textwidth}{X}
		\emph{Tous les méchants doivent être punis en ce monde ou en l'autre : } \\
		\emph{Or, il y a des méchants qui ne sont point punis en ce monde :} \\
		\emph{Donc ils le seront en l'autre.} \\
	\end{tabularx}

Il y a quelquefois trois membres dans cette sorte de syllogismes, et alors on en ôte deux pour en garder un, comme dans cet argument de saint Augustin, dans son livre du Mensonge, Chapitre 8. \emph{Aut non est credendum bonis, aut credendum est eis quos credimus debere aliquando mentiri, aut non est credendum bonos aliquando mentiri. Horum primum perniciosum est; secundum stultum. Restat ergo ut nunquam mentiantur boni}.

La seconde sorte, mais moins naturelle, est quand on prend une des parties pour ôter l'autre, comme si l'on disait :

	\begin{tabularx}{\textwidth}{X}
		\emph{Saint Bernard, témoignant que Dieu avait confirmé, par des miracles, sa prédication de la Croisade, était un saint ou un imposteur :} \\
		\emph{Or, c'était un saint :} \\
		\emph{Donc ce n'était pas un imposteur.} \\
	\end{tabularx}

Ces syllogismes disjonctifs ne sont guère faux que par la fausseté de la majeure, dans laquelle la division n'est pas exacte, se trouvant un milieu entre les membres opposés, comme si je disais :

	\begin{tabularx}{\textwidth}{X}
		\emph{Il faut obéir aux princes en ce qu'ils commandent contre la loi de Dieu, ou se révolter contre eux :} \\
		\emph{Or, il ne faut pas leur obéir en ce qui est contre la loi de Dieu :} \\
		\emph{Donc il faut se révolter contre eux :} \\
		{\space\space\space\space\space ou:} \\
		\emph{Or, il ne faut pas se révolter contre eux :} \\
		\emph{Donc il faut leur obéir en ce qui est contre la loi de Dieu.} \\
	\end{tabularx}

L'un et l'autre raisonnement est faux, parce qu'il y a un milieu dans cette disjonction qui a été observé par les premiers chrétiens, qui est de souffrir patiemment toutes choses, plutôt que de rien faire contre la loi de Dieu, sans néanmoins se révolter contre les princes.

Ces fausses disjonctions sont une des sources les plus communes des faux raisonnements des hommes.

\begin{center}\emph{\scshape\bfseries Des syllogismes copulatifs}\end{center}

Ces syllogismes ne sont que d'une sorte, qui est quand on prend une proposition copulative niante, dont ensuite on établit une partie pour ôter l'autre.

	\begin{tabularx}{\textwidth}{X}
		\emph{Un homme n'est pas tout ensemble serviteur de Dieu, et idolâtre de son argent:} \\
		\emph{Or, l'avare est idolâtre de son argent :} \\
		\emph{Donc il n'est pas serviteur de Dieu.} \\
	\end{tabularx}

Car cette sorte de syllogisme ne conclut point nécessairement, quand on ôte une partie pour mettre l'autre, comme on peut voir par ce raisonnement tiré de la même proposition :

	\begin{tabularx}{\textwidth}{X}
		\emph{Un homme n'est pas tout ensemble serviteur de Dieu, et idolâtre de l'argent :} \\
		\emph{Or les prodigues ne sont point idolâtres de l'argent :} \\
		\emph{Donc ils sont serviteurs de Dieu.} \\
	\end{tabularx}


