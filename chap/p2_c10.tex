\subsubsection{\centering \Large CHAPITRE X}
\addcontentsline{toc}{section}{\protect\numberline{}{\scshape\bfseries X} - \emph{Autres observations pour reconnaître si les propositions sont universelles ou particulières}}
\begin{center}\emph{\large\scshape Autres observations pour reconnaître si les propositions sont universelles ou particulières.}\end{center}

	\lettrine{O}{n} peut faire quelques observations semblables, et non moins nécessaires, touchant l'universalité et la particularité.

\bigbreak
{\bfseries\scshape I. Observation.} Il faut distinguer deux sortes d'universalités ; l'une qu'on peut appeler métaphysique, et l'autre morale.

J'appelle universalité métaphysique, lorsqu'une universalité est parfaite et sans exception, comme : \emph{tout homme est vivant}, cela ne reçoit pas d'exception.

Et j'appelle universalité morale celle qui reçoit quelque exception, parce que, dans les choses morales, on se contente que les choses soient telles ordinairement : \emph{ut plurimum}, comme ce que saint Paul rapporte et approuve ;

	\begin{tabularx}{\textwidth}{X}
		\emph{Cretenses semper mendaces malae bestiae, ventres pigri}. \\
	\end{tabularx}

Ou ce que dit le même Apôtre :
	\begin{tabularx}{\textwidth}{X}
		\emph{Omnes quae sua sunt quaerunt, non quae Jesu Christi}. \\
	\end{tabularx}

Ou ce que dit un ancien :
	\begin{tabularx}{\textwidth}{X}
		\emph{Omnibus hoc vitium est cantoribus, ut nolint cantare rogati, injussi nunquam desistant}. \\
	\end{tabularx}


Ou ce qu'on dit d'ordinaire :
	\begin{tabularx}{\textwidth}{X}
		\emph{Que toutes les femmes aiment à parler :} \\
		\emph{Que tous les jeunes gens sont légers et inconsistants :} \\
		\emph{Que tous les vieillards louent le temps passé}. \\
	\end{tabularx}

Il suffit dans toutes ces sortes de propositions, qu'ordinairement cela soit ainsi, et on ne doit pas aussi en conclure rien à la rigueur.

Car, comme ces propositions ne sont pas tellement générales qu'elles ne souffrent des exceptions, il pourrait se faire que la conclusion serait fausse. Comme on n'aurait pas pu conclure de chaque Crétois en particulier, qu'il aurait été un menteur et une méchante bête, quoique l'Apôtre approuve en général ce vers d'un de leurs poètes : \emph{Les Crétois sont toujours menteurs, méchantes bêtes, grands mangeurs}, parce que quelques-uns de cette île pouvaient ne pas avoir les vices qui étaient communs aux autres.

Ainsi la modération qu'on doit garder dans ces propositions qui ne sont que moralement universelles, c'est, d'une part, de n'en tirer qu'avec grand jugement des conclusions particulières, et, de l'autre, de ne pas les contredire ni ne pas les rejeter comme fausses, quoiqu'on puisse opposer des instances où elles n'ont pas de lieu, mais de se contenter, si on les étendait trop loin, de montrer qu'elles ne doivent pas se prendre à la rigueur.

\bigbreak
{\bfseries\scshape II. Observation.} Il y a des propositions qui doivent passer pour métaphysiquement universelles, quoiqu'elles puissent recevoir des exceptions, lorsque dans l'usage ordinaire ces exceptions extraordinaires ne passent point pour devoir être comprises dans ces termes universels, comme si je dis : \emph{Tous les hommes n'ont que deux bras} ; cette proposition doit passer pour vraie dans l'usage ordinaire ; et ce serait chicaner que d'opposer qu'il y a eu des monstres qui n'ont pas laissé d'être hommes, quoiqu'ils eussent quatre bras, parce qu'on voit assez qu'on ne parle pas des monstres dans ces propositions générales, et qu'on veut dire seulement que, dans l'ordre de la nature, les hommes n'ont que deux bras. On peut dire de même que tous les hommes se servent des sons pour exprimer leurs pensées, mais que tous ne se servent pas de l'écriture : et ce ne serait pas une objection raisonnable que d'opposer les muets pour trouver de la fausseté dans cette proposition, parce qu'on voit assez, sans qu'on l'exprime, que cela ne doit s'entendre que de ceux qui n'ont point d'empêchement naturel à se servir des sons, ou pour n'avoir pu les apprendre, comme ceux qui sont nés sourds, ou pour ne pouvoir les former, comme les muets.

\bigbreak
{\bfseries\scshape III. Observation.} Il y a des propositions qui ne sont universelles que parce qu'elles doivent s'entendre \emph{de generibus singulorum}, et non pas \emph{de singulis generum}, comme parlent les philosophes, c'est-à-dire de toutes les espèces de quelque genre, et non pas de tous les particuliers de ces espèces. Ainsi l'on dit que tous les animaux furent sauvés dans l'arche de Noé, parce qu'il en fût sauvé quelques-uns de toutes les espèces. Jésus-Christ dit aussi des Pharisiens, qu'ils payaient la dime de toutes les herbes, \emph{decimatis omne olus}, non qu'ils payassent la dime de toutes les herbes qui étaient dans le monde, mais parce qu'il n'y avait point de sortes d'herbes dont ils ne payassent la dime. Ainsi saint Paul dit: \emph{Sicut et ego omnibus per omnia placeo} : c'est-à-dire qu'il s'accommodait à toutes sortes de personnes, Juifs, Gentils, Chrétiens, quoiqu'il ne plût pas à ses persécuteurs, qui étaient en si grand nombre. Ainsi l'on dit d'un homme, \emph{qu'il a passé par toutes les charges}, c'est-à-dire par toutes sortes de charges.

\bigbreak
{\bfseries\scshape IV. Observation.} Il y a des propositions qui ne sont universelles que parce que le sujet doit être pris comme restreint par une partie de l'attribut ; je dis par une partie, car il serait ridicule qu'il fût restreint par tout l'attribut, comme qui prétendrait que cette proposition est vraie : \emph{Tous les hommes sont justes}, parce qu'il l'entendrait en ce sens, que tous les hommes justes sont justes, ce qui serait impertinent. Mais quand l'attribut est complexe, et a deux parties, comme dans cette proposition : \emph{Tous les hommes sont justes par la grâce de Jésus-Christ}, c'est avec raison que l'on peut prétendre que le terme de \emph{justes} est sous-entendu dans le sujet, quoiqu'il n'y soit pas exprimé, parce qu'il est assez clair qu'on veut dire seulement que tous les hommes qui sont justes ne sont justes que par la grâce de Jésus-Christ; et ainsi cette proposition est vraie en toute rigueur, quoiqu'elle paraisse fausse à ne considérer que ce qui est exprimé dans le sujet, y ayant tant d'hommes qui sont méchants et pécheurs, et qui, par conséquent, n'ont point été justifiés par la grâce de Jésus-Christ. Il y a un très grand nombre de propositions dans l'Écriture, qui doivent être prises en ce sens, et entre autres ce que dit saint Paul : \emph{Comme tous meurent par Adam, ainsi tous seront vivifiés par Jésus-Christ}. Car il est certain qu'une infinité de païens, qui sont morts dans leur infidélité, n'ont point été vivifiés par Jésus-Christ, et qu'ils n'auront aucune part à la vie de la gloire dont parle saint Paul en cet endroit ; et ainsi le sens de l'Apôtre est que, comme tous ceux qui meurent meurent par Adam, tous ceux aussi qui sont vivifiés sont vivifiés par Jésus-Christ.

Il y a aussi beaucoup de propositions qui ne sont moralement universelles qu'en cette manière, comme quand on dit : \emph{Les Français sont bons soldats; les Hollandais sont bons matelots; les Flamands sont bons peintres; les Italiens sont bons comédiens} ; cela veut dire que les Français qui sont soldats sont ordinairement bons soldats, et ainsi des autres.

\bigbreak
{\bfseries\scshape V. Observation.} Il ne faut pas s'imaginer qu'il n'y ait point d'autre marque de particularité que ces mots : \emph{quidam, aliquis, quelque}, et semblables. Car, au contraire, il arrive assez rarement que l'on s'en serve, surtout dans notre langue.

Quand la particule \emph{des} ou \emph{de} est le pluriel de l'article \emph{un} selon la nouvelle remarque de la Grammaire générale, elle fait que les noms se prennent particulièrement, au lieu que, pour l'ordinaire, ils se prennent généralement avec l'article \emph{les}. C'est pourquoi il y a bien de la différence entre ces deux propositions : \emph{Les médecins croient maintenant qu'il est bon de boire pendant le chaud de la fièvre}; et \emph{Des médecins croient maintenant que le sang ne se fait pas dans le foie}. Car \emph{les médecins} dans la première, marque le commun des médecins d'aujourd'hui, et des médecins, dans la seconde, marque seulement quelques médecins particuliers.

Mais souvent avant \emph{des}, ou \emph{de}, ou \emph{un} au singulier, on met, \emph{il y a}: comme, \emph{il y a des médecins}, et cela en deux manières.

La première est, en mettant seulement après \emph{des}, ou \emph{un}, un substantif pour être le sujet de la proposition, et un adjectif pour en être l'attribut, soit qu'il soit le premier ou le dernier: comme, \emph{Il y a des douleurs salutaires : Il y a des plaisirs funestes : Il y a de faux amis : Il y a une humilité généreuse : Il y a des vices couverts de l'apparence de la vertu}. C'est comme on exprime dans notre langue ce qu'on exprime par \emph{quelque} dans le style de l'École : \emph{Quelques douleurs sont salutaires, quelque humilité est généreuse}, et ainsi des autres.

La seconde manière est de joindre par un \emph{qui} l'adjectif au substantif : \emph{Il y a des craintes qui sont raisonnables}. Mais ce \emph{qui} n'empêche pas que ces propositions ne puissent être simples dans le sens, quoique complexes dans l'expression : car c'est comme si on disait simplement : \emph{Quelques craintes sont raisonnables}. Ces façons de parler sont encore plus ordinaires que les précédentes : \emph{Il y a des hommes qui n'aiment qu'eux-mêmes: Il y a des chrétiens qui sont indignes de ce nom}.

On se sert quelquefois en latin d'un tour semblable. {\scshape Horace}.

\begin{center}\emph{Sunt quibus in Satyra videar nimis acer, et ultra
\\Legem tendere opus}.\end{center}

Ce qui est la même chose que s'il avait dit : \emph{Quidam existimant me nimis acrem esse in satyra}.

Il y en a qui me croient trop piquant dans la Satire.

De même dans l'Écriture : \emph{Est qui nequiter se humiliat}; Il y en a qui s'humilient mal.

\emph{Omnis, tout}, avec une négation, fait aussi une proposition particulière, avec cette différence, qu'en latin la négation précède omnis, et en français elle suit tout : \emph{Non omnis qui dicit mihi, Domine, Domine, intrabit in regnum caelorum}. Tous ceux qui me disent, Seigneur, Seigneur, n'entreront point dans le royaume des cieux. \emph{Non omne peccatum est crimen}, Tout péché n'est pas un crime.

Néanmoins dans l'hébreu, \emph{non omnis} est souvent pour \emph{nullus}, comme dans le psaume : \emph{Non justificabitur in conspectu tuo omnis vivens}, nul homme vivant ne se justifiera devant Dieu. Cela vient de ce qu'alors la négation ne tombe que sur le verbe, et non point sur \emph{omnis}; au lieu qu'ordinairement dans ces façons de parler elle tombe sur l'un et l'autre, ce qu'on n'a peut-être jamais observé. Car en disant, \emph{non omnis amicus est fidelis}, Tout ami n'est pas fidèle, si la négation ne tombait que sur le verbe on nierait l'attribut de \emph{fidèle} de tout ami, ce qu'on ne veut pas faire; et si elle ne tombait point sur le verbe, mais seulement sur \emph{omnis}, la proposition serait affirmative, et on affirmerait l'attribut \emph{fidèle} de quelque ami, ce qui n'est pas l'intention de celui qui dit que tout ami n'est pas fidèle, étant clair qu'il ne veut pas dire que quelque ami est fidèle, mais que quelque ami n'est pas fidèle.

\bigbreak
{\bfseries\scshape VI. Observation.} Voilà quelques observations assez utiles quand il y a un terme d'universalité, comme \emph{tout, nul}, etc. Mais quand il n'y en a point, et qu'il n'y a point aussi de particularité, comme quand je dis, \emph{L'homme est raisonnable : L'homme est juste}, c'est une question célèbre parmi les philosophes, si ces propositions qu'ils appellent \emph{indéfinies}, doivent passer pour universelles ou pour particulières ; ce qui doit s'entendre quand elles sont sans aucune suite de discours, ou qu'on ne les a point déterminées par la suite à aucun de ces sens; car il est indubitable qu'on doit prendre le sens d'une proposition, quand elle a quelque ambiguïté, de ce qui l'accompagne dans le discours de celui qui s'en sert.

La considérant donc en elle-même, la plupart des philosophes disent qu'elle doit passer pour universelle dans une matière nécessaire, et pour particulière dans une matière contingente.

Je trouve cette maxime approuvée par de fort habiles gens, et néanmoins elle est très fausse ; et il faut dire, au contraire, que lorsqu'on attribue quelque qualité à un terme commun, la proposition indéfinie doit passer pour universelle en quelque matière que ce soit : et ainsi, dans une matière contingente, elle ne doit point être considérée comme une proposition particulière, mais comme une universelle qui est fausse; et c'est le jugement naturel que tous les hommes en font, les rejetant comme fausses lorsqu'elles ne sont pas vraies généralement, au moins d'une généralité morale, dont les hommes se contentent dans les discours ordinaires des choses du monde.

Car qui souffrirait que l'on dit, \emph{Que les ours sont blancs, Que les hommes sont noirs, Que les Parisiens sont gentilshommes; Les Polonais sont Sociniens, Les Anglais sont trembleurs}. Et cependant, selon la distinction de ces philosophes, ces propositions devraient passer pour très vraies ; puisque étant indéfinies dans une matière contingente, elles devraient être prises pour particulières. Or, il est très vrai qu'il y a quelques ours blancs, comme ceux de la nouvelle Zemble; quelques hommes qui sont noirs, comme les Éthiopiens; quelques Parisiens qui sont gentilshommes; quelques Polonais qui sont Sociniens; quelques Anglais qui sont trembleurs. Il est donc clair qu'en quelque matière que ce soit, les propositions indéfinies de cette sorte sont prises pour universelles, mais que dans une matière contingente on se contente d'une universalité morale. Ce qui fait qu'on dit fort bien : \emph{Les Français sont vaillants, les Italiens sont soupçonneux, les Allemands sont grands, les Orientaux sont voluptueux}, quoique cela ne soit pas vrai de tous les particuliers, parce qu'on se contente qu'il soit vrai de la plupart.

Il y a donc une autre distinction sur ce sujet, laquelle est plus raisonnable, qui est que ces propositions indéfinies sont universelles en matière de doctrine, comme, les anges n'ont point de corps, et qu'elles ne sont que particulières dans les faits et dans les narrations, comme quand il est dit dans l'Évangile : \emph{Milites plectentes coronam de spinis, imposuerunt capiti eius}. Il est bien clair que cela ne doit être entendu que de quelques soldats, et non pas de tous les soldats. Donc la raison est qu'en matière d'actions singulières, lors surtout qu'elles sont déterminées à un certain temps, elles ne conviennent ordinairement à un terme commun qu'à cause de quelques particuliers, dont l'idée distincte est dans l'esprit de ceux qui font tes propositions : de sorte qu'à le bien prendre, ces propositions sont plutôt singulières que particulières, comme on pourra le juger par ce qui a été dit des termes complexes dans le sens, Première partie Chapitre VI et Deuxième partie Chapitre IV.

\bigbreak
{\bfseries\scshape VII. Observation.} Les noms de \emph{corps}, de \emph{communauté}, de \emph{peuple}, étant pris collectivement, comme ils le sont d'ordinaire, pour tout le corps, toute la communauté, tout le peuple, ne font point les propositions où ils entrent, proprement universelles, ni encore moins particulières, mais singulières, comme quand je dis : \emph{Les Romains ont vaincu les Carthaginois : Les Vénitiens font la guerre aux Turcs : Les juges d'un tel lieu ont condamné un criminel}, ces propositions ne sont point universelles; autrement on pourrait conclure de chaque Romain, qu'il aurait vaincu les Carthaginois, ce qui serait faux : et elles ne sont point aussi particulières ; car cela veut dire plus que si je disais, que quelques Romains ont vaincu les Carthaginois ; mais elles sont singulières, parce que l'on considère chaque peuple comme une personne morale, dont la durée est de plusieurs siècles, qui subsiste tant qu'il compose un État, et qui agit en tous ces temps par ceux qui le composent, comme un homme agit par ses membres. D'où vient que l'on dit, que les Romains qui ont été vaincus par les Gaulois qui prirent Rome, ont vaincu les Gaulois au temps de César, attribuant ainsi à ce même terme de \emph{Romains} d'avoir été vaincus en un temps, et d'avoir été victorieux en l'autre, quoiqu'en l'un de ces temps il n'y ait eu aucun de ceux qui étaient en l'autre : et c'est ce qui fait voir sur quoi est fondée la vanité que chaque particulier prend des belles actions de sa nation, auxquelles il n'a point eu de part, et qui est aussi sotte que celle d'une oreille, qui étant sourde, se glorifierait de la vivacité de l'œil, ou de l'adresse de la main.

