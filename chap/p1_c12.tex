\subsubsection{\centering \Large CHAPITRE XII}
\addcontentsline{toc}{section}{\protect\numberline{}{\scshape\bfseries XII} - \emph{D'une autre sorte de définition de noms, par lesquels on marque ce qu'ils signifient dans l'usage}}
\begin{center}\emph{\large\scshape D'une autre sorte de définition de noms, par lesquels on marque ce qu'ils signifient dans l'usage.}\end{center}

	\lettrine{T}{out} ce que nous avons dit des définitions de noms ne doit s'entendre que de celles où l'on définit les mots dont on se sert en particulier ; et c'est ce qui les rend libres et arbitraires, parce qu'il est permis à chacun de se servir de tel son qu'il lui plaît pour exprimer ses idées, pourvu qu'il en avertisse. Mais, comme les hommes ne sont maîtres que de leur langage, et non pas de celui des autres, chacun a le droit de faire un dictionnaire pour soi,	mais on n'a pas droit d'en faire pour les autres, ni d'expliquer leurs paroles par ces significations particulières qu'on aura attachées aux mots. C'est pourquoi, quand on n'a pas dessein de faire connaître simplement en quel sens on prend un mot, mais qu'on prétend expliquer celui auquel il est communément pris, les définitions qu'on en donne ne sont nullement arbitraires, mais elles sont liées et astreintes à représenter, non la vérité des choses, mais la vérité de l'usage ; et on les doit estimer fausses, si elles n'expriment pas véritablement cet usage, c'est-à-dire si elles ne joignent pas aux sons les mêmes idées qui y sont jointes par l'usage ordinaire de ceux qui s'en servent. Et c'est ce qui fait voir aussi que ces définitions ne sont nullement exemptes d'être contestées, puisque l'on dispute tous les jours de la signification que l'usage donne aux termes.

Or, quoique ces sortes de définitions de mots semblent être le partage de grammairiens, puisque ce sont celles qui composent les dictionnaires, qui ne sont autre chose que l'explication des idées que les hommes sont convenus de lier à certains sons, néanmoins l'on peut faire sur ce sujet plusieurs réflexions très importantes pour l'exactitude de nos jugements.

La première, qui sert de fondement aux autres, est que les hommes ne considèrent pas souvent toute la signification des mots, c'est-à-dire que les mots signifient souvent plus qu'il ne semble, et que, lorsqu'on en veut expliquer la signification, on ne représente pas toute l'impression qu'ils font dans l'esprit.

Car, signifier dans un son prononcé ou écrit, n'est autre chose qu'exciter une idée liée à ce son dans notre esprit, en frappant nos oreilles ou nos yeux. Or, il arrive souvent qu'un mot, outre l'idée principale que l'on regarde comme la signification propre de ce mot, excite plusieurs autres idées qu'on peut appeler accessoires, auxquelles on ne prend pas garde, quoique l'esprit en reçoive l'impression.

Par exemple, si l'on dit à une personne : Vous en avez menti, et que l'on ne regarde que la signification principale de cette expression, c'est la même chose que si on lui disait : Vous savez le contraire de ce que vous dites; mais, outre cette signification principale, ces paroles emportent dans l'usage une idée de mépris et d'outrage, et elles font croire que celui qui nous les dit ne se soucie pas de nous faire injure, ce qui les rend injurieuses et offensantes.

Quelquefois ces idées accessoires ne sont pas attachées aux mots par un usage commun, mais elles y sont seulement jointes par celui qui s'en sert; et ce sont proprement celles qui sont excitées par le ton de la voix, par l'air du visage, par les gestes, et par les autres signes naturels qui attachent à nos paroles une infinité d'idées, qui en diversifient, changent, diminuent, augmentent la signification, en y joignant l'image des mouvements, des jugements et des opinions de celui qui parle.

C'est pourquoi, si celui qui disait qu'il fallait prendre la mesure du ton de sa voix, des oreilles de celui qui écoute, voulait dire qu'il suffit de parler assez haut pour se faire entendre, il ignorait une partie de l'usage de la voix, le ton signifiant souvent autant que les paroles mêmes. Il y a voix pour instruire, voix pour flatter, voix pour reprendre ; souvent on ne veut pas seulement qu'elle arrive jusqu'aux oreilles de celui à qui l'on parle, mais on veut qu'elle le frappe et qu'elle le perce ; et personne ne trouverait bon qu'un laquais, que l'on reprend un peu fortement, répondit : Monsieur, parlez plus bas, je vous entends bien ; parce que le ton fait partie de la réprimande, et est nécessaire pour former dans l'esprit l'idée que l'on veut y imprimer.

Mais quelquefois ces idées accessoires sont attachées aux mots mêmes, parce qu'elles s'excitent ordinairement par tous ceux qui les prononcent ; et c'est ce qui fait qu'entre des expressions qui semblent signifier la même chose, les unes sont injurieuses, les autres douces; les unes modestes, les autres impudentes: les unes honnêtes, et les autres déshonnêtes ; parce qu'outre cette idée principale en quoi elles conviennent, les hommes y ont attaché d'autres idées, qui sont cause de cette diversité.

Cette remarque peut servir à découvrir une injustice assez ordinairement à ceux qui se plaignent des reproches qu'on leur a faits, qui est de changer les substantifs en adjectifs : de sorte que, si on les a accusés d'ignorance ou d'imposture, ils disent qu'on les a appelés ignorants ou imposteurs ; ce qui n'est pas raisonnable, ces mots ne signifiant pas la même chose ; car les mots adjectifs d'ignorant ou imposteur, outre la signification du défaut qu'ils marquent, enferment encore l'idée du mépris ; au lieu que ceux d'ignorance et d'imposture marquent la chose telle qu'elle est, sans l'aigrir ni l'adoucir. L'on en pourrait trouver d'autres qui signifieraient la même chose d'une manière qui enfermerait de plus une idée adoucissante et qui témoignerait qu'on désire épargner celui à qui l'on fait ces reproches; et ce sont ces manières que choisissent les personnes sages et modérées, à moins qu'elles n'aient quelque raison particulière d'agir avec plus de force.

C'est encore par là qu'on peut reconnaître la différence du style simple et du style figuré, et pourquoi les mêmes pensées nous paraissent beaucoup plus vives quand elles sont exprimées par une figure, que si elles étaient renfermées dans des expressions toutes simples, car cela vient de ce que les expressions figurées signifient, outre la chose principale, le mouvement et la passion de celui qui parle, et impriment ainsi l'une et l'autre idée dans l'esprit ; au lieu que l'expression simple ne marque que la vérité toute nue.

Par exemple, si ce demi-vers de Virgile : \emph{Usque adeone mori miserum est !} était exprimé simplement et sans figure, de cette sorte : \emph{Non est usque adeo mori miserum}: Il est sans doute qu'il aurait beaucoup moins de force. Et la raison en est, que la première expression signifie beaucoup plus que la seconde. Car elle n'exprime pas seulement cette pensée, que la mort n'est pas un si grand mal que l'on croit ; mais elle représente de plus l'idée d'un homme qui se roidit contre la mort, et qui l'envisage sans effroi, image beaucoup plus vive que n'est la pensée même à laquelle elle est jointe. Ainsi, il n'est pas étrange qu'elle frappe davantage, parce que l'âme s'instruit par les images des vérités ; mais elle ne s'émeut guère que par l'image des mouvements.

\begin{center}\emph{Si vis me flere, dolendum est \\Primum ipsi tibi.}\end{center}

Mais, comme le style figuré signifie ordinairement, avec les choses, les mouvements que nous ressentons en les concevant et en parlant, on peut juger par là de l'usage que l'on en doit faire et quels sont les sujets auxquels il est propre. Il est visible qu'il est ridicule de s'en servir dans les matières purement spéculatives, que l'on regarde d'un œil tranquille, et qui ne produisent aucun mouvement dans l'esprit; car, puisque les figures expriment les mouvements de notre âme, celles que l'on mêle en des sujets où l'âme ne s'émeut point sont des mouvements contre la nature, et des espèces de convulsions. C'est pourquoi il n'y a rien de moins agréable que certains prédicateurs qui s'écrient indifféremment sur tout, et qui ne s'agitent pas moins sur des raisonnements philosophiques que sur les vérités les plus étonnantes et les plus nécessaires pour le salut.

Et, au contraire, lorsque la matière que l'on traite est telle qu'elle doit raisonnablement nous toucher, c'est un défaut d'en parler d'une manière sèche et froide, et sans mouvement; parce que c'est un défaut de n'être pas touché de ce que l'on doit.

Ainsi, les vérités divines n'étant pas proposées simplement pour être connues, mais beaucoup plus pour être aimées, révérées et adorées par les hommes, il est sans doute que la manière noble, élevée et figurée dont les saints Pères les ont traitées leur est bien plus proportionnée qu'un style simple et sans figure, comme celui des scolastiques, puisqu'elle ne nous enseigne pas seulement ces vérités, mais qu'elle nous représente aussi les sentiments d'amour et de révérence avec lesquels les Pères en ont parlé, et que portant ainsi dans notre esprit l'image de cette sainte disposition, elle peut beaucoup contribuer à y en imprimer une semblable; au lieu que le style scolastique étant simple, et ne contenant que les idées de la vérité toute nue, est moins capable de produire dans l'âme les mouvements de respect et d'amour que l'on doit avoir pour les vérités chrétiennes; ce qui le rend en ce point, non seulement moins utile, mais aussi moins agréable, le plaisir de l'âme consistant plus à sentir des mouvements qu'à acquérir des connaissances.

Enfin, c'est par cette même remarque qu'on peut résoudre cette question célèbre entre les anciens philosophes : s'il y a des mots déshonnêtes, et que l'on peut réfuter les raisons des Stoïciens, qui voulaient qu'on pût se servir indifféremment des expressions qui sont estimées ordinairement infâmes et impudentes.

Ils prétendent, dit Cicéron, dans une lettre qu'il a faite sur ce sujet, qu'il n'y a point de paroles sales ni honteuses; car, ou l'infamie (disent-ils) vient des choses, ou elle est dans les paroles; elle ne vient pas simplement des choses, puisqu'il est permis de les exprimer en d'autres paroles qui ne passent point pour déshonnêtes; elle n'est pas aussi dans les paroles considérées comme sons, puisqu'il arrive souvent, comme Cicéron le montre, qu'un même son signifiant diverses choses, et étant estimé déshonnête dans une signification, ne l'est point en une autre.

Mais tout cela n'est qu'une vaine subtilité qui ne naît que de ce que ces philosophes n'ont pas assez considéré ces idées accessoires que l'esprit joint aux idées principales des choses; car il arrive de là qu'une même chose peut être exprimée honnêtement par un son, et déshonnêtement par un autre, si l'un de ces sons y joint quelque autre idée qui en couvre l'infamie, et si l'autre, au contraire, la présente à l'esprit d'une manière impudente. Ainsi les mots d'adultère, d'inceste, de péché abominable, ne sont pas infâmes, quoiqu'ils représentent des actions très infâmes, parce qu'ils ne les représentent que couvertes d'un voile d'horreur, qui fait qu'on ne les regarde que comme des crimes; de sorte que ces mots signifient plutôt le crime de ces actions que les actions mêmes, au lieu qu'il y a de certains mots qui les expriment sans en donner de l'horreur, et plutôt comme plaisantes que comme criminelles, et qui y joignent même une idée d'impudence et d'effronterie, et ce sont ces mots-là qu'on appelle infâmes et déshonnêtes.

Il en est de même de certains tours par lesquels on exprime honnêtement des actions qui, quoique légitimes, tiennent quelque chose de la corruption de la nature; car ces tours sont en effet honnêtes, parce qu'ils n'expriment pas simplement ces choses, mais aussi la disposition de celui qui en parle de cette sorte, et qui témoigne par sa retenue qu'il les envisage avec peine et qu'il les couvre autant qu'il peut, et aux autres et à soi-même ; au lieu que ceux qui en parleraient d'une autre manière feraient paraître qu'ils prendraient plaisir à regarder ces sortes d'objets; et ce plaisir étant infâme, il n'est pas étrange que les mots qui impriment cette idée soient estimés contraires à l'honnêteté.

C'est pourquoi il arrive aussi qu'un même mot est estimé honnête en un temps et honteux en un autre, ce qui a obligé les docteurs hébreux de substituer en certains endroits de la Bible des mots hébreux à la marge, pour être prononcés par ceux qui la liraient, au lieu de ceux dont l'Écriture se sert : car cela vient de ce que ces mots, lorsque les prophètes s'en sont servis, n'étaient point déshonnêtes, parce qu'ils étaient liés avec quelque idée qui faisait regarder ces objets avec retenue et avec pudeur ; mais depuis, cette idée en ayant été séparée, et l'usage y en ayant joint une autre d'impudence et d'effronterie, ils sont devenus honteux; et c'est avec raison que, pour ne pas frapper l'esprit de cette mauvaise idée, les rabbins veulent qu'on en prononce d'autres en lisant la Bible, quoiqu'ils n'en changent pas pour cela le texte.

Ainsi c'était une mauvaise défense à un auteur que la profession religieuse obligeait à une exacte modestie, et à qui on avait reproché avec raison de s'être servi d'un mot peu honnête pour signifier un lieu infâme, d'alléguer que les Pères n'avaient pas fait difficulté de se servir de celui de \emph{lupanar}, et qu'on trouvait souvent dans leurs écrits les mots de \emph{meretrix}, de \emph{leno}, et d'autres qu'on aurait peine à souffrir en notre langue ; car la liberté avec laquelle les Pères se sont servis de ces mots devait lui faire connaître qu'ils n'étaient pas estimés honteux de leur temps, c'est-à-dire que l'usage n'y avait pas joint cette idée d'effronterie qui les rend infâmes, et il avait tort de conclure de là qu'il lui fût permis de se servir de ceux qui sont estimés déshonnêtes en notre langue, parce que ces mots ne signifient pas en effet la même chose que ceux dont les Pères se sont servis, puisque, outre l'idée principale en laquelle ils conviennent, ils enferment aussi l'image d'une mauvaise disposition d'esprit et qui tient quelque chose du libertinage et de l'impudence.

Ces idées accessoires étant donc si considérables et diversifiant si fort les significations principales, il serait utile que ceux qui font des dictionnaires les marquassent, et qu'ils avertissent, par exemple, des mots qui sont injurieux, civils, aigres, honnêtes, déshonnêtes, ou plutôt qu'ils retranchassent entièrement ces derniers, étant toujours plus utile de les ignorer que de les savoir.
