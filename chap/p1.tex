\subsection{\centering \huge \scshape Première Partie}
\addcontentsline{toc}{section}{\scshape\large {\bfseries Première Partie} - Contenant les réflexions sur les idées, ou sur la première action de l'esprit, qui s'appelle concevoir}
\begin{center}\emph{\Large\scshape Contenant les réflexions sur les idées, ou sur la première action de l'esprit, qui s'appelle concevoir.}\end{center}
\lettrine{C}{omme} nous ne pouvons avoir aucune connaissance de ce qui est hors de nous que par l'entremise des idées qui sont en nous, il n'y a rien de plus important dans la Logique et dans toutes les autres sciences que de bien connaître nos idées.

Pour les bien comprendre nous les considérerons en cinq manières.

La première, selon leur nature et leur origine.

La deuxième, selon la principale différence des objets qu'elles représentent.

La troisième, selon leur simplicité ou composition; où nous traiterons des abstractions et précisions d'esprit.

La quatrième, selon leur étendue ou restriction, c'est-à-dire leur universalité, particularité, singularité.

La cinquième, selon leur clarté et obscurité, ou distinction et confusion.
