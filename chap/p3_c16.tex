\subsubsection{\centering \Large CHAPITRE XVI}
\addcontentsline{toc}{section}{\protect\numberline{}{\scshape\bfseries XVI} - \emph{Division des lieux en lieux de grammaire, de logique et de métaphysique}}
\begin{center}\emph{\large\scshape Division des lieux en lieux de grammaire, de logique et de métaphysique.}\end{center}

	\lettrine{C}{eux} qui ont traité des lieux, les ont divisés en différentes matières. Celle qui a été suivie par Cicéron dans les livres de l'Invention et dans le deuxième livre de l'Orateur, et par Quintilien au cinquième livre de ses Institutions, est moins méthodique; mais elle est aussi plus propre pour l'usage des discours du barreau, auquel ils la rapportent particulièrement. Celle de Ramus est trop embarrassée de subdivisions.

En voici une qui paraît assez commode, d'un philosophe Allemand fort judicieux et fort solide nommé Clauberge, dont la Logique m'est tombée entre les mains, lorsqu'on avait déjà commencé à imprimer celle-ci.

Les lieux sont tirés, ou de la grammaire ou de la logique, ou
de la métaphysique.

\begin{center}\emph{\bfseries\scshape Lieu de Grammaire}\end{center}

	Les lieux de grammaire sont, l'étymologie, et les mots dérivés de même racine, qui s'appellent en latin \emph{conjugata} et en grec \emph{{\text{π$\alpha$ρ\'{o}$\nu$o$\mu\alpha$}}}.

On argumente par l'étymologie, quand on dit, par exemple, que plusieurs personnes du monde ne se divertissent jamais, à proprement parler; parce que se divertir, c'est se désappliquer des occupations sérieuses, et qu'ils ne s'occupent jamais sérieusement.

Les mots dérivés de même racine servent aussi à faire trouver des pensées.

	\begin{tabularx}{\textwidth}{X}
		\emph{Homo sum, humani nil a me alienum puto.} \\
		\emph{Mortali urgemur ab hoste, mortales.} \\
		\emph{Quid tam dignum misericordia quam miser ?} \\
		\emph{Quid tam indignum misericordia quam superbus miser ?} \\
	\end{tabularx}

Qu'y a-t-il de plus digne de miséricorde qu'un misérable? Et qu'y a-t-il de plus indigne de miséricorde qu'un misérable qui est orgueilleux.


\begin{center}\emph{\bfseries\scshape Lieu de Logique}\end{center}

Les lieux de logique sont les termes universels, genre, espèce, différence, propre, accident, la définition, la division ; et comme tous ces points ont été expliqués auparavant, il n'est pas nécessaire d'en traiter ici davantage.

Il faut seulement remarquer que l'on joint d'ordinaire à ces lieux certaines maximes communes, qu'il est bon de savoir, non pas qu'elles soient fort utiles, mais parce qu'elles sont communes. On en a déjà rapporté quelques-unes sous d'autres termes ; mais il est bon de les savoir sous les termes ordinaires.

\bigbreak
$1$. Ce qui s'afirme ou se nie du genre, s'affirme ou se nie de l'espèce. \emph{Ce qui convient à tous les hommes, convient aux grands ; mais ils ne peuvent pas prétendre aux avantages qui sont au-dessus des hommes}.

\bigbreak
$2$. En détruisant le genre, on détruit aussi l'espèce. \emph{Celui qui ne juge point du tout, ne juge point mal; celui qui ne parle point du tout, ne parle jamais indiscrètement}.

\bigbreak
$3$. En détruisant toutes les espèces, on détruit les genres. \emph{Les formes qu'on appelle substantielles (excepté l'âme raisonnable) ne sont ni corps ni esprit : donc ce ne sont point des substances}.

\bigbreak
$4$. Si l'on peut affirmer ou nier de quelque chose la différence totale, on en peut affirmer ou nier l'espèce. \emph{L'étendue ne convient pas à la pensée : donc elle n'est pas matière}.

\bigbreak
$5$. Si l'on peut affirmer ou nier de quelque chose la propriété, on en peut affirmer ou nier l'espèce. \emph{Étant impossible de se figurer la moitié d'une pensée, ni une pensée ronde et carrée, il est impossible que ce soit un corps}.

\bigbreak
$6$. On affirme ou on nie le défini de ce dont on affirme ou nie la définition. \emph{Il y a peu de personnes justes, parce qu'il y en a peu qui aient une ferme et constante volonté de rendre à chacun ce qui lui appartient}.

\newpage

\begin{center}\emph{\bfseries\scshape Lieu de Métaphysique}\end{center}

Les lieux de métaphysique sont certains termes généraux convenant à tous les êtres auxquels on rapporte plusieurs arguments, comme les causes, les effets, le tout, les parties, les termes opposés. Ce qu'il y a de plus utile est d'en savoir quelques divisions générales, et principalement des causes.

Les définitions qu'on donne dans l'École aux causes en général, en disant qu'\emph{une cause est ce qui produit un effet, ou ce par quoi une chose est}, sont si peu nettes, et il est si difficile de voir comment elles conviennent à tous les genres de causes, qu'on aurait aussi bien fait de laisser ce mot entre ceux que l'on ne définit point, l'idée que nous en avons étant aussi claire que les définitions qu'on en donne.

Mais la division des causes en quatre espèces, qui sont la cause finale, efficiente, matérielle et formelle, est si célèbre, qu'il est nécessaire de le savoir.

\bigbreak
On appelle {\bfseries\scshape Cause Finale} la fin pour laquelle une chose est.

Il y a des fins \emph{principales}, qui sont celles que l'on regarde principalement, et des fins \emph{accessoires}, qu'on ne considère que par surcroît.

Ce que l'on prétend faire ou obtenir est appelé \emph{finis cujus gratia}. Ainsi, la santé est la fin de la médecine, parce qu'elle prétend la procurer.

Celui pour qui l'on travaille est appelé \emph{finis cui}, l'homme est la fin de la médecine en cette manière, parce que c'est à lui qu'elle a dessein d'apporter la guérison.

Il n'y a rien de plus ordinaire que de tirer des arguments de la fin, ou pour montrer qu'une chose est imparfaite, comme qu'un discours est mal fait, lorsqu'il n'est pas propre à persuader; ou pour faire voir qu'il est vraisemblable qu'un homme a fait ou fera quelque action, parce qu'elle est conforme à la fin qu'il a accoutumé de se proposer ; d'où vient cette parole célèbre d'un juge de Rome, qu'il fallait examiner avant toutes choses, \emph{cui bono}, c'est-à-dire quel intérêt un homme aurait eu à faire une chose, parce que les hommes agissent ordinairement selon leur intérêt, ou pour montrer, au contraire, qu'on ne doit pas soupçonner un homme d'une action, parce qu'elle aurait été contraire à sa fin.

Il y a encore plusieurs autres manières de raisonner par la fin que le bon sens découvrira mieux que tous les préceptes ; ce qui soit dit aussi pour les autres lieux.

\bigbreak
{\bfseries\scshape La Cause Efficiente} est celle qui produit une autre chose. On en tire des arguments, en montrant qu'un effet n'est pas, parce qu'il n'a pas eu de cause suffisante, ou qu'il est ou sera, en faisant voir que toutes ces causes sont. Si ces causes sont nécessaires, l'argument est nécessaire ; si elles sont libres et contingentes, il n'est que probable.

Il y a diverses espèces de cause efficiente, dont il est utile de savoir les noms.

Dieu créant Adam était sa cause \emph{totale}, parce que rien ne concourait avec lui; mais le père et la mère ne sont chacun que causes partielles de leur enfant, parce qu'ils ont besoin l'un de l'autre.

Le Soleil est une cause \emph{propre} de la lumière ; mais il n'est cause qu'\emph{acci-}\emph{dentelle} de la mort d'un homme que sa chaleur aura fait mourir, parce qu'il était mal disposé.

Le père est cause \emph{prochaine} de son fils.

L'aïeul n'en est que la cause \emph{éloignée}.

La mère est une cause \emph{productive}.

La nourrice n'est qu'une cause \emph{conservante}.

Le père est une cause \emph{univoque} à l'égard de ses enfants ; parce qu'ils lui sont semblables en nature.

Dieu n'est qu'une cause éguivoque à l'égard des créatures, parce qu'elles ne sont pas de la nature de Dieu.

Un ouvrier est la cause \emph{principale} de son ouvrage, ses instruments n'en sont que la cause \emph{instrumentale}.

L'air qui entre dans les orgues est une cause \emph{universelle} de l'harmonie des orgues.

La disposition particulière de chaque tuyau, et celui qui en joue, en sont les causes \emph{particulières} qui déterminent l'universelle.

Le Soleil est une cause \emph{naturelle}.

L'homme, une cause \emph{intellectuelle} à l'égard de ce qu'il fait avec jugement.

Le feu qui brûle du bois est une cause \emph{nécessaire}.

Un homme qui marche est une cause \emph{libre}.

Le Soleil, éclairant une chambre, est la cause \emph{propre} de sa clarté; l'ouverture de la fenêtre n'est qu'une cause ou condilion, sans laquelle l'effet ne se ferait pas, \emph{conditio sine qua non}.

Le feu, brûlant une maison, est la cause \emph{physique} de l'embrasement ; l'homme qui y a mis le feu en est la cause \emph{morale}.

On rapporte encore à la cause efficiente, la cause \emph{exemplaire}, qui est le modèle que l'on se propose en faisant un ouvrage, comme le dessin d'un bâtiment par lequel un architecte se conduit ; ou généralement ce qui est cause de l'être objectif de notre idée, ou de quelque autre image que ce soit, comme le roi Louis XIV est la cause exemplaire de son portrait.

\bigbreak
{\bfseries\scshape La Cause Matérielle} est ce dont les choses sont formées, comme l'or est la matière d'un vase d'or ; ce qui convient ou ne convient pas à la matière, convient ou ne convient pas aux choses qui en sont composées.

\bigbreak
{\bfseries\scshape La Forme} est ce qui rend une chose telle et la distingue des autres, soit que ce soit un être réellement distingué de la matière, selon l'opinion de l'École, soit que ce soit seulement l'arrangement des parties. C'est par la connaissance de cette forme qu'on en doit expliquer les propriétés.

Il y a autant de différents effets que de causes, ces mots étant réciproques. La manière ordinaire d'en tirer des arguments est de montrer que si l'effet est, la cause est, rien ne pouvant être sans cause. On prouve aussi qu'une cause est bonne ou mauvaise quand les effets en sont bons ou mauvais, ce qui n'est pas toujours vrai dans les causes par accident.

On a parlé suffisamment du tout et des parties dans le chapitre de la division, et ainsi il n'est pas nécessaire d'en rien ajouter ici.

On fait de quatre sortes de termes opposés :

Les relatifs, comme père, fils; maître, serviteur.

Les contraires, comme froid et chaud ; sain et malade.

Les privatifs, comme la vie, la mort; la vue, l'aveuglement ; l'ouïe, la surdité ; la science, l'ignorance.

Les contradictoires, qui consistent dans un terme et dans la simple négation de ce terme : voir, ne voir pas. La différence qu'il y a entre ces deux dernières sortes d'opposés est que les termes privatifs enferment la négation d'une forme dans un sujet qui en est capable, au lieu que les négatifs ne marquent point cette capacité. C'est pourquoi on ne dit point qu'une pierre est aveugle ou morte, parce qu'elle n'est capable ni de la vue ni de la vie.

Comme ces termes sont opposés, on se sert de l'un pour nier l'autre. Les termes contradictoires ont cela de propre qu'en ôtant l'un, on établit l'autre.

Il y a plusieurs sortes de comparaisons : car l'on compare les choses, ou égales, ou inégales ; ou semblables, ou dissemblables. On prouve que ce qui convient ou ne convient pas à une chose égale ou semblable, convient ou ne convient pas à une autre chose à qui elle est égale ou semblable.

Dans les choses inégales, on prouve négativement que, si ce qui est plus probable n'est pas, ce qui est moins probable n'est pas à plus forte raison ; ou affirmativement que, si ce qui est moins probable est, ce qui est plus probable est aussi. On se sert d'ordinaire des différences ou des dissimilitudes pour ruiner ce que d'autres auraient voulu établir par des similitudes, comme on ruine l'argument qu'on tire d'un arrêt en montrant qu'il est donné sur un autre cas.

Voilà grossièrement une partie de ce que l'on dit des lieux. Il y a des choses qu'il est plus utile de ne savoir qu'en cette manière. Ceux qui en désireront davantage le peuvent voir dans les auteurs qui en ont traité avec plus de soin. On ne saurait néanmoins conseiller à personne de l'aller chercher dans les Topiques d'Aristote, parce que ce sont des livres étrangement confus ; mais il y a quelque chose d'assez beau sur ce sujet dans le premier livre de sa Rhétorique, où il enseigne diverses manières de faire voir qu'une chose est utile, agréable, plus grande, plus petite. Il est vrai néanmoins qu'on n'arrivera jamais par ce chemin à aucune connaissance bien solide.

