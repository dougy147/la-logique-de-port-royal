\subsubsection{\centering \Large CHAPITRE III}
\addcontentsline{toc}{section}{\protect\numberline{}{\scshape\bfseries III} - \emph{Explication plus particulière de ces règles, et premièrement de celles qui regardent les définitions}}
\begin{center}\emph{\large\scshape Explication plus particulière de ces règles, et premièrement de celles qui regardent les définitions.}\end{center}

	\lettrine{Q}{uoique} nous ayons déjà parlé dans la première partie de l'utilité des définitions des termes, néanmoins cela est si important que l'on ne peut trop l'avoir dans l'esprit; puisque par là on démêle une infinité de disputes qui n'ont souvent pour sujet que l'ambiguïté des termes, que l'un prend en un sens, et l'autre en un autre : de sorte que de très grandes contestations cesseraient en un moment, si l'un ou l'autre des disputants avait soin de marquer nettement et en peu de paroles ce qu'il entend par les termes qui sont le sujet de la dispute.

Cicéron a remarqué que la plupart des disputes entre les philosophes anciens, et surtout entre les Stoïciens et les Académiciens, n'étaient fondées que sur cette ambiguïté de paroles, les Stoïciens ayant pris plaisir, pour se relever, de prendre les termes de la morale en d'autres sens que les autres, ce qui faisait croire que leur morale était bien plus sévère et plus parfaite, quoique en effet cette prétendue perfection ne fût que dans les mots, et non dans les choses : le sage des Stoïciens ne prenant pas moins tous les plaisirs de la vie que les philosophes des autres sectes qui paraissaient moins rigoureux, et n'évitant pas avec moins de soin les maux et les incommodités, avec cette seule différence, qu'au lieu que les autres philosophes se servaient des mots ordinaires de biens et de maux, les Stoïciens, en jouissant des plaisirs, ne les appelaient pas des biens, mais des choses préférables, \emph{{\text{πρo$\eta$γ$\mu$\'{$\varepsilon$}${\nu\alpha}$}}}, et en fuyant les maux, ne les appelaient pas des maux, mais seulement des choses rejetables, \emph{{\text{\'{$\alpha$}πoπρo$\eta$γ$\mu$\'{$\varepsilon$}$\nu\alpha$}}}.

C'est donc un avis très utile de retrancher de toutes les disputes tout ce qui n'est fondé que sur l'équivoque des mots, en les définissant par d'autres termes si clairs qu'on ne puisse plus s'y méprendre.

À cela sert la première des règles que nous venons de rapporter : \emph{Ne laisser aucun terme un peu obscur ou équivoque qu'on ne le définisse}.

Mais, pour tirer toute l'utilité que l'on doit de ces définitions, il faut encore y ajouter la seconde règle.

\emph{N'employer, dans les définitions, que des termes parfaitement connus ou déjà expliqués} ; c'est-à-dire que des termes qui désignent clairement, autant qu'il se peut, l'idée qu'on veut signifier par le mot qu'on définit.

Car il faut remarquer que quoique les définitions des mots ne soient pas proprement contestables, comme nous l'avons fait voir dans la première partie, elles peuvent néanmoins être défectueuses, lorsqu'elles ne font pas l'effet pour lequel elles sont instituées. Or l'effet qu'elles doivent faire est de marquer distinctement l'idée à laquelle on attache un mot. Et par conséquent il est inutile de définir un mot, si on le laisse après l'avoir défini dans la même confusion où il était auparavant : ce qui arrivera si l'idée qu'on désigne pour l'attacher à ce mot, n'est pas désignée clairement et distinctement.

Mais de plus quand on n'a pas désigné assez nettement et distinctement l'idée à laquelle on veut attacher un mot, il est presque impossible que dans la suite on ne passe insensiblement à une autre idée que celle qu'on a désignée; c'est-à-dire, qu'au lieu de substituer mentalement à chaque fois qu'on se sert de ce mot la même idée qu'on a désignée, on n'en substitue une autre que la nature nous fournit. Et c'est ce qu'il est aisé de découvrir, en substituant expressément la définition au défini. Car cela ne doit rien changer de la proposition, si on est toujours demeuré dans la même idée, au lieu que cela la changera si on n'y est pas demeuré.

Tout cela se comprendra mieux par quelques exemples. Euclide définit l'angle plan rectiligne. \emph{La rencontre de deux lignes droites inclinées sur un même plan}. Si l'on considère cette définition comme une simple définition de mots, en sorte qu'on regarde le mot d'angle comme ayant été dépouillé de toute signification, pour n'avoir plus que celle de la rencontre de deux lignes, on ne doit point y trouver à redire ; car il a été permis à Euclide d'appeler du mot d'angle la rencontre de deux lignes : mais il a été obligé de s'en souvenir, et de ne prendre plus le mot d'angle qu'en ce sens. Or, pour juger s'il l'a fait, il ne faut que substituer, toutes les fois qu'il parle de l'angle, au mot d'angle la définition qu'il a donnée ; et si, en substituant cette définition, il se trouve quelque absurdité en ce qu'il dit de l'angle, il s'ensuivra qu'il n'est pas demeuré dans la même idée qu'il avait désignée, mais qu'il est passé insensiblement à une autre, qui est celle de la nature. Il enseigne, par exemple, à diviser un angle en deux. Substituez sa définition. Qui ne voit que ce n'est point la rencontre de deux lignes qu'on divise en deux, que ce n'est point la rencontre de deux lignes qui a des côtés, et qui a une base ou sous-tendante ; mais que tout cela convient à l'espace compris entre les lignes, et non à la rencontre des lignes.

Il est visible que ce qui a embarrassé Euclide, et ce qui l'a empêché de désigner l'angle par les mots d'espace compris entre deux lignes qui se rencontrent, est qu'il a vu que cet espace pouvait être plus grand ou plus petit, quand les côtés de l'angle sont plus longs ou plus courts, sans que l'angle en soit plus grand et plus petit ; mais il ne devait pas conclure de là que l'angle rectiligne n'était pas un espace, mais seulement que c'était un espace compris entre deux lignes droites qui se rencontrent, indéterminé selon celle de ces deux dimensions qui répond à la longueur de ces lignes, et déterminé selon l'autre par la partie proportionnelle d'une circonférence qui a pour centre le point où ces lignes se rencontrent.

Cette définition désigne si nettement l'idée que tous les hommes ont d'un angle, que c'est tout ensemble une définition de mot et une définition de la chose ; excepté que le mot d'angle comprend aussi, dans le discours ordinaire, un angle solide ; au lieu que, par cette définition, on le restreint à signifier un angle plan rectiligne : et lorsqu'on a ainsi défini l'angle, il est indubitable que tout ce que l'on pourra dire ensuite de l'angle plan rectiligne, tel qu'il se trouve dans toutes les figures rectilignes, sera vrai de cet angle ainsi défini, sans qu'on soit jamais obligé de changer d'idée, ni qu'il se rencontre jamais aucune absurdité en substituant la définition à la place du défini ; car c'est cet espace ainsi expliqué que l'on peut diviser en deux, en trois, en quatre ; c'est cet espace qui a deux côtés entre lesquels il est compris ; c'est cet espace qu'on peut terminer du côté qu'il est de soi-même indéterminé, par une ligne qu'on appelle base ou sous-tendante ; c'est cet espace qui n'est point considéré comme plus grand ou plus petit, pour être compris entre des lignes plus longues ou plus courtes, parce qu'étant indéterminé selon cette dimension, ce n'est point de là qu'on doit prendre sa grandeur et sa petitesse. C'est par cette définition qu'on trouve le moyen de juger si un angle est égal à un autre angle, ou plus grand ou plus petit : car puisque la grandeur de cet espace n'est déterminée que par la partie proportionnelle d'une circonférence qui a pour centre le point où les lignes qui comprennent l'angle se rencontrent, lorsque deux angles ont pour mesure l'aliquote pareille chacun de sa circonférence, comme la dixième partie, ils sont égaux ; et si l'un a la dixième, et l'autre la douzième, celui qui a la dixième est plus grand que celui qui a la douzième. Au lieu que, par la définition d'Euclide, on ne saurait entendre en quoi consiste l'égalité de deux angles ; ce qui fait une horrible confusion dans ses Éléments, comme Ramus a remarqué, quoique lui-même ne rencontre guère mieux.

Voici d'autres définitions d'Euclide, où il fait la même faute qu'en celle de l'angle. \emph{La raison}, dit-il, \emph{est une habitude de deux grandeurs de même genre, comparées l'une à l'autre selon la quantité; la proportion est une similitude de raisons}.

Par ces définitions, le nom de \emph{raison} doit comprendre l'habitude qui est entre deux grandeurs, lorsque l'on considère de combien l'une surpasse l'autre : car on ne peut nier que ce ne soit une habitude de deux grandeurs comparées selon la quantité : et par conséquent, quatre grandeurs auront proportion ensemble, lorsque la différence de la première à la seconde est égale à la différence de la troisième à la quatrième. Il n'y a donc rien à dire à ces définitions d'Euclide, pourvu qu'il demeure toujours dans ces idées qu'il a désignées par ces mots, et à qui il a donné les noms de \emph{raison} et de \emph{proportion}. Mais il n'y demeure pas, puisque, selon toute la suite de son livre, ces quatre membres 3, 5, 8, 10, ne sont point en proportion, quoique la définition qu'il a donnée au mot de \emph{proportion} leur convienne ; puisqu'il y a entre le premier nombre et le second, comparés selon la quantité, une habitude semblable à celle qui est entre le troisième et le quatrième.

Il fallait donc, pour ne pas tomber dans cet inconvénient, remarquer qu'on peut comparer deux grandeurs en deux manières ; l'une, en considérant de combien l'une surpasse l'autre ; et l'autre, de quelle manière l'une est contenue dans l'autre : et comme ces deux habitudes sont différentes, il fallait leur donner divers noms, donnant à la première le nom de \emph{différence}, et réservant à la seconde le nom de \emph{raison}. Il fallait ensuite définir \emph{la proportion} l'égalité de l'une ou de l'autre de ces sortes d'habitudes, c'est-à-dire de la \emph{différence} ou de la \emph{raison} ; et, comme cela fait deux espèces, les distinguer aussi par deux divers noms, en appelant l'égalité des différences \emph{proportion arithmétique}, et l'égalité des raisons \emph{proportion géométrique}. Et parce que cette dernière est d'un usage beaucoup plus grand que la première, on pouvait encore avertir que lorsque simplement on nomme \emph{proportion}, ou grandeurs proportionnelles, on entend la proportion géométrique, et que l'on n'entend l'arithmétique que quand on l'exprime. Voilà ce qui aurait démêlé toute cette obscurité, et aurait levé toute équivoque.

Tout cela nous fait voir qu'il ne faut pas abuser de cette maxime, que les définitions des mots sont arbitraires ; mais qu'il faut avoir grand soin de désigner si nettement et si clairement l'idée à laquelle on veut lier le mot que l'on définit, qu'on ne puisse s'y tromper dans la suite du discours, en changeant cette idée, c'est-à-dire en prenant le mot en un autre sens que celui qu'on lui a donné par la définition, en sorte qu'on ne puisse substituer la définition en la place du défini, sans tomber dans quelque absurdité.

