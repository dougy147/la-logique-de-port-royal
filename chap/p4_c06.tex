\subsubsection{\centering \Large CHAPITRE VI}
\addcontentsline{toc}{section}{\protect\numberline{}{\scshape\bfseries VI} - \emph{Quelques axiomes importants et qui peuvent servir de principes à de grandes vérités}}
\begin{center}\emph{\large\scshape Quelques axiomes importants et qui peuvent servir de principes à de grandes vérités.}\end{center}

	\lettrine{T}{out} le monde demeure d'accord qu'il est important d'avoir dans l'esprit plusieurs axiomes et principes, qui, étant clairs et indubitables, puissent nous servir de fondement pour connaître les choses les plus cachées ; mais ceux que l'on donne ordinairement sont de si peu d'usage qu'il est assez inutile de les savoir, car ce qu'ils appellent le premier principe de la connaissance, \emph{il est impossible que la même chose soit et ne soit pas}, est très clair et très certain; mais je ne vois point de rencontre où il puisse jamais servir à nous donner aucune connaissance. Je crois donc que ceux-ci pourront être plus utiles. Je commencerai par celui que nous venons d'expliquer.

\begin{center}{\scshape\bfseries 1. Axiome}\end{center}

	\emph{Tout ce qui est renfermé dans l'idée claire et distincte d'une chose peut en être affirmé avec vérité}.

\begin{center}{\scshape\bfseries 2. Axiome}\end{center}

	\emph{L'existence, au moins possible, est renfermée dans l'idée de tout ce que nous concevons clairement et distinctement}.

Car, dès là qu'une chose est conçue clairement, nous ne pouvons pas ne point la regarder comme pouvant être, puisqu'il n'y a que la contradiction qui se trouve entre nos idées qui nous fait croire qu'une chose ne peut être. Or, il ne peut y avoir de contradiction dans une idée, lorsqu'elle est claire et distincte; et par conséquent l'existence au moins possible est enfermée dans l'idée de tout ce que nous concevons clairement et distinctement.

\begin{center}{\scshape\bfseries 3. Axiome}\end{center}

	\emph{Le néant ne peut être cause d'aucune chose}. Il naît d'autres axiomes de celui-ci, qui peuvent en être appelés des corollaires, tels que sont les suivants.


\begin{center}{\scshape\bfseries 4. Axiome \\ou premier corollaire du troisieme}\end{center}

	\emph{Aucune chose ni aucune perfection de cette chose actuellement existante ne peut avoir le néant ou une chose non existante pour cause de son existence}.

\begin{center}{\scshape\bfseries 5. Axiome \\ou deuxieme corollaire du troisieme}\end{center}

	\emph{Toute la réalité ou perfection qui est dans une chose, se rencontre formellement ou éminemment dans sa cause première et totale}.

\newpage

\begin{center}{\scshape\bfseries 6. Axiome \\ou troisieme corollaire du troisieme}\end{center}

	\emph{Nul corps ne peut se mouvoir soi-même}, c'est-à-dire se donner le mouvement n'en ayant point.

Ce principe est si évident naturellement, que c'est ce qui a introduit les formes substantielles et les qualités réelles de pesanteur et de légèreté ; car les philosophes voyant, d'une part, qu'il était impossible que ce qui devait être mû se mût soi-même, et s'étant faussement persuadés, de l'autre, qu'il n'y avait rien hors la pierre qui poussât en bas une pierre qui tombait, ils se sont crus obligés de distinguer deux choses dans une pierre, la matière qui recevait le mouvement, et la forme substantielle aidée de l'accident de la pesanteur qui le donnait; ne prenant pas garde, ou qu'ils tombaient par là dans l'inconvénient qu'ils voulaient éviter, si cette forme était elle-même matérielle, c'est-à-dire une vraie matière; ou que si elle n'était pas matière, ce devait être une substance qui en fût réellement distincte; ce qu'il leur était impossible de concevoir clairement, à moins que de la concevoir comme un esprit, c'est-à-dire une substance qui pense, comme est véritablement la forme de l'homme, et non pas celle de tous les autres corps.

\begin{center}{\scshape\bfseries 7. Axiome \\ou quatrieme corollaire du troisieme}\end{center}

	\emph{Nul corps ne peut en mouvoir un autre, s'il n'est mû lui-même}. Car si un corps étant en repos ne peut se donner le mouvement à soi-même, il peut encore moins le donner à un autre corps.

\bigbreak

\begin{center}{\scshape\bfseries 8. Axiome}\end{center}

	\emph{On ne doit pas nier ce qui est clair et évident pour ne pouvoir comprendre ce qui est obscur}.


\begin{center}{\scshape\bfseries 9. Axiome}\end{center}

	\emph{Il est de la nature d'un esprit fini de ne pouvoir comprendre l'infini}.

\begin{center}{\scshape\bfseries 10. Axiome}\end{center}

	\emph{Le témoignage d'une personne infiniment puissante, infiniment sage, infiniment bonne et infiniment véritable, doit avoir plus de force pour persuader notre esprit que les raisons les plus convaincantes}.

Car nous devons être plus assurés que celui qui est infiniment intelligent ne se trompe pas, et que celui qui est infiniment bon ne nous trompe pas, que nous ne sommes assurés que nous ne nous trompons pas dans les choses les plus claires.

Ces trois derniers axiomes sont le fondement de la foi, de laquelle nous pourrons dire quelque chose plus bas.

\begin{center}{\scshape\bfseries 11. Axiome}\end{center}

	\emph{Les faits dont les sens peuvent juger facilement étant attestés par un très grand nombre de personnes de divers temps, de diverses nations, de divers intérêts, qui en parlent comme les sachant par eux-mêmes, et qu'on ne peut soupçonner d'avoir conspiré ensemble pour appuyer un mensonge, doivent passer pour aussi constants et indubitables que si on les avait vus de ses propres yeux}.

C'est le fondement de la plupart de nos connaissances, y ayant infiniment plus de choses que nous savons par cette voie que ne sont celles que nous savons par nous-mêmes.

