\subsubsection{\centering \Large CHAPITRE V}
\addcontentsline{toc}{section}{\protect\numberline{}{\scshape\bfseries V} - \emph{Règles, modes et fondements de la première figure}}
\begin{center}\emph{\large\scshape Règles, modes et fondements de la première figure.}\end{center}

	\lettrine{L}{a} première figure est donc celle où le moyen est sujet dans la majeure et attribut dans la mineure.

Cette figure n'a que deux règles.

\begin{center}{\bfseries\scshape\large 1. Règle}\end{center}

	\emph{Il faut que la mineure soit affirmative.}

Car si elle était négative, la majeure serait affirmative par la troisième règle générale, et la conclusion négative par la cinquième : donc le grand terme serait pris universellement dans la conclusion, parce qu'elle serait négative, et particulièrement dans la majeure, parce qu'il en est l'attribut dans cette figure, et qu'elle serait affirmative, ce qui serait contre la seconde règle, qui défend de conclure du particulier au général. Cette raison a lieu aussi dans la troisième figure, où le grand terme est aussi attribut dans la majeure.

On peut prouver cette règle par une autre raison plus essentielle, que je ne toucherai qu'en un mot, parce qu'il faudrait un trop long discours pour la rendre intelligible à tout le monde, et que ce que j'en dirai suffira pour la faire comprendre à ceux qui entrent facilement dans ces sortes de choses. Cette raison dépend principalement de cet axiome établi dans la deuxième partie. \emph{La proposition négative ne sépare pas du sujet chaque partie contenue dans la compréhension de l'attribut, mais elle en sépare seulement l'idée totale et entière composée de tous les attributs qu'il contient}. Car afin qu'une chose ne soit pas une autre, il n'est pas nécessaire qu'elle n'ait rien de commun avec elle, mais il suffit qu'elle n'ait pas tout ce que l'autre a.

On peut juger par là, que de ce que le petit terme n'est pas le moyen (qui est ce que fait voir la mineure négative), il ne s'ensuit pas que le grand terme étant contenu dans la compréhension du moyen (qui est tout ce que peut faire voir la majeure affirmative, dans la première et la troisième figure, où le moyen en est le sujet, et le grand terme l'attribut) il ne s'ensuit pas, dis-je, que le grand terme ne puisse convenir au petit terme. De ce qu'un cheval n'est pas un lion, (c'est la mineure) et qu'un lion est un animal, (c'est la majeure dans la première figure) on n'en peut pas inférer qu'un cheval n'est pas un animal.

\begin{center}{\bfseries\scshape\large 2. Règle}\end{center}

	\emph{La majeure doit être universelle}.

Car la mineure étant affirmative par la règle précédente, le moyen qui y est attribut, y est pris particulièrement : donc il doit être universel dans la majeure où il est sujet, ce qui la rend universelle; autrement il serait pris deux fois particulièrement contre la première règle générale.

\begin{center}{\bfseries Démonstration}\end{center}

\emph{Qu'il ne peut y avoir que quatre modes de la première figure.}

On a fait voir dans le chapitre précédent, qu'il ne peut y avoir que dix modes concluants. Mais de ces dix modes A, E, E, et A, O, O, sont exclus par la première règle de cette figure, qui est que la mineure doit être affirmative.

I, A, I, et O, A, O, sont exclus par la deuxième, qui est que la majeure doit être universelle.

A, A, I, et E, A, O, sont exclus par le sixième corollaire des règles générales. Car le petit terme étant sujet dans la mineure, elle ne peut être universelle que la conclusion ne puisse l'être aussi.

Et par conséquent, il ne reste que ces quatre modes :


\begin{center}
$ \text {2 Affirmatifs} \left \{
    \begin{array}{ccc}
	    \text {A,} & \text{A,} & \text{A} \\
  	    \text {A,} & \text{I,} & \text{I} \\
    \end{array}
	    \right \} $
\end{center}
\begin{center}
$ \text {2 Négatifs} \left \{
    \begin{array}{ccc}
	    \text {E,} & \text{A,} & \text{E} \\
	    \text {E,} & \text{I,} & \text{O} \\
    \end{array}
	    \right \} $
\end{center}


Ce qu'il fallait démontrer.

Ces quatre modes, pour être plus facilement retenus, ont été réduits à des mots artificiels, dont les trois syllabes marquent les trois propositions, et la voyelle de chaque syllabe marque quelle doit être cette proposition ; de sorte que ces mots ont cela de très commode dans l'École, qu'on marque clairement par un seul mot une espèce de syllogisme, que sans cela on ne pourrait faire entendre qu'avec beaucoup de discours.

\begin{center}
	\begin{tabularx}{\textwidth}{lX}
		{\large\scshape Bar—} & \emph{Tout sage est soumis à la volonté de Dieu:}            \\
		{\large\scshape Ba—} & \emph{Tout homme de bien est sage:}                          \\
		{\large\scshape Ra.} & \emph{Donc tout homme de bien et soumis à la volonté de Dieu.} \\
		{\large\scshape Ce—} & \emph{Nul péché n'est louable:}                              \\
		{\large\scshape La—} & \emph{Toute vengeance est péché:}                            \\
		{\large\scshape Rent.} & \emph{Donc nulle vengeance n'est louable.}                   \\
		{\large\scshape Da—} & \emph{Tout ce qui est au salut est avantageux:}             \\
		{\large\scshape Ri—} & \emph{Il y a des afflictions qui servent au salut:}          \\
		{\large\scshape I.} & \emph{Donc il y a des afflictions qui servent au salut.}     \\
		{\large\scshape Fe—} & \emph{Ce qui est suivi d'un juste repentir n'est jamais à souhaiter:} \\
		{\large\scshape Ri—} & \emph{Il y a des plaisir qui sont suivis d'un juste repentir:} \\
		{\large\scshape O.} & \emph{Donc il y a des plaisirs qui ne sont pas à souhaiter.} \\
	\end{tabularx}
\end{center}

\bigbreak
\bigbreak
\bigbreak

\begin{center}{\bfseries Fondement de la première figure}\end{center}

Puisque dans cette figure le grand terme est affirmé ou nié du moyen pris universellement, et ce même moyen affirmé ensuite dans la mineure du petit terme, ou sujet de la conclusion, il est clair qu'elle n'est fondée que sur deux principes ; l'un pour les modes affirmatifs, l'autre pour les modes négatifs.

\begin{center}{\bfseries Principe des modes affirmatifs}\end{center}

\emph{Ce qui convient à une idée prise universellement, convient aussi à tout ce dont cette idée est affirmée, ou qui est sujet de cette idée, ou qui est compris dans l'extension de cette idée} : car ces expressions sont synonymes.

Ainsi, l'idée \emph{d'animal} convenant à tous les hommes, convient aussi à tous les Éthiopiens. Ce principe a été tellement éclairci dans le chapitre où nous avons traité de la nature des propositions affirmatives, qu'il n'est pas nécessaire de l'éclaircir ici davantage. Il suffira d'avertir qu'on l'exprime ordinairement dans l'École en cette manière : \emph{Quod convenit consequenti, convenit antecedenti}. Et que l'on entend par terme conséquent une idée générale qui est affirmée d'une autre, et par antécédent le sujet dont elle est affirmée.

\bigbreak
\bigbreak
\begin{center}{\bfseries Principe des modes négatifs}\end{center}

\emph{Ce qui est nié d'une idée prise universellement, est nié de tout ce dont cette idée est affirmée.}

\emph{Arbre} est nié de tous les animaux ; il est donc nié de tous les hommes, parce qu'ils sont animaux. On l'exprime ainsi dans l'école : \emph{Quod negatur de consequenti, negatur de antecedenti}. Ce que nous avons dit en traitant des propositions négatives, me dispense d'en parler ici davantage.

\bigbreak
Il faut remarquer qu'il n'y a que la première figure qui conclue tout A, E, I, O.

Et qu'il n'y a qu'elle aussi qui conclue A, dont la raison est, qu'afin que la conclusion soit universelle affirmative, il faut que le petit terme soit pris généralement dans la mineure, et par conséquent qu'il en soit sujet, et que le moyen en soit l'attribut : d'où il arrive que le moyen y est pris particulièrement ; il faut donc qu'il soit pris généralement dans la majeure (par la première règle générale), et que par conséquent il en soit le sujet. Or c'est en cela que consiste la première figure, que le moyen y est sujet en la majeure, et attribut en la mineure.

