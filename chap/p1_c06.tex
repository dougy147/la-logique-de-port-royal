\subsubsection{\centering \Large CHAPITRE VI}
\addcontentsline{toc}{section}{\protect\numberline{}{\scshape\bfseries VI} - \emph{Des cinq sortes d'idées universelles, genres, espèces, différences, propres, accidents}}
\begin{center}\emph{\large\scshape Des cinq sortes d'idées universelles, genres, espèces, différences, propres, accidents.}\end{center}

	\lettrine{C}{e} que nous avons dit dans les chapitres précédents nous donne moyen de faire entendre en peu de paroles les cinq Universaux qu'on explique ordinairement dans l'École.

Car lorsque les idées générales nous représentent leurs objets comme des choses, et qu'elles sont marquées par des termes appelés substantifs ou absolus, on les appelle \emph{genres} ou \emph{espèces}.

\begin{center}{\large\scshape Du Genre}\end{center}

On les appelle genres quand elles sont tellement communes, qu'elles s'étendent à d'autres idées qui sont encore universelles, comme le quadrilatère est genre à l'égard du parallélogramme et du trapèze : la substance est genre à l'égard de la substance étendue qu'on appelle corps, et de la substance qui pense qu'on appelle esprit.

\begin{center}{\large\scshape De l'Espèce}\end{center}

Et ces idées communes, qui sont sous une plus commune et plus générale, s'appellent espèces ; comme le parallélogramme et le trapèze sont les espèces du quadrilatère, le corps et l'esprit sont les espèces de la substance.

Et ainsi la même idée peut être genre étant comparée aux idées auxquelles elle s'étend, et espèce, étant comparée à une autre qui est plus générale, comme corps, qui est un genre au regard du corps animé et du corps inanimé, et une espèce au regard de la substance; et le quadrilatère, qui est un genre au regard du parallélogramme et du trapèze, est une espèce au regard de la figure.

Mais il y a une autre notion du mot d'espèce, qui ne convient qu'aux idées qui ne peuvent être genres; c'est lorsqu'une idée n'a sous soi que des individus et des singuliers, comme le cercle n'a sous soi que des cercles singuliers qui sont tous d'une même espèce. C'est ce qu'on appelle espèce dernière, \emph{species infima}.

Et il y a un genre qui n'est point espèce ; à savoir, le suprême de tous les genres, soit que ce genre soit l'être, soit que ce soit la substance, ce qu'il est de peu d'importance de savoir, et qui regarde plus la métaphysique que la logique.

J'ai dit que les idées générales qui nous représentent leurs objets comme des choses, sont appelées genres ou espèces. Car il n'est pas nécessaire que les objets de ces idées soient effectivement des choses et des substances ; mais il suffit que nous les considérions comme des choses, en ce que, lors même que ce sont des modes, on ne les rapporte point à leurs substances, mais à d'autres idées de modes moins générales ou plus générales, comme la figure, qui n'est qu'un mode au regard du corps figuré, est un genre au regard des figures curvilignes et rectilignes, etc.

Et au contraire les idées qui nous représentent leurs objets comme des choses modifiées, et qui sont marquées par des termes adjectifs ou connotatifs, si on les compare avec les substances que ces termes connotatifs signifient confusément, quoique directement, soit que dans la vérité ces termes connotatifs signifient des attributs essentiels, qui ne sont en effet que la chose même, soit qu'ils signifient de vrais modes, on ne les appelle point alors genres ni espèces, mais, ou \emph{différences}, ou \emph{propres}, ou \emph{accidents}.

On les appelle \emph{différences}, quand l'objet de ces idées est un attribut essentiel qui distingue une espèce d'une autre; comme étendu, pendant, raisonnable.

On les appelle \emph{propres}, quand leur objet est un attribut qui appartient en effet à l'essence de la chose, mais qui n'est pas le premier que l'on considère dans cette essence, mais seulement une dépendance de ce premier, comme divisible, immortel, docile.

Et on les appelle \emph{accidents communs} quand leur objet est un vrai mode qui peut être séparé, au moins par l'esprit, de la chose dont il est dit accident; sans que l'idée de cette chose soit détruite dans notre esprit, comme rond, dur, juste, prudent. C'est ce qu'il faut expliquer plus particulièrement.

\begin{center}{\large\scshape De la différence}\end{center}

Lorsqu'un genre a deux espèces, il faut nécessairement que l'idée de chaque espèce comprenne quelque chose qui ne soit pas compris dans l'idée du genre; autrement, si chacune ne comprenait que ce qui est compris dans le genre, ce ne serait que le genre ; et comme le genre convient à chaque espèce, chaque espèce conviendrait à l'autre. Ainsi le premier attribut essentiel que comprend chaque espèce de plus que le genre, s'appelle sa différence; et l'idée que nous en avons est une idée universelle, parce qu'une seule et même idée peut nous représenter cette différence partout où elle se trouve, c'est-à-dire dans tous les inférieurs de l'espèce.

\emph{Exemple}. Le corps et l'esprit sont les deux espèces de la substance. Il faut donc qu'il y ait dans l'idée du corps quelque chose de plus que dans celle de la substance, et de même dans celle de l'esprit. Or, la première chose que nous voyons de plus dans le corps, C'est l'étendue ; et la première chose que nous voyons de plus dans l'esprit, c'est la pensée. Et ainsi la différence du corps sera l'étendue, et la différence de l'esprit sera la pensée, c'est-à-dire que le corps sera une substance étendue, et l'esprit une substance qui pense.

De là on peut voir premièrement que la différence a deux respects, l'un au genre qu'elle divise et partage, l'autre à l'espèce qu'elle constitue et qu'elle forme, faisant la principale partie de ce qui est enfermé dans l'idée de l'espèce selon sa compréhension : d'où vient que toute espèce peut être exprimée par un seul nom, comme esprit, corps ; ou par deux mots, à savoir, par celui du genre, et par celui de sa différence joints ensemble, ce qu'on appelle définition, comme substance qui pense, substance étendue.

On peut voir en second lieu que, puisque la différence constitue l'espèce et la distingue des autres espèces, elle doit avoir la même étendue que l'espèce, et ainsi qu'il faut qu'elles puissent se dire réciproquement l'une de l'autre, comme tout ce qui pense est esprit, et tout ce qui est esprit pense.

Néanmoins il arrive assez souvent que l'on ne voit dans certaines choses aucun attribut qui soit tel, qu'il convienne à toute une espèce, et qu'il ne convienne qu'à cette espèce ; et alors on joint plusieurs attributs ensemble, dont l'assemblage ne se trouvant que dans cette espèce, en constitue la différence. Ainsi les Platoniciens, prenant les démons pour des animaux raisonnables aussi bien que l'homme, ne trouvaient pas que la différence de raisonnable fût réciproque à l'homme; c'est pourquoi ils y en ajoutaient une autre, comme mortel, qui n'est pas non plus réciproque à l'homme, puisqu'elle convient aux bêtes; mais toutes deux ensemble ne conviennent qu'à l'homme. C'est ce que nous faisons dans l'idée que nous nous formons de la plupart des animaux.

Enfin, il faut remarquer qu'il n'est pas toujours nécessaire que les deux différences qui partagent un genre soient toutes deux positives, mais que c'est assez qu'il y en ait une, comme deux hommes sont distingués l'un de l'autre, si l'un a une charge que l'autre n'a pas, quoique celui qui n'a pas de charge n'ait rien que l'autre n'ait. C'est ainsi que l'homme est distingué des bêtes en général, en ce que l'homme est un animal qui a un esprit, \emph{animal mente praeditum}, et que la bête est un pur animal, \emph{animal merum}. Car l'idée de la bête en général n'enferme rien de positif qui ne soit dans l'homme ; mais on y joint seulement la négation de ce qui est en l'homme, savoir, l'esprit. De sorte que toute la différence qu'il y a entre l'idée d'animal et celle de bête est que l'idée d'animal n'enferme pas la pensée dans sa compréhension, mais ne l'exclut pas aussi et l'enferme même dans son étendue, parce qu'elle convient à un animal qui pense ; au lieu que l'idée de bête l'exclut dans sa compréhension, et ainsi ne peut convenir à l'animal qui pense.

\bigbreak
\bigbreak
\bigbreak

\begin{center}{\large\scshape Du propre}\end{center}

Quand nous avons trouvé la différence qui constitue une espèce, c'est-à-dire son principal attribut essentiel qui la distingue de toutes les autres espèces, si, considérant plus particulièrement sa nature, nous y trouvons encore quelque attribut qui soit nécessairement lié avec ce premier attribut, et qui par conséquent convienne à toute cette espèce et à cette seule espèce, \emph{omni et soli}, nous l'appelons propriété: et étant signifié par un terme connotatif, nous l'attribuons à l'espèce comme son propre; et parce qu'il convient aussi à tous les inférieurs de l'espèce, et que la seule idée que nous en avons une fois formée peut représenter cette propriété partout où elle se trouve, on en a fait la quatrième des termes communs et universaux.

\emph{Exemple}. Avoir un angle droit est la différence essentielle du triangle rectangle. Et parce que c'est une dépendance nécessaire de l'angle droit que le carré du côté qui le soutient soit égal aux carrés des deux côtés qui le comprennent, l'égalité de ces carrés est considérée comme la propriété du triangle rectangle, qui convient à tous les triangles rectangles, et qui ne convient qu'à eux seuls.

Néanmoins on a quelquefois étendu plus loin ce nom de propre, et on en a fait quatre espèces.

La première est celle que nous venons d'expliquer, \emph{quod convenit omni, soli, et semper}; comme c'est le propre de tout cercle, du seul cercle, et toujours, que les lignes tirées du centre à la circonférence soient égales.

La deuxième \emph{quod convenit omni, sed non soli}, comme on dit qu'il est propre à l'étendue d'être divisible, parce que toute étendue peut être divisée, quoique la durée, le nombre et la force le puissent être aussi.

La troisième est, \emph{quod convenit soli, sed non omni}, comme il ne convient qu'à l'homme d'être médecin ou philosophe, quoique tous les hommes ne le soient pas.

La quatrième \emph{quod convenit omni et soli, sed non semper}, dont on rapporte pour exemple le changement de la couleur du poil en blanc, \emph{canescere}; ce qui convient à tous les hommes et aux seuls hommes, mais seulement dans la vieillesse.

\begin{center}{\large\scshape De l'accident}\end{center}

Nous avons déjà dit dans le Chapitre second qu'on appelait mode ce qui ne pouvait exister naturellement que par la substance, et ce qui n'était point nécessairement lié avec l'idée d'une chose, en sorte qu'on peut bien concevoir la chose sans concevoir le mode, comme on peut bien concevoir un homme sans le concevoir prudent; mais on ne peut concevoir la prudence sans concevoir, ou un homme, ou une autre nature intelligente qui soit prudente.

Or, quand on joint une idée confuse et indéterminée de substance avec une idée distincte de quelque mode, cette idée est capable de représenter toutes les choses où sera ce mode, comme l'idée de prudent, tous les hommes prudents ; l'idée de rond, tous les corps ronds; et alors cette idée, exprimée par un terme connotatif \emph{prudent}, \emph{rond}, est ce qui fait le cinquième universel qu'on appelle accident, parce qu'il n'est pas essentiel à la chose à qui on l'attribue; car s'il l'était, il serait différence ou propre.

Mais il faut remarquer ici, comme on l'a déjà dit, que, quand on considère deux substances ensemble, on peut en considérer une comme mode de l'autre. Ainsi un homme habillé peut être considéré comme un tout composé de cet homme et de ses habits ; mais être habillé au regard de cet homme, est seulement un mode ou une façon d'être sous laquelle on le considère, quoique ses habits soient des substances. C'est pourquoi être habillé n'est qu'un cinquième universel.

En voilà plus qu'il n'en faut touchant les cinq universaux qu'on traite dans l'école avec tant d'étendue ; car il sert de très peu de savoir qu'il y a des genres, des espèces, des différences, des propres et des accidents; mais l'importance est de reconnaître les vrais genres des choses, les vraies espèces de chaque genre, leurs vraies différences, leurs vraies propriétés, et les accidents qui leur conviennent ; et c'est à quoi nous pourrons donner quelque lumière dans les chapitres suivants, après avoir dit auparavant quelque chose des termes complexes.
