\subsubsection{\centering \Large CHAPITRE VIII}
\addcontentsline{toc}{section}{\protect\numberline{}{\scshape\bfseries VIII} - \emph{De quelques défauts qui se rencontrent d'ordinaire dans la méthode des géomètres}}
\begin{center}\emph{\large\scshape De quelques défauts qui se rencontrent d'ordinaire dans la méthode des géomètres.}\end{center}

	\lettrine{N}{ous} avons vu ce que la méthode des géomètres a de bon, que nous avons réduit à cinq règles qu'on ne peut trop avoir dans l'esprit; et il faut avouer qu'il n'y a rien de plus admirable que d'avoir découvert tant de choses si cachées, et les avoir démontrées par des raisons si fermes et si invincibles, en se servant de si peu de règles. De sorte qu'entre tous les philosophes ils ont eux seuls cet avantage d'avoir banni de leur école et de leurs livres la contestation et la dispute, faisant profession de ne rien avancer qui ne soit convaincant et incontestable.

Néanmoins, si l'on veut juger des choses sans préoccupation, comme on ne peut leur ôter la gloire d'avoir suivi une voie beaucoup plus assurée que tous les autres pour trouver la vérité, on ne peut nier aussi qu'ils ne soient tombés en quelques défauts qui ne les détournent pas de leur fin, mais qui font seulement qu'ils n'y arrivent pas par la voie la plus droite et la plus commode; c'est ce que je tâcherai de montrer, en tirant d'Euclide même les exemples de ces défauts.

\begin{center}{\bfseries\scshape I. Défaut}\end{center}

	\emph{Avoir plus de soin de la certitude que de l'évidence, et de convaincre l'esprit que de l'éclairer}.

Les géomètres sont louables de n'avoir rien voulu avancer que de convaincant; mais il semble qu'ils n'ont pas assez pris garde qu'il ne suffit pas, pour avoir une parfaite science de quelque vérité d'être convaincu que cela est vrai, si de plus on ne pénètre, par des raisons prises de la nature de la chose même, pourquoi cela est vrai; car, jusqu'à ce que nous soyons arrivés à ce point-là, notre esprit n'est point pleinement satisfait, et cherche encore une plus grande connaissance que celle qu'il a : ce qui est une marque qu'il n'a point encore la vraie science. On peut dire que ce défaut est la source de presque tous les autres que nous remarquerons, et ainsi il n'est pas nécessaire de l'expliquer davantage, parce que nous le ferons assez dans la suite.

\begin{center}{\bfseries\scshape II. Défaut}\end{center}

	\emph{Prouver des choses qui n'ont pas besoin de preuves}.

Les géomètres avouent qu'il ne faut pas s'arrêter à vouloir prouver ce qui est clair de soi-même. Ils le font néanmoins souvent, parce que, s'étant plus attachés à convaincre l'esprit qu'à l'éclairer, comme nous venons de dire, ils croient qu'ils le convaincront mieux en trouvant quelque preuve des choses même les plus évidentes, qu'en les proposant simplement, et laissant à l'esprit d'en reconnaître l'évidence.

C'est ce qui a porté Euclide à prouver que les deux côtés d'un triangle pris ensemble sont plus grands qu'un seul, quoique cela soit évident par la seule notion de la ligne droite, qui est la plus courte longueur qui puisse se donner entre deux points, et la mesure naturelle de la distance d'un point à un point, ce qu'elle ne serait pas, si elle n'était aussi la plus courte de toutes les lignes qui puissent être tirées d'un point à un point.

C'est ce qui l'a encore porté à ne pas faire une demande, mais un problème qui doit être démontré, de \emph{tirer une ligne égale à une ligne donnée}, quoique cela soit aussi facile et plus facile que de faire un cercle ayant un rayon donné.

Ce défaut est venu, sans doute, de n'avoir pas considéré que toute la certitude et l'évidence de nos connaissances dans les sciences naturelles vient de ce principe : \emph{Qu'on peut assurer d'une chose, tout ce qui est contenu dans une idée claire et distincte}. D'où il s'ensuit que si nous n'avons besoin, pour connaître qu'un attribut est renfermé dans une idée, que de la simple considération de l'idée, sans y en mêler d'autres, cela doit passer pour évident et pour clair, comme nous avons déjà dit plus haut.

Je sais bien qu'il y a de certains attributs qui se voient plus facilement dans les idées que les autres. Mais je crois qu'il suffit qu'ils puissent s'y voir clairement avec une médiocre attention, et que nul homme qui aura l'esprit bien fait n'en puisse douter sérieusement, pour regarder les propositions qui se tirent ainsi de la simple considération des idées, comme des principes qui n'ont point besoin de preuves, mais au plus d'explication et d'un peu de discours. Ainsi, je soutiens qu'on ne peut faire un peu d'attention sur l'idée d'une ligne droite, qu'on ne conçoive non seulement que sa position ne dépend que de deux points (ce qu'Euclide a pris pour une de ses demandes) mais qu'on ne comprenne aussi sans peine et très clairement que si deux points d'une ligne sont également distants d'une autre ligne prolongée s'il est besoin, tous les autres points en seront aussi également distants.

D'où il s'ensuit qu'après avoir montré ce qui est facile, que la distance d'un point à une ligne se mesure par la perpendiculaire; et que deux lignes sont appelées équidistantes et parallèles, lorsque tous les points de chacune sont également distants de l'autre prolongée s'il est besoin, il suffira d'avoir trouvé dans l'une deux points également distants de l'autre, pour en conclure que tous les autres le sont aussi, et qu'ainsi elles sont parallèles.

Il est aussi à remarquer que d'excellents géomètres emploient pour principes des propositions moins claires que celles-là ; comme lorsque Archimède a établi ses plus belles démonstrations sur cet axiome : \emph{Que si deux lignes sur le même plan ont les extrémités communes, et sont courbées ou creuses vers la même part, celle qui est contenue sera moindre que celle qui la contient}.

J'avoue que ce défaut de prouver ce qui n'a pas besoin de preuves ne paraît pas grand, et qu'il ne l'est pas aussi en soi; mais il l'est beaucoup dans les suites, parce que c'est de là que naît ordinairement le renversement de l'ordre naturel dont nous parlerons plus bas ; cette envie de prouver ce qui devait être supposé comme clair et évident de soi-même, ayant souvent obligé les géomètres de traiter des choses pour servir de preuves à ce qu'ils n'auraient point dû prouver, qui ne devraient être traitées qu'après, selon l'ordre de la nature.


\begin{center}{\bfseries\scshape III. Défaut}\end{center}

	\emph{Démonstration par l'impossible}.

Ces sortes de démonstrations qui montrent qu'une chose est telle, non par ses principes, mais par quelque absurdité qui s'ensuivrait si elle était autrement, sont très ordinaires dans Euclide. Cependant il est visible qu'elles peuvent convaincre l'esprit, mais qu'elles ne l'éclairent point; ce qui doit être le principal fruit de la science : car notre esprit n'est point satisfait, s'il ne sait non seulement que la chose est, mais pourquoi elle est : ce qui ne s'apprend point par une démonstration qui réduit à l'impossible.

Ce n'est pas que ces démonstrations soient tout à fait à rejeter ; car on peut quelquefois s'en servir pour prouver des négatives qui ne sont proprement que des corollaires d'autres propositions, ou claires d'elles-mêmes, ou démontrées auparavant par une autre voie ; et alors cette sorte de démonstration, en réduisant à l'impossible, tient plutôt lieu d'explication que d'une démonstration nouvelle.

Enfin, on peut dire que ces démonstrations ne sont recevables que quand on n'en peut donner d'autres ; et que c'est une faute de s'en servir pour prouver ce qui peut se prouver positivement : or, il y a beaucoup de propositions dans Euclide qu'il ne prouve que par cette voie, qui peuvent se prouver autrement sans beaucoup de difficulté.

\begin{center}{\bfseries\scshape IV. Défaut}\end{center}

	\emph{Démonstrations tirées par des voies trop éloignées}.

Ce défaut est très commun parmi les géomètres. Ils ne se mettent pas en peine d'où les preuves qu'ils apportent sont prises, pourvu qu'elles soient convaincantes ; et cependant ce n'est que prouver les choses très imparfaitement, que de les prouver par des voies étrangères, d'où elles ne dépendent point selon leur nature.

C'est ce qu'on comprendra mieux par quelques exemples. Euclide, Livre Premier, Proposition 5, prouve qu'un triangle isocèle a les deux angles sur la base égaux en prolongeant également les côtés du triangle, et faisant de nouveaux triangles qu'il compare les uns avec les autres.

Mais il n'est pas incroyable qu'une chose aussi facile à prouver que l'égalité de ces angles, ait besoin de tant d'artifice pour être prouvée, comme s'il y avait rien de plus ridicule que de s'imaginer que cette égalité dépendit de ces triangles étrangers ; au lieu qu'en suivant le vrai ordre, il y a plusieurs voies très faciles, très courtes, et très naturelles pour prouver cette même égalité.

La quarante-septième du premier livre, où il est prouvé que le carré de la base qui soutient un angle droit est égal aux deux carrés des côtés, est une des plus estimées propositions d'Euclide ; et néanmoins il est assez clair que la manière dont elle y est prouvée n'est point naturelle, puisque l'égalité de ces carrés ne dépend point de l'égalité des triangles qu'on prend pour moyen de cette démonstration, mais de la proportion des lignes, qu'il est aisé de démontrer sans se servir d'aucune autre ligne que de la perpendiculaire du sommet de l'angle droit sur la base.

Tout Euclide est plein de ces démonstrations par des voies étrangères.

\begin{center}{\bfseries\scshape V. Défaut}\end{center}

	\emph{N'avoir aucun soin du vrai ordre de la nature}.

C'est ici le plus grand défaut des géomètres. Ils se sont imaginé qu'il n'y avait presque aucun ordre à garder, sinon que les premières propositions pussent servir à démontrer les suivantes ; et ainsi, sans se mettre en peine des règles de la véritable méthode, qui est de commencer toujours par les choses les plus simples et les plus générales, pour passer ensuite aux plus composées et aux plus particulières, ils brouillent toutes choses, et traitent pêle-mêle les lignes et les surfaces, les triangles et les carrés, prouvent, par des figures, les propriétés des lignes simples, et font une infinité d'autres renversements qui défigurent cette belle science.

Les éléments d'Euclide sont tout pleins de ce défaut. Après avoir traité de l'étendue dans les quatre premiers livres, il traite généralement des proportions de toutes sortes de grandeurs dans le cinquième. Il reprend l'étendue dans le sixième, et traite des nombres dans les septième, huitième et neuvième, pour recommencer au dixième à parler de l'étendue. Voilà pour le désordre général ; mais il est rempli d'une infinité d'autres particuliers. Il commence le premier livre par la construction d'un triangle équilatère; et vingt-deux propositions après, il donne le moyen général de faire tout triangle de trois lignes droites données, pourvu que les deux soient plus grandes qu'une seule ; ce qui emporte la construction particulière d'un triangle équilatère sur une ligne donnée.

Il ne prouve rien des lignes perpendiculaires et des parallèles que par des triangles. Il mêle la dimension des surfaces à celles des lignes.

Il prouve, Livre I, proposition 16, que le côté d'un triangle étant prolongé, l'angle extérieur est plus grand que l'un ou l'autre des opposés intérieurement; et seize propositions plus bas, il prouve que cet angle extérieur est égal aux deux opposés.

Il faudrait transcrire tout Euclide pour donner tous les exemples qu'on pourrait apporter de ce désordre.


\begin{center}{\bfseries\scshape VI. Défaut}\end{center}

	\emph{Ne point se servir de divisions et de partitions}.

C'est encore un autre défaut dans la méthode des géomètres, de ne point se servir de divisions et de partitions. Ce n'est pas qu'ils ne marquent toutes les espèces de genres qu'ils traitent ; mais c'est simplement en définissant les termes, et mettant toutes les définitions de suite, sans marquer qu'un genre a tant d'espèces, et qu'il ne peut pas en avoir davantage, parce que l'idée générale du genre ne peut recevoir que tant de différences, ce qui donne beaucoup de lumière pour pénétrer la nature du genre et des espèces.

Par exemple, on trouvera dans le premier livre d'Euclide les définitions de toutes les espèces de triangles : mais qui doute que ce ne fût une chose bien plus claire de dire ainsi :

Le triangle peut se diviser selon les côtés, ou selon les angles.

Car les côtés sont
\begin{center}\noindent	{\tiny
$ \text {ou} \left  \{
    \begin{array}{lcl}
	    \text {tous égaux, et il s'appelle} & \text{:} & \text{Équilatère.} \\
	    \text {deux seulement égaux, et il s'appelle} & \text{:} & \text{Isocèle.} \\
	    \text {tous trois inégaux, et il s'appelle} & \text{:} & \text{Scalène.} \\
    \end{array}
\text {} \right \}
    \begin{array}{ll}
    \end{array}$
	}
\end{center}

Les angles sont

\begin{center}\noindent	{\tiny
$ \text {ou} \left  \{
    \begin{array}{ll}
	    \text {tous trois aigus, et il s'appelle : Oxigone.} \\
	    \text {deux seulement aigus, et alors le troisième est } \\
    \end{array}
\text {} \right \}
    \begin{array}{ll}
    \end{array}$
}
\end{center}

\begin{center}\noindent	{\tiny
$ \text {ou} \left  \{
    \begin{array}{lcl}
	    \text {droit, et il s'appelle} & \text{:} & \text{Rectangle.} \\
	    \text {obtus, et il s'appelle} & \text{:} & \text{Amblygone.} \\
    \end{array}
\text {} \right \}
    \begin{array}{ll}
    \end{array}$
	}
\end{center}


Il est même beaucoup mieux de ne donner cette division du triangle, qu'après avoir expliqué et démontré toutes les propriétés du triangle en général ; d'où l'on aura appris qu'il faut nécessairement que deux angles au moins du triangle soient aigus, parce que les trois ensemble ne sauraient valoir plus de deux droits.

Ce défaut retombe dans celui de l'ordre, qui ne voudrait pas qu'on traitât ni même qu'on définît les espèces qu'après avoir bien connu le genre, surtout quand il y a beaucoup de choses à dire du genre, qui peut être expliqué sans parler des espèces.

