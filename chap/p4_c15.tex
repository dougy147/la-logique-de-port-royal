\subsubsection{\centering \Large CHAPITRE XV}
\addcontentsline{toc}{section}{\protect\numberline{}{\scshape\bfseries XV} - \emph{Du jugement que l'on doit faire des accidents futurs}}
\begin{center}\emph{\large\scshape Du jugement que l'on doit faire des accidents futurs.}\end{center}

	\lettrine{C}{es} règles qui servent à juger des faits passés, peuvent facilement s'appliquer aux faits à venir : car, comme l'on doit croire probablement qu'un fait est arrivé, lorsque les circonstances certaines que l'on connaît sont ordinairement jointes avec ce fait, on doit croire aussi probablement qu'il arrivera, lorsque les circonstances présentes sont telles, qu'elles sont ordinairement suivies d'un tel effet. C'est ainsi que les médecins peuvent juger du bon ou du mauvais succès des maladies, les capitaines, des événements futurs d'une guerre, et que l'on juge dans le monde de la plupart des affaires contingentes.

Mais, à l'égard des accidents où l'on a quelque part, et que l'on peut, ou procurer ou empêcher en quelque sorte par ses soins, en s'y exposant ou en les évitant, il arrive à bien des gens de tomber dans une illusion qui est d'autant plus trompeuse qu'elle leur paraît plus raisonnable. C'est qu'ils ne regardent que la grandeur et la conséquence de l'avantage qu'ils souhaitent, ou de l'inconvénient qu'ils craignent, sans considérer en aucune sorte l'apparence et la probabilité qu'il y a que cet avantage ou cet inconvénient arrive, ou n'arrive pas.

Ainsi, lorsque c'est quelque grand mal qu'ils appréhendent, comme la perte de la vie ou de tout leur bien, ils croient qu'il est de la prudence de ne négliger aucune précaution pour s'en garantir ; et si c'est quelque grand bien, comme le gain de cent mille écus, ils croient que c'est agir sagement que de tâcher de l'obtenir si le hasard en coûte peu, quelque peu d'apparence qu'il y ait qu'on y réussisse.

C'est par un raisonnement de cette sorte qu'une princesse ayant ouï dire que des personnes avaient été accablées par la chute d'un plancher, ne voulait jamais ensuite entrer dans une maison, sans l'avoir fait visiter auparavant ; et elle était tellement persuadée qu'elle avait raison, qu'il lui semblait que tous ceux qui agissaient autrement étaient imprudents.

C'est aussi l'apparence de cette raison qui engage diverses personnes en des précautions incommodes et excessives pour conserver leur santé. C'est ce qui en rend d'autres défiantes jusqu'à l'excès dans les plus petites choses, parce qu'ayant été quelquefois trompées, elles s'imaginent qu'elles le seront de même dans toutes les autres affaires : c'est ce qui attire tant de gens aux loteries : gagner, disent-ils, vingt mille écus pour un écu, n'est-ce pas une chose bien avantageuse ? Chacun croit être cet heureux à qui le gros lot arrivera; et personne ne fait réflexion que s'il est, par exemple, de vingt mille écus, il sera peut-être trente mille fois plus probable pour chaque particulier qu'il ne l'obtiendra pas, que non pas qu'il l'obtiendra.

Le défaut de ces raisonnements est que, pour juger de ce que l'on doit faire pour obtenir un bien, ou pour éviter un mal, il ne faut pas seulement considérer le bien et le mal en soi, mais aussi la probabilité qu'il arrive ou n'arrive pas, et regarder géométriquement la proportion que toutes ces choses ont ensemble; ce qui peut être éclairci par cet exemple.

Il y a des jeux où dix personnes mettant chacune un écu, il n'y en a qu'une qui gagne le tout, et toutes les autres perdent ; ainsi chacun des joueurs n'est au hasard que de perdre un écu, et peut en gagner neuf. Si l'on ne considérait que le gain et la perte en soi, il semblerait que tous y ont de l'avantage ; mais il faut de plus considérer que si chacun peut gagner neuf écus, et n'est au hasard que d'en perdre un, il est aussi neuf fois plus probable, à l'égard de chacun, qu'il perdra son écu et ne gagnera pas les neuf. Ainsi, chacun a pour soi neuf écus à espérer, un écu à perdre, neuf degrés de probabilité de perdre un écu, et un seul de gagner les neuf écus; ce qui met la chose dans une parfaite égalité.

Tous les jeux qui sont de cette sorte sont équitables, autant que les jeux peuvent l'être, et ceux qui sont hors de cette proportion sont manifestement injustes : et c'est par là qu'on peut faire voir qu'il y a une injustice évidente dans ces espèces de jeux qu'on appelle loteries, parce que le maître de loterie prenant d'ordinaire sur le tout une dixième partie pour son préciput, tout le corps des joueurs est dupé de la même manière que si un homme jouait à un jeu égal, c'est-à-dire, où il y a autant d'apparence de gain que de perte, dix pistoles contre neuf. Or, si cela est désavantageux à tout le corps, cela l'est aussi à chacun de ceux qui le composent, puisqu'il arrive de là que la probabilité de la perte surpasse plus la probabilité du gain, que l'avantage qu'on espère ne surpasse le désavantage auquel on s'expose, qui est de perdre ce qu'on y met.

Il y a quelquefois si peu d'apparence dans le succès d'une chose, que, quelque avantageuse qu'elle soit, et quelque petite que soit celle que l'on hasarde pour l'obtenir, il est utile de ne pas la hasarder. Ainsi, ce serait une sottise de jouer vingt sols contre dix millions de livres, ou contre un royaume, à condition que l'on ne pourrait le gagner, qu'au cas qu'un enfant arrangeant au hasard les lettres d'une imprimerie, composât tout d'un coup les vingt premiers vers de l'Énéide de Virgile : aussi, sans qu'on y pense, il n'y a point de moment dans la vie où l'on ne la hasarde plus qu'un prince ne hasardera son royaume en le jouant à cette condition.

Ces réflexions paraissent petites, et elles le sont en effet si on en demeure là; mais on peut les faire servir à des choses plus importantes; et le principal usage qu'on doit en tirer, est de nous rendre plus raisonnables dans nos espérances et dans nos craintes. Il y a, par exemple, beaucoup de personnes qui sont dans une frayeur excessive lorsqu'elles entendent tonner. Si le tonnerre les fait penser à Dieu et à la mort, à la bonne heure; on n'y saurait trop penser; mais si c'est le seul danger de mourir par le tonnerre qui leur cause cette appréhension extraordinaire, il est aisé de leur faire voir qu'elle n'est pas raisonnable ; car de deux millions de personnes, c'est beaucoup s'il y en a une qui meure de cette manière, et on peut dire même qu'il n'y a guère de mort violente qui soit moins commune. Puis donc que la crainte du mal doit être proportionnée, non seulement à la grandeur du mal, mais aussi à la probabilité de l'événement, comme il n'y a guère de genre de mort plus rare que de mourir par le tonnerre, il n'y en a guère aussi qui dût nous causer moins de crainte, vu même que cette crainte ne sert de rien pour nous le faire éviter.

C'est par là non seulement qu'il faut détromper ces personnes qui apportent des précautions extraordinaires et importunes pour conserver leur vie et leur santé, en leur montrant que ces précautions sont un plus grand mal que ne peut être le danger si éloigné de l'accident qu'elles craignent ; mais qu'il faut aussi désabuser tant de personnes qui ne raisonnent guère autrement dans leurs entreprises qu'en cette manière : Il y a du danger en cette affaire ; donc elle est mauvaise; il y a de l'avantage dans celle-ci, donc elle est bonne ; puisque ce n'est ni par le danger, ni par les avantages, mais par la proportion qu'ils ont entre eux qu'il faut en juger.

Il est de la nature des choses finies de pouvoir être surpassées, quelque grandes qu'elles soient, par les plus petites, si on les multiplie souvent, ou que ces petites choses surpassent plus les grandes en vraisemblance de l'événement, qu'elles n'en sont surpassées en grandeur. Ainsi, le moindre petit gain peut surpasser le plus grand qu'on puisse s'imaginer, si le petit est souvent réitéré, ou si ce grand bien est tellement difficile à obtenir, qu'il surpasse moins le petit en grandeur que le petit ne le surpasse en facilité; et il en est de même des maux que l'on appréhende, c'est-à-dire que le moindre petit mal peut être plus considérable que le plus grand mal qui n'est pas infini, s'il le surpasse par cette proportion.

Il n'y a que les choses infinies, comme l'éternité et le salut, qui ne peuvent être égalées par aucun avantage temporel, et ainsi on ne doit jamais les mettre en balance avec aucune des choses du monde. C'est pourquoi le moindre degré de facilité pour se sauver vaut mieux que tous les biens du monde joints ensemble; et le moindre péril de se perdre est plus considérable que tous les maux temporels, considérés seulement comme maux.

Ce qui suffit à toutes les personnes raisonnables pour leur faire tirer cette conclusion, par laquelle nous finirons cette Logique, que la plus grande de toutes les imprudences est d'employer son temps et sa vie à autre chose qu'à ce qui peut servir à en acquérir une qui ne finira jamais, puisque tous les biens et les maux de cette vie ne sont rien en comparaison de ceux de l'autre, et que le danger de tomber dans ces maux est très grand, aussi bien que la difficulté d'acquérir ces biens.

Ceux qui tirent cette conclusion et qui la suivent dans la conduite de leur vie, sont prudents et sages, fussent-ils peu justes dans tous les raisonnements qu'ils font sur les matières de science; et ceux qui ne la tirent pas, fussent-ils justes dans tout le reste, sont traités dans l'Écriture de fous et d'insensés, et font un mauvais usage de la logique, de la raison, et de la vie.

\finpartdeco

\bigbreak
\begin{center}
	{\Large\scshape Fin.}
\end{center}

