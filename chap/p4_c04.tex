\subsubsection{\centering \Large CHAPITRE IV}
\addcontentsline{toc}{section}{\protect\numberline{}{\scshape\bfseries IV} - \emph{Que les géomètres semblent n'avoir pas toujours bien compris la différence qu'il y a entre la définition des mots et la définition des choses}}
\begin{center}\emph{\large\scshape Que les géomètres semblent n'avoir pas toujours bien compris la différence qu'il y a entre la définition des mots et la définition des choses.}\end{center}

	\lettrine{Q}{uoiqu'il} n'y ait point d'auteurs qui se servent mieux de la définition des mots que les géomètres, je me crois néanmoins ici obligé de remarquer qu'ils n'ont pas toujours pris garde à la différence que l'on doit mettre entre les définitions des choses et les définitions des mots, qui est que les premières sont contestables, et que les autres sont incontestables ; car j'en vois qui disputent de ces définitions de mots avec la même chaleur que s'il s'agissait des choses mêmes.

Ainsi, l'on peut voir dans les commentaires de Clavius sur Euclide, une longue dispute et fort échauffée entre Pelletier et lui, touchant l'espace entre la tangente et la circonférence, que Pelletier prétendait n'être pas un angle, au lieu que Clavius soutient que c'en est un. Qui ne voit que tout cela pouvait se terminer en un seul mot, en se demandant l'un à l'autre ce qu'il entendait par le mot angle ?

Nous voyons encore que Simon Stevin, très célèbre mathématicien du prince d'Orange, ayant défini le nombre, \emph{Nombre est cela par lequel s'explique la quantité de chacune chose}, il se met ensuite fort en colère contre ceux qui ne veulent pas que l'unité soit nombre, jusqu'à faire des exclamations de rhétorique, comme s'il s'agissait d'une dispute fort solide. Il est vrai qu'il mêle dans ce discours une question de quelque importance, qui est de savoir si l'unité est au nombre comme le point est à la ligne ; mais c'est ce qu'il fallait distinguer pour ne pas brouiller deux choses très différentes : et ainsi, traitant à part ces deux questions, l'une, si l'unité est nombre, l'autre, si l'unité est au nombre ce qu'est le point à la ligne, il fallait dire, sur la première, que ce n'était qu'une dispute de mots, et que l'unité était nombre ou n'était pas nombre, selon la définition qu'on voudrait donner au nombre : qu'en le définissant comme Euclide, \emph{nombre est une multitude d'unités assemblées}, il était visible que l'unité n'était pas nombre ; mais que, comme cette définition d'Euclide était arbitraire, et qu'il était permis d'en donner une autre au nom de nombre, on pouvait lui en donner une comme est celle que Stevin apporte, selon laquelle l'unité est nombre. Par là la première question est vidée, et on ne peut rien dire, outre cela, contre ceux à qui il ne plaît pas d'appeler l'unité nombre, sans une manifeste pétition de principe, comme on peut voir en examinant les prétendues démonstrations de Stevin. La première est :

	\begin{tabularx}{\textwidth}{X}
		\emph{La partie est de même nature que le tout :} \\
		\emph{Unité est partie d'une multitude d'unités :} \\
		\emph{Donc l'unité est de même nature qu'une multitude d'unités. Et par conséquent nombre.} \\
	\end{tabularx}

Cet argument ne vaut rien du tout ; car, quand la partie serait toujours de la même nature que le tout, il ne s'ensuivrait pas qu'elle dût toujours avoir le même nom que le tout; et, au contraire, il arrive très souvent qu'elle n'a point le même nom. Un soldat est une partie de l'armée, et n'est point une armée ; une chambre est une partie d'une maison, et non point une maison ; un demi-cercle n'est point un cercle ; la partie d'un carré n'est point un carré. Cet argument prouve donc au plus que l'unité étant partie de la multitude des unités, a quelque chose de commun avec toute multitude d'unités, selon quoi on pourra dire qu'ils sont de même nature; mais cela ne prouve pas qu'on soit obligé de donner le même nom de nombre à l'unité et à la multitude d'unités, puisqu'on peut, si l'on veut, garder le nom de nombre pour la multitude d'unités, et ne donner à l'unité que son nom même d'unité ou de partie du nombre.

La seconde raison de Stevin ne vaut pas mieux :

	\begin{tabularx}{\textwidth}{X}
		\emph{Si du nombre donné l'on n'ôte aucun nombre, le nombre donné demeure :}\\
		\emph{Donc si l'unité n'était pas nombre, en ôtant un de trois, le nombre donné demeurerait, ce qui est absurde.} \\
	\end{tabularx}


Mais cette majeure est ridicule, et suppose ce qui est en question ; car Euclide niera que le nombre donné demeure, lorsqu'on n'en ôte aucun nombre, puisqu'il suffit, pour ne pas demeurer tel qu'il était, qu'on en ôte ou un nombre, ou une partie du nombre, telle qu'est l'unité : et si cet argument était bon, on prouverait de la même manière, qu'en ôtant un demi-cercle d'un cercle donné, le cercle donné doit demeurer, parce qu'on n'en a ôté aucun cercle.

Ainsi, tous les arguments de Stevin prouvent au plus qu'on peut définir le nombre en sorte que le mot de nombre convienne à l'unité, parce que l'unité et la multitude d'unités ont assez de convenance pour être signifiés par un même nom : mais ils ne prouvent nullement qu'on ne puisse pas aussi définir le nombre en restreignant ce mot à la multitude d'unités, afin de ne pas être obligé d'excepter l'unité toutes les fois qu'on explique des propriétés qui conviennent à tous les nombres, hormis à l'unité.

Mais la seconde question, qui est de savoir si l'unité est aux autres nombres comme le point est à la ligne, n'est point de même nature que la première, et n'est point une dispute de mot, mais de chose : car il est absolument faux que l'unité soit au nombre comme le point est à la ligne ; puisque l'unité ajoutée au nombre le fait plus grand, au lieu que le point ajouté à la ligne ne la fait point plus grande. L'unité est partie du nombre, et le point n'est point partie de la ligne. L'unité ôtée du nombre, le nombre donné ne demeure point ; et le point ôté de la ligne, la ligne donnée demeure.

Le même Stevin est plein de semblables disputes sur les définitions des mots, comme quand il s'échauffe pour prouver que le nombre n'est point une quantité discrète ; que la proportion des nombres est toujours arithmétique, et non géométrique, que toute racine de quelque nombre que ce soit est un nombre : ce qui fait voir qu'il n'a point compris proprement ce que c'était qu'une définition de mot, et qu'il a pris les définitions des mots, qui ne peuvent être contestées, pour les définitions des choses, que l'on peut souvent contester avec raison.


