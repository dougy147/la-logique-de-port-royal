\subsubsection{\centering \Large CHAPITRE VII}
\addcontentsline{toc}{section}{\protect\numberline{}{\scshape\bfseries VII} - \emph{Des règles qui regardent les démonstrations}}
\begin{center}\emph{\large\scshape Des règles qui regardent les démonstrations.}\end{center}

	\lettrine{U}{ne} vraie démonstration demande deux choses : l'une, que dans la matière il n'y ait rien que de certain et indubitable ; l'autre, qu'il n'y ait rien de vicieux dans la forme d'argumenter ; or, on aura certainement l'un et l'autre, si l'on observe les deux règles que nous avons posées.

Car il n'y aura rien que de véritable et de certain dans la matière, si toutes les propositions qu'on avancera pour servir de preuves, sont.

Ou les définitions des mots qu'on aura expliqués, qui étant arbitraires sont incontestables, comme nous avons fait voir dans la première partie.

Ou les axiomes qui auront été accordés, et que l'on n'a point dû supposer s'ils n'étaient clairs et évidents d'eux-mêmes par la troisième règle.

Ou des propositions déjà démontrées, et qui, par conséquent, sont devenues claires et évidentes par la démonstration qu'on en a faite.

Ou la construction de la chose même dont il s'agira lorsqu'il y aura quelque opération à faire, ce qui doit être aussi indubitable que le reste, puisque cette construction doit avoir été auparavant démontrée possible, s'il y avait quelque doute qu'elle ne le fût pas.

Il est donc clair qu'en observant la première règle, on n'avancera jamais pour preuve aucune proposition qui ne soit certaine et évidente.

Il est aussi aisé de montrer qu'on ne péchera point contre la forme de l'argumentation, en observant la seconde règle, qui est de n'abuser jamais de l'équivoque des termes, en manquant d'y substituer mentalement les définitions qui les restreignent et les expliquent.

Car s'il arrive jamais qu'on pèche contre les règles des syllogismes, c'est en se trompant dans l'équivoque de quelque terme, et le prenant en un sens dans l'une des propositions, et en un autre sens dans l'autre, ce qui arrive principalement dans le moyen du syllogisme, qui, étant pris en deux divers sens dans les deux premières propositions, est le défaut le plus ordinaire des arguments vicieux. Or, il est clair qu'on évitera ce défaut si l'on observe cette seconde règle.

Ce n'est pas qu'il n'y ait encore d'autres vices de l'argumentation outre celui qui vient de l'équivoque des termes; mais c'est qu'il est presque impossible qu'un homme d'un esprit médiocre, et qui a quelque lumière, y tombe jamais, surtout en des matières spéculatives, et ainsi il serait inutile d'avertir d'y prendre garde et d'en donner des règles ; et cela serait même nuisible, parce que l'application qu'on aurait à ces règles superflues pourrait divertir de l'attention qu'on doit avoir aux nécessaires. Aussi nous ne voyons point que les géomètres se mettent jamais en peine de la forme de leurs arguments, ni qu'ils pensent à les conformer aux règles de la logique, sans qu'ils y manquent néanmoins, parce que cela se fait naturellement et n'a pas besoin d'étude.

Il y a encore une observation à faire sur les propositions qui ont besoin d'être démontrées. C'est qu'on ne doit pas mettre de ce nombre celles qui peuvent l'être par l'application de la règle de l'évidence à chaque proposition évidente ; car si cela était, il n'y aurait presque point d'axiome qui n'eût besoin d'être démontré, puisqu'ils peuvent l'être presque tous par celui que nous avons dit pouvoir être pris pour le fondement de toute évidence : \emph{Tout ce que l'on voit clairement être contenu dans une idée claire et distincte, peut en être affirmé avec vérité}. On peut dire, par exemple :

	\begin{tabularx}{\textwidth}{X}
		\emph{Tout ce qu'on voit clairement être contenu dans une idée claire et distincte, peut en être affirmé avec vérité :} \\
		\emph{Or on voit clairement que l'idée claire et distincte qu'on a du tout, enferme d'être plus grand que sa partie :} \\
		\emph{Donc on peut affirmer avec vérité que le tout est plus grand que sa partie.} \\
	\end{tabularx}

Mais, quoique cette preuve soit très bonne, elle n'est pas néanmoins nécessaire, parce que notre esprit supplée cette majeure, sans avoir besoin d'y faire une attention particulière; et ainsi voit clairement et évidemment que le tout est plus grand que sa partie, sans qu'il ait besoin de faire réflexion d'où lui vient cette évidence ; car ce sont deux choses différentes, de connaître évidemment une chose, et de savoir d'où nous vient cette évidence.

La deuxième observation est, que quand une proposition a été démontrée généralement, elle est censée avoir été démontrée dans les cas particuliers : c'est-à-dire, que ce qui a été démontré de toutes les espèces et de tous les singuliers de chaque espèce. Car ce serait une chose ridicule de prétendre qu'après avoir démontré que tout quadrilatère a ses quatre angles égaux à quatre droits, on eût encore besoin de démontrer qu'un parallélogramme a ses quatre angles égaux à quatre droits, quoiqu'on le pût faire en cette manière :

	\begin{tabularx}{\textwidth}{X}
		\emph{Tout quadrilatère a ses quatre angles égaux à quatre droits:} \\
		\emph{Or tout parallélogramme est quadrilatère:} \\
		\emph{Donc, etc.} \\
	\end{tabularx}

On voit par là que toutes les fois qu'on prouve la différence générique, ou une propriété générique de quelque espèce, ce que les logiciens donnent souvent pour exemple des meilleurs raisonnements, comme lorsqu'on dit :

	\begin{tabularx}{\textwidth}{X}
		\emph{Tout animal a sentiment:} \\
		\emph{Tout homme est animal:} \\
		\emph{Donc tout homme a sentiment,} \\
\end{tabularx}

ce sont des arguments inutiles, et de nul usage dans les sciences; non qu'ils ne soient vrais, mais parce qu'ils sont trop vrais, et qu'ils ne prouvent rien que ce qu'on savait déjà. De sorte que sans user d'un si grand tout, on doit supposer pour prouver de chaque espèce ce qu'on a prouvé du genre. Ainsi pour montrer qu'un triangle rectangle a un de ses angles égal aux deux autres, je dirais seulement que puisque tous les trois ensemble en valent deux droits, et que l'un est droit, il faut que les autres valent aussi un droit; où je suppose sans preuve que tous les angles d'un triangle rectangle valent deux droits, parce qu'on l'a prouvé généralement du triangle : de sorte que ce serait un circuit impertinent, que de le prouver encore du triangle rectangle par cet argument de l'École :

	\begin{tabularx}{\textwidth}{X}
		\emph{Tout triangle a ses trois angles égaux à deux droits :} \\
		\emph{Or un triangle rectangle est un triangle :} \\
		\emph{Donc, etc.} \\
\end{tabularx}


