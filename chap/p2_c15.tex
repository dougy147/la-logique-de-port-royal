\subsubsection{\centering \Large CHAPITRE XV}
\addcontentsline{toc}{section}{\protect\numberline{}{\scshape\bfseries XV} - \emph{De la nature des propositions négatives}}
\begin{center}\emph{\large\scshape De la nature des propositions négatives.}\end{center}

	\lettrine{L}{a} nature d'une proposition négative ne peut s'exprimer plus clairement qu'en disant que c'est concevoir qu'une chose n'est pas une autre.

Mais, afin qu'une chose ne soit pas une autre, il n'est pas nécessaire qu'elle n'ait rien de commun avec elle, et il suffit qu'elle n'ait pas tout ce que l'autre a, comme il suffit, afin qu'une bête ne soit pas homme, qu'elle n'ait pas tout ce qu'a l'homme, et il n'est pas nécessaire qu'elle n'ait rien de ce qui est dans l'homme ; et de là on peut tirer cet axiome.



\begin{center}{\bfseries\scshape 5. Axiome}\end{center}

	\emph{La proposition négative ne sépare pas du sujet toutes les parties contenues dans la compréhension de l'attribut, mais elle sépare seulement l'idée totale et entière composée de tous ces attributs unis}.

Si je dis que la matière n'est pas une substance qui pense, je ne dis pas pour cela qu'elle n'est pas substance, mais je dis qu'elle n'est pas substance pensante, qui est l'idée totale et entière que je nie de la matière.

Il en est tout au contraire de l'extension de l'idée. Car la proposition négative sépare du sujet l'idée de l'attribut selon toute son extension : et la raison en est claire ; car être sujet d'une idée et être contenu dans son extension, n'est autre chose qu'enfermer cette idée ; et par conséquent, quand on dit qu'une idée n'en enferme pas une autre, qui est ce qu'on appelle nier, on dit qu'elle n'est pas un des sujets de cette idée.

Ainsi, si je dis que l'homme n'est pas un être insensible, je veux dire qu'il n'est aucun des êtres insensibles, et par conséquent je les sépare tous de lui; et de là on peut tirer cet autre axiome.

\begin{center}{\bfseries\scshape 6. Axiome}\end{center}

	\emph{L'attribut d'une proposition négative est toujours pris généralement}. Ce qui peut aussi s'exprimer ainsi plus distinctement: \emph{Tous les sujets d'une idée qui est niée d'une autre, sont aussi niés de cette autre idée}, c'est-à-dire qu'une idée est toujours niée selon toute extension. Si le triangle est nié des carrés, tout ce qui est triangle sera nié du carré. On exprime ordinairement dans l'École cette règle en ces termes, qui ont le même sens. \emph{Si on nie le genre, on nie aussi l'espèce}. Car l'espèce est un sujet du genre, l'homme est un sujet d'animal, parce qu'il est contenu dans son extension.

Non seulement les propositions négatives séparent l'attribut du sujet selon toute l'extension de l'attribut, mais elles séparent aussi cet attribut du sujet selon toute l'extension qu'a le sujet dans la proposition; c'est-à-dire qu'elle l'en sépare universellement si le sujet est universel, et particulièrement s'il est particulier. Si je dis que \emph{nul vicieux n'est heureux}, je sépare toutes les personnes heureuses de toutes les personnes vicieuses ; et si je dis que \emph{quelque docteur n'est pas docte}, je sépare docte de quelque docteur, et de là on doit tirer cet axiome.

\begin{center}{\bfseries\scshape 7. Axiome}\end{center}

	\emph{Tout attribut nié d'un sujet est nié de tout ce qui est contenu dans l'étendue qu'a ce sujet dans la proposition}.


