\subsubsection{\centering \Large CHAPITRE XI}
\addcontentsline{toc}{section}{\protect\numberline{}{\scshape\bfseries XI} - \emph{Principe général, par lequel sans aucune réduction aux figures et aux modes on peut juger de la bonté ou du défaut de tout syllogisme}}
\begin{center}\emph{\large\scshape Principe général, par lequel sans aucune réduction aux figures et aux modes on peut juger de la bonté ou du défaut de tout syllogisme.}\end{center}

	\lettrine{N}{ous} avons vu comme on peut juger si les arguments complexes sont concluants ou vicieux, en les réduisant à la forme des arguments plus communs, pour en juger ensuite par les règles communes; mais comme il n'y a point d'apparence que notre esprit ait besoin de cette réduction pour faire ce jugement, cela a fait penser qu'il fallait qu'il y eût des règles plus générales, sur lesquelles même les communes fussent appuyées, par où l'on reconnût plus facilement la bonté ou le défaut de toutes sortes de syllogismes : et voici ce qui en est venu dans l'esprit.

Lorsqu'on veut prouver une proposition dont la vérité ne paraît pas évidemment, il semble que tout ce qu'on a à faire soit de trouver une proposition plus connue qui confirme celle-là, laquelle, pour cette raison, on peut appeler la proposition \emph{contenante}. Mais, parce qu'elle ne peut pas la contenir expressément et dans les mêmes termes, puisque, si cela était, elle n'en serait point différente, et ainsi elle ne servirait de rien pour la rendre plus claire, il est nécessaire qu'il y ait encore une autre proposition qui fasse voir que celle que nous avons appelée \emph{contenante} contient en effet celle que l'on veut prouver ; et celle-là peut s'appeler \emph{applicative}.

Dans les syllogismes affirmatifs, il est souvent indifférent laquelle des deux on appelle \emph{contenante}, parce qu'elles contiennent toutes deux, en quelque sorte, la conclusion, et qu'elles servent mutuellement à faire voir que l'autre la contient.

Par exemple, si je doute si un homme vicieux est malheureux, et que je raisonne ainsi :

\begin{center}
	\begin{tabular}{l}
		\emph{Tout esclave de ses passions est malheureux :} \\
		\emph{Tout vicieux est esclave de ses passions :} \\
		\emph{Donc tout vicieux est malheureux.} \\
	\end{tabular}
\end{center}

Quelque proposition que vous preniez, vous pourrez dire qu'elle contient la conclusion, et que l'autre le fait voir ; car la majeure la contient, parce qu'\emph{esclave de ses passions} contient sous soi \emph{vicieux} ; c'est-à-dire que vicieux est renfermé dans son étendue, et est un de ses sujets, comme la mineure le fait voir : et la mineure la contient aussi, parce qu'\emph{esclave de ses passions} comprend, dans son idée, celle de \emph{malheureux}, comme la majeure le fait voir.

Néanmoins, comme la majeure est presque toujours plus générale, on la regarde d'ordinaire comme la proposition contenante, et la mineure comme applicative.

Pour les syllogismes négatifs, comme il n'y a qu'une proposition négative, et que la négation n'est proprement enfermée que dans la négation, il semble qu'on doive toujours prendre la proposition négative pour la contenante, et l'affirmative pour l'applicative seulement, soit que la négative soit la majeure, comme en \emph{celarent, ferio, cesare, festino} ; soit que ce soit la mineure, comme en \emph{camestres} et \emph{baroco}.

Car si je prouve par cet argument que nul avare n'est heureux :

\begin{center}
	\begin{tabular}{l}
		\emph{Tout heureux est content :} \\
		\emph{Nul avare n'est content :} \\
		\emph{Donc nul avare n'est heureux.} \\
	\end{tabular}
\end{center}

Il est plus naturel de dire que la mineure, qui est négative, contient la conclusion qui est aussi négative ; et que la majeure est pour montrer qu'elle la contient. Car cette mineure, \emph{nul avare n'est content}, séparant totalement \emph{content} d'avec \emph{avare}, en sépare aussi \emph{heureux}, puisque, selon la majeure \emph{heureux} est totalement enfermé dans l'étendue de \emph{content}.

Il n'est pas difficile de montrer que toutes les règles que nous avons données ne servent qu'à faire voir que la conclusion est contenue dans l'une des premières propositions, et que l'autre le fait voir ; et que les arguments ne sont vicieux que quand on manque à observer cela, et qu'ils sont toujours bons quand on l'observe. Car toutes ces règles se réduisent à deux principales, qui sont le fondement des autres. L'une, \emph{que nul terme ne peut être plus général dans la conclusion que dans les prémisses}. Or, cela dépend visiblement de ce principe général, \emph{que les prémisses doivent contenir la conclusion}. Ce qui ne pourrait pas être si, le même terme étant dans les prémisses et dans la conclusion, il avait moins d'étendue dans les prémisses que dans la conclusion. Car le moins général ne contient pas le plus général, \emph{quelque homme} ne contient pas \emph{tout homme}.

L'autre règle générale est, \emph{que le moyen doit être pris au moins une fois universellement}. Ce qui dépend encore de ce principe, \emph{que la conclusion doit être contenue dans les prémisses}. Car supposons que nous ayons à prouver, \emph{que quelque ami de Dieu est pauvre}, et que nous nous servions pour cela de cette proposition: \emph{quelque saint est pauvre}; je dis qu'on ne verra jamais évidemment que cette proposition contient la conclusion que par une autre proposition, où ce même moyen est \emph{saint} soit pris universellement. Car, il est visible qu'afin que cette proposition, \emph{quelque saint est pauvre}, contienne la conclusion, \emph{quelque ami de Dieu est pauvre}, il faut et il suffit que le terme \emph{quelque saint} contienne le terme \emph{quelque ami de Dieu}, puisque pour l'autre elles l'ont commun. Or, un terme particulier n'a point d'étendue déterminée; il ne contient certainement que ce qu'il enferme dans sa compréhension et dans son idée.

Et par conséquent, afin que le terme \emph{quelque saint} contienne le terme \emph{quelque ami de Dieu}, il faut qu'\emph{ami de Dieu} soit contenu dans la compréhension de l'idée de \emph{saint}.

Or, tout ce qui est contenu dans la compréhension d'une idée en peut être universellement affirmé; tout ce qui est enfermé dans la compréhension de l'idée de \emph{triangle}, peut être affirmé de \emph{tout triangle}; tout ce qui est enfermé dans l'idée \emph{d'homme}, peut être affirmé de \emph{tous homme}. Et par conséquent afin qu'\emph{ami de Dieu} soit enfermé dans l'idée de \emph{saint}, il faut que \emph{tout saint soit ami de Dieu}. D'où il s'ensuit que cette conclusion, \emph{quelque ami de Dieu est pauvre}, ne peut être contenue dans cette proposition, \emph{quelque saint est pauvre}, où le moyen \emph{saint} est pris particulièrement, qu'en vertu d'une proposition où il soit pris universellement, puisqu'elle doit faire voir qu'un \emph{ami de Dieu} est contenu dans la compréhension de l'idée de \emph{saint}. C'est ce qu'on ne peut montrer qu'en affirmant \emph{ami de Dieu} de \emph{saint} pris universellement, tout saint est ami de Dieu, et par conséquent nulle des prémisses ne contiendrait la conclusion, si le moyen étant pris particulièrement dans l'une des propositions, il n'était pris universellement dans l'autre : ce qu'il fallait démontrer.


