\subsubsection{\centering \Large CHAPITRE X}
\addcontentsline{toc}{section}{\protect\numberline{}{\scshape\bfseries X} - \emph{Du remède à la confusion qui naît dans nos pensées et dans nos discours de la confusion des mots ; où il est parlé de la nécessité et de l'utilité de définir les noms dont on se sert, et de la différence de la définition des choses d'avec la définition des noms}}
\begin{center}\emph{\large\scshape Du remède à la confusion qui naît dans nos pensées et dans nos discours de la confusion des mots ; où il est parlé de la nécessité et de l'utilité de définir les noms dont on se sert, et de la différence de la définition des choses d'avec la définition des noms.}\end{center}


	\lettrine{L}{e} meilleur moyen pour éviter la confusion des mots qui se rencontrent dans les langues ordinaires, est de faire une nouvelle langue et de nouveaux mots, qui ne soient attachés qu'aux idées que nous voulons qu'ils représentent ; mais, pour cela, il n'est pas nécessaire de faire de nouveaux sons, parce qu'on peut se servir de ceux qui sont déjà en usage, en les regardant comme s'ils n'avaient aucune signification, pour leur donner celle que nous voulons qu'ils aient, en désignant par d'autres mots simples, et qui ne soient point équivoques, l'idée à laquelle nous voulons les appliquer : comme si je veux prouver que notre âme est immortelle, le mot d'âme étant équivoque, comme nous l'avons montré, fera naître aisément de la confusion dans ce que j'aurai à dire : de sorte que pour l'éviter, je regarderai le mot d'âme comme si c'était un son qui n'eût point encore de sens, et je l'appliquerai uniquement à ce qui est en nous le principe de la pensée, en disant : \emph{j'appelle âme ce qui est en nous le principe de la pensée}.

C'est ce qu'on appelle la définition du mot, \emph{definitio nominis}, dont les géomètres se servent si utilement, laquelle il faut bien distinguer de la définition de la chose, \emph{definitio rei}.

Car dans la définition de la chose, comme peut être celle-ci : \emph{L'homme est un animal raisonnable, le temps est la mesure du mouvement}, on laisse au terme qu'on définit, comme \emph{homme} ou \emph{temps}, son idée ordinaire, dans laquelle on prétend que sont contenues d'autres idées, comme \emph{animal raisonnable}, ou \emph{mesure du mouvement}; au lieu que dans la définition du nom, comme nous avons déjà dit, on ne regarde que le son, et ensuite on détermine ce son à être signe d'une idée que l'on désigne par d'autres mots.

Il faut aussi prendre garde de ne pas confondre la définition de nom dont nous parlons ici, avec celle dont parlent quelques philosophes, qui entendent par là l'explication de ce qu'un mot signifie selon l'usage ordinaire d'une langue, ou selon son étymologie : c'est de quoi nous pourrons parler en un autre endroit ; mais ici, on ne regarde, au contraire, que l'usage particulier auquel celui qui définit un mot veut qu'on le prenne pour bien concevoir sa pensée, sans se mettre en peine si les autres le prennent dans le même sens.

Et de là il s'ensuit, premièrement, que les définitions de noms sont arbitraires, et que celles des choses ne le sont point. Car chaque son étant indifférent de soi-même et par sa nature à signifier toutes sortes d'idées, il m'est permis, pour mon usage particulier, et pourvu que j'en avertisse les autres, de déterminer un son à signifier précisément une certaine chose, sans mélange d'aucune autre; mais il en est tout autrement de la définition des choses : car il ne dépend point de la volonté des hommes que les idées comprennent ce qu'ils voudraient qu'elles comprissent ; de sorte que si, en voulant les définir, nous attribuons à ces idées quelque chose qu'elles ne contiennent pas, nous tombons nécessairement dans l'erreur.

Ainsi, pour donner un exemple de l'un et de l'autre, si, dépouillant le mot \emph{parallélogramme} de toute signification, je l'applique à signifier un triangle, cela m'est permis, et je ne commets en cela aucune erreur, pourvu que je ne le prenne qu'en cette sorte : et je pourrai dire alors que le parallélogramme a trois angles égaux à deux droits; mais si, laissant à ce mot sa signification et son idée ordinaire, qui est de signifier une figure dont les côtés sont parallèles, je venais à dire que le parallélogramme est une figure à trois lignes, parce que ce serait alors une définition de choses, elle serait très fausse, étant impossible qu'une figure à trois lignes ait ses côtés parallèles.

Il s'ensuit, en second lieu, que les définitions des noms ne peuvent pas être contestées par cela même qu'elles sont arbitraires. Car vous ne pouvez pas nier qu'un homme n'ait donné à un son la signification qu'il dit lui avoir donnée, ni qu'il n'ait cette signification dans l'usage qu'en fait cet homme, après nous en avoir avertis ; mais pour les définitions des choses, on a souvent droit de les contester, puisqu'elles peuvent être fausses, comme nous l'avons montré.

Il s'ensuit troisièmement, que toute définition de nom ne pouvant être contestée, peut être prise pour principe, au lieu que les définitions des choses ne peuvent point du tout être prises pour principes, et sont de véritables propositions qui peuvent être niées par ceux qui y trouveront quelque obscurité, et par conséquent elles ont besoin d'être prouvées comme d'autres propositions, et ne doivent pas être supposées, à moins qu'elles ne fussent claires d'elles-mêmes comme des axiomes.

Néanmoins ce que je viens de dire, que la définition du nom peut être prise pour principe, a besoin d'explication ; car cela n'est vrai qu'à cause que l'on ne doit pas contester que l'idée qu'on a désignée ne puisse être appelée du nom qu'on lui a donné; mais on n'en doit rien conclure à l'avantage de cette idée, ni croire pour cela seul qu'on lui a donné un nom, qu'elle signifie quelque chose de réel. Car, par exemple, je puis définir le mot de chimère en disant : J'appelle chimère ce qui implique contradiction ; et cependant, il ne s'ensuivra pas de là que la chimère soit quelque chose. De même, si un philosophe me dit : J'appelle pesanteur le principe inférieur qui fait qu'une pierre tombe sans que rien la pousse, je ne contesterai pas cette définition, au contraire, je la recevrai volontiers, parce qu'elle me fait entendre ce qu'il veut dire ; mais je lui nierai que ce qu'il entend par ce mot pesanteur soit quelque chose de réel, parce qu'il n'y a point de tel principe dans les pierres.

J'ai voulu expliquer ceci un peu au long, parce qu'il y a deux grands abus qui se commettent sur ce sujet dans la philosophie commune. Le premier est de confondre la définition de la chose avec la définition du nom, et d'attribuer à la première ce qui ne convient qu'à la dernière; car, ayant fait à leur fantaisie cent définitions, non de nom, mais de chose, qui sont très fausses, et qui n'expliquent point du tout la vraie nature des choses ni les idées que nous en avons naturellement, ils veulent ensuite que l'on considère ces définitions comme des principes que personne ne peut contredire ; et, si quelqu'un les leur nie, comme elles sont très niables, ils prétendent qu'on ne mérite pas de disputer avec eux, suivant cette règle, \emph{contra negantam principia non est disputandum}. Le deuxième abus est que, ne se servant presque jamais de définitions de noms, pour en ôter l'obscurité et les fixer à de certaines idées désignées clairement, ils les laissent dans leur confusion : d'où il arrive que la plupart de leurs disputes ne sont que des disputes de mots; et, de plus, qu'ils se servent de ce qu'il y a de clair et de vrai dans les idées confuses, pour établir ce qu'elles ont d'obscur et de faux ; ce qui se reconnaîtrait facilement si on avait défini les noms. Ainsi, les philosophes croient d'ordinaire que la chose du monde la plus claire est, que le feu est chaud, et qu'une pierre est pesante, et que ce serait une folie de le nier, et en effet, ils le persuaderont à tout le monde, tant qu'on n'aura point défini les noms : mais, en les définissant, on découvrira aisément si ce qu'on leur niera sur ce sujet est clair ou obscur ; car il leur faut demander ce qu'ils entendent par le mot de chaud et par le mot de pesant. Que s'ils répondent que, par chaud, ils entendent seulement ce qui est propre à causer en nous le sentiment de la chaleur, et par pesant, ce qui tombe en bas, n'étant point soutenu, ils ont raison de dire qu'il faut être déraisonnable pour nier que le feu soit chaud, et qu'une pierre soit pesante : mais, s'ils entendent par chance qui a en soi une qualité semblable à ce que nous nous imaginons quand nous sentons de la chaleur, et par pesant ce qui a en soi un principe intérieur qui le fait aller vers le centre, sans être poussé par quoi que ce soit, il sera facile alors de leur montrer que ce n'est point leur nier une chose claire, mais très obscure, pour ne pas dire très fausse, que de leur nier qu'en ce sens le feu soit chaud, et qu'une pierre soit pesante ; parce qu'il est bien clair que le feu nous fait avoir le sentiment de la chaleur par l'impression qu'il fait sur notre corps; mais il n'est nullement clair que le feu ait rien en lui qui soit semblable à ce que nous sentons quand nous sommes auprès du feu : et il est de même fort clair qu'une pierre descend en bas quand on la laisse ; mais il n'est nullement clair qu'elle y descend d'elle-même, sans que rien la pousse en bas.

Voilà donc la grande utilité de la définition des noms, de faire comprendre nettement de quoi il s'agit, afin de ne pas disputer inutilement sur des mots, que l'un entend d'une façon, et l'autre de l'autre, comme on fait si souvent, même dans les discours ordinaires.

Mais, outre cette utilité, il y en a encore une autre ; c'est qu'on ne peut souvent avoir une idée distincte d'une chose, qu'en y employant beaucoup de mots pour la désigner : or, il serait importun, surtout dans les livres de science, de répéter toujours cette grande suite de mots. C'est pourquoi, ayant fait comprendre la chose par tous ces mots, on attache à un seul mot l'idée qu'on a conçue, et ce mot tient lieu de tous les autres. Ainsi, ayant compris qu'il y a des nombres qui sont divisibles en deux également, pour éviter de répéter souvent tous ces termes, on donne un nom à cette propriété, en disant : J'appelle tout nombre qui est divisible en deux également, nombre pair : cela fait voir que toutes les fois qu'on se sert du mot qu'on a défini, il faut substituer mentalement la définition en la place du défini, et avoir cette définition si présente, qu'aussitôt qu'on nomme, par exemple, le nombre pair, on entende précisément que c'est celui qui est divisible en deux également, et que ces deux choses soient tellement jointes et inséparables dans la pensée, qu'aussitôt que le discours en exprime l'une, l'esprit y attache immédiatement l'autre. Car ceux qui définissent les termes, comme font les géomètres, avec tant de soin, ne le font que pour abréger le discours, et non pas pour abréger les idées des choses dont ils discourent; parce qu'ils prétendent que l'esprit suppléera la définition entière aux termes courts, qu'ils n'emploient que pour éviter l'embarras que la multitude des paroles apporterait.

