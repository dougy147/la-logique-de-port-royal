\section*{}
\begin{center}
	\includegraphics[scale=0.23]{images/titre-avis.png}
\end{center}
\addcontentsline{toc}{section}{\scshape\large Avis}

\begin{wrapfigure}[4]{l}{0pt}
    \includegraphics[width=0.17\textwidth]{images/enluminure_L1.png}
\end{wrapfigure}
\noindent\hspace{13cm}{\texttt{A}}\emph{ A naissance de ce petit ouvrage est due entièrement au hasard, et plutôt à une espèce de divertissement qu'à un dessein sérieux. Une personne de condition entretenant un jeune seigneur, qui dans un âge peu avancé faisait paraître beaucoup de solidité et de pénétration d'esprit, lui dit qu'étant jeune, il avait trouvé un homme qui l'avait rendu, en quinze jours, capable de répondre sur une partie de la logique. Ce discours donna occasion à une autre personne qui était présente, et qui n'avait pas grande estime pour cette science, de répondre en riant, que si Monsieur... voulait en prendre la peine, on s'engagerait bien à lui apprendre en quatre ou cinq jours tout ce qu'il y avait d'utile dans la logique. Cette proposition faite en l'air ayant servi quelque temps d'entretien, on se résolut d'en faire l'essai; mais comme on ne jugea pas les logiques ordinaires assez courtes, ni assez nettes, on eut l'idée d'en faire un petit abrégé qui ne fût que pour lui.}

\emph{C'est l'unique vue qu'on avait lorsqu'on se mit en devoir d'y travailler, et l'on ne pensait pas y employer plus d'un jour ; mais quand on voulut s'y appliquer, il vint dans l'esprit tant de réflexions nouvelles qu'on fut obligé de les écrire pour s'en décharger : ainsi, au lieu d'un jour, on y en employa quatre ou cinq, pendant lesquels on forma le corps de cette logique, à laquelle on a depuis ajouté diverses choses.}

\emph{Or, quoiqu'on y ait embrassé beaucoup plus de matières qu'on ne s'était engagé de faire d'abord, néanmoins l'essai en réussit comme on se l'était promis ; car ce jeune seigneur l'ayant lui-même réduite en quatre tables, il en apprit facilement une par jour, sans même qu'il eût presque besoin de personne pour l'entendre. Il est vrai qu'on ne doit pas espérer que d'autres que lui y entrent avec la même facilité; son esprit étant tout à fait extraordinaire dans toutes les choses qui dépendent de l'intelligence.}

\emph{Voilà la rencontre qui a produit cet ouvrage : mais, quelque sentiment qu'on en ait, on ne peut, au moins avec justice, en désapprouver l'impression, puisqu'elle a été plutôt forcée que volontaire : car plusieurs personnes en ayant tiré des copies manuscrites, ce qu'on sait assez ne pouvoir se faire sans qu'il s'y glisse beaucoup de fautes, on a eu avis que les libraires se disposaient à l'imprimer ; de sorte qu'on a jugé plus à propos de le donner au public correct et entier, que de permettre qu'on l'imprimât sur des copies défectueuses; mais c'est aussi ce qui a obligé d'y faire diverses additions qui l'ont augmenté de près d'un tiers, parce qu'on a cru devoir étendre ces vues plus loin qu'on n'avait fait en ce premier essai. C'est le sujet du discours suivant, où l'on explique la fin qu'on s'y est proposée, et la raison des matières qu'on y a traitées.}
