\subsubsection{\centering \Large CHAPITRE XIII}
\addcontentsline{toc}{section}{\protect\numberline{}{\scshape\bfseries XIII} - \emph{De la conversion des propositions : où l'on explique plus à fond la nature de l'affirmation et de la négation, dont cette conversion dépend. Et premièrement de la nature de l'affirmation}}
\begin{center}\emph{\large\scshape De la conversion des propositions : où l'on explique plus à fond la nature de l'affirmation et de la négation, dont cette conversion dépend. Et premièrement de la nature de l'affirmation.}\end{center}


	\begin{center}{\footnotesize Les chapitres suivants sont un peu difficiles à comprendre, et ne sont nécessaires que pour la spéculation. C'est pourquoi ceux qui ne voudront pas se fatiguer l'esprit à des choses peu utiles pour la pratique, peuvent les passer.}\end{center}

		\lettrine{J}{'ai} réservé jusqu'ici à parler de la conversion des propositions, parce que de là dépendent les fondements de toute l'argumentation dont nous devons traiter dans la partie suivante ; et ainsi il a été bon que cette matière ne fût pas éloignée de ce que nous avons à dire du raisonnement, quoique, pour bien la traiter, il faille reprendre quelque chose de ce que nous avons dit de l'affirmation ou de la négation, et expliquer à fond la nature de l'une et de l'autre.

Mais il est certain au moins que nous ne saurions exprimer une proposition aux autres, que nous ne nous servions de ces deux idées, l'une pour le sujet, et l'autre pour l'attribut, et d'un autre mot qui marque l'union que notre esprit y conçoit.

Cette union ne peut mieux s'exprimer que par les paroles mêmes dont on se sert pour affirmer, en disant qu'une chose est une autre chose.

Et de là il est clair que la nature de l'affirmation est d'unir et d'identifier, pour le dire ainsi, le sujet avec l'attribut, puisque c'est ce qui est signifié par le mot \emph{est}.

Et il s'ensuit aussi qu'il est de la nature de l'affirmation de mettre l'attribut dans tout ce qui est exprimé dans le sujet, selon l'étendue qu'il a dans la proposition ; comme quand je dis que \emph{tout homme est animal}, je veux dire et je signifie que tout ce qui est homme est aussi animal ; et ainsi je conçois l'animal dans tous les hommes.

Que si je dis seulement \emph{quelque homme est juste}, je ne mets pas \emph{juste} dans tous les hommes, mais seulement dans quelque homme.

Mais il faut pareillement considérer ici ce que nous avons déjà dit, qu'il faut distinguer dans les idées la compréhension de l'extension, et que la compréhension marque les attributs contenus dans une idée, et l'extension, les sujets qui participent et contiennent cette idée selon la compréhension.

Car il s'ensuit de là qu'une idée est toujours affirmée selon sa compréhension, parce qu'en lui ôtant quelqu'un de ses attributs essentiels, on la détruit et on l'anéantit entièrement, et ce n'est plus la même idée ; et, par conséquent, quand elle est affirmée, elle l'est toujours selon tout ce qu'elle comprend en soi. Ainsi, quand je dis, \emph{qu'un rectangle est un parallélogramme}, j'affirme du rectangle tout ce qui est compris dans l'idée du parallélogramme ; car, s'il y avait quelque partie de cette idée qui ne convint pas au rectangle, il s'ensuivrait que l'idée entière ne lui conviendrait pas, mais seulement une partie de cette idée : et ainsi le mot de parallélogramme, qui signifie l'idée totale, devrait être nié et non affirmé du rectangle. On verra que c'est le principe de tous les arguments affirmatifs.

Et il s'ensuit au contraire que l'idée de l'attribut n'est pas prise selon toute son extension, à moins que son extension ne fût pas plus grande que celle du sujet. Car quand je dis, par exemple, que \emph{tous les aigles volent}, je ne veux pas dire qu'il n'y eût que les aigles qui volent, mais je mets seulement l'attribut de voler dans tous les aigles, sans nier qu'il se rencontre en d'autres oiseaux.

Si je dis que \emph{tous les impudiques seront damnés}, je ne dis pas qu'ils seront eux seuls tous les damnés, mais qu'ils seront du nombre des damnés.

Ainsi, l'affirmation mettant l'idée de l'attribut dans le sujet, c'est proprement le sujet qui détermine l'extension de l'attribut dans la proposition affirmative, et l'identité qu'elle marque regarde l'attribut comme resserré dans une étendue égale à celle du sujet, et non pas dans toute sa généralité, s'il en a une plus grande que le sujet; car il est vrai que les lions sont tous animaux, c'est-à-dire que chacun des lions renferme l'idée d'animal; mais il n'est pas vrai qu'ils soient tous les animaux.

J'ai dit que l'attribut n'est pas pris dans toute sa généralité, s'il en a une plus grande que le sujet ; car n'étant restreint que par le sujet, si le sujet est aussi général que cet attribut, il est clair qu'alors l'attribut demeurera dans toute sa généralité, puisqu'il en aura autant que le sujet ; et que nous supposons que, par sa nature, il n'en peut avoir davantage.

De là on peut recueillir ces quatre axiomes indubitables.

\begin{center}{\bfseries\scshape 1. Axiome}\end{center}

	\emph{L'attribut est mis dans le sujet par la proposition affirmative, selon toute l'extension que le sujet a dans la proposition}. C'est-à-dire que si le sujet est universel, l'attribut est conçu dans toute l'extension du sujet ; et si le sujet est particulier, l'attribut n'est conçu que dans une partie de l'extension du sujet.

\begin{center}{\bfseries\scshape 2. Axiome}\end{center}

	\emph{L'attribut d'une proposition affirmative est affirmé selon toute sa compréhension}; c'est-à-dire, selon tous ses attributs. La preuve en est ci-dessus.

\begin{center}{\bfseries\scshape 3. Axiome}\end{center}

	\emph{L'attribut d'une proposition affirmative n'est point affirmé selon toute son extension, si elle est de soi-même plus grande que celle du sujet}. La preuve en est ci-dessus.


\bigbreak
\bigbreak
\begin{center}{\bfseries\scshape 4. Axiome}\end{center}

	\emph{L'extension de l'attribut est resserrée par celle du sujet, en sorte qu'il ne signifie plus que la partie de son extension qui convient au sujet}; comme quand on dit que les hommes sont animaux, le mot d'animal ne signifie plus tous les animaux, mais seulement les animaux qui sont hommes.

