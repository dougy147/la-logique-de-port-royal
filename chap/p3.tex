\subsection{\centering \huge \scshape Troisième Partie de la Logique}
\addcontentsline{toc}{section}{\scshape\large {\bfseries Troisième Partie de la Logique} - Du raisonnement}
\begin{center}\emph{\Large\scshape Du Raisonnement.}\end{center}

Cette partie que nous avons maintenant à traiter, qui comprend les règles du raisonnement, est estimée la plus importante de la logique, et c'est presque l'unique qu'on y traite avec quelque soin; mais il y a sujet de douter si elle est aussi utile qu'on se l'imagine. La plupart des erreurs des hommes, comme nous avons déjà dit ailleurs, viennent bien plus de ce qu'ils raisonnent sur de faux principes, que non pas de ce qu'ils raisonnent mal suivant leurs principes. Il arrive rarement qu'on se laisse tromper par des raisonnements qui ne soient faux que parce que la conséquence en est mal tirée, et ceux qui ne seraient pas capables d'en reconnaître la fausseté par la seule lumière de la raison, ne le seraient pas ordinairement d'entendre les règles que l'on en donne et encore moins de les appliquer. Néanmoins, quand on ne considérerait ces règles que comme des vérités spéculatives, elles serviraient toujours à exercer l'esprit ; et de plus, on ne peut nier qu'elles n'aient quelque usage en quelques rencontres, et à l'égard de quelques personnes, qui, étant d'un naturel vif et pénétrant, ne se laissent quelquefois tromper par de fausses conséquences, que faute d'attention, à quoi la réflexion qu'ils feraient sur ces règles serait capable de remédier. Quoi qu'il en soit, voilà ce qu'on en dit ordinairement, et quelque chose même de plus qu'on n'en dit.
