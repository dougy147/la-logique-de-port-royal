\subsubsection{\centering \Large CHAPITRE VI}
\addcontentsline{toc}{section}{\protect\numberline{}{\scshape\bfseries VI} - \emph{Des propositions complexes selon l'affirmation ou la négation, et d'une espèce de ces sortes de propositions que les philosophes appellent modales}}
\begin{center}\emph{\large\scshape Des propositions complexes selon l'affirmation ou la négation, et d'une espèce de ces sortes de propositions que les philosophes appellent modales.}\end{center}


	\lettrine{O}{utre} les propositions dont le sujet ou l'attribut est un terme complexe, il y en a d'autres qui sont complexes parce qu'il y a des termes ou des propositions incidentes qui ne regardent que la forme de la proposition, c'est-à-dire l'affirmation ou la négation qui est exprimée par le verbe, comme si je dis : \emph{je soutiens que la Terre est ronde ; je soutiens} n'est qu'une proposition incidente qui doit faire partie de quelque chose dans la proposition principale; et cependant il est visible qu'elle ne fait partie ni du sujet ni de l'attribut ; car cela n'y change rien du tout, et ils seraient conçus entièrement de la même sorte si je disais simplement : la Terre est ronde, et ainsi cela ne tombe que sur l'affirmation qui est exprimée en deux manières : l'une à l'ordinaire par le verbe est, \emph{la Terre est ronde} ; et l'autre plus expressément parle verbe \emph{je soutiens}.

C'est de même quand on dit, \emph{je nie; il est vrai; il n'est pas vrai}; ou qu'on ajoute dans une proposition ce qui en appuie la vérité, comme quand je dis : \emph{Les raisons d'astronomie nous convainquent que le soleil est beaucoup plus grand que la Terre}. Car cette première partie n'est que l'appui de l'affirmation.

Néanmoins il est important de remarquer qu'il y a de ces sortes de propositions qui sont ambiguës et qui peuvent être prises différemment, selon le dessein de celui qui les prononce, comme si je dis : \emph{Tous les philosophes nous assurent que les choses pesantes tombent d'elles-mêmes en bas}; si mon dessein est de montrer que les choses pesantes tombent d'elles-mêmes en bas, la première partie de cette proposition ne sera qu'incidente et ne fera qu'appuyer l'affirmation de la dernière partie; mais si, au contraire, je n'ai dessein que de rapporter cette opinion des philosophes, sans que moi-même je l'approuve, alors la première partie sera la proposition principale, et la dernière sera seulement une partie de l'attribut ; car ce que j'affirmerai ne sera pas que les choses pesantes tombent d'elles-mêmes, mais seulement que tous les philosophes l'assurent. Et il est aisé de voir que ces deux différentes manières de prendre cette même proposition la changent tellement, que ce sont deux différentes propositions, et qui ont des sens tout différents. Mais il est souvent aisé de juger par la suite auquel de ces deux sens on la prend. Car par exemple, si après avoir fait cette proposition j'ajoutais : \emph{Or les pierres sont pesantes: donc elles tombent en bas d'elles-mêmes}, il serait visible que je l'aurais prise au premier sens, et que la première partie ne serait qu'incidente. Mais si, au contraire, je concluais ainsi : \emph{Or cela est une erreur; et par conséquent il peut se faire qu'une erreur soit enseignée par tous les philosophes}, il serait manifeste que je l'aurais prise dans le second sens, c'est-à-dire que la première partie serait la proposition principale, et que la seconde ferait partie seulement de l'attribut.

De ces propositions complexes, où la complexion tombe sur le verbe, et non sur le sujet ni sur l'attribut, les philosophes ont particulièrement remarqué celles qu'ils ont appelées \emph{modales}, parce que l'affirmation ou la négation y est modifiée par l'un de ces quatre modes, \emph{possible, contingent, impossible, nécessaire}. Et parce que chaque mode peut être affirmé ou nié, comme, \emph{il est impossible, il n'est pas impossible}, et en l'une et en l'autre façon être joint avec une proposition affirmative ou négative, que \emph{la Terre est ronde}, que \emph{la Terre n'est pas ronde}, chaque mode peut avoir quatre propositions, et les quatre ensemble seize, qu'ils ont marquées par ces quatre mots : {\scshape Purpurea}, {\scshape Iliace}, {\scshape Amabimus}, {\scshape Edentuli}; dont voici tout le mystère. Chaque syllabe marque un de ces quatre modes:

\begin{center}
    $\begin{array}{lcl}
	    \text {La première} & \text{:} & \text{possible} \\
	    \text {La deuxième} & \text{:} & \text{contingent} \\
	    \text {La troisième} & \text{:} & \text{impossible} \\
	    \text {La quatrième} & \text{:} & \text{nécessaire} \\
    \end{array}$
\end{center}

Et la voyelle qui se trouve dans chaque syllabe, qui est ou A, ou E, ou I, ou U, marque si le mode doit être affirmé ou nié, et si la proposition qu'ils appellent \emph{dictum} doit être affirmée ou niée en cette manière :


\begin{table}[!htbp]\vspace{-0.4cm}
	\centering\begin{tabularx}{\textwidth}{lX}
	    A & L'affirmation du mode, et l'affirmation de la proposition. \\
	    E & L'affirmation du mode, et la négation de la proposition. \\
	    I & La négation du mode, et l'affirmation de la proposition. \\
	    U & La négation du mode, et la négation de la proposition. \\
    \end{tabularx}
\end{table}

Ce serait perdre le temps que d'en apporter des exemples qui sont faciles à trouver. IL faut seulement observer que {\scshape Purpurea} répond à l'A des propositions incomplexes : {\scshape Iliace} à E : {\scshape Amabimus} à I : {\scshape Edentuli} à U, et qu'ainsi, si on veut que les exemples soient vrais, il faut, ayant pris un sujet, prendre pour \emph{Purpurea} un attribut qui en puisse être universellement affirmé ; pour \emph{Iliace}, qui en puisse être universellement nié; pour \emph{Amabimus}, qui en puisse être affirmé particulièrement, et pour \emph{Edentuli}, qui en puisse être nié particulièrement.

Mais quelque attribut qu'on prenne, il est toujours vrai que toutes les quatre propositions d'un même mot n'ont que le même sens ; de sorte que l'une étant vraie, toutes les autres le sont aussi.
