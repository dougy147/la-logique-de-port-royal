\subsubsection{\centering \Large CHAPITRE VIII}
\addcontentsline{toc}{section}{\protect\numberline{}{\scshape\bfseries VIII} - \emph{Des propositions composées dans le sens}}
\begin{center}\emph{\large\scshape Des propositions composées dans le sens.}\end{center}

	\lettrine{I}{l} reste à dire un mot des propositions dont la composition est plus cachée. Il y en a de six sortes.

	\bigbreak
$1$. {\bfseries\scshape Exclusives} : \emph{La vertu seule est estimable. Il n'y a que la vertu d'estimable.}

$2$. {\bfseries\scshape Exceptives} : \emph{Tous les hommes sont misérables hors ceux qui sont à Dieu.}

$3$. {\bfseries\scshape Comparatives} : \emph{L'impiété est le plus grand de tous les aveuglements.}

$4$. {\bfseries\scshape Inceptives} : \emph{Celui qui se convertit à Dieu, commence à sentir le poids du péché.}

$5$. {\bfseries\scshape Désitives} : \emph{Celui qui est justifié n'est plus sous la domination du péché.}

$6$. {\bfseries\scshape Réduplicatives} : \emph{L'homme en tant qu'animal est semblable aux bêtes.}

\bigbreak
Il est aisé de voir que toutes ces propositions en enferment plusieurs dans le sens. Il ne faut qu'y faire attention pour le reconnaître; et il faut laisser quelque chose à deviner à ceux qui apprennent, afin qu'ils exercent leur esprit.

Ce qui est ici de plus remarquable, est qu'il y a souvent de ces propositions qui sont exclusives dans le sens, quoique l'exclusion n'y soit pas exprimée, surtout en latin; de sorte qu'en les traduisant en français on ne les peut exprimer dans toute leur force sans en faire des propositions exclusives, quoi qu'en latin l'exclusion n'y soit pas marquée.

Ainsi, {\scshape $2$ Corinthiens $10:17$} \emph{Qui gloriatur, in Domino glorietur}, doit être traduit : Que celui qui se glorifie, ne se glorifie qu'au Seigneur.

{\scshape Galates $6:7$} \emph{Quae seminaverit homo, haec et metet} : L'homme ne recueillera que ce qu'il aura semé.

{\scshape Ephesiens $4:5$} \emph{Unus Dominus, una fides, unum baptisma} : Il n'y a qu'un Seigneur, qu'une foi, qu'un baptême.

{\scshape Matthieu $5:46$} \emph{Si diligitis eos qui vos diligunt, quam mercedem habebitis ?} Si vous n'aimez que ceux qui vous aiment, quelle récompense en mériterez-vous ?

Sénèque, dans la Troade : \emph{Nullas habet spes Troia, si tales habet} : Si Troie n'a que cette espérance, elle n'en a point: comme s'il y avait, \emph{si tantum tales habet}.

