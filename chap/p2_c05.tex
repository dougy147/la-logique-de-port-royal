\subsubsection{\centering \Large CHAPITRE V}
\addcontentsline{toc}{section}{\protect\numberline{}{\scshape\bfseries V} - \emph{De la fausseté qui peut se trouver dans les termes complexes et dans les propositions incidentes}}
\begin{center}\emph{\large\scshape De la fausseté qui peut se trouver dans les termes complexes et dans les propositions incidentes.}\end{center}

	\lettrine{C}{e} que nous venons de dire peut servir à résoudre une question célèbre, qui est de savoir si la fausseté ne peut se trouver que dans les propositions, et s'il n'y en a point dans les idées et dans les simples termes.

Je parle de la fausseté plutôt que de la vérité, parce qu'il y a une vérité qui est dans les choses par rapport à l'esprit de Dieu, soit que les hommes y pensent ou n'y pensent pas; mais il ne peut y avoir de fausseté que par rapport à l'esprit de l'homme, ou à quelque esprit sujet à erreur, qui juge faussement qu'une chose est ce qu'elle n'est pas.

On demande donc si cette fausseté ne se rencontre que dans les propositions et dans les jugements.

On répond ordinairement que non, et cela a sa vérité; mais cela n'empêche pas qu'il n'y ait quelquefois de la fausseté, non dans les idées simples, mais dans les termes complexes, parce qu'il suffit pour cela qu'il y ait quelque jugement et quelque affirmation, ou expresse, ou virtuelle.

C'est ce que nous verrons mieux en considérant en particulier les deux sortes de termes complexes, l'un dont le \emph{qui} est explicatif, l'autre dont il est déterminatif.

Dans la première sorte de termes complexes, il ne faut pas s'étonner s'il peut y avoir de la fausseté; parce que l'attribut de la proposition incidente est affirmé du sujet auquel le \emph{qui} se rapporte : \emph{Alexandre qui est fils de Philippe}, j'affirme quoiqu'incidemment le fils de Philippe d'Alexandre, et par conséquent il y a en cela de la fausseté, si cela n'est pas.

Mais il faut remarquer deux ou trois choses importantes.

\bigbreak
{1.} Que la fausseté de la proposition incidente n'empêche pas, pour l'ordinaire, la vérité de la proposition principale. Par exemple, \emph{Alexandre, qui a été fils de Philippe, a vaincu les Perses}, cette proposition doit passer pour vraie, quand Alexandre ne serait pas fils de Philippe, parce que l'affirmation de la proposition principale ne tombe que sur Alexandre, et ce qu'on y a joint incidemment, quoique faux, n'empêche point qu'il ne soit vrai qu'Alexandre ait vaincu les Perses.

Que si néanmoins l'attribut de la proposition principale avait rapport à la proposition incidente, comme si je disais : \emph{Alexandre fils de Philippe était petit-fils d'Amintas}, ce serait alors seulement que la fausseté de la proposition incidente rendrait fausse la proposition principale.

\bigbreak
{2.} Quand on n'emploie ces propositions incidentes que pour désigner le sujet du discours, il n'est point alors nécessaire pour la vérité de ces propositions que leur attribut convienne à ce sujet dans la vérité ; mais c'est assez qu'il lui convienne dans l'opinion des hommes. Ainsi quand on dit, \emph{Alexandre fils de Philippe ou qui a été fils de Philippe}, la qualité de \emph{fils de Philippe}, qui est affirmée d'Alexandre, n'en est affirmée que selon l'opinion des hommes, et non selon la vérité des choses, de sorte que le sens est, \emph{Alexandre qui selon l'opinion des hommes a été fils de Philippe}. Et c'est pourquoi il peut être faux qu'Alexandre soit fils de Philippe, quoique l'Écriture lui donne cette qualité : \emph{Alexander Philippi, Rex Macedo}. Premier Livre des Maccabées, Chapitre 1.

\bigbreak
{3.} Les titres qui se donnent communément à certaines dignités peuvent se donner à tous ceux qui possèdent cette dignité, quoique ce qui est signifié par ce titre ne leur convienne en aucune sorte. Ainsi, parce qu'autrefois le titre de \emph{saint} et de \emph{très saint} se donnait à tous les évêques, on voit que les évêques catholiques, dans la conférence de Carthage, ne faisaient point de difficulté de donner ce nom aux évêques donatistes, \emph{sanctissimus Petilianus dixit}, quoiqu'ils sussent bien qu'il ne pouvait pas y avoir de véritable sainteté dans un évêque schismatique. Nous voyons aussi que saint Paul, dans les Actes, donne le titre de \emph{très bon} ou \emph{très excellent} à Festus, gouverneur de Judée, parce que c'était le titre qu'on donnait d'ordinaire à ces gouverneurs.

\bigbreak
{4.} Il n'en est pas de même quand une personne est l'auteur d'un titre qu'il donne à un autre, et qu'il le lui donne parlant de lui-même, non selon l'opinion des autres, ou selon l'erreur populaire ; car on peut alors lui imputer avec raison la fausseté de ces propositions. Ainsi, quand un homme dit : \emph{Aristote, qui est le prince des philosophes}, ou simplement, \emph{le prince des philosophes} a cru que l'origine des nerfs était dans le cœur, on n'aurait pas droit de lui dire que cela est faux, parce qu'Aristote n'est pas le plus excellent des philosophes ; car il suffit qu'il ait suivi en cela l'opinion commune, quoique fausse. Mais si un homme disait : \emph{M. Gassendi, qui est le plus habile des philosophes, croit qu'il y a du vide dans la nature}, on aurait sujet de disputer à cet homme la qualité qu'il voudrait donner à Gassendi, et de le rendre responsable de la fausseté qu'on pourrait prétendre se trouver dans cette proposition incidente. L'on peut donc être accusé de fausseté en donnant à la même personne un titre qui ne lui convient pas, et n'en être pas accusé en lui en donnant un autre qui lui convient encore moins dans la vérité. Par exemple, \emph{le pape Jean XII n'était ni saint, ni chaste, ni pieux}, comme Baronius le reconnaît, et cependant ceux qui l'appelaient \emph{très saint} ne pouvaient être repris de mensonge, et ceux qui l'eussent appelé \emph{très chaste} ou \emph{très pieux}, eussent été de fort grands menteurs, quoiqu'ils ne l'eussent fait que par des propositions incidentes, comme s'ils eussent dit, \emph{Jean XII, très chaste Pontife a ordonné telle chose}.

\bigbreak
Voilà pour ce qui est des premières sortes de propositions incidentes dont le \emph{qui} est explicatif; quant aux autres, dont le \emph{qui} est déterminatif, comme: \emph{Les hommes qui sont pieux; Les rois qui aiment leurs peuples}, il est certain que, pour l'ordinaire, elles ne sont pas susceptibles de fausseté, parce que l'attribut de la proposition incidente n'y est pas affirmé du sujet auquel le \emph{qui} se rapporte. Car, si l'on dit, par exemple, \emph{Que les juges qui ne font jamais rien par prière et par faveur, sont dignes de louanges}, on ne dit pas pour cela qu'il n'y ait aucun juge sur la terre qui soit dans cette perfection. Néanmoins, je crois qu'il y a toujours dans ces propositions une affirmation tacite et virtuelle, non de la convenance actuelle de l'attribut au sujet auquel le \emph{qui} se rapporte, mais de la convenance possible. Et si on se trompe en cela, je crois qu'on a raison de trouver qu'il y aurait de la fausseté dans ces propositions incidentes, comme si on disait : \emph{Les esprits qui sont carrés sont plus solides que ceux qui sont ronds}, l'idée de \emph{carré} et de \emph{rond} étant incompatible avec l'idée \emph{d'esprit} pris pour le principe de la pensée, j'estime que ces propositions incidentes devraient passer pour fausses.

Et l'on peut même dire que c'est de là que naissent la plupart de nos erreurs : car ayant l'idée d'une chose, nous y joignons souvent une autre idée incompatible, quoique par erreur nous l'ayons crue compatible, ce qui fait que nous attribuons à cette même idée ce qui ne peut lui convenir.

Ainsi, trouvant en nous-mêmes deux idées, celle de la substance qui pense, et celle de la substance étendue, il arrive souvent que lorsque nous considérons notre âme, qui est la substance qui pense, nous y mêlons insensiblement quelque chose de l'idée de la substance étendue, comme quand nous nous imaginons qu'il faut que notre âme remplisse un lieu, ainsi que le remplit un corps, et qu'elle ne serait point, si elle n'était nulle part, qui sont des choses qui ne conviennent qu'au corps; et c'est de là qu'est née l'erreur impie de ceux qui croient l'âme mortelle. On peut voir un excellent discours de saint Augustin sur ce sujet, dans le livre X de la Trinité, où il montre qu'il n'y à rien de plus facile à connaître que la nature de notre âme; mais que ce qui brouille les hommes est que, voulant la connaître, ils ne se contentent pas de ce qu'ils en connaissent sans peine, qui est que c'est une substance qui pense, qui veut, qui doute, qui sait; mais ils joignent à ce qu'elle est, ce qu'elle n'est pas, se la voulant imaginer sous quelques-uns de ces fantômes sous lesquels ils ont accoutumé de concevoir les choses corporelles.

Quand d'autre part nous considérons les corps, nous avons bien de la peine à nous empêcher d'y mêler quelque chose de l'idée de la substance qui pense ; ce qui nous fait dire des corps pesants, qu'ils veulent aller au centre; des plantes, qu'elles cherchent les aliments qui leur sont propres ; des crises d'une maladie, que c'est la nature qui s'est voulu décharger de ce qui lui nuisait; et de mille autres choses, surtout dans nos corps, que la nature veut faire ceci ou cela, quoique nous soyons bien assurés que nous ne l'avons pas voulu, n'y ayant pensé en aucune sorte, et qu'il soit ridicule de s'imaginer qu'il y ait en nous quelque autre chose que nous-même qui connaisse ce qui nous est propre ou nuisible, qui cherche l'un et qui fuie l'autre.

Je crois que c'est encore à ce mélange d'idées incompatibles qu'on doit attribuer tous les murmurés que les hommes font contre Dieu ; car il serait impossible de murmurer contre Dieu, si on le concevait véritablement selon ce qu'il est, tout-puissant, tout sage et tout bon ; mais les méchants le concevant comme tout-puissant et comme le maître souverain de tout le monde, lui attribuent tous les malheurs qui leur arrivent, en quoi ils ont raison ; et parce qu'en même temps ils le conçoivent cruel et injuste, ce qui est incompatible avec sa bonté, ils s'emportent contre lui, comme s'il avait eu tort de leur envoyer les maux qu'ils souffrent.
