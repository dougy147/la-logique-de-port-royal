\subsubsection{\centering \Large CHAPITRE XVII}
\addcontentsline{toc}{section}{\protect\numberline{}{\scshape\bfseries XVII} - \emph{Des diverses manières de mal raisonner, que l'on appelle sophismes}}
\begin{center}\emph{\large\scshape Des diverses manières de mal raisonner, que l'on appelle sophismes.}\end{center}

	\lettrine{Q}{uoique}, sachant les règles des bons raisonnements, il ne soit pas difficile de reconnaître ceux qui sont mauvais, néanmoins, comme les exemples à fuir frappent souvent davantage que les exemples à imiter, il ne sera pas inutile de représenter les principales sources des mauvais raisonnements que l'on appelle \emph{sophismes} ou \emph{paralogismes}, parce que cela donnera encore plus de facilité de les éviter, ou de s'en garder.

Je ne les réduirai qu'à sept ou huit, y en ayant quelques-uns de si grossiers, qu'ils ne méritent pas d'être remarqués.

\begin{center}{\bfseries\large I.}\end{center}
\begin{center}\emph{\scshape Prouver autre chose que ce qui est en question.}\end{center}

Ce sophisme est appelé par Aristote \emph{ignoratio elenchi}, c'est-à-dire l'ignorance de ce que l'on doit prouver contre son adversaire. C'est un vice très ordinaire dans les contestations des hommes. On dispute avec chaleur, et souvent on ne s'entend pas l'un l'autre. La passion ou la mauvaise foi fait qu'on attribue à son adversaire ce qui est éloigné de son sentiment pour le combattre avec plus d'avantage, ou qu'on lui impute les conséquences qu'on s'imagine pouvoir tirer de sa doctrine, quoiqu'il les désavoue et qu'il les nie. Tout cela peut se rapporter à cette première espèce de sophisme qu'un homme de bien et sincère doit éviter sur toutes choses.

Il eût été à souhaiter qu'Aristote, qui a eu soin de nous avertir de ce défaut, eût eu autant de soin de l'éviter ; car on ne peut dissimuler qu'il n'ait combattu plusieurs des anciens philosophes en rapportant leurs opinions peu sincèrement. Il réfute Parménides et Mélissus, pour n'avoir admis qu'un seul principe de toutes choses, comme s'ils avaient entendu par là le principe dont elles sont composées ; au lieu qu'ils entendaient le seul et unique principe dont toutes les choses ont tiré leur origine, qui est Dieu.

Il accuse tous les anciens de n'avoir pas reconnu la privation pour un des principes des choses naturelles, et il les traite sur cela de rustiques et de grossiers : mais qui ne voit que ce qu'il nous représente comme un grand mystère qui eût été ignoré jusqu'à lui ne peut jamais avoir été ignoré de personne, puisqu'il est impossible de ne pas voir qu'il faut que la matière dont on fait une table ait la privation de la forme de table, c'est-à-dire ne soit pas table avant qu'on en fasse une table? Il est vrai que ces anciens ne s'étaient pas avisés de cette connaissance pour expliquer les principes des choses naturelles, parce qu'en effet il n'y a rien qui y serve moins, étant assez visible qu'on n'en connaît pas mieux comment se fait une horloge, pour savoir que la matière dont on l'a faite a dû n'être pas horloge, avant qu'on en fît une horloge.

C'est donc une injustice à Aristote de reprocher à ces anciens philosophes d'avoir ignoré une thèse qu'il est impossible d'ignorer, et de les accuser de ne s'être pas servis, pour expliquer la nature, d'un principe qui n'explique rien ; et c'est une illusion et un sophisme que d'avoir produit au monde ce principe de la privation comme un rare secret, puisque ce n'est point ce que l'on cherche quand on tâche de découvrir les principes de la nature. On suppose comme une chose connue, qu'une chose n'est pas avant que d'être faite : mais on veut savoir de quels principes elle est composée et quelle cause l'a produite.

Aussi n'y a-t-il jamais eu de statuaire, par exemple, qui, pour apprendre à quelqu'un la manière de faire une statue, lui ait donné, pour première instruction, cette leçon par laquelle Aristote veut qu'on commence l'explication de tous les ouvrages de la nature : Mon ami, la première chose que vous devez savoir est que, pour faire une statue, il faut choisir un marbre qui ne soit pas encore cette statue que vous voulez faire.

\begin{center}{\bfseries\large II.}\end{center}
\begin{center}\emph{\scshape Supposer pour vrai ce qui est en question.}\end{center}

C'est ce qu'Aristote appelle \emph{pétition de principe}, ce qu'on voit assez être entièrement contraire à la vraie raison ; puisque, dans tout raisonnement, ce qui sert de preuve doit être plus clair et plus connu que ce qu'on veut prouver.

Cependant Galilée l'accuse, et avec justice, d'être tombé lui-même dans ce défaut, lorsqu'il veut prouver, par cet argument, que la Terre est au centre du monde.

\emph{La nature des choses pesantes est de tendre au centre du monde, et des choses légères de s'en éloigner:
\\Or, l'expérience nous fait voir que les choses pesantes tendent au centre de la Terre, et que les choses légères s'en éloignent:
\\Donc le centre de la Terre est le même que le centre du monde}.

Il est clair qu'il y a dans la majeure de cet argument une manifeste pétition de principe; car nous voyons bien que les choses pesantes tendent au centre de la Terre, mais d'où Aristote a-t-il appris qu'elles tendent au centre du monde, s'il ne suppose que le centre de la terre est le même que le centre du monde. Ce qui est la conclusion même qu'il veut prouver par cet argument.

Ce sont aussi de pures pétitions de principes que la plupart des arguments dont on se sert pour prouver un certain genre bizarre de substances, qu'on appelle dans l'École \emph{des formes substantielles}, lesquelles on prétend être corporelles, quoiqu'elles ne soient pas des corps, ce qui est assez difficile à comprendre. S'il n'y avait des formes substantielles, disent-ils, il n'y aurait point de génération ; or, il y a génération dans le monde, donc il y a des formes substantielles.

Il n'y a qu'à distinguer l'équivoque du mot de génération pour voir que cet argument n'est qu'une pure pétition de principe; car si l'on entend par le mot de génération la production naturelle d'un nouveau tout dans la nature, comme la production d'un poulet qui se forme dans un œuf, on a raison de dire qu'il y a des générations en ce sens ; mais on n'en peut pas conclure qu'il y ait des formes substantielles, puisque le seul arrangement des parties par la nature peut produire ces nouveaux touts et ces nouveaux êtres naturels. Mais si l'on entend par le mot de génération, comme ils l'entendent ordinairement, la production d'une nouvelle substance qui ne fût pas auparavant, à savoir, de cette forme substantielle, on supposera justement ce qui est en question : étant visible que celui qui nie les formes substantielles ne peut pas accorder que la nature produise des formes substantielles, et tant s'en faut qu'il puisse être porté par cet argument à avouer qu'il y en ait, qu'il doit en tirer une conclusion contraire en cette sorte : S'il y avait des formes substantielles, la nature pourrait produire des substances qui ne seraient pas auparavant; or la nature ne peut pas produire de nouvelles substances, puisque ce serait une espèce de création, et partant il n'y a point de formes substantielles.

En voici un autre de même nature : S'il n'y avait point de formes substantielles, disent-ils encore, les êtres naturels ne seraient pas des touts, qu'ils appellent, \emph{per se, totum per se}, mais des êtres par accident. Or ils sont des touts \emph{per se}: Donc il y a des formes substantielles.

Il faut encore prier ceux qui se servent de cet argument de vouloir expliquer ce qu'ils entendent par un tout \emph{per se, totum per se}. Car, s'ils entendent, comme ils font, un être composé de matière et de forme, il est clair que c'est une pétition de principe, puisque c'est comme s'ils disaient : S'il n'y avait point de formes substantielles, les êtres naturels ne seraient pas composés de matière et de formes substantielles : or ils sont composés de matière et de formes substantielles, donc il y a des formes substantielles. Que s'ils entendent autre chose, qu'ils le disent, et on verra qu'ils ne prouvent rien.

On s'est arrêté un peu en passant à faire voir la faiblesse des arguments sur lesquels on établit dans l'École ces sortes de substances qui ne se découvrent ni par le sens, ni par l'esprit, et dont on ne sait autre chose, sinon qu'on les appelle des formes substantielles ; parce que, quoique ceux qui les soutiennent le fassent à très bon-dessein, néanmoins les fondements dont ils se servent et les idées qu'ils donnent de ces formes obscurcissent et troublent des preuves très solides et très convaincantes de l'immortalité de l'âme, qui sont prises de la distinction des corps et des esprits, et de l'impossibilité qu'il y a qu'une substance qui n'est pas matière périsse par les changements qui arrivent dans la matière ; car, par le moyen de ces formes substantielles, on fournit, sans y penser, aux libertins, des exemples de substances qui périssent, qui ne sont pas proprement matière, et à qui on attribue, dans les animaux, une infinité de pensées, c'est-à-dire d'actions purement spirituelles ; et c'est pourquoi il est utile pour la religion et pour la conviction des impies et des libertins de leur ôter cette réponse, en leur faisant voir qu'il n'y a rien de plus mal fondé que ces substances périssables, qu'on appelle des formes substantielles.

On peut rapporter encore à cette sorte de sophisme la preuve que l'on tire d'un principe différent de ce qui est en question, mais que l'on sait n'être pas moins contesté que celui contre lequel on dispute. Ce sont, par exemple, deux dogmes également constants parmi les catholiques : l'un que tous les points de la foi ne peuvent pas se prouver par l'Écriture seule ; l'autre, que c'est un point de la foi, que les enfants sont capables du baptême. Ce serait donc mal raisonner à un anabaptiste de prouver contre les catholiques qu'ils ont tort de croire que les enfants soient capables du baptême, parce que nous n'en voyons rien dans l'Écriture, puisque cette preuve supposerait que l'on ne devrait croire de foi que ce qui est dans l'Écriture, ce qui est nié par les catholiques.

Enfin on peut rapporter à ce sophisme tous les raisonnements où l'on prouve une chose inconnue, par une qui est autant ou plus inconnue, où une chose incertaine par une autre qui est autant ou plus incertaine.

\begin{center}{\bfseries\large III.}\end{center}
\begin{center}\emph{\scshape Prendre pour cause ce qui n'est point cause.}\end{center}

Ce sophisme s'appelle \emph{non causa pro causa}. Il est très ordinaire parmi les hommes, et on y tombe en plusieurs manières : l'une est par la simple ignorance des véritables causes des choses. C'est ainsi que les philosophes ont attribué mille effets à la crainte du vide, qu'on a prouvé démonstrativement en ce temps, et par des expériences très ingénieuses, n'avoir pour cause que la pesanteur de l'air, comme on peut le voir dans l'excellent traité de Pascal. Les mêmes philosophes enseignent ordinairement que les vases pleins d'eau se fendent à la gelée, parce que l'eau se resserre, et ainsi laisse du vide que la nature ne peut souffrir, et néanmoins on a reconnu qu'ils ne se rompent que parce qu'au contraire l'eau étant gelée occupe plus de place qu'avant que d'être gelée, ce qui fait aussi que la glace nage sur l'eau.

On peut rapporter au même sophisme, quand on se sert de causes éloignées et qui ne prouvent rien, pour prouver des choses, ou assez claires d'elles-mêmes, ou fausses, ou au moins douteuses, comme quand Aristote veut prouver que le monde est parfait par cette raison. \emph{Le monde est parfait, parce qu'il contient des corps; le corps est parfait, parce qu'il a trois dimensions ; les trois dimensions sont parfaites, parce que trois sont tout} (quia tria sunt omnia) \emph{et trois sont tout parce qu'on ne se sert pas du mot de tout, quand il n'y a qu'une chose ou deux, mais seulement quand il y en a trois}. On prouvera par cette raison que le moindre atome est aussi parfait que le monde, puisqu'il a trois dimensions aussi bien que le monde ; mais tant s'en faut que cela prouve que le monde soit parfait, qu'au contraire tout corps, en tant que corps, est essentiellement imparfait, et que la perfection du monde consiste principalement en ce qu'il enferme des créatures qui ne sont pas corps.

Le même philosophe prouve qu'il y a trois mouvements simples, parce \emph{qu'il y a trois dimensions}. Il est difficile de voir la conséquence de l'un à l'autre.

Il prouve aussi que le ciel est inaltérable et incorruptible parce qu'il se meut circulairement, et qu'il n'y a rien de contraire au mouvement circulaire. Mais $1$. on ne voit pas ce que fait la contrariété du mouvement à la corruption ou à l'altération du corps. $2$. On voit encore moins pourquoi le mouvement circulaire, d'orient en occident, n'est pas contraire à un autre mouvement circulaire d'occident en orient.

L'autre cause qui fait tomber les hommes dans ce sophisme est la sotte vanité qui nous fait avoir honte de reconnaître notre ignorance ; car c'est de là qu'il arrive que nous aimons mieux nous forger des causes imaginaires des choses dont on nous demande raison, que d'avouer que nous n'en savons pas la cause, et la manière dont nous nous échappons de cette confession de notre ignorance est assez plaisante. Quand nous voyons un effet dont la cause nous est inconnue, nous nous imaginons l'avoir découverte, lorsque nous avons joint à cet effet un mot général de \emph{vertu} et de \emph{faculté}, qui ne forme dans notre esprit aucune autre idée, sinon que cet effet a quelque cause, ce que nous savions bien avant que d'avoir trouvé ce mot. Il n'y a personne, par exemple, qui ne sache que ses artères battent ; que le fer étant proche de l'aimant va s'y joindre, que le séné purge, et que le pavot endort. Ceux qui ne font point profession de science, et à qui l'ignorance n'est pas honteuse, avouent franchement qu'ils connaissent ces effets, mais qu'ils n'en savent pas la cause ; au lieu que les savants, qui rougiraient d'en dire autant, s'en tirent d'une autre manière, et prétendent qu'ils ont découvert la vraie cause de ces effets, qui est qu'il y a dans les artères une vertu pulsifique, dans l'aimant une vertu magnétique, dans le séné une vertu purgative, et dans le pavot une vertu soporifique. Voilà qui est fort commodément résolu, et il n'y a point de Chinois qui n'eût pu avec autant de facilité se tirer de l'admiration où on était des horloges en ce pays-là, lorsqu'on leur en apporta d'Europe, car il n'aurait eu qu'à dire qu'il connaissaït parfaitement la raison de ce que les autres trouvaient si merveilleux, et que ce n'était autre chose, sinon qu'il y avait dans cette machine une vertu \emph{indicatrice} qui marquait les heures sur le cadran, et une vertu \emph{soporifique} qui les faisait sonner; il se serait rendu aussi savant par là dans la connaissance des horloges que le sont ces philosophes dans la connaissance du battement des artères, et des propriétés de l'aimant, du séné et du pavot.

Il y a encore d'autres mots qui servent à rendre les hommes savants à peu de frais, comme de sympathie, d'antipathie, de qualités occultes ; mais encore tous ceux-là ne diraient rien de faux s'ils se contentaient de donner à ces mots de \emph{vertu} et de \emph{faculté} une notion générale de cause quelle qu'elle soit, intérieure ou extérieure, dispositive ou active. Car il est certain qu'il y a dans l'aimant quelque disposition qui fait que le fer va plutôt s'y joindre qu'à une autre pierre, et il a été permis aux hommes d'appeler cette disposition, en quoi que ce soit qu'elle consiste, \emph{vertu magnétique}, de sorte que s'ils se trompent, c'est seulement en ce qu'ils s'imaginent en être plus savants pour avoir trouvé ce mot, ou bien en ce que par là ils veulent que nous entendions une certaine qualité imaginaire, par laquelle l'aimant attire le fer, laquelle ni eux ni personne n'a jamais conçue.

Mais il y en a d'autres qui nous donnent pour les véritables causes de la nature de pures chimères, comme font les astrologues, qui rapportent tout aux influences des astres et qui ont même trouvé par là qu'il fallait qu'il y eût un ciel immobile au-dessus de tous ceux à qui ils donnent du mouvement, parce que la Terre portant diverses choses en divers pays, (\emph{Non omnis fert omnia tellus. India miitit ebur, molles sua thura Sabaei}) on n'en pouvait rapporter la cause qu'aux influences d'un ciel qui, étant immobile, eût toujours les mêmes aspects sur les mêmes endroits de la Terre.

Aussi l'un d'eux ayant entrepris de prouver par des raisons physiques l'immobilité de la Terre, fait l'une de ses principales démonstrations de cette raison mystérieuse, que si la Terre tournait autour du Soleil, les influences des astres iraient de travers, ce qui causerait un grand désordre dans le monde.

C'est par ces influences qu'on épouvante les peuples, quand on voit paraître quelque comète, ou qu'il arrive quelque grande éclipse, comme celle de l'an 1654, qui devait bouleverser le monde, et principalement la ville de Rome, ainsi qu'il était expressément marqué dans la chronologie de Helvicus, \emph{Roma fatalis}, quoiqu'il n'y ait aucune raison, ni que les comètes et les éclipses puissent avoir aucun effet considérable sur la Terre, ni que des causes générales, comme celle-là, agissent plutôt en un endroit qu'en un autre, et menacent plutôt un roi ou un prince qu'un artisan ; ainsi en voit-on cent qui ne sont suivies d'aucun effet remarquable. Que s'il arrive quelquefois des guerres, des mortalités, des pestes et la mort de quelque prince après des comètes ou des éclipses, il en arrive aussi sans comètes et sans éclipses ; et d'ailleurs ces effets sont si généraux et si communs, qu'il est bien difficile qu'ils n'arrivent tous les ans en quelque endroit du monde : de sorte que ceux qui disent en l'air que cette comète menace quelque grand de la mort, ne se hasardent pas beaucoup.

C'est encore pis quand ils donnent ces influences chimériques pour la cause des inclinations des hommes, vicieuses ou vertueuses, et même de leurs actions particulières et des événements de leur vie, sans en avoir d'autre fondement, sinon qu'entre mille prédictions il arrive par hasard que quelques-unes sont vraies ; mais si l'on veut juger des choses par le bon sens, on avouera qu'un flambeau allumé dans la chambre d'une femme qui accouche doit avoir plus d'effet sur le corps de son enfant, que la planète de Saturne en quelque aspect qu'elle le regarde, et avec quelque autre qu'elle soit jointe.

Aussi nous voyons qu'on attribue souvent des effets à la Lune, auxquels l'expérience fait voir qu'elle n'a aucune part, comme des personnes fort exactes m'ont assuré l'avoir éprouvé. On dit, par exemple, qu'il y a beaucoup de moëlles dans les os des animaux en pleine Lune, et qu'il y en a peu ou point dans la nouvelle Lune. Qu'on en fasse l'expérience, et on trouvera que cela est faux, et qu'il arrive dans tous les temps de la Lune que quelques os ont beaucoup de moëlle, et que d'autres en ont peu.

On dit aussi qu'il y a des pierres que la Lune mange, parce que ce sont celles qui sont exposées à la Lune qui se gâtent plus que les autres; mais come elles ne sauraient être exposées à la Lune qu'elles ne soient aussi exposées aux vents du midi, qui étant fort humides sont fort corrompants, il y a bien plus d'apparence d'attribuer cet effet à ces vents qu'à la Lune.

Enfin, il y en a qui apportent des causes chimériques d'effets chimériques, comme ceux qui, supposant que la nature abhorre le vide, et qu'elle fait des efforts pour l'éviter (ce qui est un effet imaginaire : car la nature n'a horreur de rien, et tous les effets qu'on attribue à cette horreur dépendent de la seule pesanteur de l'air), ne laissent pas d'apporter des raisons de cette horreur imaginaire, qui sont encore plus imaginaires. La nature abhorre le vide, dit l'un d'entre eux, parce qu'elle a besoin de la continuité des corps pour faire passer les influences, et pour la propagation des qualités. C'est une étrange sorte de science que celle-là, qui prouve ce qui n'est point par ce qui n'est point.

C'est encore à cette sorte de sophisme qu'on doit rapporter cette tromperie ordinaire de l'esprit humain, \emph{post hoc, ergo propter hoc}. Cela est arrivé ensuite de telle chose : il faut donc que cette chose en soit la cause. C'est par là que l'on a conclu que c'était une étoile nommée Canicule, qui était cause de la chaleur extraordinaire que l'on sent durant les jours que l'on appelle caniculaires ; ce qui a fait dire à Virgile, en parlant de cette étoile que l'on appelle en latin \emph{Seirius} :

	\begin{tabularx}{\textwidth}{X}
		\emph{Aut Seirius ardor.} \\
		\emph{Ille sitim morbosque ferens mortalibus aegris} \\
		\emph{Nascitur, et laevo contristat lumine caelum.} \\
	\end{tabularx}

Cependant, comme Gassendi a fort bien remarqué, il n'y a rien de moins vraisemblable que cette imagination; car cette étoile étant de l'autre côté de la ligne, ses effets devraient être plus forts sur les lieux où elle est plus perpendiculaire; et néanmoins les jours que nous appelons caniculaires ici, sont le temps de l'hiver de ce côté-là : de sorte qu'ils ont bien plus de sujet de croire en ce pays-là que la canicule leur apporte du froid, que nous n'en avons de croire qu'elle nous cause le chaud.

\begin{center}{\bfseries\large IV.}\end{center}
\begin{center}\emph{\scshape Juger d'une chose par ce qu'il ne lui convient que par accident.}\end{center}

Ce sophisme est appelé dans l'École \emph{fallacia accidentis}, qui est lorsque l'on tire une conclusion absolue, simple et sans restriction de ce qui n'est vrai que par accident. C'est ce que font tant de gens qui déclament contre l'antimoine, parce qu'étant mal appliqué il produit de mauvais effets ; et d'autres qui attribuent à l'éloquence tous les mauvais effets qu'elle produit quand on en abuse ; ou à la médecine, les fautes de quelques médecins ignorants.

C'est par là que les hérétiques de ce temps ont fait croire à tant de peuples abusés, qu'on devait rejeter comme des inventions de Satan, l'invocation des saints, la vénération des reliques, la prière pour les morts ; parce qu'il s'était glissé des abus et de la superstition parmi ces saintes pratiques autorisées par toute l'Antiquité ; comme si le mauvais usage que les hommes peuvent faire des meilleures choses les rendait mauvaises.

On tombe souvent aussi dans ce mauvais raisonnement, quand on prend les simples occasions pour les véritables causes ; comme qui accuserait la religion chrétienne d'avoir été la cause du massacre d'une infinité de personnes qui ont mieux aimé souffrir la mort que de renoncer à Jésus-Christ ; au lieu que ce n'est pas à la religion chrétienne, ni à la constance des martyrs, qu'on doit attribuer ces meurtres, mais à la seule injustice et à la seule cruauté des païens.

C'est par ce sophisme qu'on impute souvent aux gens de bien d'être cause de tous les maux qu'ils eussent pu éviter en faisant des choses qui eussent blessé leur conscience, parce que s'ils avaient voulu se relâcher dans cette exacte observance de la loi de Dieu, ces maux ne seraient pas arrivés.

On voit aussi un exemple considérable de ce sophisme dans le raisonnement ridicule des Épicuriens, qui concluaient que les dieux devaient avoir une forme humaine, parce que dans toutes les choses du monde, il n'y avait que l'homme qui eût l'usage de la raison. \emph{Les Dieux}, disaient-ils, \emph{sont très heureux; nul ne peut être heureux sans la vertu ; il n'y a point de vertu sans la raison ; et la raison ne se trouve nulle part ailleurs qu'en ce qui a la forme humaine ; il faut donc avouer que les dieux sont en forme humaine}. Mais ils étaient bien aveugles de ne pas voir que, quoique dans l'homme la substance qui pense et qui raisonne soit jointe à un corps humain, ce n'est pas néanmoins la figure humaine qui fait que l'homme pense et raisonne, étant ridicule de s'imaginer que la raison et la pensée dépendent de ce qu'il a un nez, une bouche, des joues, deux bras, deux mains, deux pieds; et ainsi c'était un sophisme puéril à ces philosophes, de conclure qu'il ne pouvait y avoir de raison que dans la forme humaine, parce que dans l'homme elle se trouvait jointe par accident à la forme humaine.

\newpage

\begin{center}{\bfseries\large V.}\end{center}
\begin{center}\emph{\scshape Passer du sens divisé au sens composé ou du sens composé au sens divisé.}\end{center}

L'un de ces sophismes s'appelle \emph{fallacia compositionis}, et l'autre \emph{fallacia divisionis}. On les comprendra mieux par des exemples.

Jésus-Christ dit, dans l'Évangile, en parlant de ses miracles : \emph{Les aveugles voient, les boiteux marchent droit, les sourds entendent}. Cela ne peut être vrai qu'en prenant ces choses séparément, et non conjointement, c'est-à-dire, dans le sens divisé, et non dans le sens composé ; car les aveugles ne voyaient pas demeurant aveugles, et les sourds n'entendaient pas demeurant sourds ; mais ceux qui avaient été aveugles auparavant et ne l'étaient plus voyaient, et de même des sourds.

C'est aussi dans le même sens qu'il est dit, dans l'Écriture, que Dieu justifie les impies. Car cela ne veut pas dire qu'il tient pour justes ceux qui sont encore impies; mais qu'il rend justes, par sa grâce, ceux qui auparavant étaient impies.

Il y a, au contraire, des propositions qui ne sont véritables qu'en un sens opposé à celui-là, qui est le sens composé. Comme quand saint Paul dit : Que les médisants, les fornicateurs, les avares n'entreront point dans le royaume des cieux ; car cela ne veut pas dire que nul de ceux qui auront eu ces vices ne seront sauvés; mais seulement que ceux qui y demeureront attachés, et qui ne les auront point quittés, en se convertissant à Dieu, n'auront point de part au royaume du ciel.

Il est aisé de voir qu'on ne peut passer, sans sophisme, de l'un de ces sens à l'autre, et que ceux-là, par exemple, raisonneraient mal qui se promettraient le ciel, en demeurant dans leurs crimes, parce que Jésus-Christ est venu pour sauver les pécheurs, et qu'il dit, dans l'Évangile, que les femmes de mauvaise vie précéderont les Pharisiens dans le royaume de Dieu; ou qui, au contraire, ayant mal vécu, désespéreraient de leur salut, comme n'ayant plus rien à attendre que la punition de leurs crimes ; parce qu'il est dit que la colère de Dieu est réservée à tous ceux qui vivent mal, et que toutes les personnes vicieuses n'ont point de part à l'héritage de Jésus-Christ. Les premiers passeraient du sens divisé au sens composé, en se promettant, quoique toujours pécheurs, ce qui n'est promis qu'à ceux qui cessent de l'être par une véritable conversion : et les derniers passeraient du sens composé au sens divisé, en appliquant à ceux qui ont été pécheurs et qui cessent de l'être en se convertissantà Dieu, ce qui ne regarde que les pécheurs qui demeurent dans leurs péchés et dans leur mauvaise vie.

\begin{center}{\bfseries\large VI.}\end{center}
\begin{center}\emph{\scshape Passer de ce qui est vrai à quelque egard, à ce qui est vrai simplement.}\end{center}

C'est ce qu'on appelle dans l'École \emph{a dicto secundum quid ad dictum simpliciter}. En voici des exemples : les Épicuriens prouvaient encore que les dieux devaient avoir la forme humaine, parce qu'il n'y en a point de plus belle que celle-là, et que tout ce qui est beau doit être en Dieu. C'était mal raisonner; car la forme humaine n'est point absolument une beauté, mais seulement au regard des corps ; et ainsi, n'étant une perfection qu'à quelque égard et non simplement, il ne s'ensuit pas qu'elle doive être en Dieu, parce que toutes les perfections sont en Dieu, n'y ayant que celles qui sont simplement perfections, c'est-à-dire qui n'enferment aucune imperfection, qui soient nécessairement en Dieu.

Nous voyons aussi, dans Cicéron, au troisième livre de la Nature des dieux, un argument ridicule de Cotta contre l'existence de Dieu, qui peut se rapporter au même défaut. \emph{Comment}, dit-il, \emph{pouvons-nous concevoir Dieu, ne pouvant lui attribuer aucune vertu? Car dirons-nous qu'il a de la prudence? Mais la prudence consistant dans le choix des biens et des maux, quel besoin Dieu peut-il avoir de ce choix, n'étant capable d'aucun mal? Dirons-nous qu'il a de l'intelligence et de la raison? Mais la raison et l'intelligence nous servent à découvrir ce qui nous est inconnu par ce qui nous est connu : or, il ne peut y avoir rien d'inconnu à Dieu. La justice ne peut aussi être en Dieu, puisqu'elle ne regarde que la société des hommes; ni la tempérance, parce qu'il n'a point de voluptés à modérer ; ni la force, parce qu'il n'est susceptible ni de douleur ni de travail, et qu'il n'est exposé à aucun péril. Comment donc pourrait être Dieu ce qui n'aurait ni intelligence, ni vertu?}

Il est difficile de rien concevoir de plus impertinent que cette manière de raisonner. Elle est semblable à la pensée d'un paysan qui, n'ayant jamais vu que des maisons couvertes de chaume, et ayant ouï dire qu'il n'y a point dans les villes de toits de chaume, en conclurait qu'il n'y a point de maisons dans les villes, et que ceux qui y habitent sont bien malheureux, étant exposés à toutes les injures de l'air. C'est comme Cotta ou plutôt Cicéron raisonne. Il ne peut y avoir en Dieu de vertus semblables à celles qui sont dans les hommes : donc il ne peut y avoir de vertus en Dieu. Et ce qui est merveilleux, c'est qu'il ne conclut qu'il n'y a point de vertu en Dieu, que parce que l'imperfection qui se trouve dans la vertu humaine ne peut être en Dieu, de sorte que ce lui est une preuve que Dieu n'a point d'intelligence, parce que rien ne lui est caché; c'est-à-dire qu'il ne voit rien, parce qu'il voit tout ; qu'il ne peut rien, parce qu'il peut tout ; qu'il ne jouit d'aucun bien, parce qu'il possède tous les biens.

\begin{center}{\bfseries\large VII.}\end{center}
\begin{center}\emph{\scshape Abuser de l'ambiguïté des mots, ce qui peut se faire en diverses manières.}\end{center}

On peut rapporter à cette espèce de sophisme tous les syllogismes qui sont vicieux, parce qu'il s'y trouve quatre termes ; soit parce que le milieu y est pris deux fois particulièrement; ou parce qu'il est pris en un sens dans la première proposition, et en un autre sens dans la seconde ; ou enfin parce que les termes de la conclusion ne sont pas pris dans le même sens dans les prémisses que dans la conclusion : car nous ne restreignons pas le mot d'ambiguïté aux seuls mots qui sont grossièrement équivoques, ce qui ne trompe presque jamais ; mais nous comprenons par là tout ce qui peut faire changer de sens à un mot, surtout lorsque les hommes ne s'aperçoivent pas aisément de ce changement, parce que diverses choses étant signifiées par le même son, ils les prennent pour la même chose. Sur quoi on peut voir ce qui a été dit vers la fin de la première partie, où l'on a aussi parlé du remède qu'on doit apporter à la confusion des mots ambigus, en les définissant si nettement qu'on n'y puisse être trompé.

Ainsi, je me contenterai d'apporter quelques exemples de cette ambiguïté, qui trompe quelquefois d'habiles gens. Telle est celle qui se trouve dans les mots qui signifient quelque tout, qui peut se prendre ou collectivement pour toutes ses parties ensemble, ou distributivement pour chacune de ses parties. C'est par là qu'on doit résoudre ce sophisme des Stoïciens, qui concluaient que le monde était un animal doué de raison, \emph{parce que ce qui a l'usage de la raison est meilleur que ce qui ne l'a point, Or, il n'y a rien, disaient-ils, qui soit meilleur que le monde : donc le monde a l'usage de la raison}. La mineure de cet argument est fausse, parce qu'ils attribuaient au monde ce qui ne convient qu'à Dieu, qui est d'être tel qu'on ne puisse rien concevoir de meilleur et de plus parfait. Mais, en se bornant dans les créatures, quoique l'on puisse dire qu'il n'y a rien de meilleur que le monde, en le prenant collectivement pour l'universalité de tous les êtres que Dieu a créés, tout ce qu'on en peut conclure au plus, est que le monde a l'usage de la raison, selon quelques-unes de ses parties telles que sont les anges et les hommes, et non pas que le tout ensemble soit un animal qui ait l'usage de la raison.

Ce serait de même mal raisonner que de dire : l'homme pense : or, l'homme est composé de corps et d'âme : donc le corps et l'âme pensent: car il suffit, afin que l'on puisse attribuer la pensée à l'homme entier, qu'il pense selon une des parties ; d'où il ne s'ensuit nullement qu'il pense selon l'autre.

Les arguments d'Aristote pour l'éternité du monde ne sont aussi fondés que sur l'ambiguïté de quelques mots. \emph{On ne saurait}, dit-il, \emph{trouver dans le temps de premier instant, parce que tout instant est la fin d'un temps précédent, et le commencement d'un suivant : Donc le temps est éternel : Donc le mouvement l'est aussi}. On prouvera par là qu'il n'y a point de mouvement de roue de moulin qui n'ait été éternel. Mais c'est une équivoque sur le mot d'\emph{instant}.

C'en est une sur le mot de \emph{privation}, lorsqu'il dit, \emph{Que le mobile n'a pas pu précéder le mouvement, parce qu'il aurait été en repos, et qu'ainsi il aurait fallu qu'avant cela il y eût eu un mouvement dont ce repos fut la privation}. Comme s'il n'y avait point d'autre privation que de perdre ce qu'on avait déjà; et que ceux qui naissent aveugles ne fussent pas privés de la vue, quoi qu'ils n'aient jamais vu clair.


\begin{center}{\bfseries\large VIII.}\end{center}
\begin{center}\emph{\scshape Tirer une conclusion générale d'une induction défectueuse.}\end{center}

On appelle induction, lorsque la recherche de plusieurs choses particulières nous mène à la connaissance d'une vérité générale. Ainsi, lorsqu'on a éprouvé sur beaucoup de mers que l'eau en est salée, et sur beaucoup de rivières que l'eau en est douce, on conclut généralement que l'eau de la mer est salée, et celle des rivières douce. Les diverses épreuves qu'on a faites que l'or ne diminue point au feu, ont fait juger que cela est vrai de tout or. Et comme on n'a point trouvé de peuple qui ne parle, on croit pour très certain que tous les hommes parlent ; c'est-à-dire se servent des sons pour signifier leur pensée.

C'est même par là que toutes nos connaissances commencent, parce que les choses singulières se présentent à nous avant les universelles, quoique ensuite les universelles servent à connaître les singulières.

Mais il est vrai néanmoins que l'induction seule n'est jamais un moyen certain d'acquérir une science parfaite, comme on le fera voir en un autre endroit, la considération des choses singulières servant seulement d'occasion à notre esprit de faire attention à ses idées naturelles, selon lesquelles il juge de la vérité des choses en général; car il est vrai, par exemple, que je ne me serais peut-être jamais avisé de considérer la nature d'un triangle, si je n'avais vu un triangle qui m'a donné occasion d'y penser : mais ce n'est pas néanmoins l'examen particulier de tous les triangles qui m'a fait conclure généralement et certainement de tous que l'espace qu'ils comprennent est égal à celui du rectangle de toute leur base, et de la moitié de leur hauteur (car cet examen serait impossible), mais la seule considération de ce qui est renfermé dans l'idée du triangle que je trouve dans mon esprit.

Quoi qu'il en soit, réservant à un autre endroit de traiter de cette matière, il suffit de dire ici que les inductions défectueuses, c'est-à-dire qui ne sont pas entières, font souvent tomber en erreur, et je me contenterai d'en rapporter un exemple remarquable.

Toutes les philosophies avaient cru jusqu'à ce temps, comme une vérité indubitable, qu'une seringue étant bien bouchée, il était impossible d'en tirer le piston sans la faire crever, et que l'on pouvait faire monter de l'eau si haut qu'on voudrait par des pompes aspirantes : ce qui le faisait croire si fermement, c'est qu'on s'imaginait s'en être assuré par une induction très certaine, en ayant fait une infinité d'expériences; mais l'un et l'autre s'est trouvé faux, parce que l'on a fait de nouvelles expériences qui ont fait voir que le piston d'une seringue, quelque bouchée qu'elle fût, pouvait se tirer, pourvu qu'on y employât une force égale au poids d'une colonne d'eau de plus de trente-trois pieds de haut de la grosseur de la seringue, et qu'on ne saurait lever de l'eau par une pompe aspirante plus haut de trente-deux à trente-trois pieds.


