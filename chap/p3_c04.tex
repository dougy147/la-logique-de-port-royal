\subsubsection{\centering \Large CHAPITRE IV}
\addcontentsline{toc}{section}{\protect\numberline{}{\scshape\bfseries IV} - \emph{Des figures et des modes des syllogismes en général. Qu'il ne peut y en avoir que quatre figures}}
\begin{center}\emph{\large\scshape Des figures et des modes des syllogismes en général. Qu'il ne peut y en avoir que quatre figures.}\end{center}

	\lettrine{A}{près} l'établissement des règles générales qui doivent être nécessairement observées dans tous les syllogismes simples, il reste à voir combien il peut y avoir de ces sortes de syllogismes.

On peut dire en général qu'il y en a autant de sortes qu'il peut y avoir de différentes manières de disposer, en gardant ces règles, les trois propositions d'un syllogisme, et les trois termes dont elles sont composées.

La disposition des trois propositions selon leurs quatre différences, A, E, I, O, s'appelle \emph{mode}.

Et la disposition des trois termes, c'est-à-dire du moyen avec les deux termes de la conclusion, s'appelle \emph{figure}.

Or, on peut compter combien il peut y avoir de modes concluants, à n'y considérer point les différentes figures selon lesquelles un même mode peut faire divers syllogismes ; car, par la doctrine des combinaisons, quatre termes (comme sont A, E, I, O), étant pris trois à trois, ne peuvent être différemment arrangés qu'en 64 manières ; mais de ces 64 diverses manières, ceux qui voudront prendre la peine de les considérer chacune à part, trouveront qu'il y en a :

Vingt-huit exclues par la troisième et la sixième règle, qu'on ne conclut rien de deux négatives et de deux particulières.

Dix-huit par le cinquième, que la conclusion suit la plus faible partie.

Six par la quatrième. Qu'on ne peut conclure négativement de deux affirmatives.

Un, à savoir I, E, O, par le troisième corollaire des règles générales.

Un, à savoir A, E, O, par le sixième corollaire des règles générales.

Ce qui fait en tout 54. Et par conséquent il ne reste que dix modes concluants.

\begin{center}
$ \text {4 Affirmatifs} \left \{
    \begin{array}{ccc}
	    \text {A,} & \text{A,} & \text{A} \\
  	    \text {A,} & \text{I,} & \text{I} \\
  	    \text {A,} & \text{A,} & \text{I} \\
  	    \text {I,} & \text{A,} & \text{I} \\
    \end{array}
	    \right \} $
\end{center}
\begin{center}
$ \text {6 Négatifs} \left \{
    \begin{array}{ccc}
	    \text {E,} & \text{A,} & \text{E} \\
	    \text {A,} & \text{E,} & \text{E} \\
	    \text {E,} & \text{A,} & \text{O} \\
	    \text {A,} & \text{O,} & \text{O} \\
	    \text {O,} & \text{A,} & \text{O} \\
	    \text {E,} & \text{I,} & \text{O} \\
    \end{array}
	    \right \} $
\end{center}

Mais cela ne fait pas qu'il n'y ait que dix espèces de syllogismes, parce qu'un seul de ces modes en peut faire diverses espèces selon l'autre manière d'où se prend la diversité des syllogismes, qui est la différente disposition des trois termes, que nous avons déjà dit s'appeler \emph{figure}.

Or, pour cette disposition des trois termes, elle ne peut regarder que les deux premières propositions, parce que la conclusion est supposée avant qu'on fasse le syllogisme pour la prouver; et ainsi, le moyen ne pouvant s'arranger qu'en quatre manières différentes avec les deux termes de la conclusion, il n'y a aussi que 4 figures possibles.

Car, ou le moyen \emph{est sujet en la majeure et attribut en la mineure}. Ce qui fait la première figure.

Ou il est \emph{attribut en la majeure et en la mineure}. Ce qui fait la deuxième figure.

Ou il est \emph{sujet en l'une et en l'autre}, ce qui fait la troisième figure.

Ou il est enfin \emph{attribut dans la majeure et sujet en la mineure}. Ce qui peut faire une quatrième figure : étant certain que l'on peut conclure quelquefois nécessairement en cette matière, ce qui suffit pour faire un vrai syllogisme.

Néanmoins, parce qu'on ne peut conclure de cette quatrième manière, qu'en une façon qui n'est nullement naturelle, et où l'esprit ne se porte jamais, Aristote et ceux qui l'ont suivi n'ont pas donné à cette manière de raisonner le nom de figure. Galien a soutenu le contraire, et il est clair que ce n'est qu'une dispute de mots, qui doit se décider en leur faisant dire de part et d'autre ce qu'ils entendent par le mot de figure.

Mais ceux-là se trompent sans doute, qui prennent pour une quatrième figure, qu'ils accusent Aristote de n'avoir pas reconnue, les arguments de la première, dont la majeure et la mineure sont transposées, comme lorsqu'on dit, \emph{Tout corps est divisible; tout ce qui est divisible est imparfait. Donc tout corps est imparfait}. Je m'étonne que Gassendi soit tombé dans cette erreur. Car il est ridicule de prendre pour la majeure d'un syllogisme, la proposition qui se trouve la première, et pour mineure, celle qui se trouve la seconde : si cela était il faudrait prendre souvent la conclusion même pour la majeure ou la mineure d'un argument, puisque c'est assez souvent la première ou la seconde des trois propositions qui le composent, comme dans ces vers d'Horace, la conclusion est la première, la mineure la seconde, et la majeure la troisième.

	\begin{tabularx}{\textwidth}{X}
		\emph{Qui melior servo qui liberior sit avarus.} \\
		\emph{In triviis fixum cum se dimittit ad assem} \\
		\emph{Non video : nam qui cupiet, metuet quoque; porro} \\
		\emph{Qui metuens vivit liber mihi non erit unquam.} \\
	\end{tabularx}


Car tout cela se réduit à cet argument :

	\begin{tabularx}{\textwidth}{X}
		\emph{Celui qui est dans de continuelles appréhensions n'est point libre :} \\
		\emph{Tout avare est dans de continuelles appréhensions :} \\
		\emph{Donc nul avare n'est libre.} \\
	\end{tabularx}

Il ne faut donc point avoir égard au simple arrangement local des propositions qui ne changent rien dans l'esprit ; mais on doit prendre pour syllogisme de la première figure tous ceux où le milieu est sujet dans la proposition où se trouve le grand terme (c'est-à-dire l'attribut de la conclusion ) et attribut dans celle où se trouve le petit terme (c'est-à-dire le sujet de la conclusion) : et ainsi il ne reste pour quatrième figure que ceux au contraire où le milieu est attribut dans la majeure et sujet dans la mineure ; et c'est ainsi que nous les appellerons, sans que personne puisse le trouver mauvais, puisque nous avertissons par avance que nous n'entendons par ce terme de figure qu'une différente disposition du moyen.

