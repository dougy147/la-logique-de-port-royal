\subsubsection{\centering \Large CHAPITRE XII}
\addcontentsline{toc}{section}{\protect\numberline{}{\scshape\bfseries XII} - \emph{De la définition qu'on appelle définition de choses}}
\begin{center}\emph{\large\scshape De la définition qu'on appelle définition de choses.}\end{center}

	\lettrine{N}{ous} avons parlé fort au long, dans la première partie, des définitions de noms, et nous avons montré qu'il ne fallait pas les confondre avec les définitions des choses; parce que les définitions des noms sont arbitraires, au lieu que les définitions des choses ne dépendent point de nous, mais de ce qui est enfermé dans la véritable idée d'une chose, et ne doivent point être prises pour principes, mais être considérées comme des propositions qui doivent souvent être confirmées par raison, et qui peuvent être combattues. Ce n'est donc que de cette dernière sorte de définition que nous parlons en ce lieu.

Il y en a de deux sortes : l'une plus exacte, qui retient le nom de définition; l'autre moins exacte, qu'on appelle description.

La plus exacte est celle qui explique la nature d'une chose par ses attributs essentiels, dont ceux qui sont communs s'appellent \emph{genre}, et ceux qui sont propres \emph{différence}.

Ainsi on définit l'homme un animal raisonnable ; l'esprit, une substance qui pense; le corps, une substance étendue; Dieu, l'être parfait. Il faut, autant que l'on peut, que ce qu'on met pour genre dans la définition, soit le genre prochain du défini, et non pas seulement le genre éloigné.

On définit aussi quelquefois par les parties intégrantes, comme lorsqu'on dit que l'homme est une chose composée d'un esprit et d'un corps. Mais alors même il y a quelque chose qui tient lieu de genre, comme le mot de chose composée, et le reste tient lieu de différence.

La définition moins exacte, qu'on appelle description, est celle qui donne quelque connaissance d'une chose par les accidents qui lui sont propres, et qui la déterminent assez pour en donner quelque idée qui la discerne des autres.

C'est en cette manière qu'on décrit les herbes, les fruits, les animaux, par leur figure, par leur grandeur, par leur couleur et autres semblables accidents. C'est de Cette nature que sont les descriptions des poètes et des orateurs.

Il y a aussi des définitions ou descriptions qui se font par les causes, par la matière, par la forme, par la fin, etc., comme si on définit une horloge, une machine de fer composée de diverses roues, dont le mouvement réglé est propre à marquer les heures.

Il y a trois choses nécessaires à une bonne définition : Qu'elle soit universelle, Qu'elle soit propre, Qu'elle soit claire.

\bigbreak
$1$. Il faut qu'une définition soit universelle, c'est-à-dire qu'elle comprenne tout le défini. C'est pourquoi la définition commune du \emph{temps}, que c'est \emph{la mesure du mouvement}, n'est peut-être pas bonne, parce qu'il y a grande apparence que le temps ne mesure pas moins le repos que le mouvement, puisqu'on dit aussi bien qu'une chose a été tant de temps en repos, comme on dit qu'elle s'est remuée pendant tant de temps; de sorte qu'il semble que le temps ne soit autre chose que la durée de la créature en quelque état qu'elle soit.

\bigbreak
$2$. Il faut qu'une définition soit propre, c'est-à-dire qu'elle ne convienne qu'au défini. C'est pourquoi la définition commune des éléments, \emph{un corps simple corruptible}, ne semble pas bonne. Car les corps célestes n'étant pas moins simples que les éléments par le propre aveu de ces philosophes, on n'a aucune raison de croire qu'il ne se fasse pas dans les cieux des altérations semblables à celles qui se font sur la Terre, puisque, sans parler des comètes, qu'on sait maintenant n'être point formées des exhalaisons de la Terre, comme Aristote se l'était imaginé, on a découvert des taches dans le soleil qui s'y forment, et qui s'y dissipent de la même sorte que nos nuages, quoique ce soient de bien plus grands corps.

\bigbreak
$3$. Il faut qu'une définition soit claire, c'est-à-dire qu'elle nous serve à avoir une idée plus claire et plus distincte de la chose qu'on définit, et qu'elle nous en fasse, autant qu'il se peut, comprendre la nature; de sorte qu'elle puisse nous aider à rendre raison de ses principales propriétés. C'est ce qu'on doit principalement considérer dans les définitions, et c'est ce qui manque à une grande partie des définitions d'Aristote.

Car qui est celui qui a mieux compris la nature du mouvement par cette définition: \emph{Actus entis in potentia quatenus in potentia}: L'acte d'un être en puissance en tant qu'il est en puissance ? L'idée que la nature nous en fournit n'est-elle pas cent fois plus claire que celle-là ? et à qui servit-elle jamais pour expliquer aucune des propriétés du mouvement ?

Les quatre célèbres définitions de ces quatre premières qualités, \emph{le sec, l'humide, le chaud, le froid}, ne sont pas meilleures.

\emph{Le sec}, dit-il, est ce qui est facilement retenu dans ses bornes, et difficilement dans celles d'un autre corps : \emph{quod suo termino facile continetur, difficulter alieno}.

Et \emph{l'humide} au contraire, ce qui est facilement retenu dans les bornes d'un autre corps, et difficilement dans les siennes : \emph{quod suo termino difficulter continetur, facile alieno}.

Mais premièrement ces deux définitions conviennent mieux aux corps durs et aux corps liquides qu'aux corps secs et aux corps humides ; car on dit qu'un air est sec et qu'un autre air est humide, quoiqu'il soit toujours facilement retenu dans les bornes d'un autre corps, parce qu'il est toujours liquide; et de plus, on ne voit pas comment Aristote a pu dire que le feu, c'est-à-dire la flamme, était sèche selon cette définition, puisqu'elle s'accommode facilement aux bornes d'un autre corps, d'où vient aussi que Virgile appelle le feu liquide : \emph{Et liquidi simul ignis}. Et c'est une vaine subtilité de dire avec Campanelle que, le feu étant enfermé \emph{aut rumpit, aut rumpitur}: car ce n'est point à cause de sa prétendue sécheresse, mais parce que sa propre fumée l'étouffe, s'il n'a de l'air. C'est pourquoi il s'accommodera fort bien aux bornes d'un autre corps, pourvu qu'il ait quelque ouverture par où il puisse chasser ce qui s'en exhale sans cesse.

Pour le \emph{chaud}, il le définit, ce qui rassemble les corps semblables et désunit les dissemblables : \emph{quod congregat homogenea, et disgregat heterogenea}.

Et le froid, ce qui rassemble les corps dissemblables et désunit les semblables, \emph{quod congregat heterogenea et disgregat homogenea}. C'est ce qui convient quelquefois au chaud et au froid, mais non pas toujours, et ce qui de plus ne sert de rien à nous faire entendre la vraie cause qui fait que nous appelons un corps chaud et un autre froid ; de sorte que le chancelier Bacon avait raison de dire que ces définitions étaient semblables à celle qu'on ferait d'un homme en le définissant ; \emph{un animal qui fait des souliers, et qui laboure les vignes}. Le même philosophe définit la nature : \emph{Principium motus et quietis in eo in quo est}: Le principe du mouvement et du repos en ce en quoi elle est. Ce qui n'est fondé que sur une imagination qu'il a eue que les corps naturels étaient en cela différents des corps artificiels, que les naturels avaient en eux le principe de leur mouvement et que les artificiels ne l'avaient que de dehors ; au lieu qu'il est évident et certain que nul corps ne peut se donner le mouvement à soi-même, parce que la matière étant de soi-même indifférente au mouvement et au repos, ne peut être déterminée à l'un ou à l'autre que par une cause étrangère, ce qui ne pouvant aller à l'infini, il faut nécessairement que ce soit Dieu qui ait imprimé le mouvement dans la matière, et que ce soit lui qui l'y conserve.

La célèbre définition de l'âme paraît encore plus défectueuse. \emph{Actus primus corporis naturalis organici potentia vitam habentis. L'acte premier du corps naturel organique qui a la vie en puissance}. $1$. On ne sait ce qu'il a voulu définir. Car si c'est l'âme en tant qu'elle est commune aux hommes et aux bêtes, c'est une chimère qu'il a définie, n'y ayant rien de commun entre ces deux choses. $2$. Il a expliqué un terme obscur par quatre ou cinq plus obscurs. Et pour ne parler que du mot de \emph{vie}, l'idée qu'on a de la vie n'est pas moins confuse que celle qu'on a de l'âme, ces deux termes étant également ambigus et équivoques.

\bigbreak
Voilà quelques règles de la division et de la définition ; mais quoiqu'il n'y ait rien de plus important dans les sciences que de bien diviser et de bien définir, il n'est pas nécessaire d'en rien dire ici davantage, parce que cela dépend beaucoup plus de la connaissance de la matière que l'on traite que des règles de la Logique.
