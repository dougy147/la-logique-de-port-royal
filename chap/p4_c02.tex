\subsubsection{\centering \Large CHAPITRE II}
\addcontentsline{toc}{section}{\protect\numberline{}{\scshape\bfseries II} - \emph{De la méthode de composition, et particulièrement de celle qu'observent les géomètres}}
\begin{center}\emph{\large\scshape De la méthode de composition, et particulièrement de celle qu'observent les géomètres.}\end{center}

	\lettrine{C}{e} que nous avons dit dans le chapitre précédent nous a déjà donné quelque idée de la méthode de composition, qui est la plus importante, en ce que c'est celle dont on se sert pour expliquer toutes les sciences.

Cette méthode consiste principalement à commencer par les choses les plus générales et les plus simples, pour passer aux moins générales et plus composées. On évite par là les redites ; puisque, si on traitait les espèces avant le genre, comme il est impossible de bien connaître une espèce sans en connaître le genre, il faudrait expliquer plusieurs fois la nature du genre dans l'explication de chaque espèce.

Il y a encore beaucoup de choses à observer pour rendre cette méthode parfaite et entièrement propre à la fin qu'elle doit se proposer, qui est de nous donner une connaissance claire et distincte de la vérité : mais, parce que les préceptes généraux sont plus difficiles à comprendre, quand ils sont séparés de toute matière, nous considérerons la méthode que suivent les géomètres comme étant celle qu'on a toujours jugée la plus propre pour persuader la vérité et en convaincre entièrement l'esprit; et nous ferons voir premièrement ce qu'elle a de bon, et en second lieu ce qu'elle semble avoir de défectueux.

Les géomètres ayant pour but de n'avancer rien que de convaincant, ils ont cru pouvoir y arriver en observant trois choses en général.

La 1. est de \emph{ne laisser aucune ambiguïté dans les termes}, à quoi ils ont pourvu par les définitions des mots dont nous avons parlé dans la première partie.

La 2. est de \emph{n'établir leurs raisonnements que sur des principes clairs et évidents}, et qui ne puissent être contestés par aucune personne d'esprit : ce qui fait qu'avant toutes choses ils posent les axiomes qu'ils demandent qu'on leur accorde, comme étant si clairs, qu'on les obscurcirait en voulant les prouver.

La 3. est de \emph{prouver démonstrativement toutes les conclusions qu'ils avancent}, en ne se servant que des définitions qu'ils ont posées, des principes qui leur ont été accordés comme étant très évidents, ou des propositions qu'ils en ont déjà tirées par la force du raisonnement, et qui leur deviennent après autant de principes.

Ainsi, on peut réduire à ces trois chefs tout ce que les géomètres observent pour convaincre l'esprit, et renfermer le tout en ces cinq règles très importantes.

\begin{center}{\bfseries Règles nécessaires pour les définitions.}\end{center}

$1$.\emph{ Ne laisser aucun des termes un peu obscurs ou équivoques, sans le définir.}
\smallbreak
$2$.\emph{ N'employer dans les définitions que des termes parfaitement connus ou déjà expliqués.}

\begin{center}{\bfseries Pour les axiomes.}\end{center}

$3$.\emph{ Ne demander en axiomes que des choses parfaitement évidentes.}

\begin{center}{\bfseries Pour les démonstrations.}\end{center}

$4$.\emph{ Prouver toutes les propositions un peu obscures, en n'employant à leur preuve que les définitions qui auront précédé, ou les axiomes qui auront été accordés, ou les propositions qui auront déjà été démontrées, ou la construction de la chose même dont il s'agira, lorsqu'il y aura quelque opération à faire.}
\smallbreak
$5$.\emph{ N'abuser jamais de l'équivoque des termes, en manquant d'y substituer mentalement les définitions qui les restreignent et qui les expliquent}.

\bigbreak
Voilà ce que les géomètres ont jugé nécessaire pour rendre les preuves convaincantes et invincibles : et il faut avouer que l'attention à observer ces règles est suffisante pour éviter de faire de faux raisonnements en traitant les sciences, ce qui sans doute est le principal, tout le reste pouvant se dire utile plutôt que nécessaire.


